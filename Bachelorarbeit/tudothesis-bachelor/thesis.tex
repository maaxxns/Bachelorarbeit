%%%%%%%%%%%%%%%%%%%%%%%%%%%%%%%%%%%%%%%%%%%%%%%%%%%%%%%%%%%%%%%%%%%%%%%%%%%%%%%%
%%%%%%%%%%%%%%%%%%   Vorlage für eine Abschlussarbeit   %%%%%%%%%%%%%%%%%%%%%%%%
%%%%%%%%%%%%%%%%%%%%%%%%%%%%%%%%%%%%%%%%%%%%%%%%%%%%%%%%%%%%%%%%%%%%%%%%%%%%%%%%

% Erstellt von Maximilian Nöthe, <maximilian.noethe@tu-dortmund.de>
% ausgelegt für lualatex und Biblatex mit biber

% Kompilieren mit
% latexmk --lualatex --output-directory=build thesis.tex
% oder einfach mit:
% make

\documentclass[
  tucolor,       % remove for less green,
  BCOR=6mm,     % 12mm binding corrections, adjust to fit your binding
  parskip=half,  % new paragraphs start with half line vertical space
  open=any,      % chapters start on both odd and even pages
  cleardoublepage=plain,  % no header/footer on blank pages
]{tudothesis}
% Seitenraender anpassen
 \usepackage{geometry}
 \geometry{
 left=3cm,
 right=3cm,
 top=3cm,
 bottom=4cm,
 bindingoffset=6mm
 }

% Warning, if another latex run is needed
\usepackage[aux]{rerunfilecheck}

% just list chapters and sections in the toc, not subsections or smaller
\setcounter{tocdepth}{1}
\renewcommand*{\chapterheadstartvskip}{\vspace*{.5\baselineskip}}
%------------------------------------------------------------------------------
%------------------------------ Fonts, Unicode, Language ----------------------
%------------------------------------------------------------------------------
\usepackage{fontspec}
\defaultfontfeatures{Ligatures=TeX}  % -- becomes en-dash etc.

% load english (for abstract) and ngerman language
% the main language has to come last
\usepackage[american]{babel}

% intelligent quotation marks, language and nesting sensitive
\usepackage[autostyle]{csquotes}

% microtypographical features, makes the text look nicer on the small scale
\usepackage{microtype}

%------------------------------------------------------------------------------
%------------------------ Math Packages and settings --------------------------
%------------------------------------------------------------------------------

\usepackage{amsmath}
\usepackage{amssymb}
\usepackage{mathtools}

% Enable Unicode-Math and follow the ISO-Standards for typesetting math
\usepackage[
  math-style=ISO,
  bold-style=ISO,
  sans-style=italic,
  nabla=upright,
  partial=upright,
]{unicode-math}
\setmathfont{Latin Modern Math}

% nice, small fracs for the text with \sfrac{}{}
\usepackage{xfrac}


%------------------------------------------------------------------------------
%---------------------------- Numbers and Units -------------------------------
%------------------------------------------------------------------------------

\usepackage[
  locale=UK,
  separate-uncertainty=true,
  per-mode=symbol-or-fraction,
]{siunitx}
\sisetup{math-micro=\text{µ},text-micro=µ}

%------------------------------------------------------------------------------
%-------------------------------- tables  -------------------------------------
%------------------------------------------------------------------------------

\usepackage{booktabs}       % \toprule, \midrule, \bottomrule, etc

%------------------------------------------------------------------------------
%-------------------------------- graphics -------------------------------------
%------------------------------------------------------------------------------

\usepackage{graphicx}
% currently broken
% \usepackage{grffile}
\usepackage{pdfpages} %append pdfs 
% allow figures to be placed in the running text by default:
\usepackage{scrhack}
\usepackage{float}
\floatplacement{figure}{htbp}
\floatplacement{table}{htbp}

% keep figures and tables in the section
\usepackage[section, below]{placeins}


%------------------------------------------------------------------------------
%---------------------- customize list environments ---------------------------
%------------------------------------------------------------------------------

\usepackage{enumitem}

%------------------------------------------------------------------------------
%------------------------------ Bibliographie ---------------------------------
%------------------------------------------------------------------------------

\usepackage[
  backend=biber,   % use modern biber backend
  autolang=hyphen, % load hyphenation rules for if language of bibentry is not
                   % german, has to be loaded with \setotherlanguages
                   % in the references.bib use langid={en} for english sources
  sorting=none,
]{biblatex}
\addbibresource{references.bib}  % the bib file to use
\DefineBibliographyStrings{english}{andothers = {{et\,al\adddot}}}  % replace u.a. with et al.

% Last packages, do not change order or insert new packages after these ones
\usepackage[pdfusetitle, unicode, linkbordercolor=tugreen, citebordercolor=tugreen]{hyperref}
\usepackage{bookmark}
\usepackage[shortcuts]{extdash}

%------------------------------------------------------------------------------
%-------------------------    Angaben zur Arbeit   ----------------------------
%------------------------------------------------------------------------------

\author{Max Koch}
\title{Generation and time-resolved detection of terahertz radiation}
\date{2022}
\birthplace{Bad Driburg}
\chair{AG Wang}
\division{Fakultät Physik}
\thesisclass{Bachelor of Science}
\submissiondate{26.07.2022}
\firstcorrector{Prof.~Dr.~Zhe Wang}
\secondcorrector{Prof.~Dr.~Mirko Cinchetti}

% tu logo on top of the titlepage
\titlehead{\includegraphics[height=1.5cm]{logos/tu-logo.pdf}}

\begin{document}
\frontmatter
% \input{content/hints.tex}
\maketitle

% Gutachterseite
\makecorrectorpage

% hier beginnt der Vorspann, nummeriert in römischen Zahlen
\thispagestyle{plain}

\section*{Abstract}
\begin{foreignlanguage}{english}
This thesis gives an overview about the time resolved generation and detection of Terahertz radiation. 

\end{foreignlanguage}

\tableofcontents

\mainmatter
% Hier beginnt der Inhalt mit Seite 1 in arabischen Ziffern
\chapter{Introduction}
For many applications the spectroscopy is an easy and efficient way to explore the properties of diffrent materials.
That is why the spectroscopy finds applications in many fields, from biology and medizin to physics and engeneering.
A big leap in the science of spectroscopy was made in 1895 when Wilhelm Conrad Röntgen first discovered the X-Ray radiation \cite{roentgen}.
With this new form of electromagnetic radiation many new applications in medizin and physics evolved and a new area of spectroscopy began.
Over the years more and more sources and detectors of various electromagnetic radiation were developed.
But there is still a gap in the spectrum.
It is known as the terahertz gap and describes the incapability of producing terahertz radiation form around $0.3-10\,\si{\tera\hertz}$ in an efficient and easy way \cite[157--159]{THzgap_applications}.
Since some years there have been efforts to close the terahertz gap and develope methodes of producing terahertz radiation.
The reason for this is that terahertz radiation has some interesting properties that would be of interest for diffrent fields.
Its non ionizing properties make it of particular interest in diffrent areas of medizin, because it does not have an harming effect on tissue\cite[161--162]{THzgap_applications}.
For instence in dermatology, where terahertz imaging processes can be used to determine the moisture content of the skin or might even give diagnostic methodes to detect skin cancer\cite{terahertz_dermatology}. 
Becaus of the same reason, and a specific spectroscopic fingerprint for most metals, the terahertz radiation is also of interest in the security sector.
Weapons like guns, knifes and even most soft explosives can be detected using terahertz spectroscopy \cite[162]{THzgap_applications}\cite{thz_explosive_detec}.


% %%%%%%%%%%%%%%%%%%%%%%%%%%%%%%%%%%%%%%%%%%%%%%%%%%%%%%%%
Für viele Anwendungen ist die Spektroskopie eine der effizientesten Möglichkeiten um gewisse Materialeigenschaften einer Probe zu bestimmen.
So haben die Wissenschaftler in vielen Bereichen auf die Methode zurück gergriffen und von ihrer relativen einfachen Umsetzung profitiert.
Ein großer Durchbruch in dem Bereich der Spektroskopie ist der Physik gelangt als Wilhelm Conrad Röntgen 1895 %(Entdeckung Röntgen Strahlung)
die Röntgenstrahlen entdeckt hat.
Durch diese neue Form der Strahlung konnte die Wissenschaft in neue Bereiche eintreten.
Über die Jahre wurden dem Spektrum mit dem Spektroskopie möglich ist immer neue Strahlungsarten hinzugefügt.
Doch eine Lücke im Spektrum besteht bis heute.
Sie ist als THz-Lücke bekannt und bezeichnet die bis heute noch ineffieziente und relativ komplizierte Produktion von THz-Strahlung und Detektion.
Seit einigen Jahren werden nun immer mehr Bestrebungen daran gesetzt diese Lücke zu schließen.
Denn THz Strahlung ist für verschiedene wissenschaftliche Bereiche von großem Interesse.
So könnten in der Medizin neue Methoden zur Dutchleuchtung organischer Materie entwickelt werden ohne das die Materie dabei durch Ionisierung zerstört wird.
Des weiteren ist sie auch für die Physik vom großem Interesse da durch elektromagnetische Wellen im THz-Frequenz bereich bestimme Gitterschwingungen angeregt werden.
Ein großer Durchbruch gelang als 1996 das erste mal THz durch optical rectification in anorganischen Kristallen erzeugt wurde. %\cite{ZnTe_Nahata_Weling_1996}
Dieser Durchbruch gelang durch das bestrahlen eines Zinc Telluride Kristalls mit einem Laser in 800 nm Bereich.


%%%%%%%%%%%%%%%%%%%%%%%%%%%%%%%%%%%%%%%%%%%%%%%%%%%%%%%%%%

At the beginning of this century a big leap in the efficient production of $\si{\tera\hertz}$ has been made.
This leap was partly made possible through the usage of non linear crytsals such as zinc telluride as emitters.
The nonlinear effect called optical rectification is exploited to produced the $\si{\tera\hertz}$-radiation in these crystalls.

% Motivation
% Hier folgt eine kurze Einleitung in die Thematik der Bachelorarbeit.
% Die Einleitung muss kurz sein, damit die vorgegebene Gesamtlänge der 
% Arbeit von 25 Seiten nicht überschritten wird. 
% Die Beschränkung der Seitenzahl sollte man ernst nehmen,
% da Überschreitung zu Abzügen in der Note führen kann. 
% Um der Längenbeschränkung zu genügen, darf auch nicht an der Schriftgröße,
% dem Zeilenabstand oder dem Satzspiegel (bedruckte Fläche der Seite) manipuliert werden.


\chapter{Theory}
\section{Terahertz radiation}

\section{Optical rectification}\label{sec:optic_ref}
To produce $\si{\tera\hertz}$ radiation we take advantage of optical rectification.
Optical rectification is a second order non linear effect and thus can be just observed in non linear materials.
This effect causes a DC polarization in the crystal that if moduled correctly causes $\si{\tera\hertz}$ radiation.
The DC polarization occurs when an outer electric field interacts with the crytal.
Because of the nonlinear structure and in turn the anharmonic potential of crystal charges, the charges oscillate further into one dircetion then the other.
The displacement of charge is what causes the DC polarization.
To discuss the effect in detail we take a look at the polarization

\begin{equation}
P = \chi(E) E \epsilon_0
\end{equation}

which is directly proportional to the elctric field $E$ and the susceptibilty $\chi(E)$.
Which in turn can be expanded to 

\begin{equation}
    \chi(E) = \chi_0 + \chi_1 E +\chi_2 E^2 + ...   \, .
\end{equation}

As describe earlier optical rectification is a second order effect and describe by the $P_\text{nl} = \chi_2 E^2$ part.
Because the laser produces electro magnetic radiation at a whole bandwith of frequencies $\Delta\omega$ we have to consider all of those electric fields inside the crystal.
To simplify we first take a look at just two electric fields.
One of those electric fields oscillats at frequency $\omega_1$ and the other at frequency $\omega_2$.
The resulting second order polarization term 

\begin{equation}
    P_\text{nl} = \chi_2 \epsilon_0 \frac{E_0^2}{2}\left(cos((\omega_1 - \omega_2)t) + cos((\omega_1 + \omega_2)t)\right)
\end{equation}

shows a cosin with a diffrence dependency $\omega_1-\omega_2$ and one with a sum dependency $\omega_1+\omega_2$.
The one with the difference dependens results in the production of $\si{\tera\hertz}$ radiation. % the one with the sum depends is important for second harmonic generation
To fully understand the effect it is necessary to describe to susceptibilty as a third rank tensor $\chi_\text{ijk}(E)$ aswell as the electric fields and polarization as vectors $\vec{E}$ and $\vec{P}_\text{i}$.
With these it is now possible to write the $\text{i}$-th component of the second order non linear polarization vector as 

\begin{equation}
    P_\text{nl, i}^{\omega_1 - \omega_2} = \chi_\text{ijk}(\omega_1-\omega_2)E_\text{j}(\omega_1)E_\text{k}(\omega_2)\epsilon_0
    \label{eq:polarization_tensor_sus}
\end{equation}

in which the two electric fields depent on the diffrente frequencies $\omega_1$ and $\omega_2$.
% chi ist tensor 
% 
% wenn omega1-omega2=0  dann DC polarisation
% wenn omega2 = 0 dann ändert sich die polarisation mit frequence omega1

\cite[289--291]{book_optical_rectification}

\section{Electro-optic sampling}\label{sec:eos}
To detect the $\si{\tera\hertz}$-electric field the electro-optic effect also known as pockels effect is used.
Through this effect a birefringence is induced in the detection crystal which in turn changes the polarization of the polarization of the probe beam.
The retardation of the polarization 
\begin{equation}
    \Gamma \propto \frac{1}{\lambda} n_0^3 l r E_\text{THz}
\end{equation}
is directly proportional to the electric field. 
It is also proportional to the wavelength $\lambda$, the crystal thickness $l$, the electrooptic coefficient $r$ and the refractive index $n_0$ \cite{wiki_book}.  

Through measurement of the change in polarization it is than possible to determine the intensity of the electric field of the $\si{\tera\hertz}$-pump beam.


\section{ZnTe}
One of the crystalls we use to generate $\si{\tera\hertz}$ radiation is zinc telluride (ZnTe). 
It allows the production of a wideband coherent $\si{\tera\hertz}$ Field through means of optical rectification \ref{sec:optic_ref} \cite{ZnTe_Nahata_Weling_1996}.
Because optical rectification is a non linear effect it is important to aline the laser to the $<110>$ achis of the crystal.


%(Nahata, A., Weling, A. S., and Heinz, T. F., A wideband coherent terahertz spectroscopy
% system using optical rectification and electrooptic sampling, Appl. Phys. Lett.,
% 69, 2321, 1996.).
% (Han, P. Y., and Zhang, X. C., Free-space coherent broadband terahertz time-domain
% spectroscopy, Meas. Sci. Technol., 12, 2001, 1747.)

\chapter{Setup and execution}
This chapter will discuss the setup that is utilized to generate and detect the $\si{\tera\hertz}$-electric field, as well as the method that is used to achieve a time-resolved measurement of the $\si{\tera\hertz}$ pulse.

\section{Setup}
\label{sec:setup}
\begin{figure}
    \centering
    \includegraphics[width=\textwidth]{Plots/Aufbau.pdf}
    \caption{The setup that generates $\si{\tera\hertz}$ radiation.
    The acronyms OAP and QWP stand for parabolic off-axis parabolic mirror and quarter-wave plate. }
    \label{fig:setup}
\end{figure}
To generate $\si{\tera\hertz}$ radiation the setup as illustrated in Figure \ref{fig:setup} is employed.
An $\SI{800}{\nano\meter}$ Ti-sapphire laser is used to generate the necessary laser radiation.
It generates a pulsed laser beam with a power of around $\SI{6.9}{\W}$ and a frequency of about $\SI{1000}{\Hz}$.
However, because the laser feeds into several setups only a fraction of the initial power reaches each setup.
This setup receives between $\SI{291}{\milli\W}$ and $\SI{579}{\milli\W}$ depending on the configurations of the other setups. 
The incoming laser beam with a pulse length of $\SI{100}{\femto\second}$ is split into a pump and a probe beam.
\\
The pump beam receives $90\%$ of the initial laser beam power, and the probe beam the remaining $10\%$.
The pump beam passes through a chopper which modulates the laser to a frequency of $\SI{280}{\hertz}$.
This way the lock-in amplifier that is utilized for the measurement can be triggered on that frequency.
After the chopper, a delay stage is placed.
With this, the pump beam path length can be altered.
This allows a time-resolved detection of the $\si{\tera\hertz}$ pulse, a detailed description can be found in section \ref{sec:time_domain}.
\\
To focus the beam on the crystal a lens with a focal length of $\SI{40}{\centi\meter}$ is positioned $\SI{15}{\centi\meter}$ in front of the crystal.
Ultimately, the beam hits the emitter crystal.
Depending on the measurement a $\SI{1}{\milli\meter}$ ZnTe or a $\SI{0.3}{\milli\meter}$ GaP crystal is used.
The GaP crystal is thinner because it has a smaller coherence length at most of the $\si{\tera\hertz}$ frequencies compared to ZnTe when an $\SI{800}{\nano\meter}$ pump laser is used.
\\
With this part of the setup $\si{\tera\hertz}$ radiation can already be generated, just the detection is missing.
\\\\
To detect the $\si{\tera\hertz}$ radiation it first needs to be separated from the laser light that is transmitted by the emitter crystal.
If the pump laser beam is not separated from the $\si{\tera\hertz}$ beam the intense pump laser beam would damage the photodiodes.
In this design, a silicon wafer blocks the laser beam and transmits the $\si{\tera\hertz}$ radiation, which is generated by the emitter crystal.
Two parabolic off-axis mirrors focus the $\si{\tera\hertz}$ beam onto the detection crystal.
\\
As the detection unit, a $\SI{1}{\milli\meter}$ ZnTe crystal is utilized.
To place the detection crystal right at the focus of the parabolic mirror, a light source reflects into the OAP.
The detection crystal is then moved to the focus of that light cone.
The same approach is used to place the emitter crystal at the correct distance from the first OAP.
\\\\
The probe beam, whose power is reduced by an optical density filter, passes through the back of the second OAP, which has a small hole in it.
From there it hits the detection crystal at roughly the same position as the pump $\si{\tera\hertz}$ beam.
\\
To make sure that this is the case, the emitter crystal and the silicon wafer are removed and a strong optical density filter is placed into the pump beam.
The filter is necessary because the pump laser beam would damage the OAPs otherwise.
Now the OAPs reflect the actual laser pump beam onto the detection crystal.
With an infrared viewing camera, it is then possible to see if the pump laser and probe laser overlap.
If that is the case the pump $\si{\tera\hertz}$ beam should be at least roughly at the same position as the pump laser beam.
This means the $\si{\tera\hertz}$ pump beam also overlaps with the probe beam.
\\
After the alignment of the pump and probe beam, the emitter crystal and the silicon wafer are placed back into the setup.
With this, the first part of the detection unit is done.
The $\si{\tera\hertz}$ radiation now changes the polarization of the probe beam.
\\\\
The next step is to measure the change in polarization.
As described in the section \ref{sec:eos}, the polarization of the probe beam changes depending on the $\si{\tera\hertz}$ electric field inside the detector.
After the change in polarization, the probe beam passes through a quarter-wave plate, which changes the linear polarization of the beam to an elliptic polarization.
The quarter-wave plate is also used to balance the signal while there is no $\si{\tera\hertz}$ electric field.
The effect of the quarter-wave plate is further described in section \ref{sec:qwp}.
\\
This way the photodiodes have an equal input if there is no $\si{\tera\hertz}$ electric field inside the detector crystal.
The beam is then split into its horizontal and vertical polarization components by a Wollaston prism, which is further explained in subsection \ref{sec:wollaston}.
The two separate beams are passed into photodiodes.
\\\\
Depending on the polarization of the probe beam after the detection crystal, the intensity of the probe beams after the Wollaston prism will change.
The signal of the photodiodes $A$ and $B$ is passed into a lock-in amplifier.
The lock-in amplifier is further explained in section \ref{sec:data_acq}.
The lock-in then calculates the difference between both signals $A-B$.
The bigger the difference $\symup{\Delta}I = A-B$ the stronger the $\si{\tera\hertz}$ electric field in the detection crystal.
\\\\
With the setup now done the signal has to be optimized.
To do this the stage is moved to the position of the strongest signal.
The goal is to lower the noise or increase the peak height of the signal, through changes in the setup.
As shown in the subsection \ref{sec:emitters} the orientation of the crystals is of great importance.
That is why the first step is to rotate the emitter crystal, to the point where the $(001)$ crystal direction is parallel to the polarization of the pump beam.
After the correct orientation is found the process is repeated for the detection crystal.
\\\\
Now the pump laser beam position on the emitter crystal needs to be adjusted.
For this, the laser mirror before the crystal is rotated until the signal is at its greatest strength.
Furthermore, the OAPs are adjusted by moving their position and orientation slightly.
The signal can also be increased by rotating the detection crystal so that the probe beam does not hit it perpendicularly.
\\
The reason is that the angle limits the effect of reflections inside the crystal.
This means the best angle would be the brewster angle however, this angle can not be utilized in this setup because the crystal surface is too small. 
The best results are achieved at an angle of $\SI{40}{\degree}$.
Finally, the overlap of the $\si{\tera\hertz}$ pump beam and laser probe beam on the detection crystal is corrected.

\FloatBarrier
\subsection{Quarter-wave plate}
\label{sec:qwp}
To balance the two photodiodes, a quarter-wave plate is utilized.
This component changes the polarization of the incoming electric field into a circular or elliptical polarization, depending on its orientation.
The quarter-wave plate introduces a phase shift of $\symup{exp}(i\frac{\pi}{2}) = i$ into the electric field of one polarization component of the wave.
This way the incoming wave $\vec{E}$, that consists of a horizontally polarized component $E_\text{h}\vec{h}$ and a vertically polarized component $E_\text{v}\vec{v}$, can be written as
\begin{equation}
    \vec{E} = (E_\text{v}\vec{v} + E_\text{h}\vec{h})\symup{e}^{i(kx-\omega t)}
\end{equation}
and changes after the quarter wave plate to 
\begin{equation}
    \vec{E} = (E_\text{v}\vec{v} + i E_\text{h}\vec{h})\symup{e}^{i(kx-\omega t)}
\end{equation}
if the phase shift is induced along the horizontally polarized axis it will result into an elliptical polarization as seen in the lower part of figure \ref{fig:qwp}.
It works the same way if the phase shift is induced at the vertical axis.
If now the electric field strengths in both polarization directions are the same $E_\text{v} = E_\text{h} = E$ the result of the phase shift is a circularly polarized wave \cite{qwp_book}
\begin{equation}
    \vec{E} = E(\vec{v} + i\vec{h})\symup{e}^{i(kx-\omega t)} \, .
\end{equation}
By rotating the quarter-wave plate it is then possible to induce the phaseshift to both polarization components with different extensions.
For the right balancing, the quarter-wave plate changes the polarization of the probe beam to a circular one, if there is no $\si{\tera\hertz}$ signal and to an elliptical, if there is a signal.
\begin{figure}
    \centering
    \includegraphics[width=0.4\textwidth]{refferenced_pic/qwp.png}
    \caption{Depending on the initial polarization the quarter-wave plate changes the polarization of the electric field to a circular (upper picture) or elliptical polarization (lower picture).}
    \label{fig:qwp}
\end{figure}
\FloatBarrier
\subsection{Wollaston prism}
\label{sec:wollaston}
The probe beam polarization, after the detector crystal, carries almost all the information about the $\si{\tera\hertz}$ electric field strength.
Because it is not possible to measure the polarization of the probe beam directly, it is necessary to split the probe beam into its two polarization components.
This happens with a Wollaston prism, shown in figure \ref{fig:wollaston}.
\\
It consists of two prisms that are glued together.
The light that passes through the prism then refracts at the point where the two prisms touch.
From the Fresnel equations, it is possible to derive, that dependent on the polarization of the light, it gets refracted differently.
This means that the beam splits into a horizontally and a vertically polarized beam, with an angle of about $\SI{20}{\degree}$.
The intensity of both beams is then measurable.
Furthermore, it is possible to derive the change in polarization that occurred at the detector crystal \cite{wollaston_prism}. 
\begin{figure}
    \centering
    \includegraphics[width=0.5\textwidth]{Plots/wollaston_prism.png}
    \caption{The picture shows a Wollaston prism through which an unpolarized beam passes.
    This beam refracts inside the prism which splits it into two beams with perpendicular polarizations s and p.
    The two beams then leave the prism with an angle between them \cite{wollaston_prism}.}
    \label{fig:wollaston}
\end{figure}

\subsection{Data acquisition}
\label{sec:data_acq}
As described in section \ref{sec:wollaston} the probe beam is split into its two polarization components.
To detect the intensity of the beams, two photodiodes are utilized.
These detect the time-integrated intensity of a laser pulse.
They put out a current which is fed into a lock-in amplifier.
\\
The lock-in amplifier is used to detect very weak signals with a specific frequency.
For this, a reference frequency $\omega_\text{chopper}$, as well as the signal $V_\text{sig}\sin(\omega_\text{chopper}t + \theta_\text{sig})$ with the phaseshift $\theta_\text{sig}$ are given into the lock-in amplifier.
The lock-in amplifier then multiplies the signal with a reference signal $V_\text{lock}\sin(\omega_\text{lock} + \theta_\text{ref})$, which is produced by the lock-in amplifier itself.
The result of the multiplication is
\begin{equation}
    V = V_\text{sig}\sin(\omega_\text{chopper}t + \theta_\text{sig})V_\text{lock}\sin(\omega_\text{chopper} + \theta_\text{ref})
\end{equation}
which can also be written as 
\begin{equation}
    V = \frac{1}{2} V_\text{sig} V_\text{lock} \left ( \cos([\omega_\text{sig} - \omega_\text{lock}]t +\theta_\text{sig} - \theta_\text{lock}) - \cos([\omega_\text{sig} + \omega_\text{lock}]t +\theta_\text{sig} + \theta_\text{lock})\right ) \, .
\end{equation}
The result is passed through a low pass filter, which removes the AC signals.
This means that only if $\omega_\text{sig}$ and $\omega_\text{lock}$ are equal does anything come out of the filter.
In that case the result 
\begin{equation}
    V = \frac{1}{2}V_\text{sig}V_\text{lock}\cos(\theta_\text{sig}-\theta_\text{lock})
\end{equation} 
can be detected and measured without the noise from other frequencies.
Because $V_\text{lock}$ is already known it is possible to calculate the signal.
The reference frequency that is used in this setup is the frequency at which the chopper modulates the pump beam.
\section{Time-resolved THz spectroscopy}
\label{sec:time_domain}
To achieve a time-resolved measurement of the $\si{\tera\hertz}$ pulse a delay stage is built into the setup.
With the delay stage, the path length of the pump beam can be changed.
This way a time delay between the pump pulse and the probe pulse is achieved.
Because the velocity of electromagnetic radiation in air is roughly the same as in a vacuum, the vacuum speed can be used to calculate the time delay   
\begin{equation}
    \symup{\Delta} t = \frac{\symup{\Delta} s}{c}
\end{equation}
with $c$ as the speed of light, and $\symup{\Delta} s$ the difference in the beam path length.
The time-resolved measurement now works by taking one measurement at the delay stage position $s_1$, which corresponds to $t_1$.
Then the stage is moved to position $s_2$ this lengthens or shortens the beam path of the pump beam.
At stage position $s_2$ a new measurement is taken which corresponds to time $t_2$.
\\
Because just the pump beam path is changed, it arrival time at the crystal changes, compared to the probe beam.
This results in a time delay $\symup{\Delta} t = t_1-t_2$.
This process is repeated until the whole $\si{\tera\hertz}$-pulse is probed.
For this, the delay stage has to be moved $\SI{4}{\milli\meter}$ in total.
To get a good resolution $1000$ measurements with a distance of $\SI{0.004}{\milli\meter}$ between them are taken.
This distance corresponds to a time resolution of $\SI{13.34}{\femto\second}$.

\section{Execution}
\label{sec:execution}
This section describes how the measurements are taken. 
It will be discussed what changes to the setup in figure \ref{fig:setup} are made to achieve the wanted results.
\\\\
Before any measurements are taken the photodiodes have to be balanced.
For this, a chopper is placed inside the probe beam, which modulates its frequency to $\SI{388}{\hertz}$.
The lock-in amplifier gets triggered on that frequency, while its inputs are currents $A$ and $B$ of the two photodiodes.
\\
It is necessary to move the stage to a point where $\si{\tera\hertz}$-pulse and probe pulse do not overlap.
So that the polarization of the probe beam does not change.
\\
Now the quarter-wave plate is being rotated until the output of both photodiodes $A-B = 0$.
To account for the noise the quarter-wave plate is rotated until the average of $A-B$ over a small time period is zero.
If that is the case the probe beam is circularly polarized after the quarter-wave plate, while there is no $\si{\tera\hertz}$ signal.
%%%%%%%%%%%%%%%%%%%%%%%%%%%%%%%%%%%%%%%%%%%%%%%%%%%%%%%%%%%%%
\subsection{Power measurements}
\label{sec:power}
The goal is to determine the effect of different pump fluences on the $\si{\tera\hertz}$ production.
For this, it is necessary to change the power of the pump beam before the emitter crystal.
For that various density filters are placed inside the pump beam path.
All the utilized filters can be seen in Table \ref{tab:filters}.
\\
After the placement behind the chopper, the pump power is measured in front of the emitter crystal.
Then the power meter is removed so that the pump beam hits the emitter crystal.
\\
Now a short measurement with a low time resolution is taken, to roughly determine the position of the signal peak.
After the peak position is found a time-resolved measurement, as described in section \ref{sec:time_domain} is taken.
For this, the stage is moved to its starting position at about $\SI{1}{\milli\meter}$ before the signal peak.
It stops $\SI{3}{\milli\meter}$ after the peak.
These positions are chosen to only scan the pulse and the oscillations that occur after the pulse.
\\
It should also be said that reflections inside the emitter and the detector crystals cause a so-called pre- and post-pulse.
These are unwanted byproducts of the initial pulse and will interfere with later calculations.
This makes it important to only measure the initial pulse but not the pre- and post-pulse.
After the measurement, the filter is swapped out with a different one.
The process is started over until all fluences are measured.
\\\\
To calculate the peak power of the $\si{\tera\hertz}$ electric field, it is necessary to measure the spot size.
Usually, this would be done with a $\si{\tera\hertz}$-optical camera, which automatically measures the spot size.
In the case of this setup, the $\si{\tera\hertz}$ electric field is too weak to be detected by the camera.
The solution to this problem is to measure the spot size directly.
\\
To do that an aperture is paced inside the probe beam.
Then the delay stage is moved to the position of the greatest electric field, where it is fixed.
Now the $A-B$ value of the output of the photodiodes is measured.
Usually the probe beam spot on the detector is bigger than the $\si{\tera\hertz}$ spot.
Through closing the aperture the probe beam spot size becomes smaller until it has the same size as the $\si{\tera\hertz}$ spot.
If the probe beam spot becomes smaller than the $\si{\tera\hertz}$ spot the signal strength starts to drop.
This means the aperture opening has to be reduced until the signal loses strength. 
At which point the opening area of the aperture is the same as the $\si{\tera\hertz}$ spot on the detector crystal.
\subsection{Electric field measurements}
\label{sec:field}
To determine the electric field of the $\si{\tera\hertz}$ radiation it is necessary to measure the intensity of both probe beams after the Wollaston prism.
With this the equation \eqref{eq:electricfield_A_B} can be used to calculate the electric field value.
Because just the maximum field strength is of particular interest just the greatest $\symup{\Delta}I = A-B$ value is necessary to measure.
For this, the delay stage is moved into the peak of the pulse.
There is the position of the greatest $A-B$ value and a measurement of the difference is taken.
This process has to be repeated for all pump power values.
\\\\
After all the differences are measured, the sum of the intensities $I = A + B$ needs to be measured.
The delay stage is moved to a position where there is no $\si{\tera\hertz}$ signal.
Because the lock-in amplifier can not calculate the sum of $A$ and $B$ it is necessary to measure them separately.
The input of the lock-in amplifier is changed to just one of the outputs of the photodiodes, either $A$ or $B$.
After one, the other is measured as well.
Because the sum $A+B$ is the same for all $\si{\tera\hertz}$ electric fields, it can be used to norm every difference.
After $A-B$ and $A$ and $B$ are measured the electric field can be calculated.

\chapter{Results}
This section will present the results of the measurements with ZnTe and GaP as emitter crystals.
Calculations will be made to determine the electricfield strength of the produced $\si{\tera\hertz}$ radiation aswell as the Power of said radiation.
Futher a comparission between the different spectras of ZnTe and GaP will be shown and the efficiency in producing radiation will be discussed.

\section{Zinc telluride}
To take this measurments a $\SI{1}{\milli\meter}$ thick ZnTe crystal is used as the emitter.
Another $\SI{1}{\milli\meter}$ thick ZnTe crystal is used as the detector.
\subsection{Fluence measurments}

\subsection{Electric field measurments}


\section{Gallium phosphide}
\subsection{Fluence measurments}
\subsection{Electric field measurments}


\section{Comparisson}
\chapter{Summary}
This thesis presented the time resolved generation and detection of $\si{\tera\hertz}$ radiation in ZnTe and GaP.
It showed the measured electro-optic sampling data and its Fourier-Transformations, to confirm the generation of $\si{\tera\hertz}$ radiation.
Besides that the pump power dependency of the $\si{\tera\hertz}$ electric field and the $\si{\tera\hertz}$ electric field power has been measured for both ZnTe and GaP as emitter crystals.
\\\\
Both ZnTe and GaP did indeed generate $\si{\tera\hertz}$ radiantion, with ZnTe being the better emitter crystal in terms of electric field strength, peak electric field power and conversion efficiency.
However the GaP crystal was $\SI{0.7}{\milli\meter}$ thinner than the ZnTe crystal, which makes a comparission between them not accurate.
It was also not ideal, that the ZnTe crystal had two burn marks on it that definetly influenced its capability of producing $\si{\tera\hertz}$ radiation.
The switch in initial laser power had a negativ effect on the consistency of the measurements and should be avoided in future experiments.
Either by lowering the pump power just by adding more density filters in the pump beam path or by taking more time for the aligment, to make sure both beams actually take the same path.
For the future more measurments with comparable crystal should be taken to better determine how the spectras and efficiency in production of ZnTe and GaP differ from one another.

\chapter*{Acknowledgments}
\thispagestyle{empty}

...

\clearpage
\appendix
% Hier beginnt der Anhang, nummeriert in lateinischen Buchstaben
\chapter{Appendix}
\section{Filters that were used for the fluence measurements}
\begin{table}
    \centering
    \begin{tabular}{ccc}
        \toprule
        Filter & Transmission (\%)  & power after filter ($\si{\milli\watt}$) \\
        \midrule
        None & 100 & 270.0 \\
        NE02A-B & 67.94 &  186.4 \\
        NE03A-B & 54.38 & 135.0  \\
        NE02A-B + NE03A-B & 36.94 & 90.5 \\
        NE07A-B & 29.26 & 81.6 \\
        NE09A-B & 20.15 & 56.4 \\
        NE13A-B & 9.01 & 24.6 \\
        \bottomrule
    \end{tabular} 
    \caption{The filters that are used to lower the power of the pump beam. In the second collum the transmission of every filter is shown. In the third collum the pump power after the filter and after the chopper is shown.
    Note that the power values corresponde to the highest laser power that is available with the given setup. Measurements with lower initial laser power are also taken. All the filters that are used are supplied by thorlabs \cite{thorlabs}.}
    \begin{tabular}{ccc}
        \toprule
        Filter & Transmission (\%)  & power after filter ($\si{\milli\watt}$) \\
        \midrule
        None & 100 & 129.5 \\
        NE02A-B & 67.94 &  75.0 \\
        NE03A-B & 54.38 & 58.2  \\
        NE02A-B + NE03A-B & 36.94 & 45.1 \\
        NE07A-B & 29.26 & 36.8 \\
        NE09A-B & 20.15 & 26.5 \\
        NE13A-B & 9.01 & 10.24\\
        \bottomrule
    \end{tabular} 
    \caption{The filters that are used to lower the power of the pump beam. In the second collum the transmission of every filter is shown. In the third collum the pump power after the filter and after the chopper is shown.
    Note that the power values corresponde to a lower laser power than the one in the upper table. All the filters that are used are supplied by thorlabs \cite{thorlabs}}
    \label{tab:filters}
\end{table}

\begin{figure}
    \centering
    \includegraphics[width=\textwidth]{Plots/burned_crystal.jpeg}
    \caption{The $\SI{1}{\milli\meter}$ ZnTe crystal that is used as the emitter.
    Two burned spots can cleary be seen on the crystal surface.}
    \label{fig:ZnTe_burned}
\end{figure}

\backmatter
\printbibliography

\cleardoublepage
\includepdf{content/Eidesstaatlich_versicherung_unterschieben_druck.pdf}
%\input{content/eid_versicherung.tex}
\end{document}
