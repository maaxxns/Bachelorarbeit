\chapter{Introduction}
For many applications the spectroscopy is an easy and efficient way to explore the properties of diffrent materials.
That is why the spectroscopy finds applications in many fields, from biology and medizin to physics and engeneering.
A big leap in the science of spectroscopy was made in 1895 when Wilhelm Conrad Röntgen first discovered the X-Ray radiation \cite{roentgen}.
With this new form of electromagnetic radiation many new applications in medizin and physics evolved and a new area of spectroscopy began.
Over the years more and more sources and detectors of various electromagnetic radiation were developed.
But there is still a gap in the spectrum.
It is known as the Terahertz gap and describes the incapability of producing Terahertz radiation form around $0.3-10\,\si{\tera\hertz}$ in an efficient and easy way \cite[157--159]{THzgap_applications}.
% Some Terahertz frequencies are not accessible to this day.
Since some years there have been efforts to close the Terahertz gap and develope methodes of producing Terahertz radiation.
The reason for this is that Terahertz radiation has some interesting properties that would be of interest for diffrent fields.
Its non ionizing properties make it of particular interest in diffrent areas of medizin, because it does not have an harming effect on tissue.
Because 


% %%%%%%%%%%%%%%%%%%%%%%%%%%%%%%%%%%%%%%%%%%%%%%%%%%%%%%%%
Für viele Anwendungen ist die Spektroskopie eine der effizientesten Möglichkeiten um gewisse Materialeigenschaften einer Probe zu bestimmen.
So haben die Wissenschaftler in vielen Bereichen auf die Methode zurück gergriffen und von ihrer relativen einfachen Umsetzung profitiert.
Ein großer Durchbruch in dem Bereich der Spektroskopie ist der Physik gelangt als Wilhelm Conrad Röntgen 1895 %(Entdeckung Röntgen Strahlung)
die Röntgenstrahlen entdeckt hat.
Durch diese neue Form der Strahlung konnte die Wissenschaft in neue Bereiche eintreten.
Über die Jahre wurden dem Spektrum mit dem Spektroskopie möglich ist immer neue Strahlungsarten hinzugefügt.
Doch eine Lücke im Spektrum besteht bis heute.
Sie ist als THz-Lücke bekannt und bezeichnet die bis heute noch ineffieziente und relativ komplizierte Produktion von THz-Strahlung und Detektion.
Seit einigen Jahren werden nun immer mehr Bestrebungen daran gesetzt diese Lücke zu schließen.
Denn THz Strahlung ist für verschiedene wissenschaftliche Bereiche von großem Interesse.
So könnten in der Medizin neue Methoden zur Dutchleuchtung organischer Materie entwickelt werden ohne das die Materie dabei durch Ionisierung zerstört wird.
Des weiteren ist sie auch für die Physik vom großem Interesse da durch elektromagnetische Wellen im THz-Frequenz bereich bestimme Gitterschwingungen angeregt werden.
Ein großer Durchbruch gelang als 1996 das erste mal THz durch optical rectification in anorganischen Kristallen erzeugt wurde. %\cite{ZnTe_Nahata_Weling_1996}
Dieser Durchbruch gelang durch das bestrahlen eines Zinc Telluride Kristalls mit einem Laser in 800 nm Bereich.


%%%%%%%%%%%%%%%%%%%%%%%%%%%%%%%%%%%%%%%%%%%%%%%%%%%%%%%%%%

At the beginning of this century a big leap in the efficient production of $\si{\tera\hertz}$ has been made.
This leap was partly made possible through the usage of non linear crytsals such as zinc telluride as emitters.
The nonlinear effect called optical rectification is exploited to produced the $\si{\tera\hertz}$-radiation in these crystalls.

% Motivation
% Hier folgt eine kurze Einleitung in die Thematik der Bachelorarbeit.
% Die Einleitung muss kurz sein, damit die vorgegebene Gesamtlänge der 
% Arbeit von 25 Seiten nicht überschritten wird. 
% Die Beschränkung der Seitenzahl sollte man ernst nehmen,
% da Überschreitung zu Abzügen in der Note führen kann. 
% Um der Längenbeschränkung zu genügen, darf auch nicht an der Schriftgröße,
% dem Zeilenabstand oder dem Satzspiegel (bedruckte Fläche der Seite) manipuliert werden.

