\chapter{Introduction}
For many applications, spectroscopy is an easy and efficient way to explore the properties of different materials.
That is why spectroscopy finds applications in many fields, from biology and medicine to physics and engineering.
A big leap in the science of spectroscopy was made in 1895 when Wilhelm Conrad Röntgen first discovered X-Ray radiation \cite{roentgen}.
With this new form of electromagnetic radiation, many new applications in medicine and physics evolved.
\\
A new era of spectroscopy began.
Over time more sources and detectors of various electromagnetic radiation were developed.
But, there was still a gap in the spectrum.
\\
It is known as the terahertz gap and describes the incapability of producing terahertz radiation from around $0.3-10\,\si{\tera\hertz}$ efficiently \cite{THzgap_applications}.
For some years there have been efforts to close the terahertz gap and develop methods of producing terahertz radiation.
The reason is, that terahertz radiation has some interesting properties that would benefit different fields.
It is non-ionizing, which makes it particularly interesting for medicine because it does not have any harmful effect on tissue \cite{THzgap_applications}.
\\
For instance in dermatology, where terahertz imaging processes can be used to determine the moisture content of skin or might even give diagnostic methods to detect skin cancer \cite{terahertz_dermatology}. 
Because of the same reason, and a specific spectroscopic fingerprint for most metals, the terahertz radiation is also of interest in the security sector.
Weapons like guns, knives, and even most soft explosives can be detected using terahertz spectroscopy \cite{THzgap_applications} \cite{thz_explosive_detec}.
The use of Terahertz radiation in the communication sector is already being analyzed, as with Terahertz bit rates of $\SI{100}{\giga\bit\per\second}$ can be reached \cite{communication}.
Furthermore, it is not only interesting for data transmission but also for data-saving.
Terahertz radiation can change the state of antiferromagnets, at much higher speeds than the operating speeds of conventional memory cards \cite{datasaving}.
\\
From 1995-1996 an advance in the generation of broadband terahertz radiation was made by using optical rectification and electro-optic sampling in non-linear crystals such as zinc telluride \cite{first_eos_wu_zhang}\cite{ZnTe_Nahata_Weling_1996}.
\\\\
This thesis will give an overview of the generation of terahertz radiation through optical rectification.
For this, the two non-linear crystal zinc telluride and gallium phosphide will be used to generate terahertz radiation.
The generated radiation will then be detected by exploiting the effect of electro-optic sampling.
\\
It is expected to generate a terahertz pulse, with frequencies at the low terahertz regime.
Through time-resolved spectroscopy, the whole time domain of the terahertz pulse will be probed.
Additionally to the spectrums of zinc telluride and gallium phosphide, the electric field, of their corresponding terahertz pulse will be calculated \cite{THZ_eltric_field}.
From the electric field, the power of the terahertz pulse can then be derived and the conversion efficiency will be calculated for both crystals.
This setup was first demonstrated by Nahata A., Weling A. S., and Heinz T. F. in 1996 \cite{ZnTe_Nahata_Weling_1996}.





% %%%%%%%%%%%%%%%%%%%%%%%%%%%%%%%%%%%%%%%%%%%%%%%%%%%%%%%%
% Für viele Anwendungen ist die Spektroskopie eine der effizientesten Möglichkeiten um gewisse Materialeigenschaften einer Probe zu bestimmen.
% So haben die Wissenschaftler in vielen Bereichen auf die Methode zurück gergriffen und von ihrer relativen einfachen Umsetzung profitiert.
% Ein großer Durchbruch in dem Bereich der Spektroskopie ist der Physik gelangt als Wilhelm Conrad Röntgen 1895 %(Entdeckung Röntgen Strahlung)
% die Röntgenstrahlen entdeckt hat.
% Durch diese neue Form der Strahlung konnte die Wissenschaft in neue Bereiche eintreten.
% Über die Jahre wurden dem Spektrum mit dem Spektroskopie möglich ist immer neue Strahlungsarten hinzugefügt.
% Doch eine Lücke im Spektrum besteht bis heute.
% Sie ist als THz-Lücke bekannt und bezeichnet die bis heute noch ineffieziente und relativ komplizierte Produktion von THz-Strahlung und Detektion.
% Seit einigen Jahren werden nun immer mehr Bestrebungen daran gesetzt diese Lücke zu schließen.
% Denn THz Strahlung ist für verschiedene wissenschaftliche Bereiche von großem Interesse.
% So könnten in der medicine neue Methoden zur Dutchleuchtung organischer Materie entwickelt werden ohne das die Materie dabei durch Ionisierung zerstört wird.
% Des weiteren ist sie auch für die Physik vom großem Interesse da durch elektromagnetische Wellen im THz-Frequenz bereich bestimme Gitterschwingungen angeregt werden.
% Ein großer Durchbruch gelang als 1996 das erste mal THz durch optical rectification in anorganischen Kristallen erzeugt wurde. %\cite{ZnTe_Nahata_Weling_1996}
% Dieser Durchbruch gelang durch das bestrahlen eines Zinc Telluride Kristalls mit einem Laser in 800 nm Bereich.
%%%%%%%%%%%%%%%%%%%%%%%%%%%%%%%%%%%%%%%%%%%%%%%%%%%%%%%%%%
% Motivation
% Hier folgt eine kurze Einleitung in die Thematik der Bachelorarbeit.
% Die Einleitung muss kurz sein, damit die vorgegebene Gesamtlänge der 
% Arbeit von 25 Seiten nicht überschritten wird. 
% Die Beschränkung der Seitenzahl sollte man ernst nehmen,
% da Überschreitung zu Abzügen in der Note führen kann. 
% Um der Längenbeschränkung zu genügen, darf auch nicht an der Schriftgröße,
% dem Zeilenabstand oder dem Satzspiegel (bedruckte Fläche der Seite) manipuliert werden.

