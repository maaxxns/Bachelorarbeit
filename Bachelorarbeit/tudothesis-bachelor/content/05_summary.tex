\chapter{Summary}
This thesis presented the time-resolved generation and detection of $\si{\tera\hertz}$ radiation in ZnTe and GaP.
It showed the measured electro-optic sampling data and its Fourier-Transformations, to confirm the generation of $\si{\tera\hertz}$ radiation.
Besides that, the pump power dependency of the $\si{\tera\hertz}$ electric field and the $\si{\tera\hertz}$ electric field power has been calculated for both ZnTe and GaP as emitter crystals.
\\\\
Both ZnTe and GaP did indeed generate $\si{\tera\hertz}$ radiation, with ZnTe being the better emitter crystal in terms of electric field strength, peak electric field power, and conversion efficiency.
However, the GaP crystal was $\SI{0.7}{\milli\meter}$ thinner than the ZnTe crystal, which makes a comparison between them not accurate.
It was also not ideal, that the ZnTe crystal had two burn marks on it that influenced its capability of generating $\si{\tera\hertz}$ radiation.
The switch in initial laser power harmed the consistency of the measurements and should be avoided in future experiments.
Either by lowering the pump power just by adding more density filters in the pump beam path or by taking more time for the alignment, to make sure both beams take the same path.
In the future, more measurements with comparable crystals should be taken to better determine how the spectrums and efficiency in the generation of ZnTe and GaP differ from one another.
