\chapter{Summary}
This chapter will summarize the results that were made in this thesis.
Furthermore an outlook for future measurements will be given and discussed.
\\\\
This thesis presented the time-resolved generation and detection of $\si{\tera\hertz}$ radiation in ZnTe and GaP.
The setup that was used to measure the $\si{\tera\hertz}$ radiation was explained and a detailed description of the execution was given.
The thesis demonstrated the measured electro-optic sampling data and its Fourier-Transformations, to confirm the generation of $\si{\tera\hertz}$ radiation.
Besides that, the pump power dependency of the $\si{\tera\hertz}$ electric field and the $\si{\tera\hertz}$ electric field power were calculated for both ZnTe and GaP as emitter crystals.
\\\\
Both ZnTe and GaP did indeed generate $\si{\tera\hertz}$ radiation, with ZnTe being the better emitter crystal in terms of electric field strength, peak electric field power, and conversion efficiency.
The maximum electric field strength reached is $\SI{9.59(15)}{\kilo\V\per\centi\meter}$.
The highest conversion efficiency reached lies at around $\SI{2.08e-05}{}$.
The highest electric field reached with GaP is $\SI{3.38(5)}{\kilo\V\per\centi\meter}$.
Its highest conversion efficiency is at around $\SI{2.71e-6}{}$.
However, the GaP crystal was $\SI{0.7}{\milli\meter}$ thinner than the ZnTe crystal, which makes a comparison between them not accurate.
Moreover, It was also not ideal, that the ZnTe crystal had two burn marks on it.
This influenced its capability of generating $\si{\tera\hertz}$ radiation.
The switch in initial laser power harmed the consistency of the measurements and needs to be avoided in future experiments.
\\\\
In the future, more measurements with comparable crystals should be taken to better determine how the spectrums and efficiency in the generation of ZnTe and GaP differ from one another.
Different crystal thicknesses should be tested as emitters.
Besides that, the detector crystal should also be swapped out with other crystals of different thicknesses.
The low phonon mode of ZnTe at $\SI{5.3}{\tera\hertz}$ limits its ability to sufficiently detected higher $\si{\tera\hertz}$ frequencies.
A good option as a detector crystal would be GaP as it is already available and also has a phonon mode that lies at a much higher frequency of $\SI{11}{\tera\hertz}$.
\\
Furthermore, measurements of the polarization dependence could be taken with the $\si{\tera\hertz}$ radiation.
This was not possible with the given setup because the distance between the two OAPs was too small to fit in two polarization filters.
Moreover, transmission measurements of $\si{\tera\hertz}$ radiation through different materials could be taken.
As it is possible to derive the dielectric function of the material by measuring its $\si{\tera\hertz}$ transmission.
Depending on the probe size these measurements could already be taken with the given setup.
\\\\
The presented setup was built from scratch by the author of this thesis and his supervisor.
All necessary programs were self-written by the author and supervisor and can be accessed at \cite{github}.