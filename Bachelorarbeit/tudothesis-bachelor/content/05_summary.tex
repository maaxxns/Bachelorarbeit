\chapter{Summary}
This chapter will summarize the results that were made in this thesis.
Furthermore and outlook for future measurements will be given and discussed.
\\
This thesis presented the time-resolved generation and detection of $\si{\tera\hertz}$ radiation in ZnTe and GaP.
The setup that was used to measure the $\si{\tera\hertz}$ radiation was explained and a detailed description of the execution was given.
The thesis demonstrated the measured electro-optic sampling data and its Fourier-Transformations, to confirm the generation of $\si{\tera\hertz}$ radiation.
Besides that, the pump power dependency of the $\si{\tera\hertz}$ electric field and the $\si{\tera\hertz}$ electric field power has been calculated for both ZnTe and GaP as emitter crystals.
\\\\
Both ZnTe and GaP did indeed generate $\si{\tera\hertz}$ radiation, with ZnTe being the better emitter crystal in terms of electric field strength, peak electric field power, and conversion efficiency.
The maximum electric field strength that could be reached is $\SI{9.59(15)}{\kilo\V\per\centi\meter}$.
The highest conversion efficiency that is reached lies at around $\SI{2.08e-05}{}$.
The highest electric field that could be reached with GaP is $\SI{3.38(5)}{\kilo\V\per\centi\meter}$.
Its highest conversion efficiency is at around $\SI{2.71e-6}{}$.
However, the GaP crystal was $\SI{0.7}{\milli\meter}$ thinner than the ZnTe crystal, which makes a comparison between them not accurate.
It was also not ideal, that the ZnTe crystal had two burn marks on it that influenced its capability of generating $\si{\tera\hertz}$ radiation.
The switch in initial laser power harmed the consistency of the measurements and should be avoided in future experiments.
Either by lowering the pump power just by adding more density filters in the pump beam path or by taking more time for the alignment, to make sure both beams take the same path.
\\\\
In the future, more measurements with comparable crystals should be taken to better determine how the spectrums and efficiency in the generation of ZnTe and GaP differ from one another.
Different crystal thicknesses should be tested as emitters.
Besides that, the detector crystal should also be swapped out with other crystals of different thickness.
The low phonon mode of ZnTe at $\SI{5.3}{\tera\hertz}$ limits its ability to sufficiently detected higher $\si{\tera\hertz}$ frequencies.
A good option as detector crystal would be GaP as it is already available and also has a phonon mode that lays at a much higher frequency of $\SI{11}{\tera\hertz}$.
\\
Furthermore, measurements on the polarization dependence could be taken with the $\si{\tera\hertz}$ radiation.
This was not possible with the given setup because the distance between the two OAP was too small to fit in two polarization filters.
Moreover, transmission measurements of $\si{\tera\hertz}$ radiation through different materials could be taken.
As it is possible to derive the dieletric function of the material by measuring its $\si{\tera\hertz}$ transmission.
Depending on the probe size these measurements could already be taken with the given setup.
The time limitation that was given, was the only reason to not take those measurements.
\\\\
The setup that was used was build from scratch by the author of this thesis and his supervisor.
All necessary programms were self-written by the author and advisor.
The presented measurements were all taken by the author and its advisor over the course of this thesis.