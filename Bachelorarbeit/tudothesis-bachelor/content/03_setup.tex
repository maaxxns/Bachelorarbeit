\chapter{Setup and execution}
This chapter will discuss the setup that is used to detect the $\si{\tera\hertz}$-electric field.
Aswell as the methode that ís used to achieve a time resolved measurment of the $\si{\tera\hertz}$ pulse.

\section{Setup}
\label{sec:setup}
To produce $\si{\tera\hertz}$ radition the setup as shown in Figure \ref{fig:setup} is used.
An incoming $\SI{0.5}{\W}$ laser with a pulselength of $\SI{100}{\femto\second}$ is splitt into pump and probe laser.
The pump beam receives $90\%$ of the initial laser beam power, the probe beam the remaining $10\%$.
Now the pump beam passes through a chopper which modulates the laser to a frequency of $\SI{280}{\hertz}$.
This way the Lock in Amplifier that is used for the measurment can be triggered on that frequency.
After the chopper a delay stage is placed.
With this the pump beam path length can be altered.
This allows a time resolved detection of the $\si{\tera\hertz}$ pulse, a detailed description can be found in section \ref{sec:time_domain}.
Atlast the beam is reflected onto the emitter crystal.
Depending on the measurment a zinc telluride (ZnTe) or a gallium phosphite (GaPh) crystal is used.
A sillicon wafer blocks the transmitted laser beam and transmitts the $\si{\tera\hertz}$ radiation, which is generated by the emitter crystal.
Two parabolic mirrors focus the $\si{\tera\hertz}$ beam onto the detection crystal.
Just as the emitter crystal either ZnTe or GaPh are used as detection crystal.
The probe beam, whose power is reduced by a optical density filter, hits the detection crystal at roughly the same position as the pump $\si{\tera\hertz}$ beam.
As dicussed in section \ref{sec:electrooptic_sampling}, the polarization of the probe beam is change depending on the $\si{\tera\hertz}$-electric field.
After the change in polarization the probe beam passes thorugh a quater-wave-plate, which changes the linear polarization of the beam to an elliptic polarization.
The beam is then splitt into its horizontal and vertical polarization components by a wallston-prism.
The two seperated beams are passed into photodiodes.
\\\\
Depening on the change of polarization the insenity of probe beams after the wallston-prism change.
The signal of the photodiodes $A$ and $B$ is passed into a Lock in Amplifier, which calculates the diffrence of both signals $A-B$.
The bigger the diffrence $A-B$ the stronger is the $\si{\tera\hertz}$-electric field in the detection crystal.

\section{Time resolved THz spectroscopy}
\label{sec:time_domain}
To achieve a time resolved measurment of the $\si{\tera\hertz}$-pulse a delay stage is build into the setup.
With the delay stage the path length of the pump beam can be changed.
This way a time delay between pump pulse and probe pulse is achieved.
Because the velocity of electromagentic radiation in air is roughly the same as in a vacuum we can calculated the time delay   
\begin{equation}
    \Delta t = \frac{\Delta s}{c}
\end{equation}
with $c$ as the speed of ligth, and $\Delta s$ the diffrence in beam path.
