\chapter{Setup and execution}
This chapter will discuss the setup that is utilized to generate and detect the $\si{\tera\hertz}$-electric field, as well as the method that is used to achieve a time-resolved measurement of the $\si{\tera\hertz}$ pulse.

\section{Setup}
\label{sec:setup}
\begin{figure}
    \centering
    \includegraphics[width=\textwidth]{Plots/Aufbau.pdf}
    \caption{The setup that generates $\si{\tera\hertz}$ radiation.
    The acronyms OAP and QWP stand for parabolic off-axis parabolic mirror and quarter-wave plate. }
    \label{fig:setup}
\end{figure}
To generate $\si{\tera\hertz}$ radiation the setup as illustrated in Figure \ref{fig:setup} is employed.
An $\SI{800}{\nano\meter}$ Ti-sapphire laser is used to generate the necessary laser radiation.
It generates a pulsed laser beam with a power of around $\SI{6.9}{\W}$ and a frequency of about $\SI{1000}{\Hz}$.
However, because the laser feeds into several setups only a fraction of the initial power reaches each setup.
This setup receives between $\SI{291}{\milli\W}$ and $\SI{579}{\milli\W}$ depending on the configurations of the other setups. 
The incoming laser beam with a pulse length of $\SI{100}{\femto\second}$ is split into a pump and a probe beam.
\\
The pump beam receives $90\%$ of the initial laser beam power, and the probe beam the remaining $10\%$.
The pump beam passes through a chopper which modulates the laser to a frequency of $\SI{280}{\hertz}$.
This way the lock-in amplifier that is utilized for the measurement can be triggered on that frequency.
After the chopper, a delay stage is placed.
With this, the pump beam path length can be altered.
This allows a time-resolved detection of the $\si{\tera\hertz}$ pulse, a detailed description can be found in section \ref{sec:time_domain}.
\\
To focus the beam on the crystal a lens with a focal length of $\SI{40}{\centi\meter}$ is positioned $\SI{15}{\centi\meter}$ in front of the crystal.
Ultimately, the beam hits the emitter crystal.
Depending on the measurement a $\SI{1}{\milli\meter}$ ZnTe or a $\SI{0.3}{\milli\meter}$ GaP crystal is used.
The GaP crystal is thinner because it has a smaller coherence length at most of the $\si{\tera\hertz}$ frequencies compared to ZnTe when an $\SI{800}{\nano\meter}$ pump laser is used.
\\
With this part of the setup $\si{\tera\hertz}$ radiation can already be generated, just the detection is missing.
\\\\
To detect the $\si{\tera\hertz}$ radiation it first needs to be separated from the laser light that is transmitted by the emitter crystal.
If the pump laser beam is not separated from the $\si{\tera\hertz}$ beam the intense pump laser beam would damage the photodiodes.
In this design, a silicon wafer blocks the laser beam and transmits the $\si{\tera\hertz}$ radiation, which is generated by the emitter crystal.
Two parabolic off-axis mirrors focus the $\si{\tera\hertz}$ beam onto the detection crystal.
\\
As the detection unit, a $\SI{1}{\milli\meter}$ ZnTe crystal is utilized.
To place the detection crystal right at the focus of the parabolic mirror, a light source reflects into the OAP.
The detection crystal is then moved to the focus of that light cone.
The same approach is used to place the emitter crystal at the correct distance from the first OAP.
\\\\
The probe beam, whose power is reduced by an optical density filter, passes through the back of the second OAP, which has a small hole in it.
From there it hits the detection crystal at roughly the same position as the pump $\si{\tera\hertz}$ beam.
\\
To make sure that this is the case, the emitter crystal and the silicon wafer are removed and a strong optical density filter is placed into the pump beam.
The filter is necessary because the pump laser beam would damage the OAPs otherwise.
Now the OAPs reflect the actual laser pump beam onto the detection crystal.
With an infrared viewing camera, it is then possible to see if the pump laser and probe laser overlap.
If that is the case the pump $\si{\tera\hertz}$ beam should be at least roughly at the same position as the pump laser beam.
This means the $\si{\tera\hertz}$ pump beam also overlaps with the probe beam.
\\
After the alignment of the pump and probe beam, the emitter crystal and the silicon wafer are placed back into the setup.
With this, the first part of the detection unit is done.
The $\si{\tera\hertz}$ radiation now changes the polarization of the probe beam.
\\\\
The next step is to measure the change in polarization.
As described in the section \ref{sec:eos}, the polarization of the probe beam changes depending on the $\si{\tera\hertz}$ electric field inside the detector.
After the change in polarization, the probe beam passes through a quarter-wave plate, which changes the linear polarization of the beam to an elliptic polarization.
The quarter-wave plate is also used to balance the signal while there is no $\si{\tera\hertz}$ electric field.
The effect of the quarter-wave plate is further described in section \ref{sec:qwp}.
\\
This way the photodiodes have an equal input if there is no $\si{\tera\hertz}$ electric field inside the detector crystal.
The beam is then split into its horizontal and vertical polarization components by a Wollaston prism, which is further explained in subsection \ref{sec:wollaston}.
The two separate beams are passed into photodiodes.
\\\\
Depending on the polarization of the probe beam after the detection crystal, the intensity of the probe beams after the Wollaston prism will change.
The signal of the photodiodes $A$ and $B$ is passed into a lock-in amplifier.
The lock-in amplifier is further explained in section \ref{sec:data_acq}.
The lock-in then calculates the difference between both signals $A-B$.
The bigger the difference $\symup{\Delta}I = A-B$ the stronger the $\si{\tera\hertz}$ electric field in the detection crystal.
\\\\
With the setup now done the signal has to be optimized.
To do this the stage is moved to the position of the strongest signal.
The goal is to lower the noise or increase the peak height of the signal, through changes in the setup.
As shown in the subsection \ref{sec:emitters} the orientation of the crystals is of great importance.
That is why the first step is to rotate the emitter crystal, to the point where the $(001)$ crystal direction is parallel to the polarization of the pump beam.
After the correct orientation is found the process is repeated for the detection crystal.
\\\\
Now the pump laser beam position on the emitter crystal needs to be adjusted.
For this, the laser mirror before the crystal is rotated until the signal is at its greatest strength.
Furthermore, the OAPs are adjusted by moving their position and orientation slightly.
The signal can also be increased by rotating the detection crystal so that the probe beam does not hit it perpendicularly.
\\
The reason is that the angle limits the effect of reflections inside the crystal.
This means the best angle would be the brewster angle however, this angle can not be utilized in this setup because the crystal surface is too small. 
The best results are achieved at an angle of $\SI{40}{\degree}$.
Finally, the overlap of the $\si{\tera\hertz}$ pump beam and laser probe beam on the detection crystal is corrected.

\FloatBarrier
\subsection{Quarter-wave plate}
\label{sec:qwp}
To balance the two photodiodes, a quarter-wave plate is utilized.
This component changes the polarization of the incoming electric field into a circular or elliptical polarization, depending on its orientation.
The quarter-wave plate introduces a phase shift of $\symup{exp}(i\frac{\pi}{2}) = i$ into the electric field of one polarization component of the wave.
This way the incoming wave $\vec{E}$, that consists of a horizontally polarized component $E_\text{h}\vec{h}$ and a vertically polarized component $E_\text{v}\vec{v}$, can be written as
\begin{equation}
    \vec{E} = (E_\text{v}\vec{v} + E_\text{h}\vec{h})\symup{e}^{i(kx-\omega t)}
\end{equation}
and changes after the quarter wave plate to 
\begin{equation}
    \vec{E} = (E_\text{v}\vec{v} + i E_\text{h}\vec{h})\symup{e}^{i(kx-\omega t)}
\end{equation}
if the phase shift is induced along the horizontally polarized axis it will result into an elliptical polarization as seen in the lower part of figure \ref{fig:qwp}.
It works the same way if the phase shift is induced at the vertical axis.
If now the electric field strengths in both polarization directions are the same $E_\text{v} = E_\text{h} = E$ the result of the phase shift is a circularly polarized wave \cite{qwp_book}
\begin{equation}
    \vec{E} = E(\vec{v} + i\vec{h})\symup{e}^{i(kx-\omega t)} \, .
\end{equation}
By rotating the quarter-wave plate it is then possible to induce the phaseshift to both polarization components with different extensions.
For the right balancing, the quarter-wave plate changes the polarization of the probe beam to a circular one, if there is no $\si{\tera\hertz}$ signal and to an elliptical, if there is a signal.
\begin{figure}
    \centering
    \includegraphics[width=0.4\textwidth]{refferenced_pic/qwp.png}
    \caption{Depending on the initial polarization the quarter-wave plate changes the polarization of the electric field to a circular (upper picture) or elliptical polarization (lower picture).}
    \label{fig:qwp}
\end{figure}
\FloatBarrier
\subsection{Wollaston prism}
\label{sec:wollaston}
The probe beam polarization, after the detector crystal, carries almost all the information about the $\si{\tera\hertz}$ electric field strength.
Because it is not possible to measure the polarization of the probe beam directly, it is necessary to split the probe beam into its two polarization components.
This happens with a Wollaston prism, shown in figure \ref{fig:wollaston}.
\\
It consists of two prisms that are glued together.
The light that passes through the prism then refracts at the point where the two prisms touch.
From the Fresnel equations, it is possible to derive, that dependent on the polarization of the light, it gets refracted differently.
This means that the beam splits into a horizontally and a vertically polarized beam, with an angle of about $\SI{20}{\degree}$.
The intensity of both beams is then measurable.
Furthermore, it is possible to derive the change in polarization that occurred at the detector crystal \cite{wollaston_prism}. 
\begin{figure}
    \centering
    \includegraphics[width=0.5\textwidth]{Plots/wollaston_prism.png}
    \caption{The picture shows a Wollaston prism through which an unpolarized beam passes.
    This beam refracts inside the prism which splits it into two beams with perpendicular polarizations s and p.
    The two beams then leave the prism with an angle between them \cite{wollaston_prism}.}
    \label{fig:wollaston}
\end{figure}

\subsection{Data acquisition}
\label{sec:data_acq}
As described in section \ref{sec:wollaston} the probe beam is split into its two polarization components.
To detect the intensity of the beams, two photodiodes are utilized.
These detect the time-integrated intensity of a laser pulse.
They put out a current which is fed into a lock-in amplifier.
\\
The lock-in amplifier is used to detect very weak signals with a specific frequency.
For this, a reference frequency $\omega_\text{chopper}$, as well as the signal $V_\text{sig}\sin(\omega_\text{chopper}t + \theta_\text{sig})$ with the phaseshift $\theta_\text{sig}$ are given into the lock-in amplifier.
The lock-in amplifier then multiplies the signal with a reference signal $V_\text{lock}\sin(\omega_\text{lock} + \theta_\text{ref})$, which is produced by the lock-in amplifier itself.
The result of the multiplication is
\begin{equation}
    V = V_\text{sig}\sin(\omega_\text{chopper}t + \theta_\text{sig})V_\text{lock}\sin(\omega_\text{chopper} + \theta_\text{ref})
\end{equation}
which can also be written as 
\begin{equation}
    V = \frac{1}{2} V_\text{sig} V_\text{lock} \left ( \cos([\omega_\text{sig} - \omega_\text{lock}]t +\theta_\text{sig} - \theta_\text{lock}) - \cos([\omega_\text{sig} + \omega_\text{lock}]t +\theta_\text{sig} + \theta_\text{lock})\right ) \, .
\end{equation}
The result is passed through a low pass filter, which removes the AC signals.
This means that only if $\omega_\text{sig}$ and $\omega_\text{lock}$ are equal does anything come out of the filter.
In that case the result 
\begin{equation}
    V = \frac{1}{2}V_\text{sig}V_\text{lock}\cos(\theta_\text{sig}-\theta_\text{lock})
\end{equation} 
can be detected and measured without the noise from other frequencies.
Because $V_\text{lock}$ is already known it is possible to calculate the signal.
The reference frequency that is used in this setup is the frequency at which the chopper modulates the pump beam.
\section{Time-resolved THz spectroscopy}
\label{sec:time_domain}
To achieve a time-resolved measurement of the $\si{\tera\hertz}$ pulse a delay stage is built into the setup.
With the delay stage, the path length of the pump beam can be changed.
This way a time delay between the pump pulse and the probe pulse is achieved.
Because the velocity of electromagnetic radiation in air is roughly the same as in a vacuum, the vacuum speed can be used to calculate the time delay   
\begin{equation}
    \symup{\Delta} t = \frac{\symup{\Delta} s}{c}
\end{equation}
with $c$ as the speed of light, and $\symup{\Delta} s$ the difference in the beam path length.
The time-resolved measurement now works by taking one measurement at the delay stage position $s_1$, which corresponds to $t_1$.
Then the stage is moved to position $s_2$ this lengthens or shortens the beam path of the pump beam.
At stage position $s_2$ a new measurement is taken which corresponds to time $t_2$.
\\
Because just the pump beam path is changed, it arrival time at the crystal changes, compared to the probe beam.
This results in a time delay $\symup{\Delta} t = t_1-t_2$.
This process is repeated until the whole $\si{\tera\hertz}$-pulse is probed.
For this, the delay stage has to be moved $\SI{4}{\milli\meter}$ in total.
To get a good resolution $1000$ measurements with a distance of $\SI{0.004}{\milli\meter}$ between them are taken.
This distance corresponds to a time resolution of $\SI{13.34}{\femto\second}$.

\section{Execution}
\label{sec:execution}
This section describes how the measurements are taken. 
It will be discussed what changes to the setup in figure \ref{fig:setup} are made to achieve the wanted results.
\\\\
Before any measurements are taken the photodiodes have to be balanced.
For this, a chopper is placed inside the probe beam, which modulates its frequency to $\SI{388}{\hertz}$.
The lock-in amplifier gets triggered on that frequency, while its inputs are currents $A$ and $B$ of the two photodiodes.
\\
It is necessary to move the stage to a point where $\si{\tera\hertz}$-pulse and probe pulse do not overlap.
So that the polarization of the probe beam does not change.
\\
Now the quarter-wave plate is being rotated until the output of both photodiodes $A-B = 0$.
To account for the noise the quarter-wave plate is rotated until the average of $A-B$ over a small time period is zero.
If that is the case the probe beam is circularly polarized after the quarter-wave plate, while there is no $\si{\tera\hertz}$ signal.
%%%%%%%%%%%%%%%%%%%%%%%%%%%%%%%%%%%%%%%%%%%%%%%%%%%%%%%%%%%%%
\subsection{Power measurements}
\label{sec:power}
The goal is to determine the effect of different pump fluences on the $\si{\tera\hertz}$ production.
For this, it is necessary to change the power of the pump beam before the emitter crystal.
For that various density filters are placed inside the pump beam path.
All the utilized filters can be seen in Table \ref{tab:filters}.
\\
After the placement behind the chopper, the pump power is measured in front of the emitter crystal.
Then the power meter is removed so that the pump beam hits the emitter crystal.
\\
Now a short measurement with a low time resolution is taken, to roughly determine the position of the signal peak.
After the peak position is found a time-resolved measurement, as described in section \ref{sec:time_domain} is taken.
For this, the stage is moved to its starting position at about $\SI{1}{\milli\meter}$ before the signal peak.
It stops $\SI{3}{\milli\meter}$ after the peak.
These positions are chosen to only scan the pulse and the oscillations that occur after the pulse.
\\
It should also be said that reflections inside the emitter and the detector crystals cause a so-called pre- and post-pulse.
These are unwanted byproducts of the initial pulse and will interfere with later calculations.
This makes it important to only measure the initial pulse but not the pre- and post-pulse.
After the measurement, the filter is swapped out with a different one.
The process is started over until all fluences are measured.
\\\\
To calculate the peak power of the $\si{\tera\hertz}$ electric field, it is necessary to measure the spot size.
Usually, this would be done with a $\si{\tera\hertz}$-optical camera, which automatically measures the spot size.
In the case of this setup, the $\si{\tera\hertz}$ electric field is too weak to be detected by the camera.
The solution to this problem is to measure the spot size directly.
\\
To do that an aperture is paced inside the probe beam.
Then the delay stage is moved to the position of the greatest electric field, where it is fixed.
Now the $A-B$ value of the output of the photodiodes is measured.
Usually the probe beam spot on the detector is bigger than the $\si{\tera\hertz}$ spot.
Through closing the aperture the probe beam spot size becomes smaller until it has the same size as the $\si{\tera\hertz}$ spot.
If the probe beam spot becomes smaller than the $\si{\tera\hertz}$ spot the signal strength starts to drop.
This means the aperture opening has to be reduced until the signal loses strength. 
At which point the opening area of the aperture is the same as the $\si{\tera\hertz}$ spot on the detector crystal.
\subsection{Electric field measurements}
\label{sec:field}
To determine the electric field of the $\si{\tera\hertz}$ radiation it is necessary to measure the intensity of both probe beams after the Wollaston prism.
With this the equation \eqref{eq:electricfield_A_B} can be used to calculate the electric field value.
Because just the maximum field strength is of particular interest just the greatest $\symup{\Delta}I = A-B$ value is necessary to measure.
For this, the delay stage is moved into the peak of the pulse.
There is the position of the greatest $A-B$ value and a measurement of the difference is taken.
This process has to be repeated for all pump power values.
\\\\
After all the differences are measured, the sum of the intensities $I = A + B$ needs to be measured.
The delay stage is moved to a position where there is no $\si{\tera\hertz}$ signal.
Because the lock-in amplifier can not calculate the sum of $A$ and $B$ it is necessary to measure them separately.
The input of the lock-in amplifier is changed to just one of the outputs of the photodiodes, either $A$ or $B$.
After one, the other is measured as well.
Because the sum $A+B$ is the same for all $\si{\tera\hertz}$ electric fields, it can be used to norm every difference.
After $A-B$ and $A$ and $B$ are measured the electric field can be calculated.
