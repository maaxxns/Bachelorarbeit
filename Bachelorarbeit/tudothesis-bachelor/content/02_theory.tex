\chapter{Theory}

\section{Optical rectification}\label{sec:optic_ref}
To produce $\si{\tera\hertz}$ radiation we take advantage of optical rectification.
This effect is caused by a change of polarization from the incoming electric field of the laser pulse.
The polarization

\begin{equation}
P = \xi(E) E
\end{equation}

is directly proportional to the elctric field $E$ and the susceptibilty $\xi(E)$.
Which in turn can be expanded to 

\begin{equation}
    \xi(E) = \xi_1 + \xi_2 E +\xi_3 E^2 + ...   \, .
\end{equation}

As describe earlier optical rectification is a second order effect in thus describe by the $P_\text{el-op} = \xi_2 E^2$ part.
Because we dont have just one electric field inside the crytsal, but two we have to take in account both fields.
On of those electric fields oscillats at frequency $\omega_1$ and the other at frequency $\omega_2$.
The resulting  second order polarization term 

\begin{equation}
    P_\text{el-op} = \xi_2 \frac{E_0^2}{2}\left(cos((\omega_1 - \omega_2)t) + cos((\omega_1 + \omega_2)t)\right)
\end{equation}

shows a cosin with a diffrence depends $\omega_1-\omega_2$ and one with a sum depends $\omega_1+\omega_2$.
The one with the diffrence depends results in the production of $\si{\tera\hertz}$ radiation.

\section{Electro-optic sampling}\label{sec:eos}


\section{ZnTe}
One of the crystalls we use to generate $\si{\tera\hertz}$ radiation is zinc telluride (ZnTe). 
It allows the production of a wideband coherent $\si{\tera\hertz}$ Field through means of optical rectification \ref{sec:optic_ref} \cite{ZnTe_Nahata_Weling_1996}.
Because optical rectification is a non linear effect it is important to aline the laser to the $<110>$ axis of the crystal.


%(Nahata, A., Weling, A. S., and Heinz, T. F., A wideband coherent terahertz spectroscopy
% system using optical rectification and electrooptic sampling, Appl. Phys. Lett.,
% 69, 2321, 1996.).
% (Han, P. Y., and Zhang, X. C., Free-space coherent broadband terahertz time-domain
% spectroscopy, Meas. Sci. Technol., 12, 2001, 1747.)
