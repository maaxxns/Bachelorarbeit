\chapter{Theory}
\section{Terahertz radiation}
This thesis will show means of generating Terahertz ($\si{\tera\hertz}$) radiation.
As the name describes, its frequency regime lies at about $(0.3-30)\cdot10^{12}\,\si{\hertz}$.
Its wavelength can be easily calculated by
\begin{equation}
    \lambda = \frac{c}{\nu}
\end{equation}
with the speed of light $c$ and its frequency $\nu$.
Which results in a wavelength of about $\SI{100}{\micro\meter}-\SI{1}{\milli\meter}$.
This means it lies between infrared and microwave radiation.
$\si{\tera\hertz}$ radiation is caused by various non-linear effects which only occur at relatively high intensities.
The need for high intensities, aswell as the lack of emitters, makes it hard to generate it efficiently.
The result is the so-called $\si{\tera\hertz}$ gap that is slowly being closed by new techniques and advances in science.
Some of these are the generation of $\si{\tera\hertz}$ radiantion through in- and anorganic crystals, aswell as the use of semiconductors.
Water can absorb $\si{\tera\hertz}$ radiation which will be shown later.
%%%%%%%%%%%%%%%%%%%%%%%%%%%%%%%%%%%%%%%%%%%%%%%%%%%%%%%%%%%%%%%%%%%%%%%%%%%%%%%%%%


%%%%%%%%%%%%%%%%%%%%%%%%%%%%%%%%%%%%%%%%%%%%%%%%%%%%%%%%%%%%%%%%%%%%%%%%%%%%%%%%%%
\section{Non-linear crystals}
Optical non-linear crystals exhibit a nonlinearity in their second-order susceptibility $\chi_2$ or in even higher orders.
For this, they show special non-linear effects at high electric field strengths.
For different purposes, different kinds of crystals are in use.
Almost all of these crystals are artificially generated crystals.
The two crystals that are used in this experiment both show zincblende structure.


\subsection{Zinc telluride}
\label{sec:znte}
One of the crystals we use to generate $\si{\tera\hertz}$ radiation is zinc telluride (ZnTe). 
It allows the production of a wideband coherent $\si{\tera\hertz}$ field through means of optical rectification \ref{sec:optic_ref} \cite{ZnTe_Nahata_Weling_1996}.
It should be mentioned that ZnTe has a phonon resonance at $\SI{5.3}{\tera\hertz}$ \cite{phonon_modes}, which limits the bandwidth of emitted radiation and its ability to detect radiation of that or higher frequencies.
Because optical rectification is a non-linear effect it is important to aline the laser to the $<110>$ axis of the crystal.
The angle between the $<001>$ direction and the polarization of the pump beam is also from importance.
It can be seen in figure \ref{fig:polarization_dependence_angle} that two angles give the strongest signal.
\begin{figure}
    \centering
    \includegraphics[width=0.4\textwidth]{refferenced_pic/degreedepenceZnTe.png}
    \caption{The plot shows the angle dependent emission of $\si{\tera\hertz}$ radiation in ZnTe.
    The angle is between the $<001>$ direction of the crystal and the polarization of the pump beam. An angle of $\theta = \SI{90}{\degree}$ however relates to the pump beam polarization being parallel to the $<001>$ axis.
    The plot was taken from source \cite{selig}.}
    \label{fig:polarization_dependence_angle}
\end{figure}
It is also used as a detection crystal for the $\si{\tera\hertz}$ field.
The ZnTe changes the polarization of the incoming probe beam dependent on the field strength of the $\si{\tera\hertz}$ radiation.
This effect is discussed in section \ref{sec:eos}.
The refractive index of ZnTe at $\SI{800}{\nano\meter}$ is $n_ZnTe = 2.85$.

\textbf{Platzhalter für Bild von ZnTe struktur}

\subsection{Gallium phosphide}
The other crystal that is used to generate $\si{\tera\hertz}$ radiation is Gallium phosphide (GaP).
It is a common emitter of $\si{\tera\hertz}$ radiation with a broad spectrum.
This crystal also exhibits a phonon resonance, but a much higher frequency of about $\SI{11}{\tera\hertz}$ \cite[60]{wiki_book}.
Aswell as ZnTe, Gap needs to stimulated by the laser along its $<110>$ axis.
The refractive index of GaP at $\SI{800}{\nano\meter}$ is $n_GaP = 3.193$.

\textbf{Platzhalter für Bild von GaP struktur}
%%%%%%%%%%%%%%%%%%%%%%%%%%%%%%%%%%%%%%%%%%%%%%%%%%%%%%%%%%%%%%%%%%%%%%%%%%%%%%%%%%


%%%%%%%%%%%%%%%%%%%%%%%%%%%%%%%%%%%%%%%%%%%%%%%%%%%%%%%%%%%%%%%%%%%%%%%%%%%%%%%%%%
\section{Optical rectification}\label{sec:optic_ref}
To produce $\si{\tera\hertz}$ radiation we take advantage of optical rectification or rather a kind of difference frequency mixing that is very similar to optical rectification.
Optical rectification is a second-order non-linear effect and thus can be just observed in non-linear materials.
This effect causes a DC polarization in the crystal that if moduled correctly causes $\si{\tera\hertz}$ radiation.
The DC polarization occurs when an outer electric field interacts with the crystal.
Because of the nonlinear structure and in turn the anharmonic potential of crystal charges, the charges oscillate further in one direction than the other.
The displacement of charge is what causes the DC polarization.
To discuss the effect in detail we take a look at the polarization

\begin{equation}
P = \chi(E) E \epsilon_0
\end{equation}

which is directly proportional to the electric field $E$ and the susceptibility $\chi(E)$.
Which in turn can be expanded to 

\begin{equation}
    \chi(E) = \chi_0 + \chi_1 E +\chi_2 E^2 + ...   \, .
\end{equation}

As described earlier optical rectification is a second-order effect and described by the $P_\text{nl} = \chi_2 E^2$ part.
Because the laser produces electromagnetic radiation at the whole bandwidth of frequencies $\omega + \Delta\Omega$, in a gaussian profile, it is necessary to consider all of those electric fields mixing inside the crystal.
To simplify its best to first take a look at just two electric fields.
One of those electric fields oscillates at frequency $\omega_1$ and the other at frequency $\omega_2$.
The resulting second-order polarization term 

\begin{equation}
    P_\text{nl} = \chi_2 \epsilon_0 \frac{E_0^2}{2}\left[cos((\omega_1 - \omega_2)t) + cos((\omega_1 + \omega_2)t)\right]
\label{eq:two_freq_mixing}
\end{equation}

shows a \textbf{cosine} with a difference dependency $\omega_1-\omega_2$ and one with a sum dependency $\omega_1+\omega_2$.
The one with the difference depends results in the production of $\si{\tera\hertz}$ radiation. % the one with the sum depends is important for second harmonic generation
The sum part of equation \eqref{eq:two_freq_mixing} is not important for $\si{\tera\hertz}$ production and will be neglected in further calculations \cite[45--46]{wiki_book}.
If the whole bandwidth $\omega + \Delta\Omega$ is now taken into account not just two frequencies mix but all.
This results in a polarization dependency on the bandwidth $\Delta\Omega$ as such
\begin{equation}
    P_{\Delta\Omega} = \chi_2 \epsilon_0 \frac{E_0^2}{2}cos(\Delta\Omega t) \, .
    \label{eq:polarization_depens}
\end{equation}
With a bandwidth in the femtosecond regime the resulting change in polarization produces $\si{\tera\hertz}$ radiation \cite[289--291]{book_optical_rectification}\cite[46]{wiki_book}.
\\\\
% Because the change in polarization is also dependent on the pump beam electric field, as seen in equation \ref{eq:polarization_depens}, part of the experiment is to change the pump fluence.
% The fluence is defined as 
% \begin{equation}
%     H = \int_{0}^{t} E_\text{irr} \symup{d}t
%     \label{eq:fluence}
% \end{equation}
% with $E_\text{irr}$ being the irradiance.
% The irradiance 
% \begin{equation}
%     E_\text{irr} = \frac{\Phi}{A}
%     \label{eq:irradiance}
% \end{equation} 
% is defined as the radiant flux $\Phi$ per area $A$.
% \\\\
The process of $\si{\tera\hertz}$ production can also be explained by the excitation of higher energy levels.
It is visualized in figure \ref{fig:freq_mix}.
\begin{figure}
    \centering
    \includegraphics[width=\textwidth]{refferenced_pic/diffrence_frequency_mixing.PNG}
    \caption{Graphic a) shows the two frequencies $\omega_\text{i} $ and $\omega_\text{j}$ going into the nonlinear medium with susceptibility $\chi_2$.
    Through difference frequency mixing the medium emits radiation at frequency $\Delta\Omega$.
    Graphic b) shows the interaction of frequency $\omega_\text{i} $ and $\omega_\text{j}$ inside the medium.
    Here $\omega_\text{i}$ excites a higher virtual energy level, from which $\omega_\text{j}$ gets substracted which leaves $\Delta\Omega$.}
    \label{fig:freq_mix}
\end{figure}
The frequency $\omega_\text{i}$ that goes into the crystal excites a higher energy level.
Now the other frequency $\omega_\text{j}$ that goes into the crystal lowers the energy state again.
Because of the conservation of energy the crystal puts out a photon of the resulting difference

\begin{equation}
    \Delta\Omega = \omega_\text{i} - \omega_\text{j} \, .
\end{equation}


%%%%%%%%%%%%%%%%%%%%%%%%%%%%%%%%%%%%%%%%%%%%%%%%%%

%Hier noch Grafik rein und vielleicht nochmal was ändern
%Patrick hat das ganze ja eher durch das anregen von virtuellen Energielevel erklärt.
%Dazu wäre eine Grafik auch gut

%%%%%%%%%%%%%%%%%%%%%%%%%%%%%%%%%%%%%%%%%%%%%%%%%%

\section{Electro-optic sampling}\label{sec:eos}
To detect the $\si{\tera\hertz}$-electric field Electro-optic sampling (EOS) is used.
Which makes use of the electro-optic effect also known as the pockels effect.
Through this effect, a birefringence is induced in the detection crystal which in turn changes the polarization of the probe beam.
The phase shift 
\begin{equation}
    \text{sin}(\theta) = \frac{2\pi}{\lambda} n_0^3 l r E_\text{THz}
\end{equation}
is directly proportional to the electric field $E_\text{THz}$. 
It is also proportional to the wavelength $\lambda$, the crystal thickness $l$, the electrooptic coefficient $r$, and the refractive index $n_0$ \cite{wiki_book}. 
Through measurement of the change in polarization, it is then possible to determine the electric field strength of the $\si{\tera\hertz}$-pump beam.
For this two photodiodes are used, which measure the probe beam intensity in dependence on its polarization.
It is explained in section \ref{sec:setup} how the beam is split into its two polarization-dependent parts.
One photodiode just measures the intensity of the horizontally polarized probe beam $A$ and one the vertically polarized part $B$.
With the normed difference 

\begin{equation}
    \frac{A-B}{A+B} = \text{sin}(\theta) = \frac{2\pi}{\lambda} n_0^3 l r E_\text{THz}
    \label{eq:electricfield_A_B}
\end{equation}

it is than possible to determine the field strength $E_\text{THz}$ \cite[7]{THZ_eltric_field}.
With the electricfield calculated, the peak power of the $\si{\tera\hertz}$-electric field is then attainable.
For this the intensity
\begin{equation}
    I = c \epsilon_0 E_\text{THz}^2
    \label{eq:intensity}
\end{equation}
is calculated. With $\epsilon_0$ beeing the vacuum permittivity.
The power 
\begin{equation}
    P = \int I \symup{d}A
    \label{eq:power}
\end{equation}
can than be derived by integrating the intensity $I$ over the area of the $\si{\tera\hertz}$ spot $A$.
With the power of the electricfield a conversion efficiency $C$ can be calculated by formular
\begin{equation}
    C = \frac{P_\text{THz}}{P_\text{pump}}
    \label{eq:conversion}
\end{equation}
in which $P_\text{THz}$ is the peak electricfield power and $P_\text{pump}$ is the pump power.
%%%%%%%%%%%%%%%%%%%%%%%%%%%%%%%%%%%%%%%%%%%%%%%%%%%%%%%%%%%%%%%%%%%%%%%%


%%%%%%%%%%%%%%%%%%%%%%%%%%%%%%%%%%%%%%%%%%%%%%%%%%%%%%%%%%%%%%%%%%%%%%%%
\section{Coherence length}
Because the refractive indices of the $\SI{800}{\nano\meter}$ laser pump beam and the generated $\si{\tera\hertz}$ radiation differs from one another, the pulses travel at different speeds inside the crystal.
If the mismatch is too big the efficiency of production and detection of $\si{\tera\hertz}$ radiation through the crystals suffers.
The effective length at which the velocity mismatch can be tolerated is called the coherence length

\begin{equation}
    l(\omega_{\si{\tera\hertz}}) = \frac{\pi c}{\omega_{\si{\tera\hertz}} \left | n_\text{opt eff}(\omega_0) - n_{\si{\tera\hertz}}(\omega_{\si{\tera\hertz}})\right |}
\end{equation}

with 

\begin{equation}
    n_{\text{opt eff}} = n_\text{opt}(\omega) - \lambda_\text{opt}\frac{\partial n_\text{opt}}{\partial \lambda}\big{|}_{\lambda_\text{opt}}   
\end{equation}

here c is the velocity of light, $\omega_{\si{\tera\hertz}} = \frac{2\pi}{\nu_{\si{\tera\hertz}}}$ is the frequency of $\si{\tera\hertz}$ radiation, $\omega_0$ is the frequency of the laser, $n_{\si{\tera\hertz}}$ is the refractive index of $\si{\tera\hertz}$ radiation in the medium, $n_\text{opt}$ is the refractive index of the pump laser wavelength $\lambda_\text{opt}$ and $n_\text{opt eff}$ is the refractive index of the group velocity of the pump laser radiation \cite[3]{coherence_legnth}.
Because the refractive index of $\si{\tera\hertz}$ radiation and $\SI{800}{\nano\meter}$ laser light in ZnTe is very similar for $\si{\tera\hertz}$ frequencies up $\SI{2.5}{\tera\hertz}$ \cite{coherence_legnth} the coherence length for those frequencies are long enough for the application in the setup described in section \ref{sec:setup}.
The relation between coherence length and $\si{\tera\hertz}$ frequency in ZnTe is shown in figure \ref{fig:coherence_legnth}.

\begin{figure}
    \centering
    \includegraphics[width=0.5\textwidth]{refferenced_pic/coherence_length_ZnTe.png}
    \caption{The coherence length of an $\SI{800}{\nano\meter}$ laser pulse and $\si{\tera\hertz}$ radiation in dependence of the $\si{\tera\hertz}$ frequency.
    The solid line includes the effect of dispersion at optical frequencies. The dotted line neglects the dispersion at optical frequencies.
    The plot was taken from source \cite{coherence_legnth}.}
    \label{fig:coherence_legnth}
\end{figure}
\textbf{Platzhalter für Grafik mit kohärenzlänge von GaP falls ich dazu was finde}
% I need a paper to refernce the coherence length of GaP
\FloatBarrier
