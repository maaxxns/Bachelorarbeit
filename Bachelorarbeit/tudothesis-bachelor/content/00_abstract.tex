\thispagestyle{plain}

\section*{Abstract}
\begin{foreignlanguage}{english}
This thesis demonstrates the production of broadband terahertz radiation, through optical rectification, with a $\SI{1}{\milli\meter}$ zinc telluride and a $\SI{0.3}{\milli\meter}$ gallium phosphide crystal.
A $\SI{1000}{\hertz}$ Ti:sapphire tabletop laser source with $\SI{800}{\nano\meter}$ wavelength, a bandwidth of around $\SI{100}{\femto\second}$, and varying laser powers is used to showcase the dependence of the pump power on the terahertz generation.
The terahertz radiation is detected in the time-domain by time-resolved electro-optic sampling.
The electric field strength, power of the electric field, and the conversion efficiency from pump power to terahertz electric field power are calculated for both emitters and then compared.
Terahertz electric field strengths of about $\SI{9.6}{\kilo\V\per\centi\meter}$ are reached with the zinc telluride crystal.
The highest conversion efficiency reached is around $\SI{2.1e-05}{}$.
The showcased measurements encourage the further use of zinc telluride crystals to generate broadband terahertz radiation.
The result of this thesis is that a $\SI{1}{\milli\meter}$ zinc telluride generates terahertz radiation at higher electric field strengths and has a greater conversion efficiency than a $\SI{0.3}{\milli\meter}$ gallium phosphide crystal. 
\end{foreignlanguage}
