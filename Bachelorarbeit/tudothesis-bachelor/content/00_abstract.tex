\thispagestyle{plain}

\section*{Abstract}
\begin{foreignlanguage}{english}
This thesis demonstrates on the production of broadband terahertz radiation, through optical rectification, with a $\SI{1}{\milli\meter}$ zinc telluride and a $\SI{0.3}{\milli\meter}$ gallium phosphide crystal.
A $\SI{1000}{\hertz}$ Ti:sapphire table top laser source with $\SI{800}{\nano\meter}$ wavelength, a bandwith of around $\SI{100}{\femto\second}$ and varying laser powers is used to showcase the dependence of the pump power strength on the terahertz generation.
The terahertz radiation will be detected in the timedomain through means of time-resolved electro-optic sampling.
The electric field strength, power of the electric field and the conversion effiency from pump power to terahertz electric field power will be calculted for both emitters, which will be then compared.
Terahertz electric field strengths of about $\SI{9.59(15)}{\kilo\V\per\centi\meter}$ were reached with the zinc telluride crystal.
The highest conversion efficiency that could be reached was around $\SI{2.08(7)e-05}{}$.
The showcased measurements encourage the further use of zinc telluride crystals as a way of producing broadband terahertz radiation.
The results of this thesis is that a $\SI{1}{\milli\meter}$ zinc telluride porduces terahertz radiation at higher electric field strengths and has a greater conversion efficiency than a $\SI{0.3}{\milli\meter}$ gallium phosphide crystal. 
\end{foreignlanguage}
