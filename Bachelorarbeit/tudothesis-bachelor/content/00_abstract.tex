\thispagestyle{plain}

\section*{Abstract}

This thesis demonstrates the generation of broadband terahertz radiation, through optical rectification, with a $\SI{1}{\milli\meter}$ zinc telluride and a $\SI{0.3}{\milli\meter}$ gallium phosphide crystal.
A Ti:sapphire tabletop laser source with $\SI{800}{\nano\meter}$ wavelength, a pulselength of around $\SI{100}{\femto\second}$, and a laser power of around $\SI{6.9}{\W}$ are used to showcase the influence of the pump power on the terahertz generation.
The terahertz radiation is detected in the time-domain by time-resolved electro-optic sampling.
The electric field strength, power of the electric field, and the conversion efficiency from pump power to terahertz electric field power are calculated for both emitters and are compared subsequently.
Terahertz electric field strengths of about $\SI{9.6}{\kilo\V\per\centi\meter}$ are reached with the zinc telluride crystal.
The highest conversion efficiency reached is around $\SI{2.1e-05}{}$.
The highest electric field reached with gallium phosphide is $\SI{3.38(5)}{\kilo\V\per\centi\meter}$.
Its highest conversion efficiency is at around $\SI{2.71e-6}{}$.
The showcased measurements encourage the further use of zinc telluride crystals to generate broadband terahertz radiation.
The main result of this thesis is that a $\SI{1}{\milli\meter}$ zinc telluride generates terahertz radiation at higher electric field strengths and has a greater conversion efficiency than a $\SI{0.3}{\milli\meter}$ gallium phosphide crystal. 
\section*{Kurzfassung}
\begin{foreignlanguage}{ngerman}
Diese Arbeit beschäftigt sich mit der Erzeugung von Breitband-Terahertz Strahlung, durch den elektrooptischen Effekt mit einem $\SI{1}{\milli\meter}$ Zinktellurid und einem $\SI{0.3}{\milli\meter}$ Galliumphosphid Kristall.
Ein Titan:Saphir-Laser mit $\SI{800}{\nano\meter}$ Wellenlänge, einer Pulselänge von ungefähr $\SI{100}{\femto\second}$ und einer Leistung von circa $\SI{6.9}{\W}$ wird genutzt, um den Einfluss der optischen Pumpleistung auf die Terahertz Erzeugung zu zeigen.
Die Terahertzstrahlung wird in der Zeitdomäne durch zeitaufgelöstes elektrooptische Abtasten gemessen.
Die elektrische Feldstärke, die Leistung des elektrischen Feldes und die Umwandlungseffizienz von Pumpleistung zu der Leistung des elektrischen Feldes der Terahertzstrahlung werden für beide Emitter berechnet und danach verglichen.
Elektrische Feldstärke der Terahertzstrahlung von bis zu $\SI{9.6}{\kilo\V\per\centi\meter}$ werden mit Zinktellurid erreicht.
Die höchste Umwandlungseffizienz liegt für Zinktellurid bei ungefähr $2.1\cdot 10^{-5}$.
Das höchste elektrische Feld, welches mit Galliumphosphid erreicht wird, liegt bei $\SI{3.38(5)}{\kilo\V\per\centi\meter}$.
Die höchste Umwandlungseffizienz für Galliumphosphid ist circa $2.71\cdot 10^{-6}$.
Die gezeigten Messungen unterstützen den weiteren Gebrauch von Zinktellurid als Quelle für Breitband Terahertzstrahlung.
Das vorrangige Ergebnis dieser Arbeit ist, dass ein  $\SI{1}{\milli\meter}$ Zinktellurid Kristall stärkere elektrische Felder erzeugt und eine höhere Umwandlungseffizienz hat, also ein $\SI{0.3}{\milli\meter}$ Galliumphosphid Kristall.
\end{foreignlanguage}