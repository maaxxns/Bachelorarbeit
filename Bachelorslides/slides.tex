\PassOptionsToPackage{unicode}{hyperref}
\documentclass[aspectratio=1610, 9pt]{beamer}

% Load packages you need here
\usepackage{polyglossia}
\setmainlanguage{english}

\usepackage{csquotes}
    

\usepackage{amsmath}
\usepackage{amssymb}
\usepackage{mathtools}

\usepackage{hyperref}
\usepackage{bookmark}
\usepackage[
  locale=UK,
  separate-uncertainty=true,
  per-mode=symbol-or-fraction,
]{siunitx}
\usepackage[
  backend=biber,   % use modern biber backend
  autolang=hyphen, % load hyphenation rules for if language of bibentry is not
  % german, has to be loaded with \setotherlanguages
  % in the references.bib use langid={en} for english sources
  sorting=none,
  ]{biblatex}
  \addbibresource{references.bib}  % the bib file to use
  \DefineBibliographyStrings{english}{andothers = {{et\,al\adddot}}}  % replace u.a. with et al.
  
  
% load the theme after all packages
\usetheme[
  showtotalframes, % show total number of frames in the footline
]{tudo}

% Put settings here, like
\unimathsetup{
  math-style=ISO,
  bold-style=ISO,
  nabla=upright,
  partial=upright,
  mathrm=sym,
}
\setbeamertemplate{caption}{\raggedright\insertcaption\par}

\title{Generation and time-resolved detection of
terahertz radiation}
\subtitle{Bachelorthesis presentation}
\author[M.~Koch]{Max Koch}
\institute[AG Wang]{Arbeitsgruppe Wang \\  Fakultät Physik}
%\titlegraphic{\includegraphics[width=0.3\textwidth, angle=90]{images/setup.jpeg}}


\begin{document}

\maketitle

\section{Intro}

\begin{frame}{The THz Gap}
  \subsection{The spectrum}
  \begin{center}
    \begin{figure}
      \includegraphics[width=0.7\textwidth]{images/spectrum.png}
      \caption{\textcolor{tugreen}{The electromagntic spectrum} from G. P. Williams, Rep. Prog. Phys, \textbf{69} (2005)\nocite{spectrum_pic}.}
    \end{figure}
  \end{center}
\end{frame}


\begin{frame}{Terahertz}
  \subsection{Applications for THz}
  \begin{center}
    \begin{minipage}[c]{0.5\linewidth}
      So why do we need terahertz radiation?
      \vspace{0.2in}
      \begin{itemize}
        \item medicine \nocite{THzgap_applications}
        \vspace{0.1in}
        \item security \nocite{thz_explosive_detec}
        \vspace{0.1in}
        \item data transmission \& saving \nocite{communication,datasaving}
        \vspace{0.1in}
        \item physics \nocite{wiki_book}
      \end{itemize}
    \end{minipage}
\end{center}
\end{frame}

\begin{frame}{Outline}
  \tableofcontents
\end{frame}

\section{Setup}

\begin{frame}{But how to generate THz?}
  \includegraphics[width=\textwidth]{images/Aufbau.pdf}
\end{frame}

\section{Theory}
\subsection{Optical-rectification}
\begin{frame}{Optical-rectification}
 Polarisation \nocite{wiki_book}
  \begin{align}
    \frac{P}{\epsilon_0} =& \chi_1E + \chi_2 E^2 + \chi_3 E^3 + ... \\
    \frac{P_\text{nl}}{\epsilon_0} =& \chi_2 \frac{E_0^2}{2} \left[ \cos((\omega_\text{i} - \omega_\text{j})t) + \cos((\omega_\text{i} + \omega_\text{j})t) \right ]
  \end{align}
  Energy conservation \nocite{boyd2020nonlinear}
  \begin{columns}
    \begin{column}{.4\textwidth}
      \begin{itemize}
        \item two photons go in
        \item higher energy level
        \item two photon emission 
        \item energy drops to groundstate
      \end{itemize}
    \end{column}
    \begin{column}{.5\textwidth}
      \includegraphics[width=\textwidth]{images/diffrence_frequency_mixing.PNG}
    \end{column}
  \end{columns}
\end{frame}

\subsection{Electro-optic sampling}

\begin{frame}[t]{Electro-optic sampling (EOS)}
\begin{columns}
  \begin{column}{.5\textwidth}
  And how to detect it?\\
  \vspace{0.3in}
  $\rightarrow$\textbf{Pockels effect!}\nocite{THZ_eltric_field}
  \vspace{0.1in}
    \begin{equation}
      \frac{\symup{\Delta}I}{I} = \text{sin}(\theta) = \frac{2\pi}{\lambda} n_0^3 l r E_\text{THz}
    \end{equation}
  \end{column}
  \begin{column}{.5\textwidth}
    \vspace{0.3in}
    \begin{figure}
      \centering
      \includegraphics[width=0.8\textwidth]{images/detectionunit.png}
    \end{figure}
  \end{column}
\end{columns}
\end{frame}

\section{Results}
\subsection{ZnTe}
\begin{frame}{EOS ZnTe}
\begin{figure}
  \includegraphics[width=0.8\textwidth]{images/2_11_30_20normalX.pdf}
  \caption{\textcolor{tugreen}{EOS} of ZnTe with $\SI{135.0}{\milli\W}$.}
\end{figure}
\end{frame}

\begin{frame}{Spectrum ZnTe with FFT}
  \begin{columns}
    \begin{column}{.5\textwidth}
  \begin{figure}
    \includegraphics[width=\textwidth]{images/2_11_30_20normalFX.pdf}
    \caption{\textcolor{tugreen}{Spectrum} of ZnTe with $\SI{135.0}{\milli\W}$.}
  \end{figure}
  \end{column}
  \begin{column}{.5\textwidth}
    \begin{figure}
      \includegraphics[width=\textwidth]{images/2_11_30_20normallog(FX).pdf}
      \caption{\textcolor{tugreen}{Spectrum} of ZnTe with $\SI{135.0}{\milli\W}$ with log y-axis.}
    \end{figure}    
  \end{column}
  \end{columns}
\end{frame}

\subsection{GaP}
\begin{frame}{EOS GaP}
  \begin{figure}
    \includegraphics[width=0.8\textwidth]{images/GaP14_55_42normalX.pdf}
    \caption{\textcolor{tugreen}{EOS} of GaP with $\SI{124.2}{\milli\W}$.}
  \end{figure}
\end{frame}

\begin{frame}{Spectrum GaP with FFT}
  \begin{columns}
    \begin{column}{.5\textwidth}
  \begin{figure}
    \includegraphics[width=\textwidth]{images/GaP14_55_42normalFX.pdf}
    \caption{\textcolor{tugreen}{Spectrum} of GaP with $\SI{124.2}{\milli\W}$.}
  \end{figure}
  \end{column}
  \begin{column}{.5\textwidth}
    \begin{figure}
      \includegraphics[width=\textwidth]{images/GaP14_55_42normallog(FX).pdf}
      \caption{\textcolor{tugreen}{Spectrum} of GaP with $\SI{124.2}{\milli\W}$ with log y-axis.}
    \end{figure}    
  \end{column}
  \end{columns}
\end{frame}

\subsection{Comparison}

\begin{frame}{Comparison Spectra}
  \begin{columns}
    \begin{column}{.5\textwidth}
      \begin{center}
        \begin{figure}
          \includegraphics[width=\textwidth]{images/2_11_30_20normalFX.pdf}
          \caption{\textcolor{tugreen}{Spectrum} of ZnTe with $\SI{135.0}{\milli\W}$ pump power.}
        \end{figure}
      \end{center}
  \end{column}
  \begin{column}{.5\textwidth}
    \begin{center}
      \begin{figure}
        \includegraphics[width=\textwidth]{images/GaP14_55_42normalFX.pdf}
        \caption{\textcolor{tugreen}{Spectrum} of GaP with $\SI{124.2}{\milli\W }$ pump power.}
      \end{figure}  
    \end{center}  
  \end{column}
  \end{columns}
\end{frame}

\begin{frame}{Filter}
  Switch filter for different pump powers:
  \begin{center}
  \begin{figure}
    \begin{overprint}
      \onslide<1>\includegraphics[width=\textwidth]{images/Aufbau.pdf}
      \onslide<2>\includegraphics[width=\textwidth]{images/Aufbau_blue_square.pdf}
    \end{overprint}
  \end{figure}  
\end{center}
\end{frame}

\begin{frame}{Electric field}
  \begin{equation}
    \frac{\symup{\Delta}I}{I} = \text{sin}(\theta) = \frac{2\pi}{\lambda} n_0^3 l r E_\text{THz}
  \end{equation}
  \begin{columns}
    \begin{column}{.5\textwidth}
      \begin{center}
      \begin{figure}
        \centering
        \begin{overprint}
          \onslide<1>\includegraphics[width=\textwidth]{images/2_11_30_20normalX.pdf}\caption{\textcolor{tugreen}{EOS signal} of ZnTe with $\SI{135.0}{\milli\W}$ pump power.}
          \onslide<2>\includegraphics[width=0.75\textwidth]{images/eltric_field_ZnTe.pdf}\caption{\textcolor{tugreen}{Peak electric fields} of ZnTe with different pump powers.}
        \end{overprint}
      \end{figure}
    \end{center}
  \end{column}
  \begin{column}{.5\textwidth}
    \begin{center}
    \begin{figure}
      \centering
      \begin{overprint}
        \onslide<1>\includegraphics[width=\textwidth]{images/GaP14_55_42normalX.pdf}\caption{\textcolor{tugreen}{EOS signal} of GaP with $\SI{124.2}{\milli\W }$.}
        \onslide<2>\includegraphics[width=0.75\textwidth]{images/eltric_field_GaP.pdf}\caption{\textcolor{tugreen}{Peak electric fields} of GaP with different pump powers.}
      \end{overprint}
    \end{figure}
  \end{center}
  \end{column}
  \end{columns}
\end{frame}

\begin{frame}{THz Power}
 \nocite{griffiths}
  \begin{align}
    I_\text{THz} =& c \epsilon_0 E_\text{THz}^2 \\
    P =& \int\, I_\text{THz} \,\symup{d}A_\text{spot} \approx I_\text{THz}\cdot A_\text{spot}
\end{align}
  \begin{columns}
    \begin{column}{.5\textwidth}
  \begin{figure}
    \includegraphics[width=\textwidth]{images/Powerznte.pdf}
    \caption{ZnTe \textcolor{tugreen}{peak electric field powers} and \textcolor{tugreen}{conversion effiency} for different pump powers.}
  \end{figure}
  \end{column}
  \begin{column}{.5\textwidth}
    \begin{figure}
      \includegraphics[width=\textwidth]{images/Powergap.pdf}
      \caption{GaP \textcolor{tugreen}{peak electric field powers} and \textcolor{tugreen}{conversion effiency} for different pump powers.}
    \end{figure}    
  \end{column}
  \end{columns}
\end{frame}

\begin{frame}{Conclusion}
  \begin{center}
  \begin{columns}
    \begin{column}{0.5\textwidth}
      \textbf{ZnTe reached}\\
      \only<1->{electric field strengths $\SI{9.59}{\kilo\V\per\centi\meter}$\\}
      \only<3->{conversion effiencies $\SI{2.08e-5}{}$}
    \end{column}
    \begin{column}{.5\textwidth}
      \textbf{GaP reached}\\
      \only<2->{electric field strengths $\SI{3.38}{\kilo\V\per\centi\meter}$\\}
      \only<4>{conversion effiencies of $\SI{2.71e-6}{}$}
    \end{column}
  \end{columns}
  \vspace{0.2in}
  \begin{itemize}
  \item recommend further use of ZnTe as easy to use broadband THz-source
  \item GaP seems to be not suited for the generation of THz radation with a $\SI{800}{\nano\meter}$ Laser
  \item Further measurments with more comparable crystal size are recommend
  \end{itemize}
  \end{center}
\end{frame}

\section*{Thank you}
\begin{frame}{}
  \begin{center}
  \textbf{\textcolor{tugreen}{Thank you all for your attention!}}
  \end{center}
\end{frame}

\section{References }
\printbibliography

\begin{frame}{Lower/Higher initial Power}
  \begin{figure}
    \centering
    \includegraphics[width=\textwidth]{images/Aufbau_vor.pdf}
    \caption{The beam path before it reaches the shown setup.}
  \end{figure}
\end{frame} 

\begin{frame}{Coherence-length}
Defined as \nocite{coherence_legnth}
\begin{equation}
    l(\omega_{\text{THz}}) = \frac{\pi c}{\omega_{\text{THz}} \left | n_\text{opt eff}(\omega_0) - n_{\text{THz}}(\omega_{\text{THz}})\right |}
\end{equation}
with 
\begin{equation}
    n_{\text{opt eff}} = n_\text{opt}(\omega) - \lambda_\text{opt}\frac{\partial n_\text{opt}}{\partial \lambda}\biggl{|}_{\lambda_\text{opt}}  
\end{equation}
\begin{center}
  \begin{columns}
    \begin{column}{.4\textwidth}
    \begin{figure}
      \includegraphics[width=0.9\textwidth]{images/coherence_length_ZnTe.png}
      \caption{\tiny{\textcolor{tugreen}{(ZnTe) With $\SI{800}{\nano\meter}$ pump laser} from  A. Nahata and A. S. Weling, A wideband coherent terahertz spectroscopy system using optical rectification and electro-optic sampling, Appl. Phys. Lett., \textbf{69}, (2014)\nocite{coherence_legnth}.}}
    \end{figure}%
    \end{column}
    \begin{column}{.4\textwidth}
    \begin{figure}
      \includegraphics[width=0.8\textwidth]{images/GAP_coherencelength.png}
      \caption{\tiny{\textcolor{tugreen}{(GaP) With various pump laser wavelengths} from L. Jiang and L. Chai, Efficient terahertz wave generation from GaP crystals pumped by chirp-controlled pulses from femtosecond photonic crystal fiber amplifier,  Appl. Phys. Lett. \textbf{104}, (2014)\nocite{GaP_coherence_length}.}}
    \end{figure}
    \end{column}
  \end{columns}
\end{center} 
\end{frame}

\begin{frame}{Other GaP measurments}
  \begin{figure}
    \includegraphics[width=0.8\textwidth]{images/GaP14_20_20normalX.pdf}
    \caption{\textcolor{tugreen}{EOS} of GaP with $\SI{24.2}{\milli\W}$ pump power.}
  \end{figure}
\end{frame}


\end{document}

