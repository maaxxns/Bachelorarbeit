\documentclass[twoside,a4paper,12pt,german]{report}
    \usepackage{fontspec}
    \defaultfontfeatures{Ligatures=TeX}  % -- becomes en-dash etc.

    % \usepackage[ansinew]{inputenc}
    \usepackage[american, ngerman]{babel}
    \usepackage[autostyle]{csquotes}
    \usepackage[active]{srcltx} % SRC Specials for DVI Searching
   % \usepackage{amssymb} % AMS-LaTeX Symbole werden eingebunden
   % \usepackage{amsmath}
   \usepackage[centertags]{amsmath}
   \usepackage{amsfonts}
   \usepackage{amssymb}
   \usepackage{amsthm}
    \usepackage{a4}
    \usepackage{graphicx} % Zum Einbinden von BMP - Bitmaps
    \usepackage[dvips]{epsfig} % Zum Einbinden von EPS- Vektorgrafiken
    \usepackage{multicol}
    \usepackage{alltt} % Programmlisting am Ende der Datei kann dargestellt werden
    \usepackage{makeidx} \makeindex % Erstellen eines Index
    \usepackage{fancyhdr}  % Die neue Version von fancyheadings zur Erstellung von Kopfzeilen
    \usepackage{longtable}
    \pagestyle{fancyplain}


    \usepackage{graphicx}
    \DeclareGraphicsRule{.wmf}{bmp}{tif}{}% declare WMF filename extension




%ich habe einen eigenen Bildbefehl definiert: \bild der wie unten
%funktioniert. Das Bild selbst heisst 'test,wmf', 270 w�re die
%Breite des Bildes, 180 die H�he, und 'beispiel' der Text, der
%unter das Bild gesetzt wird.

%\bild{htb}{test.wmf}{270}{180}{beispiel}

\newcommand{\bild}[5]{ % wo htb , name , breite , hoehe , Unterschrift
\begin{figure}[#1]
\unitlength1mm
\begin{center}
\includegraphics*[bb=0 0 #3 #4]{#2}
\\[-2mm]
\addtolength{\tbreite}{-10mm}
\begin{minipage}[t]{\tbreite}
\caption[#2, Breite: #3mm, Höhe: #4mm] %[#6]
{\label{#2} \sl #5}
\end{minipage}
\end{center}
\setlength{\tbreite}{\textwidth}
\end{figure}
\setlength{\tbreite}{\textwidth}}
