\thispagestyle{empty}

\noindent
\underline{\bf Verzeichnis der verwendeten Abk*rzungen und Symbole:} \rm


\begin{tabbing}
HITRAN \quad \=test \=\kill
BAT \> biologische Arbeitsstoff--Toleranz\\
CLS \> Classical Least Squares\\
DFT \> Diskrete Fourier--Transformation\\
DOAS \> Differential Optical Absorption Spectroscopy\\
EPA \> U.S. Environmental Protection Agency\\
FCKW \> Fluorchlorkohlenwasserstoffe\\
FFT \> Fast Fourier--Transformation\\
FTIR \> Fourier--Transformation Infrarot\\
GEISA \> Gestion et *tude des  Informations\\
\> Spectroscopiques Atmosph*riques--Datenbank\\
HITRAN \> High Resolution Transmission\\
\> Molecular Absorption--Datenbank\\
IR \> infrarot\\
i.p. \> in press\\
ISAS \> Institut f*r Spektrochemie und angewandte\\
\> Spektroskopie, Dortmund\\
K300 \> in dieser Arbeit verwendetes FTIR--Spektrometer f*r Messungen\\
\> in der offenen Atmosph*re (Aufl*sung routinem**ig 0.2 cm\up{-1})\\
KT \> Kayser--Threde GmbH, M*nchen\\
LIDAR \> Light Detecting and Ranging\\
LUA \> Landesumweltamt NRW, Essen\\
MAK \> Maximale Arbeitsplatz--Konzentration\\
MCT \> Mercium Cadmium Tellurid\\
NIST \> National Institute of Standards and Technology, U.S.A.\\
NWG \> Nachweisgrenze\\
OPD \> Optical Path Difference\\
PLS \> Partial Least Squares\\
PPB \> Parts per Billion\\
PPM \> Parts per Million\\
STD \> Spectral Standard Deviation\\
TDLAS \> Tunable Diode Laser Absorption Spectroscopy\\
TO--16 \> Compendium of Methods for the Determination of Toxic\\
\> Organic Compounds in Ambient Air\\
VDI \> Verein deutscher Ingenieure\\
\> \\
\> \\
$\bf A$ \> Matrizen werden mit fetten Gro*buchstaben dargestellt\\
$\bf x$ \> Vektoren werden mit fetten Kleinbuchstaben dargestellt\\
$\bf A^T$ \> transponierte Matrix\\
$\bf A^{-1}$ \> inverse Matrix\\
$\bf A^*$ \> adjungierte Matrix\\
$\alpha$ \> Absorptionskoeffizient\\
\bf COV \rm \> Kovarianzmatrix\\
E \> Extinktion\\
$\epsilon$ \> Extinktionskoeffizient\\
I \> Intensit*t\\
$k_{\stackrel{\sim}{\nu}}$ \> Absorptionsquerschnitt\\
\bf K \rm \> Referenzspektren--Matrix\\
$\lambda$ \> Wellenl*nge\\
$\mu$ \> Dipolmoment\\
$\sigma$ \> spektrale Standardabweichung\\
t \> Zeit\\
T \> Temperatur\\
$\stackrel{\sim}{\nu}$ \> Wellenzahl\\

\end{tabbing}

\cleardoublepage
