\chapter{\label{grundlagen}Grundlagen FTIR und alternative Messmethoden}

FTIR--Spektrometer besitzen im mittleren Infrarotbereich eine
ganze Reihe von Vorteilen gegenüber dispersiven Spektrometern
(siehe auch Kap. \ref{vorteile} und \cite{griffiths77}), benötigen
aber für die zu rechnende Fourier--Transformation leistungsstarke
Rechner. Seit den letzten zwei Jahrzehnten stehen solche Rechner
zur Verfügung und die Fourier--Transformations--Spektroskopie hat
die dispersive Spektroskopie aufgrund ihrer überlegenheit im
mittleren Infrarotbereich vollständig verdrängt. In dieser Arbeit
konnten deutliche Verbesserungen bei der quantitativen Auswertung
von FTIR--Spektren aus Messungen in der offenen Atmosphäre erzielt
werden. Um das Verständnis der folgenden Kapitel zu erleichtern,
werden zuerst die Grundlagen der FTIR--Spektroskopie kurz
zusammengestellt und im letzten Abschnitt mit alternativen
Techniken zur Messung von Schadgasen in der offenen Atmosphäre
verglichen. Für eine ausführliche Darstellung der Grundlagen der
FTIR--Spektroskopie sei auf die einschlägigen Lehrbücher verwiesen
(z.B. \cite{griffiths86}, \cite{günzler96}, \cite{gottwald97} oder
\cite{schrader95}).

\section{\label{molekuelbau}Molekülbau und Wechselwirkung mit
 IR--\-Strahl\-ung}

In der IR--Spektroskopie wird üblicherweise die Wellenzahl
$\stackrel{\sim}{\nu}$ als Einheit zur Beschreibung der
elektromagnetischen Strahlung verwendet. Sie gibt die Anzahl der
elektromagnetischen Wellen pro cm an. Für die Umrechnung der
Wellenlänge $\lambda$ [$\mu$m] in die Wellenzahl
$\stackrel{\sim}{\nu}$ [cm\up{-1}] gilt:
\begin{equation}\label{eqlanu}
  \stackrel{\sim}{\nu}=\frac{10000}{\lambda}
\end{equation}

Der infrarote Bereich im elektromagnetischen Spektrum erstreckt
sich von 10--12800 cm\up{-1} (1000--0.78 $\mu$m). Dieser Bereich
wird noch einmal unterteilt in
\begin{itemize}
  \item das nahe IR (4000--12800 cm\up{-1}),
  \item das mittlere IR (200--4000 cm\up{-1})
  \item und das ferne IR (10--200 cm\up{-1}).
\end{itemize}


\subsection{\label{rotschwing}Rotations-- Schwingungsspektren}

In der Praxis wird das zu analysierende Gas mit einer
breitbandigen Infrarotstrahlungsquelle, üblicherweise ein
Silicium--Carbid--Stift, auch Globar genannt, mit einer
Be\-triebs\-tem\-pe\-ra\-tur von 1200--1600 K bestrahlt. Die
IR--Strahlung, die bestimmte Atomgruppen im betrachteten Molekül
zum Schwingen anregt, wird beim Durchgang durch das zu
analysierende Gas absorbiert und liefert die charakteristischen
Absorptionsbanden im Spektrum. Für das Auftreten von
detektierbaren Absorptionssignalen im Spektrum ist eine
periodische änderung des Dipolmomentes $\mu$ erforderlich. Die
änderung der Sig\-nal\-in\-ten\-si\-tät I ist dabei direkt
proportional zum Quadrat der Dipolmomentänderung $\Delta \mu^2$.
\begin{equation}\label{eqmu}
  I \sim \Delta \mu^2
\end{equation}

Moleküle, bei denen die Schwingungsanregung mit einer
Dipolmomentänderung verbunden ist, werden \it IR--aktiv \rm
genannt (z.B. NH\down{3}, N\down{2}O). Demgegenüber sind z.B. die
Schwingungen von homöonuklearen zweiatomigen Molekülen \it
IR--inaktiv \rm , da hier keine Dipolmomentänderung gegeben ist.\\

Die Anzahl der Schwingungsfreiheitsgrade Z (\it
Normalschwingungen\rm ) berechnen sich in einem linearen Molekül
(z.B. CO\down{2}) zu (N--Anzahl der Atome im Molekül):
\begin{equation}\label{eqlin}
 \text{Z}=3 \cdot \text{N} - 5
\end{equation}
und bei einem nichtlinearen Molekül (z.B. H\down{2}O) zu:
\begin{equation}\label{eqnlin}
 \text{Z}=3 \cdot \text{N} - 6
\end{equation}

Neben diesen Normalschwingungen können in einem Spektrum auch
intensitätsmä{\ss}ig schwächere \it Oberschwingungsbanden \rm
auftreten, die von übergängen mit einer
Schwingungsquantenzahländerung $\mid\!\Delta v\!\mid>1$ in andere
Schwingungszustände resultieren. Weiterhin können auch Banden
auftreten, die durch eine kombinierte Anregung von
Normalschwingungen entstehen (\it Kombinationsbanden\rm ).\\

Weisen zwei verschiedene Normalschwingungen die gleiche Frequenz
auf, so nennt man diese \it entartet\rm . Nun können aber auch
Grund-- und Oberschwingungen verschiedener Schwingungsformen
zufällig dieselbe Energie besitzen. Diese werden dann \it zufällig
entartet \rm genannt. Die Folge der zufälligen Entartung (z.B. bei
gleicher Symmetrie der Schwingungsniveaus) ist die Absto{\ss}ung der
Energieniveaus und ein Auseinanderrücken beider Schwingungsbanden
(\it Fermi--Resonanz\rm ) und ein Angleichen der
Bandenintensitäten.\\

Für die Wellenzahl $\stackrel{\sim}{\nu}$ eines
Rotationsschwingungsübergangs für ein lineares zweiatomiges
Molekül mit parallel zur Molekülachse schwingendem Dipolmoment
gilt mit der Schwingungsquantenzahl v und der Rotationsquantenzahl
J in guter erster Näherung:

\begin{equation}\label{eqnu}
  \stackrel{\sim}{\nu} = G(v')-G(v'')+B'_vJ'(J'+1)-B''_vJ''(J''+1)
\end{equation}
mit: $G(v)\,=\,\stackrel{\sim}{\nu}(v+\frac{1}{2})$ und
$B_v=\frac{h}{8\pi^{2}c \cdot I}-\alpha(v+\frac{1}{2})$
(I--Trägheitsmoment,
$\alpha$--Rotations--Schwingungs--Wechselwirkungskonstante,
c--Lichtgeschwindigkeit).\\

Als Auswahlregeln gelten für die Schwingungsquantenzahl $\Delta v
= \pm 0,1,2,3 \ldots$ und für die Rotationsquantenzahl $\Delta
J=\pm 1$. Banden mit $\Delta J=-1$ werden \it P--Zweig \rm und mit
$\Delta J=1$ \it R--Zweig \rm genannt.\\

Für lineare Moleküle mit senkrecht zur Molekülachse ausgeführten
Schwingungen sind die Auswahlregeln zu erweitern; zugelassen ist
in diesem Fall auch $\Delta J=0$ (\it Q--Zweig\rm ). Für
komplexere Molekülmodelle sei auf \cite{hollas92} verwiesen.\\

Die für ein Molekül charakteristischen Rotations-- und
Schwingungsübergänge liefern vorwiegend Banden auch im mittleren
Infrarot im Bereich von 650--4000 cm\up{-1}. Aus diesem Grund sind
die in dieser Arbeit verwendeten Geräte auf diesen
Wellenzahlbereich optimiert.\\


\subsection{\label{qualitativ}Qualitative Spektrenaussagen}

Ein IR--Spektrum ist für jede Substanz charakteristisch und kann
zu ihrer Identifizierung dienen. Die gesamte Information des
molekularen Stoffaufbaus steckt im IR--Spektrum, so dass im
Prinzip eine Gesamtspektrenanalyse möglich ist. Da die Spektren
aber mannigfaltig sind wie die Anzahl der chemischen Substanzen,
gelingt eine Modellierung mit einem erheblichen Rechenaufwand nur
für relativ einfach gebaute Moleküle. Oft helfen aber
Interpretationen der Spektren aufgrund von empirischen Regeln, um
qualitative Aussagen machen zu können.\\

Bei organischen Verbindungen besteht das Grundgerüst aus C--, N--
und O--Atomen, deren Einfachbindungsstärke ähnlich gro{\ss} ist.
Schwingungen, die solche molekularen Teilgruppen betreffen,
befinden sich im Bereich von 1000--1430 cm\up{-1}. Vielfach lässt
sich eine Zuordnung zu Teilstrukturen schwierig treffen. Der
Bereich ist aber sehr gut geeignet, mit Vergleichsspektren einer
Bibliothek die Substanz zu identifizieren (\it
Fingerprint--Bereich \rm). Grundsätzlich lassen sich folgende
Aussagen treffen:\\

\noindent\underline{Bereich von 1000--1270 cm\up{-1}}:\ Starke
Banden weisen auf C--O--haltige Verbindungen wie Ester, Ether,
Alkohole und Phenole hin.\\

\noindent\underline{Bereich von 1350--1440 cm\up{-1}}:\ In diesem
Bereich deuten Deformationsschwingungen von
Kohlenwasserstoffgruppen z.B. auf Methyl-- oder Methylengruppen
hin, wobei an einem Kohlenstoffatom hängende Dimethyl-- oder
Trimethylgruppen typische Dublette ausbilden.\\

\noindent\underline{Bereich von 1500--1900 cm\up{-1}}:\ Alkene
besitzen im Bereich von 1630--1685 cm\up{-1}, aromatische Ringe
zwischen 1460 und 1600 cm\up{-1} charakteristische Banden.
Zwischen 1660 und 1950 cm\up{-1} weisen Carbonylgruppen starke
Banden auf.\\

\noindent\underline{Bereich von 2250--3800 cm\up{-1}}:\ Gesättigte
Kohlenwasserstoffe zeigen CH--Streck\-schwing\-ungs\-ban\-den im
Bereich von 2850--3000 cm\up{-1}, aromatische oder ungesättigte
Grundgerüste im Bereich 3000--3100 cm\up{-1}.\\

Nach dieser ersten Grobeinteilung der Banden eines Spektrums,
können dann anhand sehr viel feinerer Korrelationstabellen
Aussagen über das Gerüst und die anhängenden funktionellen Gruppen
getroffen werden (siehe z.B. \cite{günzler96},\\
\noindent\cite{gottwald97}).\\


\section{\label{ftir}FTIR--Spektroskopie}

\subsection{\label{spektrometertypen}Michelson--Interferometer}

Dispersive Spektrometer zerlegen die einfallende Strahlung mittels
eines Prismas oder eines Gitters in ihre Spektralfarben.
Wellenzahlabhängig wird dann in der Spektrometrie am Detektor die
Strahlungsleistung sowohl mit als auch ohne Probe gemessen.\\

% \bild{htb}{michelso.wmf}{400}{300}{\it Schema eines
% Fourier--Transform--Spektrometers mit klassischem
% Michelson--Interferometer.}

Fourier--Transform--Spektrometer hingegen, die in den
überwiegenden Fällen auf einem
Mi\-chel\-son--\-In\-ter\-fe\-ro\-me\-ter basieren (siehe Abb.
\ref{michelso.wmf}), nutzen die gesamte auffallende
Strahlungsleistung aus. Eine zunächst einmal monochromatische
Strahlungsquelle G emittiere Strahlung, die auf den
halbdurchlässigen Strahlteiler trifft. Die eine Hälfte der
Strahlungsleistung wird in Richtung des festen Spiegels
reflektiert, die andere Hälfte fällt auf den beweglichen Spiegel,
der sich mit annähernd konstanter Geschwindigkeit um den Punkt
gleichen Abstandes beider Spiegel vom Strahlteiler bewegt. Bei
gleichem Abstand der beiden Spiegel vom Strahlteiler interferieren
die reflektierten Strahlen mit gleicher Phase konstruktiv mit
einem resultierenden grö{\ss}tmöglichem Maximum an Intensität (zero
OPD--\bf o\rm ptical \bf p\rm ath \bf d\rm ifference). Bewegt sich
der Spiegel und wird der OPD grö{\ss}er, so bewegen sich die zwei
Strahlen fortschreitend in und aus der Phase. Die Intensität am
Detektor variiert in Abhängigkeit einer Kosinusfunktion mit
änderung des OPDs. Die Aufzeichnung der änderung dieser Intensität
nach Abzug eines konstanten Terms wird \it Interferogramm \rm
genannt.\\

Bei polychromatischen Strahlungsquellen, wie z.B. einem Globar,
zeigt jede Frequenz für sich dieses Phänomen und das resultierende
Interferogramm \bf int \rm ist dann die Summe der Intensitäten
$I(\stackrel{\sim}{\nu})$ über alle Frequenzen
$\stackrel{\sim}{\nu}$ in Abhängigkeit vom Spiegelweg x:
\begin{equation}\label{eqint}
  \bf int\rm = \int I(\stackrel{\sim}{\nu})\: \cos (2\pi \stackrel{\sim}{\nu}
  x)\;
  \text{d}\stackrel{\sim}{\nu}
\end{equation}

Parallel mit der polychromatischen IR--Strahlungsquelle wird auch
das Licht eines He--Ne--Lasers durch das Interferometer geschickt
und erzeugt ein entsprechendes Interferogramm. Das Abtastintervall
$\Delta x$, mit dem die Interferogramme digitalisiert werden,
entspricht dann einem ganzzahligen Vielfachen der Nulldurchgänge
des Lasersinussignals. $\Delta x$ bestimmt damit aus
signaltheoretischen Gründen die maximale Wellenzahl, die noch
eindeutig detektiert werden kann (\it Nyquist--Kriterium\rm ):
\begin{equation}\label{eqnyquist}
  \stackrel{\sim}{\nu}_{max}-\stackrel{\sim}{\nu}_{min} = \frac{1}{2\Delta
  x}
\end{equation}

Das Michelson--Interferometer kann in den kommerziell erhältlichen
Spektrometern auch modifiziert auftreten. So wurden die
Planspiegel beim in dieser Arbeit benutzten K300--Spektrometer
durch Kubusecken ersetzt und die Spektrometeranordnung pendelnd
aufgehängt. Dadurch konnte der Spiegelweg im Gegensatz zum
klassischen Interferometer auf ein Viertel verkürzt werden.
Zusätzlich wird durch die Kubusecken eine Verkippung und ein
Versatz der Spiegel kompensiert. Beide Modifikationen garantieren
eine höhere Stabilität gegenüber Temperaturänderungen und
Vibrationen.\\

Der Detektor wandelt das optische Signal in ein elektrisches
Signal um. Neben der Empfindlichkeit (änderung des erzeugten
elektrischen Signals in Abhängigkeit von der Strahlungsleistung),
sind vor allem auch die Ansprechgeschwindigkeit und das
Detektorrauschen als Kriterien für den Einsatz in der Praxis
wichtig. Zwei Gruppen von Detektoren werden heute in
IR--Spektrometern verwendet. Die \it thermischen Detektoren \rm
wandeln die durch die auffallende IR--Strahlung erzeugte Wärme in
elektrische Energie um. \it Quantendetektoren \rm machen sich den
inneren und äu{\ss}eren lichtelektrischen Effekt zunutze. In der
Atmosphärenspektroskopie wird eine hohe Ansprechgeschwindigkeit
und Empfindlichkeit des Detektors benötigt. MCT (\bf M\rm ercium
\bf C\rm admium \bf T\rm ellurid) --Halbleiterdetektoren sind
hierzu sehr gut geeignet. Sie müssen allerdings im Betrieb mit
flüssigem Stickstoff gekühlt werden.\\


\subsection{\label{vorteile}Vorteile von FT--Spektrometern gegenüber dispersiven
Geräten}

FT--Spektrometer haben im wesentlichen drei ganz entscheidende
Vorteile gegenüber dispersiven Spektrometern im mittleren
Infrarot:
\begin{itemize}
  \item Beim FT--Spektrometer empfängt der Detektor während der
  ganzen Messzeit alle Frequenzbestandteile der von der IR--Quelle
  ausgesandten Strahlung, was bei gleicher spektraler Auflösung
  und gleichem Signal---/Rauschverhältnis eine um den Faktor 1/N
  (N--Anzahl der Frequenzelemente) kürzere Messzeit im Vergleich zu dispersiven Spektrometern
  bedeutet (\it Multiplex-- \rm oder \it Fellgett--\rm Vorteil).
  \item Bei FT--Spektrometern kann gegenüber dispersiven Spektrometern ein wesentlich grö{\ss}erer
  Durchsatz an Strahlungsleistung bei gleicher spektraler
  Auflösung für den Strahlengang verwendet werden, was eine
  deutliche Verbesserung des Signal/Rauschverhältnisses
  bedeutet. Der Lichtleitwert \cite{schrader95} einer kreisförmigen Blende ist höher
  als der der zwei Spalte innerhalb eines Monochromators
  (\it Jacquinot-- \rm oder \it Throughput--\rm Vorteil).
  \item Der Einsatz eines He--Ne--Referenzlasers sorgt für eine
  hohe Wellenzahlgenauigkeit in den FT--Spektren (<0.01
  cm\up{-1}). Das Interferogramm wird an den Nulldurchgängen des
  Interferenzmusters des Lasers digitalisiert (\it Connes--\rm
  Vorteil).
\end{itemize}



\subsection{\label{ft}Fourier--Transformation und Einkanalspektrenberechnung}

Im Gegensatz zum Spektrum, in dem jeder Wellenzahl genau ein
Messwert zugeordnet ist, enthält das Interferogramm in jedem
Datenpunkt Informationen über den gesamten Spektralbereich. Um
eine qualitative oder quantitative Auswertung der Daten zu
ermöglichen, muss das Interferogramm in ein Spektrum überführt
werden. Dies geschieht mittels der Fourier--Transformation.\\

Das Prinzip der Fourier--Transformation besagt, dass eine gegebene
Funktion als Summe von Kosinus-- und Sinusfunktionen beschrieben
werden kann:
\begin{equation}\label{eqft}
  C(\stackrel{\sim}{\nu})= \int_{-\infty}^{\infty} I(x)\,
  \exp{(i\,2\pi \stackrel{\sim}{\nu} x})\; \text{dx}
\end{equation}

Für digital abgetastete Signale I(n $\Delta x$) muss die diskrete
Fouriertransformation (DFT) verwendet werden:

\begin{equation}\label{eqdft}
  C (k \cdot
  \Delta\!\stackrel{\sim}{\nu})=\sum_{n=0}^{N-1}I(n\,\Delta x)\:
  \exp (\frac{i\, 2\,\pi\, n\, k}{N})
\end{equation}
Die kontinuierlichen Variablen x und $\stackrel{\sim}{\nu}$ werden
dabei durch die diskreten Stützpunkte $n\cdot \Delta x$ und $k
\cdot \Delta\!\stackrel{\sim}{\nu}$ ersetzt. Die neue Funktion,
die Fouriertransformierte C($k \cdot \Delta x$), besteht aus den
Koeffizienten dieser Entwicklung nach Sinus-- und
Kosinusfunktionen. Die von der Stellung des beweglichen Spiegels
und somit von der optischen Wegdifferenz abhängige
Interferogrammfunktion ist in die von der Wellenzahl abhängige
Spektrenfunktion transformiert worden.\\

In der Praxis wird Gleichung \ref{eqdft} meist nicht verwendet, da
ihre Berechnung sehr rechenzeitintensiv ist. Schnellere
Algorithmen basieren darauf, die Anzahl der notwendigen komplexen
Multiplikationen und Sinus--/Kosinusberechnungen erheblich zu
reduzieren. Oft eingesetzt wird der \it Cooley--Tukey\rm
--Algorithmus. Einschränkung hierbei ist, dass die Gesamtzahl der
Datenpunkte N eine Potenz von 2 sein muss (\cite{higham96}).\\

% \bild{htb}{int2spe1.wmf}{400}{300}{\bf A \it mit einem
% FT-Spektrometer aufgenommene Interferogramme (Atmosphärenspektrum
% oben, Eigenstrahlungsspektrum unten), \bf B \it daraus nach der
% Fourier--Transformation erhaltene Einkanalspektren. Zusätzlich ist
% über dem Atmosphäreneinkanalspektrum ein berechnetes
% Hintergrundeinkanalspektrum (in der Ordinate versetzt) gezeigt.
% Die Intensitäten sind in willkürlichen Einheiten dargestellt.}

Fällt keine der Abtastpositionen genau mit dem Maximum des
Interferogramms zusammen, oder führen Optik, Detektor oder
Verstärker zu wellenzahlabhängigen Phasenverschiebungen, so ist
die Fouriertransformierte in Gleichung \ref{eqdft} nicht mehr
reell, sondern komplex. Es muss eine Phasenkorrektur durchgeführt
werden, um das reelle Spektrum $S(\stackrel{\sim}{\nu})$ aus dem
komplexen Ergebnis C($\stackrel{\sim}{\nu}$) zu berechnen. Dies
geschieht entweder durch die Berechnung des \it
Amplitudenspektrums \rm P($\stackrel{\sim}{\nu}$) (oft auch
"`Powerspektrum"' genannt), wobei keine Phasenkorrektur erfolgt
\begin{equation}\label{eqpow}
  P(\stackrel{\sim}{\nu})=\sqrt{C(\stackrel{\sim}{\nu}) \cdot
  C^*(\stackrel{\sim}{\nu})}
\end{equation}
oder mit Hilfe der aufwendigeren Mertz--Methode \cite{mertz67} als
multiplikative Phasenkorrektur. Das in dieser Arbeit verwendete
K300--Spektrometer liefert beidseitige Interferogramme, die um das
Maximum beim Zero OPD aufgenommen werden. Die Phasenabweichungen
aufgrund der optischen Bauteile können durch die Berechnung des
Powerspektrums ausreichend eliminiert werden.\\

Für die quantitative Analyse der Spektren wird weiterhin zum
Spektrum mit der Probe auch ein Spektrum ohne Probe (\it
Hintergrundspektrum\rm ) benötigt, um die wellenzahlabhängigen
Einflüsse von IR--Strahler, optischen Komponenten und Detektor
mitberücksichtigen zu können. Zusätzlich zu beiden Spektren wird
jedesmal noch das Spektrum der IR--Schwarzkörperstrahlung der
Umgebung und der Spektrometeroptik bei abgeschalteter
Strahlungsquelle benötigt. Dieses Spektrum wird \it
Eigenstrahlungsspektrum \rm genannt. In Abbildung
\ref{int2spe1.wmf} \bf A \rm sind die Interferogramme einer
Atmosphärenmessung und der zugehörigen Eigenstrahlung gezeigt, in
\bf B \rm die zugehörigen Einkanalspektren. Man sieht, dass der
Eigenstrahlungsanteil deutlich kleiner ist als die Intensität des
Atmosphärenspektrums. Da die Eigenstrahlungsspektren auch die
spektralen Eigenschaften des Detektors und der optischen
Komponenten des Aufbaus beinhalten, ist zusätzlich noch ein
berechnetes Hintergrundeinkanalspektrum mit einem Offset
dargestellt, das frei von atmosphärischen Absorptionsbanden ist
und zur Berechnung des Atmosphärentransmissionsspektrums dient.\\


\subsection{\label{faltung}Auflösung, Faltungseffekte und Zerofilling}

Zwei Absorptionslinien sind im Spektrum dann noch voneinander zu
unterscheiden, wenn zwei Absorptionsmaxima ungefähr gleicher Höhe
durch ein Absorptionsminimum getrennt sind, welches eine um etwa
20 \% höhere Transmission besitzt. Dieser Wert ist willkürlich
gesetzt und wird in der IR--Spektroskopie nicht einheitlich
gehandhabt. Grundsätzlich gilt in Interferogrammen für
breitbandige Strahlungsemission, dass sie bei einem
Weglängenunterschied der beiden Interferometerspiegel $\Delta x=0$
ein ausgeprägtes Intensitätsmaximum haben. Die Intensität schwächt
sich bei grö{\ss}er werdendem $\Delta x$ schnell ab. Liegen zwei
Frequenzen im Spektrum weit auseinander, so werden sie mit grö{\ss}er
werdendem $\Delta x$ im Interferogramm schnell au{\ss}er Phase
geraten. Je näher sie allerdings beieinander liegen, desto grö{\ss}er
muss $\Delta x$ sein, um sie noch voneinander trennen zu können.
Für die Auflösung $\Delta \stackrel{\sim}{\nu}$ gilt:

\begin{equation}\label{eqaufl}
  \Delta \stackrel{\sim}{\nu} = \frac{1}{2\,x_{max}}
\end{equation}

Die Auflösung ist also direkt mit der maximalen Länge der
Interferogramme x\down{max} korreliert.\\

Ein zweiter Effekt ist bei einer endlichen Länge der
Interferogramme zu berücksichtigen. Die Theorie der
Fourier--Transformation in Kapitel \ref{ft} gilt für
Interferogramme unendlicher Länge. Die endlich vorliegenden
Interferogramme können als Produkt des Interferogramms mit einer
speziellen Gewichtungsfunktion, nämlich der Rechteckfunktion (sog.
\it Boxcar--\rm Funktion), beschrieben werden. Man kann zeigen,
dass die Fouriertransformation des Produktes zweier Funktionen
einer Faltungsoperation mit beiden Fouriertransformierten jener
Funktionen entspricht (siehe z.B. \cite{griffiths86}).
Mathematisch wird die Faltung eines Spektrums
S($\stackrel{\sim}{\nu}$) mit einer Apparatefunktion
f($\stackrel{\sim}{\nu}$) folgenderma{\ss}en beschrieben:
\begin{eqnarray}\label{eqconv}
  G(\stackrel{\sim}{\nu}) & = & S(\stackrel{\sim}{\nu})*f(\stackrel{\sim}{\nu})\nonumber\\
& = & \int^{\infty}_{-\infty} S(\stackrel{\sim}{\nu}^{\:\prime})
\; f(\stackrel{\sim}{\nu}-\stackrel{\sim}{\nu}^{\:\prime})\;
\text{d}\!\stackrel{\sim}{\nu}^{\:\prime}
\end{eqnarray}

Das resultierende Spektrum G($\stackrel{\sim}{\nu}$) dieser
Faltung enthält aber bei Verwendung einer Boxcar--Funktion
unerwünschte Seitenmodulationen im Spektrum einer schmalen
Absorptionslinie (\it Leakage--Effekt\rm ). So wird aus einer
endlichen Kosinusfunktion nach der Fouriertransformation eine
sinc-- statt einer Deltafunktion. Diese Seitenmodulationen kann
man durch eine Gewichtung der Interferogrammpunkte verringern,
z.B. durch Multiplikation mit einer Dreiecksfunktion. Diese
Gewichtungsfunktion wird \it Apodisation \rm genannt. Die
Fouriertransformation der Apodisationsfunktion wird \it
instrumentelle Linienfunktion \rm oder auch \it Apparatefunktion
\rm genannt. Sie hat allerdings zur Folge, dass die
Halbwertsbreite der einzelnen Spektrallinien grö{\ss}er und somit die
Auflösung schlechter wird. Es sind in der Literatur
(\cite{zhu198}, \cite{zhu298}) neben der Dreiecksfunktion weitere
Funktionen vorgeschlagen worden, die die Seitenmodulation bei
möglichst geringer Vergrö{\ss}erung der Halbwertsbreite weiter
verkleinern.\\

Bei Fouriertransform--Spektrometern muss weiterhin auch die \it
Selbstapodisation \rm berücksichtigt werden. Die Strahlung einer
Quelle endlicher Ausdehnung, die durch eine Blende endlicher
öffnung in das Spektrometer fällt und auf den Detektor fokussiert
wird, ist nicht optimal kollimiert, sondern leicht divergent.
Zwischen der Strahlung im Zentrum und den äu{\ss}eren Lichtstrahlen
besteht eine Phasenverschiebung, was zu einer zusätzlichen
Apodisation führt und die Auflösung somit leicht verschlechtert.
Es kann weiterhin gezeigt werden \cite{griffiths86}, dass die
Wellenlänge des äu{\ss}eren Strahls leicht grö{\ss}er ist als die des
Zentralstrahls. In einer ersten Näherung ist dann die
Wellenzahlposition der betrachteten Linie die Mittelung aus den
Wellenzahlpositionen des äu{\ss}eren und des zentralen Strahls. Es
tritt somit eine leichte Wellenzahlverschiebung der Bande auf.
Angemerkt sei, dass in Fouriertransform--Spektrometern zusätzlich
die Strahlung eines HeNe--Lasers durch das Interferogramm geführt
wird, um das Interferometersignal bei den Nulldurchgängen des
Lasersignals zu digitalisieren. Ist der Laser nun etwas
unterschiedlich im Strahlengang des Interferometers im Vergleich
zum IR--Strahl geführt, hat dies ebenfalls eine
Wellenzahlverschiebung der Banden im Spektrum zur Folge.\\

Die Faltung der Spektren erfolgt in der Transmissionsdomaine.
Durch Logarithmierung dieser Transmissionsspektren erhält man
danach die zugehörigen Extinktionsspektren. Abhängig von
Auflösung, gewählter Apodisationsfunktion und Halbwertsbreite der
betrachteten Bande, wird dann die Funktion der gemessenen
Extinktion in Abhängigkeit von der wahren Extinktion für gro{\ss}e
Extinktionen (ab bestimmten Extinktionswerten z.B. im
Bandenmaximum) nichtlinear. Bei gleicher Apodisationsfunktion ist
diese Nichtlinearität für schlechter aufgelöste Spektren grö{\ss}er
als bei höher aufgelösten Spektren. Siehe hierzu auch die
Abbildung im Anhang A für das Prüfgas Kohlenstoffmonoxid, wo diese
nichtlineare Abhängigkeit für die Bande bei 2169.3 cm\up{-1} bei
Auflösungen von 0.2 und 0.5 cm\up{-1} dargestellt ist.\\

Diese Faltungseffekte sind sehr früh auch schon für dispersive
Spektrometer gezeigt worden. Für dispersive Spektrometer wird die
\it Dreiecks--\rm Funktion bei nicht--beugungsbegrenzten Spalten
als Apparatefunktion verwendet. \cite{ramsay52} zeigt
beispielhaft, dass bei der Wahl einer Spaltbreite, für die die
Halbwertsbreite der zugehörigen Dreiecksfunktion halb so gro{\ss} ist
wie die wahre Linienbreite der Absorptionsbande, sich der wahre
und der gemessene Extinktionskoeffizient im Bandenmaximum um
$\approx$ 20 \% unterscheiden. Der Unterschied der integralen
Intensität beträgt jedoch nur $\approx$ 2--3 \%. Der Abfall der
Linienintensität bei schlechterer Auflösung geht also mit einer
Verbreiterung der Bande einher. In Abbildung \ref{no2.wmf} ist
dieser Sachverhalt für ein NO\down{2}--Spektrum für eine Auflösung
von 0.2 und 1.0 cm\up{-1} gezeigt.\\

Ein weiterer unerwünschter Effekt in den Spektren ist die nicht
ausreichende Dichte digitalisierter Punkte. Im Spektrum werden
z.B. dann die Maxima von schmalen Banden nicht optimal abgetastet.
Dieser Effekt lässt sich dadurch verringern, bzw. ganz vermeiden,
wenn an das Ende des Interferogramms Nullen angefügt werden (\it
Zerofilling\rm ). Dadurch wird die Anzahl der Punkte pro
Wellenzahl im Spektrum erhöht, was einer Interpolation entspricht
(siehe auch Abb. \ref{h2ouni.wmf}). Eine Verdoppelung der
Interferogrammlänge entspricht dabei einem zusätzlichen
Spektrendatenpunkt.\\


\subsection{\label{lambertbeer}Lambert--Beer'sches Gesetz und Extinktionsspektrenberechnung}

Bei IR--spektroskopischen Messungen wird der Anteil der durch die
Probe absorbierten elektromagnetischen Strahlung gemessen und zur
quantitativen Auswertung genutzt. Lambert fand 1760 heraus, dass
die Intensität der Strahlung beim Durchgang durch das homogene
absorbierende Medium proportional mit zunehmender Weglänge
abnimmt. Beer erkannte 1852 zudem, dass die Strahlungsintensität
ebenso proportional zur steigenden Konzentration des
absorbierenden Mediums abgeschwächt wird.\ Das Lambert--Beer'sche
Gesetz lautet in der Exponentialform:

\begin{equation}\label{eqlb}
  I=I_0 \cdot e^{-\epsilon^{\:\prime} \cdot c \cdot d}
\end{equation}

\begin{tabbing}
mit \quad \=-- \=test \=\kill

mit \> I \> -- \> Strahlungsintensität\\

\> $I_0$ \> -- \> Intensität der auf die Probe fallenden
Strahlung\\

\> $\epsilon^{\:\prime}$ \> -- \> Extinktionskoeffizient zur Basis
e\\

\> c \> -- \>Konzentration \\

\> d \> -- \> Schichtdicke\\
\end{tabbing}

% \bild{htb}{int2spe2.wmf}{400}{300}{\bf A \it durch Division eines
% Atmosphärenspektrums, aufgenommen für eine optische Weglänge von
% 130 m, und Hintergrundeinkanalspektrum erhaltenes
% Transmissionsspekrum und \bf B \it aus dem Transmissionsspektrum
% durch Logarithmierung erhaltenes Extinktionsspektrum.}

Um eine der Konzentration proportionale lineare Grö{\ss}e zu erhalten,
wird die Extinktion E eingeführt. üblicherweise wird zu deren
Berechnung der dekadische Logarithmus verwendet, wobei der Faktor
log e = 0.434 im Extinktionskoeffizienten $\epsilon$
berücksichtigt wird.

\begin{equation}\label{eqlbext}
  E=-log \frac{I}{I_0} = \epsilon \cdot c \cdot d
\end{equation}

Der Extinktionskoeffizient $\epsilon$ ist abhängig von Wellenzahl,
Temperatur und Druck. Zu beachten ist, dass aufgrund der
Logarithmierung bei intensiven Banden eine kleine änderung der
Transmission $\frac{I}{I_0}$ eine gro{\ss}e Extinktions-- und somit
Konzentrationsänderung zur Folge hat.\\

In Abbildung \ref{int2spe2.wmf} ist die Berechnung eines
Transmissionsspektrums und des zugehörigen Extinktionsspektrums
dargestellt. Vom Atmosphären-- und Hintergrundeinkanalspektrum aus
Abbildung \ref{int2spe1.wmf} \bf B \rm wurden zur Berechnung des
Transmissionsspektrums in Abb. \ref{int2spe2.wmf} \bf A \rm die
zugehörigen Eigenstrahlungsspektren abgezogen und die
resultierenden Spektren durcheinander dividiert. Abb.
\ref{int2spe2.wmf} \bf B \rm zeigt das durch Logarithmierung
erhaltene Extinktionsspektrum. Eingezeichnet sind zusätzlich die
breiten Spektralbereiche, die in der Atmosphäre von H\down{2}O--
und CO\down{2}--Banden dominiert werden und in denen die
Auswertung anderer Komponenten oft nicht möglich ist. Das
zusätzliche Fenster zeigt beispielhaft einen Spektralbereich mit
Extinktionen, in dem üblicherweise Komponenten ohne Probleme
ausgewertet werden (Extinktion<1).

\subsection{\label{openpath}FTIR--Messaufbau zur Schadgasbestimmung
in der offenen Atmosphäre}

Der Einsatz von optischen Messverfahren in der offenen Atmosphäre,
im angelsächsischen Sprachgebrauch auch \it open--path monitoring
\rm genannt, hat eine Reihe von Vorteilen gegenüber extraktiven
Probenahmeverfahren. Da die Messanordnung lediglich aus einer
Strahlungsquelle und einer Empfangseinheit besteht, ist eine
gesonderte Probenahmevorrichtung nicht erforderlich. Es können
Messstrecken von wenigen Metern bis zu einigen hundert Metern
realisiert werden, was eine wesentlich höhere Verlässlichkeit der
Messwerte über die jeweiligen Messstrecken bedeutet als dies bei
Punktmessverfahren der Fall ist. Die Messwerte sind dann
Mittelwerte über die gesamte Messstrecke. Die Messungen erfolgen
dabei zerstörungsfrei und \it in--situ\rm , wobei eine weit
schnellere Messzeit und quasi on--line Auswertung ermöglicht wird,
allerdings auch eine Probenaufbereitung wie z.B. Trocknung zur
Entfernung der stark absorbierenden Wassers in der Atmosphärenluft
ausschlie{\ss}t. Um kleine Emissionsquellen zu lokalisieren, muss aber
zudem bei optischen Verfahren mit längeren Messstrecken oft ein
grö{\ss}erer Aufwand betrieben werden.\\

In der FTIR--Spektroskopie wird über eine ganze Anzahl von
Interferogrammen gemittelt, um das Signal/Rauschverhältnis zu
verbessern. Bei dem in dieser Arbeit genutzten K300--Spektrometer
mit 0.2 cm\up{-1} Auflösung haben sich 100 Mittelungen als guter
Kompromiss zwischen Signal/Rauschverhältnis und Messzeit erwiesen.
Die Fouriertransformation wird dann am über 100 Interferogrammen
gemittelten Interferogramm vorgenommen. Die Dauer der Aufnahme
eines Interferogramms liegt bei 1 s, welches danach auf der
Festplatte abgespeichert wird. Für 100 Mittelungen benötigt der
verwendete Pentium90 mit 32 MB Arbeitsspeicher ca. 3$\frac{1}{2}$
min. In dieser Zeit können bei Messungen in der offenen Atmosphäre
Dichteschwankungen der zu analysierenden Gase auftreten. Dies
führt, vor allem bei änderungen des stark absorbierenden Wassers,
zu Veränderungen der auf den Detektor fallenden
Strahlungsintensität. \cite{kyle84} et al. zeigen, dass diese
Intensitätsänderungen auf den Detektor mathematisch einer
zusätzlichen Faltung des gemittelten Interferogramms entspricht,
welche die Auflösung ein wenig verschlechtert. Bei normalerweise
in der Atmosphäre vorliegenden Dichteschwankungen ist die
Verschlechterung der Auflösung jedoch vernachlässigbar
\cite{kyle84}.\\

FTIR--Messungen haben weiterhin den Vorteil, dass eine Vielzahl
von Molekülen (z.B. aromatische und halogenierte
Kohlenwasserstoffe, Alkane, Alkene, Aldehyde, Ketone, Alkohole,
Ester, Anorganika) simultan detektiert werden können. Weiterhin
emittieren heiße Abgase oberhalb von ca. 70 C genügend
IR--Strahlung, womit eine Analyse dieser Gase auch ohne
zusätzliche IR--Quelle möglich ist (\it passive FTIR--Messungen\rm
). Die Nachweisgrenzen für die einzelnen Stoffe liegen bei einem
aktivem Aufbau mit IR--Strahlungsquelle im unteren ppb--Bereich
(siehe auch Kap. \ref{nwg}), für passive Aufbauten, abhängig von
der Hintergrundtemperatur, um 3 Grö{\ss}enordnungen darunter
\cite{chaffin99}, \cite{mattuip}. Zu berücksichtigen ist, dass mit
FTIR--Messungen Moleküle, die kein Dipolmoment besitzen (wie
N\down{2}, O\down{2} und die Edelgase) nicht detektiert werden
können.\\

% \bild{htb}{aufbau.bmp}{400}{230}{\it Schematischer Aufbau eines
% sogenannten bistatischen FTIR--Systems für Messungen in der
% offenen Atmosphäre.}

Abbildung \ref{aufbau.bmp} zeigt einen typischen Aufbau eines
FTIR--Systems für Messungen in der offenen Atmosphäre. Aufbauten,
wie in dieser Abbildung zu sehen, bei denen IR--Strahlungsquelle
und Interferometer räumlich voneinander getrennt sind, werden \it
bistatische \rm Messanordnungen genannt. Das Probevolumen wird
dabei durch den Abstand von Strahlungquelle und Interferometer
bestimmt.\\

Bei \it monostatischen Systemen \rm hingegen bilden Sender und
Empfänger eine Einheit. Die IR--Quelle ist im Spektrometer
integriert und ein zusätzlicher Strahlteiler trennt den
einkoppelnden von dem auskoppelnden Strahl. Ein Retroreflektor
sorgt für die Rückführung des IR--Strahls zum Interferometer. Das
Probevolumen wird in diesem Fall zweimal durchlaufen.\\

Der Vorteil von monostatischen Systemen liegt hauptsächlich darin,
dass sie einen modulierten Infrarotstrahl aussenden und
detektieren. In diesem Fall liefert die unmodulierte Strahlung
(Eigenstrahlung) der Umgebung nur einen Gleichspannungsanteil zum
Signal. Weiterhin ist auch aufgrund des Einsatzes des
Retroreflektors keine zusätzliche zweite Stromversorgung für den
IR--Strahler notwendig, was bei Feldmessungen Probleme bereiten
kann. Die Justage des Retroreflektors bei monostatischen Systemen
ist unproblematischer als bei bistatischen Systemen, wo die
optischen Achsen vom Parabolspiegel des IR--Strahlers und des
Teleskopes des FT--Spektrometers zusammenfallen müssen. Der
Einsatz eines Retroreflektors erlaubt zudem auch unproblematische
Messungen in explosionsgeschützten Bereichen. Aufgrund des zweiten
Strahlteilers werden diese Vorteile jedoch mit einem
Strahlungsverlust erkauft, der bei gleicher Messstrecke aufgrund
des zweimaligen Durchlaufens sich noch zusätzlich zur bistatischen
Anordnung halbiert. Ein weiterer Nachteil der monostatischen
Systeme liegt darin, dass Streulicht mit gemessen wird
\cite{richardson298}. Im allgemeinen sind monostatische Systeme in
der Anschaffung teurer als bistatische Systeme.\\

Das in dieser Arbeit verwendete K300--Spektrometer arbeitet in der
bistatischen Konfiguration. Die Einsatzmöglichkeiten solcher
Systeme zur Messung diffuser Emissionen sind vielfältig:
\begin{itemize}
  \item Messung von Emissionen in oder neben Fabriken
  \item Messung über Kläranlagen
  \item Altlastendeponien (z.B. Aushubarbeiten)
  \item Arbeitsplatzmessungen
  \item Allgemeine Luftanalyse (Qualität der Atmosphäre)
  \item Katastrophenfälle, Chemieunfälle, Brände
  \item Messung von kontrollierten Verbrennungen
  \item Automobile Abgase an Stra{\ss}en und Tunneln
  \item Lackieranlagen und Tankstellen
  \item Erfassung wichtiger Daten zur Erstellung von
  Gas--Ausbreitungsmodellen
\end{itemize}

Der Einsatz eines solchen Verfahrens erfordert allerdings
allgemein bindende Richtlinien, um vergleichbare und validierte
Messergebnisse erhalten zu können. In den U.S.A. gibt es diese
Richtlinien seit 1997 \cite{to1697}, in Deutschland liegen die
entsprechenden VDI--Richtlinien zur Zeit im Gründruck vor
\cite{vdi98} und sollen Ende 1999 erscheinen. Im Rahmen dieser
Arbeit wurde ein entscheidender Beitrag zur Verbesserung der
quantitativen Genauigkeit der Analyse geleistet. Hierzu wurde
durchgängig ein hochauflösendes Spektrometer mit 0.2
cm\up{-1}--Auflösung verwendet. In der Atmosphärenspektroskopie
sind Auflösungen von 0.5, 1 cm\up{-1} oder schlechter üblich. Die
Erkenntnisse dieser Arbeit konnten in eine Expertensoftware
umgesetzt werden, womit eine neue Messqualität in der
Atmosphärenspektrometrie erschlossen wird. Diese Erkenntnisse
werden in den folgenden Kapiteln vorgestellt.\\


\section{\label{anderemesstechniken}Vergleich mit alternativen
optischen Messtechniken}

Neben der FTIR--Spektroskopie gibt es noch andere optische
Verfahren, die für Gasmessungen in der offenen Atmosphäre
eingesetzt werden. Die wichtigsten dieser Verfahren werden in den
nächsten Abschnitten vorgestellt und miteinander verglichen.


\subsection{\label{doas}DOAS}

Das DOAS--Verfahren (\bf D\rm ifferential \bf O\rm ptical \bf A\rm
bsorption \bf S\rm pectroscopy) ist ein weiteres Verfahren,
welches mit breitbandigen Strahlungsquellen arbeitet und womit
Spektren im UV/VIS--Bereich von 200--500 nm ausgewertet werden
können. Als Strahlungsquellen dienen dabei Glühlampen, die mit
Temperaturen um die 3000 K oder Lichtbogenlampen (z.B. Xenon), die
mit Temperaturen um die 6000--10000 K arbeiten. Anforderung an die
Strahlungsquellen ist vor allem, dass sich ihre
Spektralcharakteristik nur langsam ändert. Die Auswertung erfolgt
in der Weise, dass ein Spektrum simultan oder kurz hintereinander
mit zwei verschiedenen Auflösungen aufgenommen wird. Das
hochaufgelöste Spektrum (z.B. 0.3 nm) enthält dann die Strukturen
der zu messenden Banden, im schlecht aufgelösten Spektrum (z.B. 25
nm) sind die Feinstrukturen nicht mehr zu erkennen und damit ist
näherungsweise nur der spektrale Hintergrund enthalten. Die
Differenz dieser beiden Spektren kann damit quantitativ
ausgewertet werden.\\

Zur Aufnahme der Spektren werden verschiedene
Spektrometeranordnungen genutzt \cite{platt94}. Oft werden
Spektrometer in
Czer\-ny--\-Tur\-ner--\-Mo\-no\-chro\-ma\-tor\-an\-ordn\-ung
verwendet (welche neben den Gittern zusätzlich mit zwei weiteren
Spiegeln ausgerüstet sind). Es können in anderen
Spektrometerkonfigurationen aber auch holographische oder
Echelle--Gitter verwendet werden \cite{becker-ross97}. Auch wurden
Fourier--Transform--Spektrometern eingesetzt, deren Vorteil im
erhöhten Lichtleitwert liegt, wenn mit höherer Auflösung gemessen
wird (siehe \cite{platt94} und darin zitierte Literatur). Abhängig
vom Wellenzahlbereich und der gewünschten Empfindlichkeit werden
ebenfalls eine ganze Reihe von unterschiedlichen Detektoren
eingesetzt. Sie reichen dabei von nichtdispersiven
Halbleiterdetektoren über Photomultiplier bis hin zu
Photodiodenarrays.\\

Der Vorteil von Messungen im UV--Bereich liegt vor allem darin,
dass die Absorptionskoeffizienten in diesem Bereich deutlich
grö{\ss}er sind als im IR--Bereich. Zudem können aufgrund der
leistungsstarken Strahlungsquellen Wegstrecken bis zu mehreren
Kilometern realisiert werden. Allerdings ist die Anzahl der
Spezies, die schmale Banden im UV--Bereich besitzen, beschränkt.
So können vorwiegend nur NO\down{x}, SO\down{2}, O\down{3},
NH\down{3}, CS\down{2}, Benzol, HNO\down{2}, BrO und ClO gemessen
werden, dies aber mit sehr guten Nachweisgrenzen mit
Stoffmengenanteilen im ppb bis ppt--Bereich.


\subsection{\label{diodenlaser}Diodenlaser}

Die Messanordnung für die Absorptionsspektroskopie mit
durchstimmbaren Diodenlasern (TDLAS--\bf T\rm unable \bf D\rm iode
\bf L\rm aser \bf A\rm bsorption \bf S\rm pectroscopy) besteht
vorwiegend aus einem durchstimmbaren Laser auf der einen und zwei
Photodioden auf der anderen Seite der Messstrecke. Das Laserlicht
wird an einem Strahlteiler geteilt und ein Strahl durch die
kontaminierte Messstrecke, der zweite an ihr vorbei geleitet. Das
Differenzsignal der beiden Photodioden gibt das
Absorptionsspektrum. Durch den Einsatz von
Mehrfachreflexionsküvetten kann die Nachweisgrenze bei
Punktmessungen deutlich verbessert werden. Bei niedrigem Druck in
der Zelle kann zudem die Druckverbreiterung reduziert werden. Der
kleinste noch messbare Wert für den Absorptionskoeffizienten
$\alpha$ hängt vom Verhältnis des spektral noch auflösbaren
Intervalls $\Delta\omega$ zur Linienbreite des
Absorptionsübergangs $\Delta\omega_D$ ab.\\

Im Gegensatz zu den vorgenannten Verfahren mit breitbandigen
Strahlungsquellen, emittiert der Diodenlaser monochromatisch und
ist somit eine sehr schmalbandige Strahlungsquelle mit einer oft
um Grö{\ss}enordnungen höheren Leistungsdichte. Dies bringt folgende
Vorteile \cite{demtröder93}:
\begin{itemize}
  \item die spektrale Auflösung zur Messung von Absorptionen ist höher
  als bei den anderen optischen Verfahren und vorwiegend durch die
  Doppler--Breite der Absorptionslinie bestimmt
  \item aufgrund der hohen spektralen Leistungsdichte treten
  Rauschprobleme des Detektors in den Hintergrund
  \item aufgrund der guten räumlichen Bündelung der Laserstrahlung
  können längere Absorptionswege, z.B.bei Mehrfachreflexionen in
  Absorptionszellen, realisiert werden.
  \item die Laserfrequenz kann sehr schnell durchgestimmt werden.
  Mit Diodenlasern sind Durchstimmbereiche über einige
  Wellenzahlen im Mikrosekundenbereich möglich.
\end{itemize}

Die Nachweisgrenzen lassen sich noch einmal um Grö{\ss}enordnungen
verbessern, wenn die Frequenz $\omega$ des Lasers während des
Durchstimmens um einen Betrag $\Delta\omega$ moduliert wird. Mit
Hilfe eines phasenempfindlichen Verstärkers erhält man dann mit
Einstellung auf die n--te Harmonische der Modulationsfrequenz
$\Delta\omega$ jeweils die n--te Ableitung des
Absorptionskoeffizienten $\alpha(\omega)$ gegenüber reinen
Absorptionsmessungen mit durchstimmbaren Lasern. Dadurch lassen
sich spektrale Untergründe und technisches Rauschen eliminieren
bzw. vermindern und somit die Nachweisempfindlichkeit steigern.
Oft wird die zweite Harmonische zur Detektion herangezogen, weil
die Signale symmetrisch und noch genügend hoch sind. Die optimale
Modulationsfrequenz hängt von der Frequenzhalbwertsbreite der zu
analysierenden Bande ab und beträgt in etwa das 2.2--fache dieses
Wertes \cite{silver92}. Eine weitere Minimierung des Untergrundes
ist mit Doppelmodulationsverfahren zu erzielen. Hierbei wird die
Laserfrequenz mit einer zweiten zusätzlichen Frequenz moduliert
und die Detektion findet z.B. auf der Summenfrequenz statt
\cite{zybin95}.\\

Die erreichbaren Nachweisgrenzen für Messungen in der offenen
Atmosphäre mittels TDLAS hängen von mehreren Faktoren ab (siehe
auch den übersichtsartikel von Werle \cite{werle98} oder eine
übersicht möglicher Einsatzgebiete \cite{grisar92} ). Sie sind
deutlich besser, wenn eine Mehrfachreflexionszelle eingesetzt und
dadurch die optische Pfadlänge vergrö{\ss}ert und der Druck reduziert
werden kann. Es ist in dem Fall aber zu prüfen, ob die untersuchte
Gasmenge auch repräsentativ für das Messszenario ist. Für das zu
detektierende Gas ist ein geeigneter Laser auszusuchen, der bei
einer starken, möglichst schmalbandigen und von anderen
Komponenten nicht überlagerten Absorptionslinie emittiert.
Mittlerweile gibt es für den mittleren und fernen IR--Bereich
(sog. Bleisalzlaser) und für den sichtbaren Spektralbereich
Halbleiterlaser. Die Halbleiterlaser sind in
Wellenlängenbereichen, die in der Kommunikationselektronik
benötigt werden, in sehr guter Qualität zu erhalten.\\

Die Halbleiterlaser, die Laserstrahlung bis zu Wellenlängen von
etwa 2.8 $\mu$m emittieren, können bei Raumtemperatur betrieben
werden, mit Wellenlängen darüber müssen sie mit Stickstoff oder
Helium gekühlt werden. Abhängig von diesen Randbedingungen sind
Nachweisgrenzen im ppb-- bis in den ppt--Bereich möglich
\cite{werle98}.


\subsection{\label{lidar}LIDAR}

Ein optisches Verfahren, welches besonders für die Messung von
Gasen in gro{\ss}er Höhe bzw. an unzugänglichen Orten geeignet ist,
ist das LIDAR--Verfahren (\bf L\rm ight \bf D\rm etecting \bf A\rm
nd \bf R\rm anging). Das Prinzip des LIDAR--Verfahrens beruht
darauf, dass ein Laserpuls in die Atmosphäre ausgesandt und die
rückgestreute Strahlung registriert und analysiert wird. Dabei
gibt es verschiedene Arten von Wechselwirkung zwischen
elektromagnetischer Strahlung und atmosphärischen Bestandteilen,
die im Experiment ausgenutzt werden können. Der dominante Prozess
ist die elastische Streuung von Photonen an Aerosolen und
Staubteilchen (Mie--Streuung) und an Molekülen
(Rayleigh--Streuung). Weiterhin sind auch nicht--elastische
Streuprozesse wie die Raman--Streuung möglich.\\

Das Rückstreusignal $S(\lambda,t)$ wird bei Messungen im
UV/VIS--Spektralbereich von einem Photomultiplier zeitaufgelöst
gemessen. Die Zeit $t=2 \cdot R/c$ hängt von der Entfernung R der
streuenden Teilchen ab (c--Lichtgeschwindigkeit). Für die
zurückgestrahlte Leistung in einem Zeitraum $\Delta t$ und somit
eines bestimmten Probenvolumens gilt:
\begin{equation}\label{eqlidar1}
  S(\lambda , t_1)=\int^{t_1+\Delta t/2}_{t_1-\Delta t/2}
  S(\lambda , t) \: {\rm dt}
\end{equation}
Diese rückgestreute Leistung $S(\lambda,t)$ hängt dabei von der
Ausgangsleistung des Laserpulses $P_0$, von der Abschwächung durch
den Absorptionskoeffizienten $\alpha(\lambda)$ auf Hin-- und
Rückweg, der Teilchenzahldichte N, dem Rückstreukoeffizienten
$\sigma_{streu}$ und dem vom Teleskop mit dem Durchmesser D
erfassten Raumwinkel D\up{2}/R\up{2} ab \cite{measures84}:
\begin{equation}\label{eqlidar2}
  S(\lambda,t)=P_0(\lambda)\, N\, \sigma_{streu}(\lambda)\,
  \frac{D^2}{R^2}\: \exp[-2\,\alpha(\lambda)\, R]
\end{equation}
Beim differentiellen Absorptions--LIDAR (DIAL) werden zwei
Laserpulse mit unterschiedlichen Wellenlängen benötigt. Eine
Wellenlänge wird so gewählt, dass sie mit dem
Absorptionslinienmaximum des zu messenden Moleküls übereinstimmt.
Die andere Wellenlänge wird möglichst nah der ersten gewählt, aber
so, dass sie nicht im Absorptionsbereich des zu messenden Gases
liegt. Für genügend kleine $\lambda_1-\lambda_2$ ändern sich dann
die Mie--Rückstreukoeffizienten nicht merklich. Durch
Quotientenbildung der beiden gemessenen Rückstreusignale lässt
sich der gesuchte Volumenanteil der zu bestimmenden Moleküle
ortsaufgelöst bestimmen.\\

Die Ortsauflösung beträgt beim Einsatz von
Nanosekunden--Laserpulsen heute ca. 5--10 m in Abständen, je nach
Justage, ab etwa 200 m vom Messsystem entfernt. Die Ortsauflösung
kann weiter verbessert werden, wenn Pikosekunden-- oder
Femtosekunden--Laser genutzt werden. In heutigen Systemen werden
sowohl gepulste UV/VIS-- (z.B. Titan--Saphir--Laser,
frequenzverdoppelte Farbstofflaser) wie auch IR--Laser (z.B.
CO\down{2}--Laser, optisch parametrische Oszillatoren (OPO))
verwendet. Je nach benötigter Reichweite gibt es Systeme mit
Ausgangsleistungen von wenigen mJoule bis in den Joule--Bereich
hinein. Mit den Lasern im UV/VIS--Bereich können die anorganischen
Komponenten H\down{2}O, O\down{3}, SO\down{2}, NO und NO\down{2}
sowie aromatische Kohlenwasserstoffe wie Benzol gemessen werden,
wobei die Nachweisgrenzen im unteren ppb--Bereich liegen. Im
Gegensatz zu den UV/VIS--Systemen ist mit den IR--Systemen eine
grö{\ss}ere Anzahl von organischen Substanzen messbar (Olefine,
organische Lösungsmittel, halogenierte Kohlenwasserstoffe), sowie
Ammoniak, SF\down{6}, O\down{3} und H\down{2}O. Die
Nachweisgrenzen liegen hier im unteren ppm--Bereich.\\

LIDAR--Systeme können aber auch für Absorptionsmessungen genutzt
werden. Durch den Einsatz von \it Hardtargets \rm (d.h. festen
Zielen im Hintergrund wie z.B. Bäume, Häuser etc.) kann eine sehr
viel effizientere Rückstreuung erfolgen, wobei dann aber keine
ortsaufgelöste Messung mehr möglich ist. Die Nachweisgrenzen
können sich dann um bis zu zwei Grö{\ss}enordnungen verbessern.\\


\subsection{\label{vergl}Vergleich der Methoden untereinander}

Bei der Auswahl der bestgeeigneten optischen Methode für
Gasmessungen in der offenen Atmosphäre für eine gegebene
Problemstellung sind mehrere Kriterien zu beachten. Es ist zu
klären, welche und wieviele Komponenten gleichzeitig mit welcher
Nachweisgrenze gemessen werden sollen. Dabei sind auch die Länge
der zu realisierenden Messstrecke und eventuelle Beschränkungen
beim Aufbau der Strahlungsquelle mit zu berücksichtigen. Die
Zeitdauer einer Messung und ein möglicher Dauerbetrieb sind neben
den Kosten weitere Entscheidungskriterien.\\

Das FTIR-- und das DOAS--Verfahren haben als Vollspektrenmethoden
den Vorteil, dass mehrere Komponenten simultan bestimmt werden
können. Querempfindlichkeiten können dabei in der Auswertung
mitberücksichtigt werden. Mit der FTIR--Methode können dabei aber
weitaus mehr Stoffe gemessen werden als beim DOAS--Verfahren.
Dafür bietet die Absorptionsmessung im UV--Bereich bessere
Nachweisgrenzen, und gerade Gase wie die NO\down{x}, SO\down{2}
und O\down{3}, die im mittleren IR--Bereich nur relativ schwache
Absorptionsbanden besitzen und z.T. stark von Wasser überlagert
werden, können im UV--Bereich sehr gut gemessen werden. Beide
Verfahren benötigen aber genügend lange Messwege und die
Integrationszeiten für Messungen liegen im Minutenbereich. Für das
mittlere Infrarot gibt es mittlerweile sehr umfangreiche
Datenbanken, die trotz ihrer Nachteile (siehe Kap.
\ref{bibliotheken}) für eine quantitative Analyse herangezogen
werden können. Die Möglichkeit, Referenzspektren für den
UV--Bereich zu erhalten, sind z.Z. noch deutlich eingeschränkter,
was sich in den kommenden Jahren weiter verbessern wird. Beide
Verfahren eignen sich vor allem auch für die Dauerüberwachung von
diffusen Emissionsquellen,\\

Das TDLAS--Verfahren hat aufgrund der hohen Leistungsdichte von
Lasern um bis zu zwei bis drei Grö{\ss}enordnungen bessere
Nachweisgrenzen. Zur Detektion mehrerer Stoffe müssen allerdings
in der jeweiligen Messanordnung die Laser ausgetauscht werden, was
kostspielig sein kann. Querempfindlichkeiten mit anderen
Komponenten können oft im Signal erkannt, selten aber bei der
quantitativen Bestimmung herausgerechnet werden, so wie das beim
FTIR--Verfahren möglich ist (natürlich wird versucht, solche
Banden, wenn möglich, zu meiden). Um Nachweisgrenzen bis in den
unteren ppt--Bereich zu gewährleisten, ist der Einsatz von
Mehrfachreflexionszellen unerlässlich. Hier ist sehr genau zu
prüfen, ob das in der Zelle angesammelte Gasvolumen repräsentativ
für die jeweilige Messaufgabe ist. Wird die Messzelle nicht
benötigt, können Messungen im zeitlichen Abstand von Sekunden und
darunter durchgeführt werden. Diodenlaser sind somit besonders für
die Detektion von einem oder wenigen Gasen in zeitlich schnell
veränderlichen Prozessen geeignet, bei denen zusätzlich auch nur
kurze Wegstrecken realisiert werden können (z.B. Messung von
Autoabgasen auf Prüfständen \cite{hirschbergerip}).\\

Das LIDAR--Verfahren hingegen eignet sich gut für räumlich und
zeitlich hochaufgelöste Bestimmung von Einzelgaskonzentrationen an
Orten, die für die anderen Verfahren nicht zugänglich sind, wie
z.B. die Höhenprofilmessung eines bestimmten Gases. Je nach
verwendetem Lasersystem ist die Anzahl der möglichen zu
detektierenden Spezies allerdings eingeschränkt.
Querempfindlichkeiten mit anderen Komponenten müssen von
vorneherein ausgeschlossen werden, da sie bei der Auswertung der
gemessenen Intensitätssignale nicht erkannt werden können. Die
Nachweisgrenzen liegen im allgemeinen nur im ppm--Bereich, was für
viele Anwendungen aber ausreichend ist. Da das rückgestreute
Signal mit einem Empfangsteleskop gesammelt und auf den Detektor
fokussiert wird, ist es i.a. nicht möglich (genaugenommen hängt
dies von der Justage des Teleskops ab), Konzentrationsbestimmungen
für die ersten 200 m der Messstrecke vorzunehmen. Abhängig von der
Laserleistung können aber Wegstrecken bis 30 km realisiert werden.
Der Aufbau von LIDAR--Systemen ist noch sehr kostspielig.\\

\cleardoublepage
