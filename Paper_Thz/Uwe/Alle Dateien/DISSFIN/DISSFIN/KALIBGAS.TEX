\chapter{\label{kalibriergas}Kalibriergasmessungen}

F\"{u}r die multivariate Auswertung von FTIR--Atmosph\"{a}renspektren ist
es notwendig, die Referenzspektren der zu analysierenden
Komponenten vorliegen zu haben. F\"{u}r eine quantitativ hochwertige
Analyse m\"{u}ssen die zur Verf\"{u}gung stehenden Bibliotheken bewertet
und gegebenenfalls eigene Referenzspektren gemessen werden, um
quantitativ bessere Auswertungsergebnisse zu gew\"{a}hrleisten.\\

\section{\label{bibliotheken}Existierende FTIR--Gasspektrenbibliotheken}

F\"{u}r die Analyse von Spektren, die mittels
Gaschromatographie/IR--Verfahren im mittleren Infrarot erhalten
wurden, stehen mittlerweile sehr gro{\ss}e Datenbanken (5000-10000
Spektren) mit schlecht aufgel\"{o}sten (4--8 cm\up{-1} Aufl\"{o}sung)
Spektren zur Verf\"{u}gung. Die Anzahl der Komponenten, von denen
hochaufgel\"{o}ste Spektren zu erhalten sind, ist hingegen recht
beschr\"{a}nkt.\\

Grunds\"{a}tzlich kann man zwischen gemessenen Spektrenbibliotheken
und Parameterdatenbanken unterscheiden, mit denen Spektren mit den
gew\"{u}nschten Bedingungen bzgl. Temperatur, Druck, Abtastung und
Aufl\"{o}sung berechnet werden k\"{o}nnen.\\


\subsection{\label{spektrenbibliotheken}Spektrenbibliotheken}

\subsubsection{\label{qasoft}QASoft--Spektrenbibliothek}

Die QASoft--Spektrenbibliothek ist eine kommerzielle Datenbank der
Infrared Analysis Inc., Anaheim, U.S.A.. Sie enth\"{a}lt mittlerweile
quantitative Referenzspektren von 269 relevanten Schadgasen,
welche mit den Aufl\"{o}sungen 0.25, 0.5, 1.0 und 2.0 cm\up{-1}
erh\"{a}ltlich sind. Auf Anfrage sind 151 Gasspektren auch mit einer
besseren Aufl\"{o}sung von 0.125 cm\up{-1} zu erhalten. Erst seit
kurzer Zeit liegen das H\down{2}O--Spektrum in 20 und das
CO\down{2}--Spektrum in 12 verschiedenen Konzentrationen vor.
Vormals waren die Spektren beider Stoffe nur mit einer sehr
geringen Konzentration zu erhalten. welche bei \"{u}blichen Wegl\"{a}ngen
von 100 m in der Atmosph\"{a}re aber eindeutig zu klein war und
Nichtlinearit\"{a}tseffekte nicht ber\"{u}cksichtigte.\\

Eine Vielzahl der Spektren dieser Datenbank wurden mit einem
Analect RFX--65--Spektrometer mit einer Aufl\"{o}sung von 0.125
cm\up{-1} aufgenommen. Weitere Spektren wurden mit einem Digilab-
und einem MIDAC-Spektrometer mit 0.5 cm\up{-1}, wieder andere mit
einem Analect--Ger\"{a}t mit 0.25 cm\up{-1} Aufl\"{o}sung aufgenommen
\cite{hanst97}. Die Interferogramme der Spektren sind nicht
verf\"{u}gbar. Somit ist es nicht m\"{o}glich, eine andere
Apodisationsfunktion als die vorgegebene Dreiecksapodisation zu
w\"{a}hlen.\\

Die meisten Spektren liegen mit Volumenanteilen von ca. 100
ppm$\cdot$m vor. F\"{u}r Spurengase ist das im Normalfall ungef\"{a}hr die
in der Praxis gemessene Gr\"{o}{\ss}enordnung. Die unzureichende
CLS--Anpassung mit Referenspektren f\"{u}r Molek\"{u}le mit schmalen
Banden wie CO und N\down{2}O und f\"{u}r H\down{2}O ist beispielhaft
in Abbildung \ref{rescls.wmf} gezeigt.\\

Zus\"{a}tzlich wird mit der Datenbank das kleine Unterprogramm RIAS
(\bf R\rm eg\-ion \bf I\rm n\-te\-gra\-tion \bf A\rm nd \bf S\rm
ub\-trac\-tion) vertrieben, welches unter dem
Spektrenbearbeitungsprogramm GRAMS (Galactic Industries) l\"{a}uft.
Dieses Programm f\"{u}hrt eine Subtraktion gew\"{a}hlter Referenzspektren
aus dem Atmosph\"{a}renspektrum durch. Dabei wird \"{u}ber eine vom
Benutzer zu w\"{a}hlende Bande integriert, um den Skalierungsfaktor zu
erhalten. Desweiteren steht auch ein CLS--Modul zur Verf\"{u}gung, bei
welchem der Benutzer Referenzspektren und den auszuwertenden
Bereich selbst zu w\"{a}hlen hat. Eine Spektrenvorbearbeitung ist
nicht m\"{o}glich.\\

Eine generelle Aussage \"{u}ber die Qualit\"{a}t der Spektren zu machen,
ist schwierig. Tatsache ist jedoch, dass Spektren (z.B. Wasser),
ganz offensichtlich in Bereichen mit schwachen Banden in h\"{o}herer
Konzentration gemessen und dann herunterskaliert wurden. Dies ist
vor allem dann problematisch, wenn dadurch nichtlineare Anteile
nicht mitber\"{u}cksichtigt werden. Weiterhin gibt es Bereiche in den
Spektren, die herausgeschnitten und durch eine Gerade ersetzt
wurden (siehe auch Abb. \ref{datbvgl.wmf}). Ebenso weisen die
Spektren Unterschiede in der Wellenzahlgenauigkeit auf (dazu siehe
auch Kap. \ref{wellenzahlstab}), was zu weiteren Fehlern in der
quantitativen Auswertung f\"{u}hren kann. Dennoch ist die QASoft
derzeit f\"{u}r die hochaufl\"{o}sende Atmosph\"{a}renspektroskopie die
einzige Datenbank, die eine hohe Anzahl an Gasspektren in
vertretbarer Qualit\"{a}t f\"{u}r die quantitative Analyse bereitstellt.\\

\noindent(Internetadresse: http://www.infrared--analysis.com)

\subsubsection{\label{nist}NIST--Spektrendatenbank}

Die NIST (\bf N\rm ational \bf I\rm nstitute of \bf S\rm tandards
and \bf T\rm echnology, U.S.A.)--Datenbank enth\"{a}lt zur Zeit 21
Spektren im mittleren Infrarot von fl\"{u}chtigen organischen
Komponenten. Diese Spektren sind in 5 verschiedenen Aufl\"{o}sungen
von 0.125--2.00 cm\up{-1} und 5 verschiedenen
Apodisationsfunktionen erh\"{a}ltlich. F\"{u}r jedes Spektrum wurde
jeweils \"{u}ber 9 Transmissionsspektren gemittelt. Nichtlinearit\"{a}ten
wurden nicht ber\"{u}cksichtigt. Eine Beschreibung der Datenbank
findet sich bei \cite{chu98} oder eine ausf\"{u}hrliche Dokumentation
auch auf der Internetseite.\\

Diese Datenbank lag uns nicht vor, daher kann eine Bewertung nicht
vorgenommen werden. Da diese Spektren aber von der NIST
zertifiziert werden, kann davon ausgegangen werden, dass die Gase
hochgenau vermessen worden sind. Ein ganz offensichtlicher
Nachteil ist aber die doch stark begrenzte Anzahl der gemessenen
Molek\"{u}le und die Nichtber\"{u}cksichtigung von
Nichtlinearit\"{a}tseffekten bei hohen Werten f\"{u}r das Produkt aus
Konzentration und Wegl\"{a}nge.\\

\noindent(Internetadresse: http://www.nist.gov/srd/nist79.htm)


\subsubsection{\label{epa}EPA--Spektrendatenbank}

Die EPA (U.S. \bf E\rm nvironmental \bf P\rm rotection \bf A\rm
gency) hat eine Datenbank mit gemessenen Infrarotspektren von \"{u}ber
100 Schadgasen und der Aufl\"{o}sung von 0.25 cm\up{-1} erstellt. Die
meisten Gase sind mit mehreren Konzentrationen und
unterschiedlichen Temperaturen erh\"{a}ltlich. Die Spektren k\"{o}nnen
kostenlos im Internet kopiert werden.\\

Obwohl in den letzten Jahren Spektren einzelner Gase ausgetauscht
wurden, um deren Qualit\"{a}t zu verbessern, gilt immer noch f\"{u}r viele
Spektren, dass dort deutlich sichtbare H\down{2}O-- und
CO\down{2}--Absorptionsbanden und unbefriedigende Basislinien
vorhanden sind (siehe auch Abb. \ref{datbvgl.wmf}). Die
quantitative Genauigkeit ist hier nicht untersucht worden, die
mangelhafte Spektrenaufbearbeitung l\"{a}sst aber Ungenauigkeiten
vermuten, die f\"{u}r eine hochgenaue Analyse nicht akzeptabel sind.
F\"{u}r eine qualitative Analyse von Atmosph\"{a}renspektren sind diese
Spektren als Hilfsmittel zur Identifikation unbekannter
Komponenten aber durchaus brauchbar.\\

\noindent(Internetadresse:
http://www.epa.gov/ttnernc01/ftir/epaadata.html)

\subsection{\label{parametersammlungen}Parametersammlungen zur Referenzspektrenberechnung}

Die Parametersammlungen zur Referenzspektrenberechnung enthalten
folgende Informationen f\"{u}r jede Absorptionslinie eines gegebenen
Molek\"{u}ls:
\begin{itemize}
  \item die Wellenzahl [cm\up{-1}]
  \item die Intensit\"{a}t [cm\up{-1}/(Molek\"{u}l$\cdot$cm\up{-1}]
  \item der Energiegrundzustand der betrachteten \"{U}berg\"{a}nge
  [cm\up{-1}]
  \item die Quantenzahlen der Schwingungs--, Rotations-- und
  elektronischen \"{U}berg\"{a}nge und die Hyperfeinstrukturaufspaltung
  \item das \"{U}bergangsmoment [debye]
  \item die Halbwertsbreite aufgrund der Verbreiterung durch
  atmosph\"{a}rische Druckeffekte [cm\up{-1}/atm]
  \item die Druckverschiebung der Bande [cm\up{-1}/atm]
  \item Fehlerabsch\"{a}tzungen f\"{u}r Wellenzahl, Intensit\"{a}t und
  Halbwertsbreite
  \item Codes f\"{u}r die einzelnen Molek\"{u}le und Isotope, Datum der
  Eintragung in die Datenbank
\end{itemize}

Die Motivation f\"{u}r die Erstellung der im folgenden aufgelisteten
drei Parametersammlungen war am Anfang unterschiedlich.
Mittlerweile wurden die Datenbest\"{a}nde aber mehr und mehr einander
angeglichen.\\

\subsubsection{\label{hitran}HITRAN--Datenbank}

Das urspr\"{u}ngliche Ziel der HITRAN-- (\bf Hi\rm gh Resolution \bf
Tran\rm smission Molecular Absorption database) Datenbank Ende der
60er Jahre war die Zusammenstellung detaillierter Informationen
\"{u}ber die Infraroteigenschaften terrestrischer Molek\"{u}le in der
Atmosph\"{a}re. Die Datenbank enth\"{a}lt in der bislang letzten Ausgabe
von 1996 Parameter von 35 kleinen Molek\"{u}len im Wellenzahlbereich
von 0--22656 cm\up{-1}, das entspricht in etwa 1 000 000
Spektrallinien \cite{rothman98}. Zus\"{a}tzlich liefert die Datenbank
Informationen zu Isotopen der enthaltenen Molek\"{u}le,
Brechungsindizes von Aerosolen und einigen ionisierten
Komponenten. F\"{u}r hohe Temperaturen liegen diese Parameter in der
HITEMP (\bf Hi\rm gh \bf Temp\rm erature spectroscopic absorption
parameters)--Datenbank vor.\\

F\"{u}r schwere Molek\"{u}le wie die Fluorchlorkohlenwasserstoffe liegen
die Informationen nicht mit den o.g. Parametern vor, sondern als
Absorptionsquerschnitt $k_{\stackrel{\sim}{\nu}}$:
\begin{equation}\label{eqabsquer}
  k_{\stackrel{\sim}{\nu}}=\frac{ln[I_0(\stackrel{\sim}{\nu})/I(\stackrel{\sim}{\nu})]}{n\cdot d}
\end{equation}
mit der Wellenzahl $\stackrel{\sim}{\nu}$,
I\down{0}($\stackrel{\sim}{\nu}$) und I($\stackrel{\sim}{\nu}$)
den Intensit\"{a}ten des einfallenden bzw. transmittierten Strahls,
n--der Molek\"{u}ldichte und d--der durchstrahlten Wegstrecke.
HITRAN96 enth\"{a}lt die Absorptionsquerschnitte von 8 zus\"{a}tzlichen
Molek\"{u}len.\\

Mittlerweile wird die HITRAN--Datenbank an der Atomic and
Molecular Physics Division am Harvard--Smithonian Center for
Astrophysics weiterentwickelt.

\noindent(Internetadresse: http://www.hitran.com)

\subsubsection{\label{geisa}GEISA--Datenbank}

Neben der Zusammenstellung von IR--Daten terrestrisch relevanter
Molek\"{u}le, war es Ziel der GEISA--Datenbank (\bf G\rm estion et \bf
\'{E}\rm tude des \bf I\rm nformations \bf S\rm pectroscopiques \bf
A\rm tmosph\'{e}riques), auch Atmosph\"{a}renmolek\"{u}le anderer Planeten
miteinzubeziehen. Erste Arbeiten hierzu begannen 1974 am
Laboratoire de M\'{e}t\'{e}orologie Dynamique du CNRS in Palaiseau/Cedex
in Frankreich. Die Datenbank umfasst mittlerweile Parameter von 42
Molek\"{u}len (und 96 Isotopen) im Wellenzahlbereich von 0--22656
cm\up{-1} mit insgesamt 1 346 266 Eintragungen. Die letzte
aktualisierte Ausgabe von 1997 \cite{husson99} enth\"{a}lt zus\"{a}tzlich
die Absorptionsquerschnitte von 23 Molek\"{u}len im Spektralbereich
von 556--1763 cm\up{-1}.\\

\noindent(Internetadresse: http://www.ara.polytechnique.fr)


\subsubsection{\label{atmos}ATMOS--Datenbank}

Die ATMOS--Datenbank (\bf A\rm tmospheric \bf T\rm race \bf M\rm
olecule \bf S\rm pectroscopy) wurde erstellt, um Infrarotspektren
der Erdatmosph\"{a}re, aufgenommen bei mittlerweile vier Fl\"{u}gen des
Space Shuttle 1985, 1992, 1993 und 1994, auswerten zu k\"{o}nnen. Sie
besteht im wesentlichen aus einer \"{u}berarbeiteten Version der
HITRAN--Ausgabe von 1992 mit zus\"{a}tzlichen Linien aus der GEISA von
1991, enth\"{a}lt rund 700 000 Linienparameter von 49 Molek\"{u}len
zwischen 0 und 10 000 cm\up{-1} und die Absorptionsquerschnitte
von 8 weiteren Molek\"{u}len. Die Hauptarbeiten hierzu wurden im Jet
Propulsion Laboratory am California Institute of Technology in
Pasadena geleistet. Eine ausf\"{u}hrliche Darstellung der Datenbank
und die Unterschiede zu den anderen Datenbanken findet sich bei
\cite{brown96}. Im Gegensatz zur HITRAN und GEISA wird der
Datenbestand der ATMOS z.Z. nicht mehr ver\"{a}ndert.\\

\noindent(Internetadresse: http://remus.jpl.nasa.gov/atmos/)

\subsubsection{\label{malt}Spektrenberechnungsprogramme HAWKS,
GEISA--PC, AirSentry und MALT}

Die Programme HAWKS (\bf H\rm ITRAN \bf A\rm tmospheric \bf W\rm
or\bf ks\rm tation) und GEISA--PC, welche den Datenbanken
beiliegen, erlauben die Spektrenberechnung aus den Parametern der
Datenbanken. Sie sind allerdings in der Bedienung noch nicht sehr
benutzerfreundlich. An Verbesserungen bei beiden Programmen wird
gearbeitet. Die AirSentry--Software aus dem Laboratory of
Theoretical Spectroscopy (LTS) am Institute of Atmospheric Optics
in Tomsk/Russland ist in der Lage, sowohl die Softwareformate der
HITRAN wie auch der GEISA zu bearbeiten. Die
Benutzerfreundlichkeit zur Spektrenberechnung ist aber weitaus
h\"{o}her. Unterschiede bei den Resultaten gibt es nicht.\\

Die MALT (\bf M\rm ultiple \bf A\rm tmospheric \bf L\rm ayer \bf
T\rm ransmission)--Software von Griffith et al. erlaubt
dar\"{u}berhinaus die Berechnung der Spektren unter Ber\"{u}cksichtigung
der instrumentellen Linienparameter, so dass die Aufl\"{o}sung und
eventuelle Linienverschiebungen miteinbezogen werden
\cite{griffith96}. MALT unterst\"{u}tzt z.Z. nur das HITRAN--Format.


\paragraph{\label{parabewert}Kurze Bewertung der
Parameterdatenbanken:}

Alle drei Parameterdatenbanken beinhalten sehr zuverl\"{a}ssiges
Informationsmaterial, wenn es um die Berechnung von Spektren der
in ihr enthaltenen Molek\"{u}le geht. Aufgrund der engen
Zusammenarbeit der drei federf\"{u}hrenden Gruppen bei der Erstellung
dieser Datenbanken, ist die Redundanz der Daten untereinander aber
recht hoch. Nicht mehr aktuelle Vergleiche der Datenbanken finden
sich bei \cite{husson93} und \cite{brown96}, ein aktueller, bisher
nicht publizierter Vergleich auf der Internetseite der GEISA.\\

Der entscheidende Nachteil der Parameterdatenbanken liegt darin,
dass sie auf ca. 40 Molek\"{u}le beschr\"{a}nkt sind. Dies ist f\"{u}r die
Spurengasanalytik der Atmosph\"{a}re ausreichend, nicht aber f\"{u}r die
Schadgasanalyse von Luftinhaltsstoffen wie sie bei der diffusen
Emission z.B. aus industriellen Quellen auftreten. Es gibt Gase,
bei denen bisher nur in bestimmten spektralen Intervallen
charakterisiert sind (z.B. Ethan, siehe Abb. \ref{datbvgl.wmf},
HF, SF\down{6}). Bei der CLS--Auswertung in diesen Bereichen sind
diese Spektren dann nicht verwendbar. Im \"{u}brigen soll es Probleme
bei der Genauigkeit der Intensit\"{a}tsberechnung bei hohen
Temperaturen (um die 1000�C) geben, was aber f\"{u}r die
atmosph\"{a}rische Schadgasbestimmung nicht relevant ist.\\



\section{\label{motivation}Motivation f\"{u}r eigene Kalibriermessungen}

Zu Beginn des Projektes standen zur CLS--Auswertung ausschlie{\ss}lich
Spektren der QASoft--Datenbank (Kap. \ref{qasoft}) zur Verf\"{u}gung.
Es zeigte sich recht schnell, dass deren Qualit\"{a}t f\"{u}r eine
zufriedenstellende quantitative Auswertung nicht ausreichend war.
Die Spektren wurden mit verschiedenen Spektrometern aufgenommen
(siehe oben). All diese Ger\"{a}te zeigen Unterschiede in der
photometrischen Genauigkeit untereinander und zu dem im Projekt
benutzten K300. Die Gr\"{u}nde hierf\"{u}r liegen in der spektralen
Aufl\"{o}sung, der Verwendung unterschiedlicher Detektoren und in der
Selbstapodisation (siehe auch Kap. \ref{photometgen}). Da das
K300--Spektrometer mit einer Aufl\"{o}sung von 0.2 cm\up{-1} arbeitet,
m\"{u}ssen die Spektren mit 0.125 cm\up{-1} Aufl\"{o}sung mit einer
instrumentellen Linienfunktion auf 0.2 cm\up{-1}--Aufl\"{o}sung
umgerechnet werden, was zu Fehlern f\"{u}hren kann. Hinzu kommen
weitere Abweichungen durch die Interpolations--Algorithmen, die
die QASoft--Spektren auf den Punktabstand der K300--Spektren
bringen. Die restlichen Spektren stehen nur mit einer schlechteren
Aufl\"{o}sung von 0.5 cm\up{-1} zur Verf\"{u}gung und weisen somit noch
gr\"{o}{\ss}ere Abweichungen zu mit K300--Spektrometer gemessenen Spektren
auf.\\

\bild{htb}{datbvgl.wmf}{420}{345}{\it Beispiele f\"{u}r qualitative
Defizite von Datenbanken f\"{u}r hochaufgel\"{o}ste IR--Gasspektren. Diese
Spektren werden verglichen mit in dieser Arbeit gemessenen
Spektren (ISAS). Bis auf die untersten sind die Spektren in \bf
A\it , \bf B \it und \bf C \it mit einem Offset dargestellt.}

Ein weiterer entscheidender Nachteil ist, dass die Interferogramme
der QASoft--Datenbank nicht verf\"{u}gbar sind. Somit ist es nicht
m\"{o}glich, eine andere Apodisationsfunktion als die vorgegebene
Dreiecksapodisation zu w\"{a}hlen. Es hat sich gezeigt (siehe Kap.
\ref{faltung}), dass die Wahl anderer Apodisationsfunktionen zu
besseren Ergebnissen, insbesondere zu einem erweiterten linearen
Extinktionsbereich hinsichtlich der Abh\"{a}ngigkeit der Bandenmaxima
von der Konzentration, f\"{u}hren kann. Die meisten Spektren liegen in
der QASoft mit Volumenanteilen von ca. 100 ppm$\cdot$m vor, was
oft ausreichend ist, aber auch bedeutet, dass Nichtlinearit\"{a}ten
bei h\"{o}heren Volumenanteilen nicht ber\"{u}cksichtigt werden.\\

Abbildung \ref{datbvgl.wmf} \bf C \rm und \bf D \rm illustrieren,
wie bei der Bearbeitung einiger Spektren der Datenbank vorgegangen
worden ist. Die gezeigten Beispiele sind nicht der Normalfall und
auch nicht \"{a}u{\ss}erst gravierend. Sie zeigen dennoch, dass bei der
Beurteilung der Qualit\"{a}t der Spektren vorsichtig vorgegangen
werden muss. Im Anhang A sind bei s\"{a}mtlichen in dieser Arbeit
vermessenen Pr\"{u}fgasen auch die Referenzwerte der st\"{a}rksten Bande
im linearen Extinktionsbereich mit den Werten der QASoft
verglichen worden und in den jeweiligen Abbildungen \bf A \rm
eingezeichnet. Die Werte stimmen i.a. gut \"{u}berein.\\

Die Parametersammlungen sind f\"{u}r die Molek\"{u}le, die letztendlich in
ihnen enthalten sind, generell eine gute Alternative. Aber auch
sie haben das Problem, dass einige Linien nicht beschrieben sind
(siehe z.B. Abb. \ref{datbvgl.wmf} \bf A\rm ). Die
EPA--Spektrendatenbank ist zum jetzigen Zeitpunkt f\"{u}r eine
hochwertige quantitative Analyse nicht geeignet (Abb.
\ref{datbvgl.wmf} \bf B\rm ).\\

Ziel unserer eigenen Kalibriergasmessungen war es somit, die
Fehler aufgrund der Spektrometereinfl\"{u}sse zu eliminieren und durch
Messung eines gro{\ss}en Konzentrationsbereiches auch das nichtlineare
Verhalten mitber\"{u}cksichtigen zu k\"{o}nnen (siehe auch Kap.
\ref{pruefgasauswertung}). Desweiteren sollten Basislinieneffekte
minimiert und H\down{2}O-- und CO\down{2}--Banden optimal
eliminiert werden.\\

Folgende Pr\"{u}fgase wurden in dieser Arbeit untersucht: Ammoniak,
Acetaldehyd, Benzol, Distickstoffoxid, Ethen, Isobuten,
Kohlenstoffmonoxid, Methan, Methanol, Propen, p--Xylol,
Schwefeldioxid, Stickstoffdioxid, Toluol  und zus\"{a}tzlich Wasser
und Kohlenstoffdioxid.\\


\section{\label{aufbau}Aufbau einer Kalibriereinrichtung}

\bild{htb}{laborg2.bmp}{375}{530}{\it Laboraufbau mit
Blendensystem zur Verd\"{u}nnung von Pr\"{u}fgasen, Multireflexions--Zelle
und K300--Spektrometer}

F\"{u}r die Erstellung von qualitativ hochwertigen Referenzspektren
wurde ein pr\"{a}zises, vielseitig einsetzbares, robustes und
kosteng\"{u}nstiges Kalibrierverfahren f\"{u}r die
infrarotspektrometrische Analytik von Spurengasen gesucht.
Erfahrungen aus fr\"{u}heren Arbeiten \cite{heise85} zeigen, dass
Apparaturen zur Verd\"{u}nnung von Pr\"{u}fgasen unter Verwendung von
Mikroblenden hierf\"{u}r sehr gut geeignet sind (siehe Abb.
\ref{laborg2.bmp}). Verwendet wurden Pr\"{u}fgase der Firma
Messer--Griesheim, wobei die Spurengase in Stickstoff vorlagen. Um
die Pr\"{u}fgase zu niedrigeren Konzentrationen heruntermischen und
diese variieren zu k\"{o}nnen, wurde ein Zwei--Leitungsblendensystem
aufgebaut, durch das definierte Gasvolumenstr\"{o}me von Stickstoff
und dem Pr\"{u}fgas flie{\ss}en und vor der Multireflexionszelle
(White--Zelle) zusammengef\"{u}hrt werden k\"{o}nnen. Die Gasfl\"{u}sse h\"{a}ngen
hierbei einzig vom Vordruck der verwendeten Gase vor den
entsprechenden Blenden ab (Druckmessung 1). Um einen m\"{o}glichst
gro{\ss}en Konzentrationsbereich abdecken zu k\"{o}nnen, stehen insgesamt
5 Blenden (0.15, 0.20, 0.36, 0.60 und 0.96 mm) zur Verf\"{u}gung, die
bei demselben Vordruck (2.4 bis 6 bar) parallel geschaltet werden
und f\"{u}r die Stickstoff--Verd\"{u}nnung genutzt werden k\"{o}nnen. Das
Pr\"{u}fgas wird durch das zweite Leitungssystem, das mit nur einer
Blende ausger\"{u}stet ist, geschickt. Die Blenden sind einfach
austauschbar. F\"{u}r das Pr\"{u}fgas stehen Mikroblenden mit den Gr\"{o}{\ss}en
0.10, 0.15, 0.20 und 0.36 mm, ausgelegt ebenfalls f\"{u}r den
Druckbereich 2.4 bis 6 bar, zur Verf\"{u}gung. Beide Leitungssysteme
werden vor der White--Zelle zusammengef\"{u}hrt, wo eine Mischung
stattfindet. Es muss darauf geachtet werden, dass die
Leitungsquerschnitte gro{\ss} genug sind, so dass kein Druckaufbau in
den Leitungen entsteht. Dies wird bei der Druckmessung an der
Stelle 2 \"{u}berpr\"{u}ft. Das System wurde mit 3 unabh\"{a}ngigen Verfahren
\"{u}berpr\"{u}ft, um genaue Gasvolumenstr\"{o}me in Abh\"{a}ngigkeit vom Vordruck
zu erhalten. Hierzu wurde zum einen das Multicon--System der Firma
Draeger (L\"{u}beck) genutzt, das den Gasfluss anhand einer
Differenzdruckmessung an einer Gleichrichterr\"{o}hre misst, zum
zweiten das Gilibrator--2--System der Firma Sensidyne (Clearwater,
Fl.) , welches die Geschwindigkeit einer vom Gasvolumen gebildeten
Seifenhaut mittels zweier Lichtschranken misst und zum dritten
eine handels\"{u}bliche Gasuhr der Firma Elster (Wiesbaden) verwendet.
Bei s\"{a}mtlichen Blenden konnten im verwendeten Druckbereich, der
eine laminare Str\"{o}mung hinter der Blende garantiert, sehr gute
\"{U}bereinstimmungen in der Kalibration bzgl. Gasflusseinstellung und
Messung erzielt werden.\\

Als K\"{u}vette wurde eine Multireflexionsk\"{u}vette (White--Zelle) der
Firma Bastian--Feinmechanik aus Wuppertal genutzt. Diese Zelle hat
ein Volumen von 50 L und erlaubt die Einstellung von optischen
Wegl\"{a}ngen von 6.92 m bis 110.72 m in 6.92 m--Abst\"{a}nden (zur
Arbeitsweise von White--Zellen siehe auch \cite{hanst94}). Eine
angeschlossene Pumpe mit K\"{u}hlfalle erlaubte die Evakuierung der
Zelle von Atmosph\"{a}rendruck auf 2$\cdot$10\up{-3} bar innerhalb von
7 min. Der Zelleninnendruck kann durch die Druckmessung an der
Stelle 3 (siehe Abb. \ref{laborg2.bmp}) bestimmt werden. Die
Voroptik konnte st\"{a}ndig mit Stickstoff gesp\"{u}lt werden.\\

Abbildung \ref{snrpeak.wmf} zeigt, dass das Peak zu Peak--Rauschen
in Extinktionseinheiten, bestimmt durch zwei direkt aufeinander
folgende Messungen in der bei einem Druck von 1013 hPa Stickstoff
gef\"{u}llten White--Zelle, bis 90 m nur sehr moderat ansteigt und
somit eine problemlose Messung bis zu solchen Wegl\"{a}ngen
erm\"{o}glicht, wodurch sehr gut praxisrelevante Situationen simuliert
werden k\"{o}nnen.\\

\bildlinks{htb}{snrpeak.wmf}{300}{250}{85}{85}{-35}{\it Peak zu
Peak--Extinktionsrauschen, bestimmt durch zwei direkt
aufeinanderfolgende Messungen mit 1 bar Stickstoff in der
White--Zelle in Abh\"{a}ngigkeit von der Wegl\"{a}nge (Globar bei 6.92 m:
12.02 V Betriebsspannung, bei den \"{u}brigen Wegl\"{a}ngen 13.12 V).}

Die Datenaufnahme erfolgte auf einem handels\"{u}blichen PC mit
Pentium90--Prozessor und 32 MB RAM Arbeitsspeicher unter
Verwendung des mitgelieferten Spektrometerprogramms unter der
GRAMS--Software. Die Aufnahme eines Spektrums unter Akkumulation
von 100 Interferogrammen bei einer spektralen Aufl\"{o}sung von 0.2
cm\up{-1} ben\"{o}tigte ca. 3$\frac{1}{2}$ min.\\


\section{\label{messablauf}Messablauf}

Ein Ziel der Pr\"{u}fgasmessungen war es, einen m\"{o}glichst breiten
Konzentrationsbereich abzudecken und hierbei auch in einen Bereich
relativ hoher Konzentrationen vorzusto{\ss}en, um nichtlineare
Extinktionsabh\"{a}ngigkeiten ber\"{u}cksichtigen zu k\"{o}nnen. Weiterhin
sollten qualitativ hochwertige Spektren hinsichtlich der
Vermessung von Absorptionen atmosph\"{a}rischer Komponenten erhalten
werden. Dazu war es n\"{o}tig, den in der Voroptik- und
Spektrometeratmosph\"{a}re enthaltenen H\down{2}O-- und
CO\down{2}--Gehalt m\"{o}glichst gering und konstant zu halten.
Sichergestellt wurde dies durch eine kontinuierliche Sp\"{u}lung von
Spektrometer und Voroptik mit Stickstoff der Reinheit 5.0 von
$2\frac{1}{2}$ h vor bis zum Ende der Messungen. Gleichzeitig
wurde ebenfalls weit vor Beginn der Messungen der Globar
eingeschaltet und das K300--Spektrometer optimal justiert, um zu
Beginn der Messungen m\"{o}glichst stabile Interferometerverh\"{a}ltnisse
vorliegen zu haben, da sonst durch Temperatur\"{a}nderungen geringe
optische Dejustagen des Interferometers auftreten, die eine \"{u}ber
l\"{a}ngere Zeit gleichbleibende Interferogrammaufnahme verhindern.\\

Desweiteren sollten f\"{u}r die sp\"{a}tere Mittelung von Spektren
m\"{o}glichst viele Messungen vorliegen, wobei sich folgendes Vorgehen
als vorteilhaft herausstellte. F\"{u}r jedes Pr\"{u}fgas wurden 2
Messreihen mit jeweils 14 Spektren aufgenommen, so dass
letztendlich 28 Spektren f\"{u}r jedes Pr\"{u}fgas zur Verf\"{u}gung standen.
Jeweils zu Beginn und zum Ende einer Messreihe wurde ein
Hintergrundspektrum von einer mit Stickstoff gef\"{u}llten K\"{u}vette und
das dazugeh\"{o}rige Eigenstrahlungsspektrum aufgenommen. F\"{u}r die
Pr\"{u}fgasmessungen wurde ein bestimmter Volumenstrom eingestellt
(\"{u}blicherweise 7--9 l/min als Gesamtvolumenstrom) und die Wegl\"{a}nge
von 6.92 bis 89.96 m schrittweise ver\"{a}ndert. Durch diese
Volumenstr\"{o}me lassen sich eventuelle Wandadsorptionseffekte auch
bei polaren Gasen in den vorliegenden Konzentrationen wie z.B.
beim Ammoniak erheblich reduzieren. Um sie aber sicher
auszuschlie{\ss}en, wurde am Ende der Messreihe noch einmal ein
weiteres Spektrum bei 6.92 m Wegl\"{a}nge aufgenommen. H\"{a}tten
Wandeffekte vorgelegen, w\"{u}rden sich das erste und letzte Spektrum
der Messreihe voneinander unterscheiden. Desweiteren wurde darauf
geachtet, dass sich 3--4 Konzentrationsmesspunkte in beiden
Messreihen \"{u}berlagerten. So war eine zus\"{a}tzliche Kontrolle der
Vordr\"{u}cke und verwendeten Blenden zur Einstellung reproduzierbarer
Gasfl\"{u}sse m\"{o}glich.\\


\section{\label{spekaufbereitung}Spektrenaufbereitung}

Um eine z\"{u}gige Aufarbeitung der gemessenen Kalibrationsspektren zu
erreichen, wurde eine Auswertung mittels selbst erstellter
Software (ca. 7000 Programmzeilen unter MATLAB 5.2) vorgenommen.\\

Grunds\"{a}tzlich wurden w\"{a}hrend der Messreihen die Interferogramme
abgespeichert. Dies erlaubte die Nichtlinearit\"{a}tskorrektur
s\"{a}mtlicher Einkanalspektren. Aus Kompatibilit\"{a}tsgr\"{u}nden zu den
Referenzspektren aus der QASoft--\-Dat\-en\-bank wurde weiterhin
die Dreiecksapodisation gew\"{a}hlt.\\

\bildlinks{htb}{ethbas.wmf}{300}{250}{80}{85}{-35}{\it Ethen, 61
ppm$\cdot$m, \bf A \it Extinktionsspektrum mit beispielhafter
Basislinie und Splineanpassung durch die vorgegebenen St\"{u}tzstellen
(mit Offset). \bf B \it Basislinienkorrigiertes
Extinktionsspektrum.}

Zur Bildung des Extinktionsspektrums standen die zwei
Hintergrund-- und Eigenstrahlungsspektren vor und nach der
Messreihe zur Verf\"{u}gung. Es wurde jeweils das Hintergrundspektrum
gew\"{a}hlt, welches die geringsten Basislinieneffekte bei der
Transmissionsspektrenberechnung verursachte. Leichte
Konzentrationsunterschiede von atmosph\"{a}rischen Komponenten, die zu
verbleibenden H\down{2}O-- und CO\down{2}--Absorptionsbanden
f\"{u}hrten, waren nicht relevant, da diese in einem folgenden Schritt
noch eliminiert wurden. Abbildung \ref{ethbas.wmf} \bf A \rm zeigt
eine typische auftretende Basislinie. F\"{u}r jedes Pr\"{u}fgas wurden
verschiedene Punkte festgelegt, die die St\"{u}tzstellen von
Splinefunktionen abgaben. Somit konnte die Basislinienberechnung
f\"{u}r s\"{a}mtliche Spektren einer Messung automatisiert werden. Der
Teil \bf B \rm zeigt das basislinienkorrigierte Spektrum.\\

Im basislinienkorrigierten Spektrum in Abbildung \ref{ethbas.wmf}
\bf B \rm sind noch deutliche H\down{2}O-- und
CO\down{2}--Absorptionssignaturen zu erkennen. Es hat sich
herausgestellt, dass das Absorptionsspektrum von H\down{2}O und
CO\down{2}, das aus beiden Hintergrundspektren vor und nach der
Messreihe gebildet wurde, hervorragend geeignet war, um eine
Eliminierung dieser Absorptionsbanden in den Pr\"{u}fgasspektren
vorzunehmen. Zum einen ist der Konzentrationsbereich passend,
desweiteren sind leichte Wellenzahlverschiebungen, Temperatur--
und Druckunterschiede nicht relevant, um signifikante spektrale
Residuen zu erzeugen. Abbildung \ref{ethwater.wmf} zeigt zwei
typische Spektren zur Eliminierung der H\down{2}O-- und
CO\down{2}--Signaturen. Die Extinktionen in Spektren der Messreihe
2 in \bf B \rm sind deswegen kleiner, weil seit Beginn der Sp\"{u}lung
von Spektrometer und Voroptik mit Stickstoff schon mehr Zeit
vergangen war, womit die Volumenanteile von H\down{2}O und
CO\down{2} weiter reduziert, jedoch nicht vollst\"{a}ndig eliminiert
werden konnten.\\

\bildlinks{htb}{ethwater.wmf}{300}{250}{80}{85}{-35}{\it Typische
Extinktionsspektren, die zur Eliminierung von H\down{2}O-- und
CO\down{2}--Absorptionen in den Pr\"{u}fgasspektren verwendet wurden,
\bf A \it Messreihe 1 (Spektrum ist mit Offset dargestellt) und
\bf B \it Messreihe 2.}

\bildlinks{htb}{ethwatab.wmf}{300}{250}{80}{85}{-35}{\it Typisches
Ethenspektrum \bf A \it ohne Abzug, \bf B \it nach H\down{2}O--
und CO\down{2}--Absorptionsbanden Kompensation und \bf C \it nach
H\down{2}O--, CO\down{2}--Bandenabzug und Spikeeliminierung. Die
Spektren \bf A \it und \bf B \it sind mit einem Offset
dargestellt.}

Die Subtraktion der H\down{2}O-- und CO\down{2}--Absorptionsbanden
erfolgt in den 4 relevanten Bereichen 630-760, 1256-2180,
2299-2400 und 2950-4000 cm\up{-1}. Hierzu werden in jedem Bereich
f\"{u}r jedes Pr\"{u}fgas individuell kleine Bereiche von 10-20 cm\up{-1}
Breite, in denen keine Banden des Pr\"{u}fgases, aber intensive
Absorptionslinien von H\down{2}O und CO\down{2} vorliegen,
herausgesucht und die zu subtrahierenden Banden mittels
Least--Squares--Fit an das individuelle Pr\"{u}fgasspektrum angepasst.
Auch diese Prozedur konnte automatisiert werden. Abbildung
\ref{ethwatab.wmf} \bf A \rm und \bf B \rm zeigt ein typisches
Ethenspektrum vor und nach dem H\down{2}O-- und
CO\down{2}--Bandenabzug.\\

\"{A}u{\ss}erst kleine auftretende Unterschiede in der Wellenzahlachse
f\"{u}hren zu Bildung von Spikes, d.h. normalerweise einzeln
auftretenden Punkten, die keiner Bande entsprechen. Auch diese
werden eliminiert. Hierzu wurde ein Programm entwickelt, welches
diese Spektralpunkte erkennt und mittels Anpassung einer Geraden
an die Nachbarpunkte eliminiert (siehe auch Abb.
\ref{ethwatab.wmf} \bf C \rm und Abbildung \ref{ethspike.wmf}).
Hierbei ist zu beachten, dass in Bereichen, in denen das Pr\"{u}fgas
eigene Banden besitzt, sehr sorgf\"{a}ltig vorzugehen ist.\\


\bildlinks{htb}{ethspike.wmf}{300}{250}{80}{85}{-35}{\it
Ausschnitt aus einem typischen Ethenspektrum vor und nach der
Spikeeliminierung (Spikes hellgrau).}


\section{\label{pruefgasauswertung}Auswertung und Diskussion}

Die Auswertung der 28 pro Pr\"{u}fgas erhaltenen Spektren erfolgt in
zwei Schritten. Anhand der univariaten Auswertung der jeweils
st\"{a}rksten Bande ist zu ersehen, ab welchem Extinktionsmaximum die
Auswertung keine sinnvollen Ergebnisse mehr liefert, da das Gas
optisch dick ist, d.h. eine Transmission unter 3$\cdot$10\up{-3}
Transmissionseinheiten vorliegt ("`Totalabsorption"').
Gleichzeitig ist hierdurch auch eine Qualit\"{a}tskontrolle m\"{o}glich.
Bei groben Abweichungen von einer Kalibrierfunktion, die im
Extremfall durch ein kubisches Polynom dargestellt werden kann,
ist dies ein Hinweis auf einen systematischen Fehler bei der
Aufnahme dieses Spektrums. Bei Banden mit Extinktionsmaxima um 2.5
(d.h. die Transmission liegt nahe der Totalabsorption) ist es
sinnvoll, in diesem Fall zus\"{a}tzlich eine Bande mit deutlich
geringeren Extinktionen auszuwerten, um eventuelle systematische
Fehler bei der Pr\"{u}fgasspektrenmessung zu detektieren.\\

F\"{u}r die Auswertung in der Praxis w\"{a}re es w\"{u}nschenswert, spektrale
Parameter vorliegen zu haben, mit denen man ein Spektrum
gew\"{u}nschter Konzentration und optischer Wegl\"{a}nge berechnen kann,
in dem auch Abweichungen vom Lambert--Beer'schen Gesetz,
darstellbar durch nichtlineare Terme, ber\"{u}cksichtigt werden. Aus
diesem Grund wurde im Anschluss an die Qualit\"{a}tskontrolle eine
Parametrisierung der Spektren vorgenommen. Die univariate
Auswertung der intensivsten Bande gab f\"{u}r jedes Pr\"{u}fgas eine
individuelle Grenze der zu ber\"{u}cksichtigenden Extinktionen vor.
Jeder Wellenzahlpunkt aller Pr\"{u}fgasspektren wurde dann einem
linearen Fit unterzogen. Spektren, die bei der betrachteten
Wellenzahl \"{u}ber dem vorgegebenen Extinktionslimit lagen, wurden
nicht ber\"{u}cksichtigt. In einem zweiten Schritt wurde nun
festgestellt, an welchen Wellenzahlpunkten ebenfalls ein
quadratischer Term zu ber\"{u}cksichtigen war. Hierzu wurde das
Residuum aus dem gemessenen Pr\"{u}fgasspektrum mit der h\"{o}chsten
Konzentration und dem aus der linearen Anpassung f\"{u}r diese h\"{o}chste
Konzentration berechneten Spektrum gebildet. Punkte, die \"{u}ber
2$\sigma$ der Rauschfunktion lagen, wurden auch mit einem
quadratischen Polynom angepasst. Die gleiche Vorgehensweise wurde
noch f\"{u}r eine Anpassung mit einem kubischen Polynom gew\"{a}hlt. Somit
liegt nun f\"{u}r jedes Pr\"{u}fgas eine 3xN--Matrix vor, mit der f\"{u}r eine
beliebige Konzentration ein Spektrum berechnet werden kann, bei
dem innerhalb der durch die Kalibration vorgegebenen maximalen
Grenzen sowohl lineare, wie auch quadratische und kubische Terme
ber\"{u}cksichtigt werden.\\

\bildlinks{htb}{ethsnr.wmf}{300}{250}{80}{85}{-35}{\it Vergleich
zwischen zwei typischen Extinktionsrauschspektren von einem
berechneten gemittelten Referenzspektrum (oben, mit Offset) und
einem gemessenen einzelnen Spektrum (unten) bei gleicher
Konzentration.}

Im Anhang A findet sich f\"{u}r alle 14 gemessenen Pr\"{u}fgase eine
Dokumentation dieser Auswertungen. Die jeweils erste Abbildung
zeigt die univariate Auswertung der st\"{a}rksten Bande, die zweite
Abbildung zeigt den linearen, quadratischen und kubischen Term f\"{u}r
eine in der Atmosph\"{a}re m\"{o}glich vorkommende Konzentration und die
dritte Abbildung zeigt das Residuum des berechneten mit dem
einzeln gemessenen Spektrum. F\"{u}r Kohlenstoffmonoxid wurde
beispielhaft die Auswertung auch f\"{u}r eine spektrale Aufl\"{o}sung von
0.5 cm\up{-1} vorgenommen. Der Vergleich der Auswertungen der
st\"{a}rksten Bande bei 2169.3 cm\up{-1} in der Abbildung \bf A \rm im
Anhang zeigt eindrucksvoll die gr\"{o}{\ss}eren Nichtlinearit\"{a}ten in den
Spektren bei schlechterer spektraler Aufl\"{o}sung.\\

Ein weiterer Vorteil dieser Auswertung liegt in dem durch die
Mittelung verbesserten Signal/Rauschverh\"{a}ltnis. Bei den gemessenen
Spektren h\"{a}ngt dieses vorwiegend von der in der White--Zelle
eingestellten optischen Wegl\"{a}nge ab. Abbildung \ref{ethsnr.wmf}
zeigt eine durchschnittliche Verbesserung um den Faktor 10
gegen\"{u}ber einem Spektrum, das bei 20.76 m optischer Wegl\"{a}nge in
der White--Zelle gemessen wurde.\\

Die relative Analysengenauigkeit der Pr\"{u}fgase von der Firma
Messer--Griesheim (D\"{u}sseldorf) betrug 2 \% (bei Acetaldehyd und
p--Xylol 5 \%). Weiterhin trugen zum Fehler der hier vermessenen
Pr\"{u}fgase die einzelnen Fehler der Wegl\"{a}ngenmessung, der
Gesamtdruckmessung, der Temperaturmessung, der Druckmessungen vor
den Mikroblenden von Pr\"{u}fgas und Stickstoff und die Kalibrierung
der Volumenstrommessungen durch das System bei. Unter Verwendung
des Fehlerfortpflanzungsgesetzes resultiert damit f\"{u}r die
gemittelten Pr\"{u}fgasspektren (hierbei wurde ein gemittelter Fehler
in der optischen Wegl\"{a}ngenbestimmung angenommen) ein maximaler
relativer Fehler von $\pm 3 \%$ (bei Acetaldehyd und p--Xylol $\pm
6 \%$). Dies ist ein \"{a}u{\ss}erst zufriedenstellende Pr\"{a}zision, wenn
man bedenkt, dass sich die spektralen Extinktionsangaben in den
o.g. Datenbanken teilweise um 10 \% und mehr voneinander
unterscheiden.\\

\cleardoublepage
