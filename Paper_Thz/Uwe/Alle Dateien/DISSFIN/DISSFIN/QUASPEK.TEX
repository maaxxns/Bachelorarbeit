\chapter{\label{quantitativ}Quantitative Spektroskopie}

In diesem Kapitel werden verschiedene Strategien zur quantitativen
Auswertung von FT--IR Atmosph\"{a}renspektren vorgestellt. Um
zuverl\"{a}ssige und richtige Ergebnisse bei dieser quantitativen
Auswertung zu erhalten, m\"{u}ssen die Einfl\"{u}sse bekannt sein, die zu
systematischen Fehlern f\"{u}hren k\"{o}nnen. Dabei muss zwischen
Spektrometereinfl\"{u}ssen (Kap. \ref{spektrometereinfluesse}) und
\"{A}nderungen in den atmosph\"{a}rischen Bedingungen (Kap.
\ref{austempdr}) unterschieden werden. Die klassischen
multivariaten Auswertungsmethoden erfordern zudem ein
Hintergrundspektrum und die Kenntnis s\"{a}mtlicher Komponenten im
Gasgemisch, die im Vergleich zum spektralen Rauschen signifikante
Beitr\"{a}ge zum Atmosph\"{a}renspektrum aufweisen. In den Kapiteln
\ref{hintergrund} und \ref{crosscorr} werden verschiedene bei
Messungen im Feld auftretende Probleme und L\"{o}sungsm\"{o}glichkeiten
aufgezeigt.

\section{\label{spektrometereinfluesse}Spektrometerabh\"{a}ngige
Einfl\"{u}sse}

\subsection{\label{photometgen}Photometrische Genauigkeit mit
FT--IR--Spektrometern}

Eine m\"{o}gliche Quelle f\"{u}r systematische Fehler bei der
quantitativen Auswertung liegt in der Berechnung der IR--Spektren
mittels Fourier--Trans\-for\-ma\-tion aus den nur begrenzt
vorliegenden Interferogrammen. Hier ist positiverweise anzumerken,
dass die sonst \"{u}bliche Phasenkorrektur, die bei einseitig
aufgenommenen Interferogrammen standardm\"{a}{\ss}ig nach der
Mertz--Methode vorgenommen wird und die bei starken Absorptionen
zu betr\"{a}chtlichen photometrischen Fehlern f\"{u}hren kann
\cite{chase82}, f\"{u}r die mit dem K--300 Spektrometer aufgenommenen
Interferogramme entf\"{a}llt. Diese Interferogramme liegen beidseitig
vor, so dass kein N\"{a}herungsverfahren eingesetzt werden muss, um
die realen Amplitudenanteile nach der Fourier--Transformation der
Interferogramme zu berechnen. Da die berechneten Spektren
\"{u}blicherweise nur Anteile gr\"{o}{\ss}er Null aufweisen, f\"{u}hrt die
Berechnung des sogenannten Power-Spektrums \bf ps \rm ($\bf
fft_{int}$--Fourier--Transformierte des Interferogramms) zum
gew\"{u}nschten Einkanalspektrum:

\begin{equation}\label{eqps}
  \bf ps(\stackrel{\sim}{\nu})=\sqrt{\bf fft_{int}(\stackrel{\sim}{\nu}) \cdot fft_{int}^*(\stackrel{\sim}{\nu})}
\end{equation}

\bild{htb}{nlkorr1.wmf}{350}{430}{\bf A \it nicht korrigiertes und
\bf B \it Einkanalspektrum mit Korrektur der
Detektornichtlinearit\"{a}t einer SF\down{6}--Probe mit den
dazugeh\"{o}rigen Eigenstrahlungsspektren (optische Pfadl\"{a}nge 41.5 m,
SF\down{6} 31 hPa in N\down{2}, Gesamtdruck 1010 hPa). In den
Nebenbildern ist die Differenz von Einkanal-- und
Eigenstrahlungsspektrum gezeigt. Die Intensit\"{a}ten sind in
willk\"{u}rlichen Einheiten dargestellt.}

Ein weiterer Punkt, der zu ber\"{u}cksichtigen ist, betrifft die
Spektrenabtastung, d.h. mit welcher Nyquist--Frequenz gerechnet
werden kann. Standardm\"{a}{\ss}ig wird beim K300 das Interferogramm nicht
wie sonst \"{u}blich mit einem Abstand von 633 nm digitalisiert,
sondern nach jedem dritten Nulldurchgang des
He--Ne--Laser--Sinussignals, was zur Digitalisierung des
Interferogramms dient. Ein entsprechendes Zerofill im
Interferogramm (Anf\"{u}gen von Nullen im Interferogramm, was einer
Interpolation entspricht) bringt dann standardm\"{a}{\ss}ig einen
spektralen Punktabstand von 0.0904 cm\up{-1}. Pro berechnetem
Spektrum stehen somit insgesamt 65536 Datenpunkte im
Spektralbereich 0 - 5924 cm\up{-1} zur Verf\"{u}gung. F\"{u}r die
multivariate Auswertung der Spektren (Kap. \ref{multivariat}) ist
dieser Punktabstand ausreichend, f\"{u}r die univariate Auswertung
(Kap. \ref{univariat}) von schmalen Rotations--Schwingungsbanden
kleiner Molek\"{u}le ist allerdings eine zus\"{a}tzliche Interpolation
notwendig, die jedoch von schmalen spektralen Intervallen ausgehen
kann (siehe auch Abb. \ref{h2ouni.wmf}).\\

\bild{htb}{nlkorr2.wmf}{300}{370}{\bf A \it Laborspektrum
SF\down{6}, gemessen in einer White-Zelle (31 hPa SF\down{6} in
N\down{2}, Gesamtdruck 1010 hPa, 6.9 m Wegl\"{a}nge). \bf B \it
Kalibrationskurven von SF\down{6}--Laborspektren bei
unterschiedlichen optischen Wegl\"{a}ngen und unter Verwendung von
verschiedenen Korrekturma{\ss}nahmen zur Spektrenberechnung (ESA:
Eigenstrahlungsabzug, DNLK: Detektornichtlinearit\"{a}tskorrektur,
RSK: Responsivitykorrektur).}

Um entsprechende spektrale Signal/Rauschverh\"{a}ltnisse in
angemessenen Messzeiten im Minutenbereich zu erhalten, m\"{u}ssen als
Detektoren empfindliche gek\"{u}hlte Halbleiterdetektoren eingesetzt
werden. Die K\"{u}hlung erfolgt \"{u}ber einen Joule--Thompson--K\"{u}hler, um
die Verwendung von fl\"{u}ssigem Stickstoff zur Detektork\"{u}hlung bei
Feldmessungen von vornherein zu vermeiden. Ein Nachteil dieser
Photodetektoren ist, dass sie f\"{u}r h\"{o}here Strahlungsleistungen
nichtlineare elektrische Signale liefern. Diese Situation tritt
insbesondere f\"{u}r das zu messende Interferogrammmaximum auf, so
dass hierdurch photometrische Fehler nach der \"{u}blichen
Spektrenberechnung bedingt sind (siehe Abb. \ref{nlkorr1.wmf}).\\

F\"{u}r die bistatische K300--Spektrometeranordnung wurde im Rahmen
dieser Arbeit der Einfluss der Nichtlinearit\"{a}tskorrektur des
Detektorsignals und die \"{A}nderungen der Detektor--Empfindlichkeit,
die aufgrund der sehr unterschiedlichen, auf den Detektor
treffenden Strahlungsintensit\"{a}ten auftreten, untersucht. Die
Nichtlinearit\"{a}tskorrektur wird mittels Auswertung des gemessenen
Interferogramms vorgenommen, wobei die Annahme gemacht wird, dass
sich die Nichtlinearit\"{a}t des Detektorsignals durch ein Polynom 2.
Grades approximieren l\"{a}sst (\cite{keens90}, \cite{keens93}).\\

Die Auswirkungen des nichtlinearen Detektorverhaltens sind in
Abbildung \ref{nlkorr1.wmf} anhand des positiven Signals unterhalb
von ungef\"{a}hr 550 cm\up{-1} deutlich zu sehen, obwohl unterhalb
dieser Wellenzahl aufgrund der halbleiterbedingten Bandl\"{u}cke kein
MCT--Detektorsignal messbar ist. In den Bildeinsch\"{u}ben sind die
Einkanalspektren nach Abzug der Eigenstrahlung gezeigt. Abbildung
\ref{nlkorr1.wmf} \bf B \rm zeigt die signifikante Verbesserung
nach der Detektor--Nichtlinearit\"{a}tskorrektur der gemessenen
Interferogrammsignale.\\

Bei bistatischen Anordnungen muss weiterhin die Eigenstrahlung
ber\"{u}cksichtigt werden, deren Spektrum aufgrund der geringeren
Intensit\"{a}ten im Vergleich zum Atmosph\"{a}renspektrum, welches mit dem
Globarstrahler aufgenommen wurde, kaum von der
Detektornichtlinearit\"{a}t beeinflusst wird. Auch nach der
Nichtlinearit\"{a}tskorrektur des Probeneinkanalspektrums erreichen
die Gebiete der vollst\"{a}ndigen Absorption bei hohen
Signalintensit\"{a}ten noch nicht die Intensit\"{a}t des zugeh\"{o}rigen
Eigenstrahlungsspektrums. Die Subtraktion beider Spektren ergibt
dennoch gering negative Werte in dem f\"{u}r die Eigenstrahlung
relevanten Spektralbereich. Dies ist ein Hinweis darauf, dass das
Polynom 2. Grades, welches zur Nichtlinearit\"{a}tskorrektur genutzt
wird, im Fall hoher Signalintensit\"{a}ten zur Modellierung
m\"{o}glicherweise nicht ganz ausreichend ist. Weiterhin k\"{o}nnen
leichte \"{A}nderungen in der Detektortemperatur (z.B. durch
unterschiedliche Strahlungsleistungen bedingt) seine
Empfindlichkeit geringf\"{u}gig \"{a}ndern. Die Multiplikation des
Probeneinkanalspektrums mit einem Faktor wenig gr\"{o}{\ss}er als eins
bringt eine bessere \"{U}bereinstimmung der Intensit\"{a}ten von Proben-
und zugeh\"{o}rigem Eigenstrahlungsspektrum (Empfindlichkeitskorrektur
des Detektors).\\

Die Auswirkungen der verschiedenen Korrekturm\"{o}glichkeiten auf die
photometrische Genauigkeit bei Auswertung des
SF\down{6}--Bandenmaximums in Abbildung \ref{nlkorr2.wmf} \bf A
\rm sind im Bild \bf B \rm beispielhaft gezeigt. Bei konstantem
Partialdruck von SF\down{6} in der White--Zelle wurde die optische
Pfadl\"{a}nge ge\"{a}ndert. Die gr\"{o}{\ss}ten photometrischen Fehler treten
nat\"{u}rlich bei Nichtber\"{u}cksichtigung der Eigenstrahlung auf
(unterste Kurve in Abb. \ref{nlkorr2.wmf} \bf B\rm ). Die
zweitunterste Kurve (Linie mit Punkten) zeigt die Auswertung des
betrachteten Bandenmaximums nach Abzug der Eigenstrahlung, aber
ohne Nichtlinearit\"{a}tskorrektur. Die oberste Kurve ber\"{u}cksichtigt
die Nichtlinearit\"{a}tskorrektur, die erhaltenen Intensit\"{a}tswerte f\"{u}r
die Spektralbereiche vollst\"{a}ndiger Absorption stimmen aber noch
nicht ganz mit den Intensit\"{a}ten der zugeh\"{o}rigen
Eigenstrahlungsspektren \"{u}berein. Die zweite Kurve von oben (nach
unten zeigende Dreiecke) ber\"{u}cksichtigt auch die f\"{u}r den Detektor
notwendige Empfindlichkeitskorrektur und stellt die genaueste
Vorbearbeitung der Spektren zur sp\"{a}teren quantitativen Analyse
dar.\\

Abbildung \ref{nlkorr2.wmf} zeigt sehr deutlich, dass, abh\"{a}ngig
vom Produkt aus Volumenanteil $\cdot$ Wegl\"{a}nge, bei unzureichender
Spektrenvorbearbeitung vor allem bei Banden mit hohen
Extinktionswerten gro{\ss}e systematische Fehler bei der quantitativen
Analyse auftreten k\"{o}nnen.\\

\cite{richardson198} stellt weitere Strategien zur Korrektur von
Detektornichtlinearit\"{a}ten vor, die sowohl durch
Hardwareerweiterung als auch durch Softwareimplementierung
erreicht werden k\"{o}nnen. Grunds\"{a}tzlich sind solche
Hardwareerweiterungen sehr aufwendig. Die von Richardson et al.
weiterhin vorgestellte Softwareimplementierung korrigiert die
Nichtlinearit\"{a}ten in den Extinktionsspektren mit Hilfe einer
Funktion ersten Grades. Die oben in dieser Arbeit vorgestellte
Nichtlinearit\"{a}tskorrektur hingegen setzt mit Hilfe der Software
bei der Ursache an und korrigiert das Interferogramm. Dies
erscheint sinnvoller und ist zudem universeller und problemloser
einsetzbar, weil die Nichtlinearit\"{a}tskorrektur der
Extinktionsspektren von der jeweilig benutzten
Apodisationsfunktion, den Halbwertsbreiten und maximalen
Extinktionen der betrachteten Banden abh\"{a}ngt.\\


\subsection{\label{wellenzahlstab}Stabilit\"{a}t der Wellenzahlskala}

Die Wellenzahlstabilit\"{a}t von Spektren, die \"{u}ber FT--Spektrometer
erhalten werden, ist auch \"{u}ber l\"{a}ngere Zeiten deutlich h\"{o}her als
wenn dispersive Spektrometer eingesetzt werden. Dies liegt darin
begr\"{u}ndet, dass die Frequenzskala des FT--Ger\"{a}tes \"{u}blicherweise an
die 633 nm-Wellenl\"{a}nge des He--Ne--Lasers gekn\"{u}pft ist, der die
Digitalisierungsschritte f\"{u}r jedes Interferogramm liefert. Die
Reproduzierbarkeit der Wellenzahlwerte wird \"{u}blicherweise mit etwa
0.01 cm\up{-1} angegeben, was f\"{u}r wiederholte Messungen mit einem
Ger\"{a}t vorteilhaft ist. F\"{u}r die \"{U}bertragbarkeit der Ergebnisse
zwischen verschiedenen Ger\"{a}ten m\"{u}ssen jedoch besondere
Anstrengungen unternommen werden, um systematische Fehler bedingt
durch Verschiebungen in der Wellenzahlskala bei der quantitativen
Auswertung zu vermeiden. Diese sind insbesondere bei aufgel\"{o}ster
Rotationsfeinstruktur und schmalen Absorptionslinien besonders
gegeben. So sind zwangsl\"{a}ufig schon bei geringen
Wellenzahlverschiebungen extrem gro{\ss}e spektrale Residuen zu
erwarten.\\

\tabelle{htb}{1.2}{wellenzahlgenauigkeit}{{|l c c c c|}\hline  &
\rule[-4mm]{0cm}{1cm}\underline{$\stackrel{\sim}{\nu}$
[cm\up{-1}]} & &
\underline{$\bf\Delta\!\stackrel{\sim}{\nu}/\stackrel{\sim}{\nu}
\cdot 10^{-5}$} &
\\ \raisebox{2.7ex}[-1.5ex]{\bf Komponente} & \bf HITRAN96 & \bf
Spektr.1 & \bf Spektr.2  & \bf Spektr.3 \\ \cline{1-5} \bf
H\down{2}O & 576.112 & 5.0 & 5.2 & 2.7\\ \bf H\down{2}O & 1596.236
& 4.6 & 5.6 & 2.3\\ \bf H\down{2}O & 1601.208 & 4.8 & 5.1 & 1.4\\
\bf CO\down{2} & 2382.466 & 4.9 & 4.8 & 1.5\\ \bf CO\down{2} &
2383.331 & 4.1 & 4.2 & 1.4\\ \bf H\down{2}O & 3969.082 & 4.8 & 5.2
& 2.2\\ \cline{1-5}}{Wellenzahlgenauigkeit verschiedener
K300--Spektrometer mit 0.2 cm\up{-1} Aufl\"{o}sung}

Um die Wellenzahlgenauigkeit von drei K300--Spektrometern
abzusch\"{a}tzen, wurde \"{u}ber den gesamten mittleren IR--Bereich die
exakte Wellenzahlposition von in Atmosph\"{a}renspektren immer
vorhandenen H\down{2}O-- und CO\down{2}--Banden bestimmt (Tab.
\ref{wellenzahlgenauigkeit}). Hierzu wurde die Abtastung der
ausgesuchten symmetrischen Linie durch Erg\"{a}nzung von Nullen in der
Fourierdom\"{a}ne um den Faktor 16 verbessert und anschlie{\ss}end der
Schwerpunkt der Linie bestimmt \cite{cameron82}. Als Referenzwerte
wurden die Linienpositionen der HITRAN96--Bibliothek
(\cite{rothman98}, Kap. \ref{hitran}) herangezogen, die als sehr
genau gelten.\\

Es zeigt sich, dass die Quotienten aus relativer
Wellenzahlverschiebung und der genauen Linienposition f\"{u}r jedes
Spektrometer ann\"{a}hernd konstant sind. Beim Vergleich der
Spektrometer untereinander zeigt sich allerdings, dass
Spektrometer 1 und 2  ungef\"{a}hr gleiche, wohingegen Spektrometer 3
um den Faktor 2 kleinere Abweichungen zu den exakten Werten hat.\\

Aus diesem Grunde wurden die Auswirkungen von unterschiedlich
gro{\ss}en Wellenzahlverschiebungen exemplarisch an verschiedenen
simulierten Spektren mit spektraler Aufl\"{o}sung von 0.2 cm\up{-1}
untersucht (Tab. \ref{relkonzerror}), um die Toleranzen bei den
Korrekturstrategien absch\"{a}tzen zu k\"{o}nnen.\\

\tabelle{htb}{1.2}{relkonzerror}{{|l c c c c c c c|}\hline  &
\rule[-4mm]{0cm}{1cm}{\underline{\bf Bereich}} & \underline{\bf V
$\cdot$ L\up{\star}} & & & \underline{\bf
$\Delta\!\stackrel{\sim}{\nu}$ [cm\up{.1}]} & & \\
\raisebox{2.7ex}[-1.5ex]{\bf Kompon.} & \bf [cm\up{-1}] & \bf [ppm
$\cdot$ m] & \bf 0.003 & \bf 0.01 & \bf 0.05 & \bf 0.09 \bf & \bf
0.18\\ \cline{1-8} \bf Hexan & 2900-2965 & 120 & -0.02 & -0.08 &
-0.35 & -0.60 & -1.18\\ \bf Toluol & 960-1080 & 180 & 0.03 & 0.10
& 0.88 & 2.17 & 6.73\\ \bf Ammoniak & 840-960 & 85 & 0.01 & 0.06 &
1.31 & 3.71 & 12.34\\ \bf Methan & 2900-2965 & 180 & 0.01 & 0.07 &
1.65 & 4.61 & 15.13\\ \cline{1-8}}{Relativer
Konzentrationsvorhersagefehler in \% aufgrund von
Wellenzahlverschiebungen. Die Auswertung erfolgte mittels CLS, die
Spektren wurden f\"{u}r eine nominale Aufl\"{o}sung von 0.2 cm\up{-1} und
Dreiecksapodisation berechnet (QASoft). (\up{\star} V$\cdot$ L --
Volumenanteil $\cdot$ Wegl\"{a}nge).}

Die gr\"{o}{\ss}te der in Tabelle \ref{wellenzahlgenauigkeit} auftretenden
Wellenzahlabweichungen liegt bei 0.18 cm\up{-1}. Da aber nicht die
Abweichung zu den exakten Werten, sondern die zu den verwendeten
Referenzspektren von Bedeutung ist, k\"{o}nnen auch sehr viel kleinere
Abweichungen auftreten. Um die Auswirkungen der
Wellenzahlabweichung abzusch\"{a}tzen, wurden aus der
QASoft--Referenzdatenbank die Spektren von Hexan und Toluol (als
Komponenten mit breiten Banden) und Ammoniak und Methan (als
Komponenten mit schmalen Banden) f\"{u}r eine nominale Aufl\"{o}sung von
0.2 cm\up{-1} und Dreiecksapodisation entnommen und mittels CLS
f\"{u}r die aufgef\"{u}hrten Wellenzahlverschiebungen ausgewertet. Man
sieht deutlich, dass die prozentualen Fehler gerade bei schmalen
Banden, f\"{u}r gro{\ss}e Wellenzahlverschiebungen nicht zu
vernachl\"{a}ssigen sind.\\

Aus diesem Grund muss eine Strategie gefunden werden, die diese
Wellenzahlverschiebungen zu korrigieren vermag. Da die relative
Wellenzahlabweichung
$\Delta\!\stackrel{\sim}{\nu}$/$\stackrel{\sim}{\nu}$ f\"{u}r ein
Spektrometer hinreichend konstant ist, hat es sich gezeigt, dass
es ausreichend ist, die auszuwertenden Wellenzahlbereiche in
Unterbereiche von 50 cm\up{-1} aufzuteilen. Die
Wellenzahlverschiebung in diesem Unterbereich berechnet sich dann
aus der Multiplikation der f\"{u}r jedes Spektrum mit Hilfe der o.g.
H\down{2}O-- und CO\down{2}--Linien zu bestimmenden relativen
Wellenzahlverschiebung und der mittleren Wellenzahl des
Unterbereichs. Die dadurch erhaltene Verschiebung kann dann in
diesen Unterbereichen linear ausgeglichen werden. Es ist dabei
darauf zu achten, dass die Spektrenpunkte durch Interpolation auf
die Originalachse gebracht werden m\"{u}ssen.\\


\subsection{\label{nichtlin}Lambert--Beer'sch\-es Gesetz und m\"{o}gliches
nichtlineares Extinktionsverhalten}

\markright{\sl NICHTLINEARES VERHALTEN BZGL. LAMBERT--BEER}

Gl. \ref{eqlbext} zeigt, dass im Lambert--Beer'schen Gesetz die
Extinktion linear abh\"{a}ngig von Konzentration und Schichtdicke ist.
Im folgenden werden Einfl\"{u}sse der Messmethode gezeigt, die zu
Abweichungen von dieser Linearit\"{a}t und damit zu systematischen
Fehlern bei der Auswertung f\"{u}hren k\"{o}nnen.\\

Die Auswirkungen der instrumentellen Linienfunktion, die die
spektrale Aufl\"{o}sung des FT--Spektrometers bestimmt, auf die
gemessenen Linien- und Bandenextinktionswerte ist bekannt (siehe
Kap. \ref{faltung}). Dies l\"{a}sst sich durch Faltung der wahren
Linienprofile mit den jeweiligen instrumentellen Linienfunktionen
in der Transmissionsdomaine realisieren. Je h\"{o}her die spektrale
Aufl\"{o}sung im Vergleich zur wahren Halbwertsbreite der vorliegenden
Absorptionsbanden ist, desto geringer fallen die Abweichungen von
der Linearit\"{a}t in den konzentrationsabh\"{a}ngigen maximalen
Linienextinktionen aus. Nun wird die Verwendung einer hohen
spektralen Aufl\"{o}sung mit einem geringeren spektralen
Signal/Rauschverh\"{a}ltnis erkauft, so dass f\"{u}r die
Atmosph\"{a}renspektroskopie Kompromisse eingegangen werden m\"{u}ssen.
Die bei Atmosph\"{a}rendruck vorliegenden Linienhalbwertsbreiten von
kleinen zwei- oder dreiatomigen Molek\"{u}len bewegen sich \"{u}berwiegend
zwischen 0.1 und 0.2 cm\up{-1}, so dass die beim K300 routinem\"{a}{\ss}ig
gew\"{a}hlte spektrale Aufl\"{o}sung von 0.2 cm\up{-1} f\"{u}r eine Vielzahl
von F\"{a}llen einen zufriedenstellenden Linearit\"{a}tsbereich liefert.
Hinzu kommt, dass speziell f\"{u}r den Bereich der atmosph\"{a}rischen
H\down{2}O-- und CO\down{2}-- Absorptionen bei Verwendung einer
solch hohen Aufl\"{o}sung die Querempfindlichkeiten der zu
analysierenden Stoffe stark abnehmen und auch zus\"{a}tzliche
Datenpunkte f\"{u}r die quantitative Auswertung zur Verf\"{u}gung stehen.
Mit zunehmender Verbesserung der spektralen Aufl\"{o}sung nimmt im
allgemeinen die Selektivit\"{a}t zu, wenn dem nicht eine entsprechende
Verschlechterung des Signal/Rauschverh\"{a}ltnisses entgegensteht. Auf
der anderen Seite l\"{a}sst sich durch eine Verringerung der
spektralen Aufl\"{o}sung der nutzbare dynamische Konzentrationsbereich
trotz einer nichtlinearen funktionellen Abh\"{a}ngigkeit vergr\"{o}{\ss}ern
\cite{jaakola97}. Im Verlauf der Arbeit musste in der Praxis von
dieser Option allerdings kein Gebrauch gemacht werden, da das
Signal/Rauschverh\"{a}ltnis bei einer spektralen Aufl\"{o}sung von 0.2
cm\up{-1} f\"{u}r alle Anwendungen ausreichend war.\\

Anzumerken ist, dass das verwendete K--300 Spektrometer keine
variable Eingangsaperturblende besitzt, da bereits bei der
h\"{o}chsten spektralen Aufl\"{o}sung die Detektorfl\"{a}che voll
ausgeleuchtet ist und somit keine weitere Vergr\"{o}{\ss}erung des
optischen Lichtleitwertes (Throughput) sinnvoll ist, die bei
schlechterer spektraler Aufl\"{o}sung sonst im allgemeinen erreichbar
ist (siehe auch \cite{jaakola97} und darin zitierte Literatur).
Dies hat zur Folge, dass das erreichbare Signal/Rauschverh\"{a}ltnis
sich nur linear mit einer Reduzierung der Aufl\"{o}sung verbessert.\\

Bei der FTIR--Spektroskopie sind eine Reihe von verschiedenen
instrumentellen Linienfunktionen m\"{o}glich, die sich durch die Wahl
einer geeigneten Gewichtsfunktion f\"{u}r das Interferogramm, der
sogenannten Apodisationsfunktion, ergeben (Kap. \ref{faltung}). Da
durch die Boxcar--Apodsation (keine Interferogrammgewichtung)
recht erhebliche Linienseitenmodulationen bestehen, verwendet man
\"{u}blicherweise andere Apodisationsfunktionen. Am weitesten
verbreitet ist die Dreiecksapodisation, die auch bei der
Erstellung einer umfangreichen kommerziell verf\"{u}gbaren
Spektrenbibliothek von der Fa. Infrared Analysis (QASoft)
ber\"{u}cksichtigt wurde. Daten dieser Spektralbibliothek wurden f\"{u}r
die Bereitstellung von Referenzspektren genutzt, so dass auch f\"{u}r
die weiteren Untersuchungen diese Apodisationsfunktion eingesetzt
wurde. In der Literatur gibt es Untersuchungen \"{u}ber die Verwendung
anderer Apodisationsfunktionen, die einen erweiterten
Linearit\"{a}tsbereich im Vergleich zu den auf einer
Dreiecksapodisationsfunktion basierenden Ergebnissen mit sich
bringen (siehe \cite{zhu298}).\\

Aufgrund unterschiedlicher Linienhalbwertsbreiten in den vorliegenden
Atmosph\"{a}renspektren existieren verschiedene Linearit\"{a}tsbereiche f\"{u}r
einzelne Linien, so dass eine allgemeine Grenze f\"{u}r die G\"{u}ltigkeit des
linearen Lambert--Beer'schen Gesetzes nicht gegeben ist. In der
Auswertungssoftware (Kap. \ref{expertensystem}) wurde als eine Strategie
ein flexibel vorgebbarer Extinktionsgrenzwert gew\"{a}hlt, bei denen
Extinktionen gr\"{o}{\ss}er als dieser Grenzwert bei der Auswertung ausgeblendet
werden. Analytisch sinnvoller ist es nat\"{u}rlich, auch diese Punkte f\"{u}r die
Auswertung nutzen zu k\"{o}nnen. In diesem Fall m\"{u}ssen die Nichtlinearit\"{a}ten
schon in den Referenzspektren bei der Auswertung mitber\"{u}cksichtigt werden
(siehe auch Kap. \ref{kalibriergas}). Dieser Weg wird in dieser Arbeit
bei der Aufnahme eigener Referenzspektren konsequent verfolgt werden.\\

\markright{\sl ABWEICHUNGEN VON STANDARDTEMPERATUR UND --DRUCK}

\section{\label{austempdr}Auswirkungen durch Abweichungen von
Standard\-temperatur und --druck}

\markright{\sl ABWEICHUNGEN VON STANDARDTEMPERATUR UND --DRUCK}

Zus\"{a}tzlich zu den spektrometerabh\"{a}ngigen Einfl\"{u}ssen, k\"{o}nnen auch
Abweichungen von Temperatur und Druck von den Standardbedingungen
(296 K, 1013 hPa), bei denen \"{u}blicherweise die Referenzdaten
vorliegen, bei Messungen in der offenen Atmosph\"{a}re zu
systematischen Fehlern bei der Auswertung f\"{u}hren. Dabei haben
sowohl Temperatur und Druck Auswirkungen auf die Halbwertsbreite
der Banden und auf das Bandenmaximum.\\

Die Bandenintensit\"{a}t h\"{a}ngt vornehmlich von der Besetzungszahl der
betrachteten \"{U}berg\"{a}nge ab. Die Besetzungszahl der angeregten
Rotations-- und Schwingungsniveaus in Molek\"{u}len folgt der
Boltzmann--Verteilung. Bei Raumtemperatur sind angeregte
Rotationsniveaus schon besetzt, die Schwingungs\"{u}berg\"{a}nge erfolgen
aber i.a. vom Grundzustand aus. Lediglich bei kleinen
Anregungsenergien ist ein signifikanter Anteil bereits bei
Raumtemperatur angeregt. Abh\"{a}ngig von der Temperatur T muss
aufgrund der \"{A}nderung dieser Besetzungszahlen die integrierte
Linienst\"{a}rke S einer Bande zu T=296 K korrigiert werden, wenn die
Referenzspektren unter Normalbedingungen gemessen wurden:

\begin{equation}\label{eqlinienstaerke}
S(T) = S(296) \cdot \frac{Q(296)}{Q(T)} \cdot \frac{{\rm exp}
\left(-\frac{c_2E_0}{T} \right)}{{\rm exp}
\left(-\frac{c_2E_0}{296} \right)} \cdot \frac{\left(1-{\rm
exp}\left(-\frac{c_2
\stackrel{\sim}{\nu}_0}{T}\right)\right)}{\left(1-{\rm
exp}\left(-\frac{c_2 \stackrel{\sim}{\nu}_0}{296}\right)\right)} ,
\end{equation}
dabei sind die Q--Funktionen die Zustandssummen, $c_2$ die zweite
Strahlungskonstante (=hc/k=1.439 cm K) und $E_0$ die Energie des
betrachteten \"{U}bergangs f\"{u}r $\stackrel{\sim}{\nu}_0$ (siehe auch
\cite{griffith96}).\\

Zur Linienverbreiterung tragen im wesentlichen zwei Punkte bei:
Der Grund f\"{u}r die \it Doppler--Verbreiterung \rm liegt in der
temperaturbedingten Geschwindigkeitsverteilung der Molek\"{u}le
bedingt. Im thermischen Gleichgewicht haben die Molek\"{u}le eines
Gases eine Geschwindigkeitsverteilung, die der Maxwell'schen
Statistik folgt. Mit der wahrscheinlichsten Geschwindigkeit $v_W$
(k--Boltzmannkonstante, T--Temperatur, m--Molek\"{u}lmasse)

\begin{equation}\label{eqdopl1}
  v_W=\sqrt{\frac{2kT}{m}}
\end{equation}
und $\gamma_G=\frac{\stackrel{\sim}{\nu}_0}{c}v_W$ der
Gauss--Halbwertsbreite bei halber H\"{o}he folgt f\"{u}r das Gauss'sche
Linienprofil $I_G$ bei einer Wellenzahl $\stackrel{\sim}{\nu}$:

\begin{equation}\label{eqdopl1}
  I_G(\stackrel{\sim}{\nu}) = \frac{1}{\gamma_G \sqrt{\pi}}
  \quad {\rm exp} \left(-\frac{(\stackrel{\sim}{\nu}-\stackrel{\sim}{\nu}_o)^2}{\gamma_G^2}\right)
\end{equation}

Die Ursache f\"{u}r die Druckverbreiterung (\it Lorentz--Verbreiterung\rm )
liegt in den St\"{o}{\ss}en des absorbierenden Molek\"{u}ls mit s\"{a}mtlichen anderen
begr\"{u}ndet, was zu einer Verschiebung der Energieniveaus der Molek\"{u}le
f\"{u}hrt. F\"{u}r das daraus resultierende Lorentzprofil gilt mit $\gamma_L$ der
Lorentz--Halbwertsbreite bei halber H\"{o}he:
\begin{equation}\label{eqlor}
  I_L(\stackrel{\sim}{\nu})=\frac{\gamma_L/2\pi}{(\stackrel{\sim}{\nu}-\stackrel{\sim}{\nu}_0)^2+(\gamma_L/2)^2}
\end{equation}

F\"{u}r Molek\"{u}le mittlerer Gr\"{o}{\ss}e bei Raumtemperatur und Normaldruck
liegen typische Werte f\"{u}r $\gamma_G$ bei 0.003 cm\up{-1} und
$\gamma_L$ bei 0.07 cm\up{-1}. Somit dominiert in der
Atmosph\"{a}renspektroskopie der Lorentzbeitrag zur Breite des
Gesamtprofils, welches eine Faltung aus Doppler-- und
Lorentzverbreiterung ist. Das resultierende Linienprofil wird \it
Voigt--Profil \rm genannt.\\

\tabelle{htb}{1.2}{tabtempdr}{{|l c c c|}\hline & &
\rule[-4mm]{0cm}{1cm}\underline{\bf $\Delta c_{rel}/\Delta P \cdot 10^3
$} & \underline{\bf $\Delta c_{rel}/\Delta T \cdot 10^3
$}\\\raisebox{2.7ex}[-1.5ex]{\bf Komponente} &
\raisebox{2.7ex}[-1.5ex]{\bf Aufl\"{o}sung} & \bf [ppm$\cdot$m/hPa] & \bf
[ppm$\cdot$m/�C] \\ \cline{1-4} \bf CH\down{3}CL & wahre Linienbreite &
0.79 & 0.51\\ \bf (652 ppm$\cdot$m) & 0.2 cm\up{-1} & 0.89 & 0.35\\ & 1.0
cm\up{-1} & 0.96 & 0.25\\ \bf C\down{2}H\down{6} & wahre Linienbreite &
0.53 & 1.12\\\bf (260 ppm$\cdot$m) & 0.2 cm\up{-1} & 0.70 & 0.75\\& 1.0
cm\up{-1} & 0.91 & 0.29 \\\bf CH\down{4} & wahre Linienbreite & 0.77 &
0.83\\\bf (128 ppm$\cdot$m) & 0.2 cm\up{-1} & 0.87 & 0.60\\& 1.0
cm\up{-1} & 0.97 & 0.50\\ \bf CO & wahre Linienbreite & 0.46 & 0.50\\\bf
(225 ppm$\cdot$m) & 0.2 cm\up{-1} & 0.76 & -0.36\\& 1.0 cm\up{-1} & 0.95
& -0.91 \\ \cline{1-4} \multicolumn{4}{l}{\small Die Simulationen wurden
mit der HITRAN96-Datenbank durchgef\"{u}hrt, wobei}\\
\multicolumn{4}{l}{\raisebox{1.5ex}[-1.5ex]{\small der Druckbereich von
960-1030 hPa und Temperaturen von 0-40�C ber\"{u}ck-}}\\
\multicolumn{4}{l}{\raisebox{3ex}[-3ex]{\small sichtigt wurden.
Ausgewertet wurde in den Bereichen 2900-3200 cm\up{-1}
(CH\down{3}CL),}}\\ \multicolumn{4}{l}{\raisebox{4.5ex}[-4.5ex]{\small
2950--3050 cm\up{-1} (C\down{2}H\down{6}), 2650--3250 cm\up{-1}
(CH\down{4}) und 2000--2300 cm\up{-1} (CO).}}
 }{Temperatur-- und Druckeffekte
bei der CLS--Auswertung von mit der HITRAN96 berechneten Spektren mit
unterschiedlichen Druck- und Temperaturbedingungen und Referenzspektren,
berechnet mit den Standardbedingungen (1013 hPa, 20�C).}

Der erfolgreiche Einsatz berechneter Referenzspektren mit den
Linienparametern aus der HITRAN--Datenbank wurde u.a. von
\cite{griffith96} und \cite{schaefer95} gezeigt. In diesem Zusammenhang
ist ein weiterer wichtiger Punkt f\"{u}r die quantitative Auswertung der
Einfluss von atmosph\"{a}rischen Druck-- und Temperaturschwankungen (siehe
Tab. \ref{tabtempdr}) verglichen mit den Standardbedingungen 1013 hPa und
20�C. Solche Effekte sind besonders signifikant f\"{u}r Substanzen mit
aufgel\"{o}ster Rotations--Vibrationsfeinstruktur zu erwarten. Exemplarisch
wurden die Spektren der Kohlenwasserstoffe Methan und Ethan, sowie der
chlorierte Kohlenwasserstoff Methylchlorid und Kohlenmonoxid untersucht.
Alle hier betrachteten synthetischen Spektren wurden mit der
HITRAN96--Datenbank berechnet.\\

F\"{u}r Tabelle \ref{tabtempdr} wurden Spektren im Druckbereich von
960-1030 hPa in 10 hPa--Schritten bei konstanter Temperatur von
20�C und im Temperaturbereich von 0-40�C in 10�C--Schritten bei
konstantem Druck von 1013 hPa f\"{u}r jedes Gas im in der
Tabellenunterschrift angegebenen Spektralbereich und f\"{u}r die in
der Tabelle angegebene Konzentration berechnet. Diese Spektren
wurden dann mit Referenzspektren, berechnet f\"{u}r die
Standardbedingungen 1013 hPa und 20�C, mittels CLS ausgewertet. In
Tabelle \ref{tabtempdr} sind die Steigungen der
Regressionsgeraden, die die relativen Abweichungen der ermittelten
Konzentrationen pro Kelvin und pro hPa von den wahren
Konzentrationen angeben, dargestellt.\\

Zur Abweichung von den wahren Konzentrationswerten tragen im
wesentlichen zwei Effekte bei. Zum einen \"{a}ndert sich aufgrund des
idealen Gasgesetzes die Dichte des betrachteten Gases. Dieser
Effekt liefert zu den Steigungen in Tabelle \ref{tabtempdr} einen
Beitrag von $\Delta c/(c \cdot \Delta P)=0.99\cdot10^{-3}$ f\"{u}r die
Druck\"{a}nderungen und von $\Delta c/(c \cdot \Delta
T)=-0.34\cdot10^{-2}$ f\"{u}r die Temperatur\"{a}nderungen. Desweiteren
findet aufgrund der Temperatur\"{a}nderungen eine molek\"{u}labh\"{a}ngige
Besetzungszahl\"{a}nderung der angeregten Rotationszust\"{a}nde im
Schwingungsgrundzustand statt. Dies bedeutet, dass die
Rotationslinien mit h\"{o}heren J--Quantenzahlen st\"{a}rker besetzt
werden. F\"{u}r Druck\"{a}nderungen bei konstanter Temperatur sind nur die
Linienverbreiterungen zu ber\"{u}cksichtigen.\\

Ein dritter Effekt tritt auf, wenn mit endlicher spektraler
Aufl\"{o}sung gemessen wird. Bei gegebenen Zustandsbedingungen f\"{u}r das
Gas liegt eine Verteilung von Linienhalbwertsbreiten vor. Bei
Temperatur\"{a}nderungen gibt es Verschiebungen in der von den
Rotationsquantenzahlen abh\"{a}ngigen Intensit\"{a}tsverteilung der
Linien. Druck\"{a}nderungen haben rotationsquantenzahlabh\"{a}ngige
Linienhalbwertsbreiten\"{a}nderungen zur Folge. Bei der Faltung ist
das Verh\"{a}ltnis aus wahren Linienhalbwertsbreiten und der
Halbwertsbreite der instrumentellen Linienfunktion f\"{u}r die
maximale Linienintensit\"{a}t von entscheidender Bedeutung. Durch die
Least--Squares Auswertung ergibt sich f\"{u}r die Druckabh\"{a}ngigkeit
eine Zunahme und f\"{u}r die Temperaturabh\"{a}ngigkeit eine Abnahme der
Steigungen.\\

Beim CO sind die temperatur-- und druckabh\"{a}ngigen Effekte sehr
gro{\ss}, da hier besonders gro{\ss}e Unterschiede in den
Linienhalbwertsbreiten vorliegen. Rotationslinien mit gro{\ss}em J
besitzen im Vergleich zu denen mit niedrigem J, die bei tieferen
Temperaturen st\"{a}rker besetzt sind, deutlich geringere
Linienhalbwertsbreiten \cite{hartmann88}, so dass die
Extinktionsmaxima solcher Linien nach der Faltung st\"{a}rker abnehmen
als die \"{u}brigen Linien nahe der Schwingungfrequenz. Insgesamt
f\"{u}hrt dieser Effekt zu den negativen Werten in Tabelle
\ref{tabtempdr}.\\

\bild{htb}{cotemp.wmf}{300}{370}{\bf A \it CO--Atmosph\"{a}renspektrum
mit der HITRAN96 berechnet, Datenpunktabstand 0.0904 cm\up{-1},
T=30�C,  Produkt aus Volumenanteil und Wegl\"{a}nge = 57 ppm m (\bf
a\it), und Differenzen zu einem Spektrum mit 20�C (\bf b\it) und
10�C (\bf c\it). Die Spektren \bf a \it und \bf b \it sind mit
einem Offset dargestellt. \bf B \it Relative Abweichung in Prozent
zwischen dem CO--Spektrum bei 30�C und denen bei 20�C und 10�C
(Auswertung des Linienmaximums).}

Bei der Auswahl der Linien f\"{u}r die univariate Auswertung ist vor
allem hinsichtlich der Temperatureffekte Sorge zu tragen, wenn
nicht explizit die Temperaturabh\"{a}ngigkeit ber\"{u}cksichtigt werden
kann. Abbildung \ref{cotemp.wmf} zeigt, wie gro{\ss} die Abweichungen
f\"{u}r die CO--Auswertung sein k\"{o}nnen, wenn nicht geeignete Linien
f\"{u}r die Auswertung genommen werden. Angemerkt sei, dass deutlich
zu sehen ist, dass die Abtastung $\Delta\stackrel{\sim}{\nu}$ in
diesem Fall nicht ausreicht, um die Einh\"{u}llende der Linien exakt
zu beschreiben.\\



\section{\label{hintergrund}Hintergrundproblematik}

Da die mit einem FTIR--Spektrometer gemessenen Einkanalspektren
auch die spektralen Eigenschaften des Detektors und der optischen
Komponenten des Aufbaus beinhalten, m\"{u}ssen diese Einkanalspektren
f\"{u}r eine quantitative Analyse mittels multivariater Verfahren
durch ein weiteres Einkanalspektrum, aufgenommen unter gleichen
Bedingungen, aber ohne die zu quantifizierenden Komponenten,
dividiert werden, um Transmissionsspektren zu erhalten (siehe auch
Kap. \ref{ft}). Dieses weitere Einkanalspektrum wird
Hintergrundspektrum genannt. Eines der gr\"{o}{\ss}ten Probleme bei der
Auswertung von FTIR--Spektren liegt in der Messung oder
Generierung eines geeigneten Hintergrundspektrums.\\

In der Literatur wurden hierzu in den letzten Jahren eine Vielzahl
von Vorschl\"{a}gen gemacht. Eine \"{U}bersicht findet sich bei
\cite{giese97}, weitere Ans\"{a}tze z.B. bei \cite{espinoza98} und
\cite{phillips96}. Grunds\"{a}tzlich kann man zwischen der
experimentellen Messung und der synthetischen Berechnung eines
Hintergrundspektrums unterscheiden.\\


\subsection{\label{hintergrundexp}Experimentelle Methoden}

Die in der Praxis wohl immer noch g\"{a}ngigste Methode zur Bestimmung
eines experimentellen Hintergrundes in der
Atmosph\"{a}renspektroskopie ist die Messung eines Spektrums unter
gleichen meteorologischen und optischen Bedingungen wie sie f\"{u}r
das zu analysierende Gasspektrum vorliegen, aber bei Abwesenheit
der zu messenden Komponenten. Diese Hintergrundmessung sollte
gleich im Anschluss an die Probenmessung erfolgen und wird
\"{u}blicherweise entweder im  "`Upwind"'  -- oder im "`Sidewind"'
--Verfahren erm\"{o}glicht. Bei einer z.B. angenommenen
Nord--S\"{u}d--Windrichtung wird im Upwind--Verfahren eine Messstrecke
gleicher L\"{a}nge n\"{o}rdlich zu der zu messenden Gaswolke aufgebaut.
Beim Sidewind--Verfahren wird westlich oder \"{o}stlich der Gaswolke
gemessen. Bei beiden Verfahren ist sorgf\"{a}ltig darauf zu achten,
dass keine Anteile der Gaswolke in der Messstrecke liegen. Dieses
Verfahren hat den Vorteil, dass die gemessenen H\down{2}O-- und
CO\down{2}--Konzentrationen etwa denen im zu analysierenden
Probengasspektrum entsprechen und Wellenzahlverschiebungen
gew\"{o}hnlich nicht auftreten. Dadurch k\"{o}nnen H\down{2}O und
CO\down{2} sehr gut kompensiert werden. Weiterhin muss kein
zus\"{a}tzlicher apparativer oder rechnerischer Aufwand getrieben
werden, um das Extinktionsspektrum zu berechnen.\\

Ein sehr entscheidender Nachteil ist jedoch, dass es in der Praxis
h\"{a}ufig sehr aufwendig ist, die Apparatur zu verlegen, wobei Globar
und Spektrometer noch einmal aufeinander justiert werden m\"{u}ssen.
Diese Unterschiede in der Justage k\"{o}nnen zu Basislinieneffekten im
Extinktionsspektrum f\"{u}hren. Diese Nachteile bringen es mit sich,
dass gew\"{o}hnlich im Verlauf einer Messreihe nur sehr wenige
Hintergrundspektren aufgenommen werden und somit meteorologische
und optische Ver\"{a}nderungen im Verlauf einer Messreihe nicht
ber\"{u}cksichtigt werden. Desweiteren ist es aufgrund der \"{o}rtlichen
und meteorologischen Gegebenheiten oft nicht m\"{o}glich, eine
geeignete Wegstrecke f\"{u}r die Hintergrundmessung zu finden.\\

Einen Ausweg hierf\"{u}r bietet die Messung eines "`Kurzweg"'
--Hintergrundes. Dazu wird die IR--Strahlungsquelle direkt vor das
Spektrometer gebracht. Um die S\"{a}ttigung des Detektors zu
vermeiden, muss die Strahlungsintensit\"{a}t abgeschw\"{a}cht werden. Da
eine Reduzierung der Globartemperatur eine \"{A}nderung der
Strahlungscharakteristik bedeutet, die im Nachhinein rechnerisch
korrigiert werden muss, ist der Einsatz z.B. eines Gitternetzes
zur Strahlungsabschw\"{a}chung die vorzuziehende Alternative. Der
Vorteil des Upwind--/Sidewind--Verfahrens zur optimalen
H\down{2}O-- und CO\down{2}--Kompensation ist in diesem Fall nicht
mehr gegeben, weiterhin muss aber noch der sehr aufwendige Umbau
stattfinden. Da das K300--Spektrometer einen zus\"{a}tzlichen zweiten
optischen Port besitzt, ist alternativ auch die Messung eines
Kurzweghintergrundspektrums mit Hilfe eines an das Spektrometer
befestigten Globarstrahlers m\"{o}glich. Weiterhin ist auch eine
Anordnung mit einem zweiten zus\"{a}tzlichen Globarstrahler und einer
Umlenkvorrichtung direkt vor dem Spektrometer m\"{o}glich. In diesem
Fall entf\"{a}llt der aufwendige Umbau.\\


\subsection{\label{hintergrundber}Methoden zur Berechnung eines
Hintergrundspektrums}

Die Nachteile bei der experimentellen Aufnahme von
Hintergrundspektren lassen Methoden in den Vordergrund treten, die
deren Nachteile nicht aufweisen. Es gibt mehrere einfache Methoden
zur Hintergrundberechnung. So l\"{a}sst sich f\"{u}r einen betrachteten,
meist recht schmalen Wellenzahlbereich, ein Polynom niedriger
Ordnung anpassen, mit dem die Basislinie i.a. gut repr\"{a}sentiert
wird. F\"{u}r schmale Banden ist es auch m\"{o}glich, die Bereiche von
Peaks auszublenden und die Punkte in den Flanken der Bande durch
eine Gerade zu verbinden. Eine weitere M\"{o}glichkeit zur iterativen
Auswertung einer Messreihe besteht darin, ein in einer anderen
Messreihe gewonnenes Hintergrundspektrum heranzuziehen, um die
Messreihe auszuwerten. Im allgemeinen wird hier eine noch stark
variierende Basislinie resultieren. Genommen wird nun das
Einkanalspektrum mit der kleinsten Konzentration der zu
analysierenden Komponente. Dieses Einkanalspektrum wird nun in
einem zweiten Schritt als Hintergrundspektrum f\"{u}r die \"{u}brigen
Spektren genommen, wobei die Konzentration der zu analysierenden
Komponenten im Nachhinein um den Wert, der bei dem als
Hintergrundspektrum genommenen Einkanalspektrum abgesch\"{a}tzt wurde,
zu korrigieren ist.\\

Ein weiteres iteratives Verfahren startet ebenfalls mit einem
entweder im Labor oder in der offenen Atmosph\"{a}re bereits
anderweitig gewonnenen Hintergrundspektrum. Es wird das
Extinktionsspektrum berechnet und eine der zu analysierenden
Komponenten wird aus diesem mittels skaliertem Bibliotheksspektrum
spektral subtrahiert. Anschlie{\ss}end wird aus dem resultierenden
Extinktionspektrum wieder ein Einkanalspektrum berechnet. Dieses
Einkanalspektrum wird in einer zweiten Iteration als neues
Hintergrundspektrum genutzt. Mit jeder Iteration wird i.a. das
Hintergrundspektrum optimiert. Probleme gibt es dabei aber vor
allem bei Komponenten mit geringen Konzentrationen.\\

Es existieren weiterhin noch aufwendigere Methoden zur Umgehung
der Hintergrundproblematik. So kann man anstatt des
Extinktionsspektrums, gebildet mit einem anderweitig erhaltenen
Hintergrund, auch die n--te Ableitung dieses Spektrums auswerten.
Lineare Anteile der Basislinie treten z.B. im Fall n=1 nur noch
als Offset auf. Desweiteren k\"{o}nnen \"{u}berlappende Peaks besser
unterschieden werden. Je h\"{o}her die Ableitung desto besser die
Unterscheidung \"{u}berlappender Peaks. Auf der anderen Seite geht mit
jeder Erh\"{o}hung der Ordnung der Ableitung etwa eine Verdopplung des
Rauschens einher, was Ordnungen gr\"{o}{\ss}er als n=2 oft als nicht
geeignet f\"{u}r die atmosp\"{a}rische Infrarotspektroskopie erscheinen
lassen.\\

Um dieses Rauschproblem zu umgehen, wurde die sogenannte
Shift--Methode vorgeschlagen \cite{xiao93}. Als
Hintergrundeinkanalspektrum wird in diesem Fall das
wellenzahlverschobene, zu analysierende Einkanalspektrum genommen.
Je breiter die zu analysierende Bande, desto gr\"{o}{\ss}er muss auch die
benutzte Verschiebung sein. Vergleiche mit den oben vorgestellten
Methoden zeigen leicht bessere Ergebnisse. Eine erfolgreiche
Anwendung dieser Methode erfordert jedoch auch eine gro{\ss}e
Erfahrung des Spektroskopikers mit diesem Verfahren
\cite{giese97}.\\

Espinoza et al. \cite{espinoza98} schlagen ein anderes Verfahren
zur Hintergrundberechnung vor. Als Grundlage wird das
Interferogramm des zu analysierenden Gemischspektrums genommen.
Dies wird um etwa den Faktor 100 gek\"{u}rzt und mit einer
Gewichtsfunktion (z.B. der Gaussfunktion) multipliziert
(Apodisierung). Das nach der FFT resultierende Einkanalspektrum
(die schmalen Strukturen sind nun nicht mehr aufgel\"{o}st) wird dann
als Hintergrundspektrum genommen. Auch dieses Verfahren erfordert
bei der Auswahl der genauen Punkteanzahl im Interferogramm und der
Gewichtsfunktion, die von der konkreten Messsituation abh\"{a}ngen,
eine gro{\ss}e Erfahrung und ist nur bei Banden mit Halbwertsbreiten
kleiner 2 cm\up{-1} erfolgreich.\\

Phillips et al. \cite{phillips96} verfolgen einen anderen Ansatz,
dessen Voraussetzung ebenfalls ein Frequenzabstand in den Signalen
der Basislinie und des Komponentenspektrums ist. Sie gehen davon
aus, dass ein Einkanalspektrum die Summe dreier Komponenten ist:
eine sich relativ wenig \"{a}ndernde Basislinie, eine Reihe schmaler
Absorptionssignale und Rauschen. In einem ersten Iterationsschritt
wird das erwartete Komponentenspektrum durch Addition
verschiedener Referenzspektren erzeugt und mit einer die
Spektrometerfunktion repr\"{a}sentierenden Funktion gefaltet. Das
gesch\"{a}tzte Hintergrundspektrum wird mit einer breiten
Gl\"{a}ttungsfunktion gefaltet und beide Spektren voneinander
abgezogen. Das erhaltene Residuum gibt Aufschluss dar\"{u}ber, wie gut
die Sch\"{a}tzung des Komponentenspektrums war. Die Iteration wird nun
solange wiederholt, bis das Residuum nur noch Rauschen zeigt. Da
die L\"{o}sung nicht eindeutig ist, wird eine zweite Bedingung, die
Entropie, eingef\"{u}hrt, welche in diesem Fall maximal sein muss.
Dieses Verfahren ist sehr rechenintensiv und erfordert schon im
vorhinein m\"{o}glichst viele Informationen \"{u}ber Zusammensetzung und
Konzentrationen der Probe.\\

Die bisher genannten Verfahren zur Hintergrunderzeugung sind f\"{u}r
die Integration in ein Expertensystem wenig geeignet, da sie
schlecht automatisierbar sind und aufgrund ihrer Komplexit\"{a}t und
ihrer starken Abh\"{a}ngigkeit von der jeweiligen Messsituation ein
hohes Ma{\ss} an individueller Erfahrung erfordern. Zudem sind sie
meist sehr rechenzeitintensiv. Die multivariate CLS-Auswertung, zu
der das Hintergrundspektrum ben\"{o}tigt wird, erfolgt in 6 Segmenten
(siehe auch Kap. \ref{bereichsauswahl}). Gesucht wird ein
Verfahren, bei dem automatisch der Hintergrund in diesen 6
Segmenten angepasst wird. Es w\"{a}re w\"{u}nschenswert, wenn dies ohne
Einschr\"{a}nkung f\"{u}r alle Spektrometer einer Baureihe zu realisieren
w\"{a}re.\\

\bild{htb}{ballsw.wmf}{300}{370}{\it verschiedene
Einkanalcharakteristiken, \bf A \it eines Atmosph\"{a}renspektrums,
\bf B \it eines N\down{2}--Laborspektrums, \bf C \it bearbeitetes
Spektrum aus \bf B\it . Die Intensit\"{a}t ist in willk\"{u}rlichen
Einheiten angegeben.}

\bild{htb}{spektrsw.wmf}{380}{456}{\it Einkanalcharakteristiken
verschiedener K300--Spektrometer. Die Intensit\"{a}t ist in
willk\"{u}rlichen Einheiten angegeben.}

Ein erfolgversprechender Weg besteht darin, ein Laborspektrum
einer sehr gut mit Stickstoff gesp\"{u}lten K\"{u}vette aufzunehmen und zu
\"{u}berpr\"{u}fen, welche Basislinieneffekte in den zu betrachtenden
Segmenten auftreten. Abbildung \ref{ballsw.wmf} \bf A \rm zeigt
die Einkanalcharakteristik eines Atmosph\"{a}renspektrums, welches zur
Auswertung ansteht. \bf B \rm ist ein mit demselben Spektrometer
aufgenommenes Laborspektrum, welches jedoch in einer sehr gut mit
N\down{2} gesp\"{u}lten Multireflexionsk\"{u}vette aufgenommen wurde. Ein
solches Spektrum weist nur noch sehr geringe H\down{2}O- und
CO\down{2}-Banden auf. Diese H\down{2}O-- und CO\down{2}--Banden
wurden in \bf C \rm gegl\"{a}ttet, so dass damit ein optimales
Laborspektrum f\"{u}r die weitere Auswertung zur Verf\"{u}gung steht.\\

Abbildung \ref{spektrsw.wmf} zeigt die
Einkanalspektrencharakteristik von sechs verschiedenen
K300--Spektrometern. Man sieht, dass vor allem im wichtigen
Fingerprint--Bereich (1000--1500 cm\up{-1}) schon deutliche
Unterschiede in den Signaturen, haupts\"{a}chlich durch
MCT\---De\-tek\-tor\-en unterschiedlicher Chargen bedingt, zu
erkennen sind. Negativ machen sich hier vor allem die gegen\"{u}ber
dem KT--K300--1 und LUA--K300--1 bei den anderen Spektrometern
beobachteten gro{\ss}en Intensit\"{a}tseinbr\"{u}che bei 1200 cm\up{-1}
bemerkbar. Daraus resultiert ein erheblich schlechteres
Signal/Rauschverh\"{a}ltnis in diesem wichtigen Bereich. Die
unterschiedlichen Charakteristiken zeigen aber auch, dass die
Strategie, ein ideales Laborspektrum als Hintergrundspektrum zu
nehmen, nur f\"{u}r jedes Spektrometer individuell funktionieren
kann.\\

Da nicht f\"{u}r alle Spektrometer die M\"{o}glichkeit besteht, ein
optimales Laborspektrum mittels Langwegzelle aufzunehmen, wurde
eine zweite Strategie zur Berechnung eines synthetischen
Hintergrundes entwickelt. Hierbei wird die erste Ableitung des
Atmosph\"{a}ren--Einkanalspektrums und des optimalen
Laborhintergrundspektrums gebildet. An Stellen mit einer
Abweichung der beiden Werte um mehr als 10\%  sind Banden im
Spektrum anzunehmen. Diese Punkte werden eliminiert und durch die
\"{u}brigen eine Splinefunktion approximiert, wodurch ein
synthetischer Hintergrund bereitgestellt werden kann. Um diese
Strategie auf andere Spektrometer des gleichen Typs anwenden zu
k\"{o}nnen, ist ein Laborspektrum eines Spektrometers einer Baureihe
ausreichend. Je schmaler die Absorptionsbanden, desto exakter ist
diese Hintergrundanpassung. Sie ist aber nicht, wie bei anderen
Verfahren, auf Bandenhalbwertsbreiten kleiner 2 cm\up{-1}
beschr\"{a}nkt und es ist damit auch eine Anpassung sehr
strukturierter Basislinien m\"{o}glich (siehe z.B. Segment 2 in Abb.
\ref{backse21.wmf}). Die Auswertung mittels CLS erfolgt in
Wellenzahlsegmenten. Darauf wird n\"{a}her in Kapitel
\ref{bereichsauswahl} eingegangen.\\

\bild{htb}{backse21.wmf}{330}{445}{\bf A \it
Atmosph\"{a}ren--Einkanalspektren im Segment 2. Spektrum \bf a \it
zeigt das gemessene Feldspektrum und das zugeh\"{o}rige berechnete
synthetische Hintergrundspektrum (mit Offset 2), \bf b \it das
ideale Labor-- und \bf c \it das gemessene
Feld--Hintergrundspektrum. In Teil \bf B \it sind die dazu
geh\"{o}rigen Extinktionsspektren dargestellt. Hintergrundspektren bei
der Berechnung der Extinktionsspektren waren in \bf a \it  das
synthetische, in \bf b \it das ideale Labor-- und in \bf c \it das
gemessene Feld--Hintergrundspektrum. Die Intensit\"{a}ten in \bf A \it
sind in willk\"{u}rlichen Einheiten angegeben, die Spektren in \bf B b
\it und \bf c \it sind mit einem Offset dargestellt. }

\bild{htb}{backse22.wmf}{340}{227}{\it Resultierende Basislinien
eines Atmosph\"{a}ren--Extinktionsspektrums unter Verwendung
verschiedener Hintergrundspektren: \bf a \it gemessenes
Hintergrundspektrum und \bf b \it ideales
Laborhintergrundspektrum. In \bf c \it und \bf d \it werden die
Atmosp\"{a}renspektren zweier Spektrometer gleicher Bauart betrachtet.
\bf c \it zeigt die resultierende Basislinie f\"{u}r zwei zeitlich und
\"{o}rtlich parallel gemessene Feldspektren zweier baugleicher
Spektrometer, \bf d \it die resultierende Basislinie f\"{u}r ein
Feldspektrum von Spektrometer 2 mit dem idealen
Labor--Hintergrundspektrum von Spektrometer 1. Die Spektren \bf
b\it , \bf c \it und \bf d \it sind mit einem Offset dargestellt.}

In dem Segment 2 treten f\"{u}r CO\down{2} und im Segment 3 f\"{u}r
N\down{2}O sehr breite Banden auf, f\"{u}r die durch den Algorithmus
kein ad\"{a}quates Hintergrundspektrum angepasst wird. Daher muss f\"{u}r
die Auswertung dieser Gase, die in allen Atmosph\"{a}renspektren
enthalten sind, f\"{u}r diese Segmente bei Verwendung des Algorithmus
weiterhin die Information \"{u}ber die Lage dieser breiten Bande
implementiert werden, um in diesem Bereich eine lineare Basislinie
anzupassen, was in beiden Segmenten ausreichend ist.\\

Abbildung \ref{backse21.wmf} zeigt die Hintergrundproblematik im
Segment 2 anhand eines auf einer M\"{u}lldeponie gemessenen
Atmosph\"{a}reneinkanalspektrums (\bf A a\rm). Aufgrund der \"{o}rtlichen
Gegebenheiten war es nicht m\"{o}glich, ein Hintergrundspektrum (\bf A
c\rm ) ohne die gesuchten Komponenten zu messen. Das ideale
Laborspektrum ist in \bf A b \rm dargestellt, das synthetische
Hintergrundspektrum direkt \"{u}ber dem gemessenen
Atmosph\"{a}renspektrum. In \bf B \rm sind die aus den verschiedenen
Hintergrundspektren resultierenden Extinktionsspektren
dargestellt. Das obere Spektrum \bf c \rm ist mit dem gemessenen
und das mittlere Spektrum \bf b \rm mit dem idealen
Labor-Hintergrund berechnet worden. Man sieht deutlich, dass im
oberen Spektrum nicht nur die Absorptionsbanden von H\down{2}O und
CO\down{2} kompensiert worden sind, sondern auch gesuchte
Zielkomponenten. Die Basislinie ist im Fall des gemessenen
Hintergrundspektrums akzeptabel, da bei der multivariaten
CLS-Auswertung grunds\"{a}tzlich eine Basislinie bis zum quadratischen
Term mitangepasst wird (siehe auch Kap. \ref{clstheorie}). Bei
Auswertung des Spektrums \bf b \rm hingegen ist die Basislinie
nicht vollst\"{a}ndig mit diesen Termen anpassbar.\\

Das Extinktionsspektrum, welches mit dem synthetischen Hintergrund
berechnet wurde, zeigt keinerlei Basislinieneffekte, und somit
k\"{o}nnen s\"{a}mtliche Komponenten mit ihren wahren Konzentrationen
analysiert werden.\\

Die resultierende Basislinie, die abh\"{a}ngig vom verwendeten
Hintergrundspektrum ist, ist f\"{u}r eine  quantitative Auswertung von
gro{\ss}er Bedeutung. Je besser die Struktur dieser Basislinie durch
einen Offset, einen linearen und einen quadratischen Term
angepasst werden kann, desto kleiner ist auch der systematische
Fehler bei der multivariaten Auswertung. In Abbildung
\ref{backse22.wmf} sind verschiedene Basislinien dargestellt, die
diese Problematik deutlich machen. Die Spektren \bf a \rm und \bf
b \rm zeigen noch einmal die Basislinien aus Abbildung
\ref{backse21.wmf} \bf B\rm , wobei beide Basislinien mit Spektren
desselben Spektrometers erhalten wurden. Spektrum \bf c \rm zeigt
die aufgrund unterschiedlicher Einkanalcharakteristiken
resultierende Basislinie aus Atmosph\"{a}renspektren von zwei
Spektrometern gleicher Bauart. Ein gemessenes
Atmosph\"{a}ren--Hintergrundspektrum mit Spektrometer 1 w\"{u}rde somit
bei der Auswertung eines Atmosph\"{a}renspektrums, gemessen mit
Spektrometer 2, zu einer Basislinie f\"{u}hren, die mit den oben
beschriebenen drei Termen nicht mehr anpassbar ist. Das gleiche
gilt auch f\"{u}r \bf d\rm , wo anstatt des gemessenen
Atmosph\"{a}ren--Hintergrundes, ein Labor--Hintergrundspektrum von
Spektrometer 1 genommen wird. Wird allerdings der mathematische
Algorithmus zur Berechnung eines Hintergrundspektrums verwendet,
so lassen sich auch bei Spektrometer 2 die guten Ergebnisse aus
Abb. \ref{backse21.wmf} \bf a \rm reproduzieren, auch wenn dann
das Laborspektrum von Spektrometer 1 als Referenz genommen wird.
Grunds\"{a}tzlich ist also festzuhalten, dass die in dieser Arbeit
vorgestellte Methode zur Berechnung von Hintergrundspektren f\"{u}r
Spektrometer der gleichen Bauart eingesetzt werden kann und
bez\"{u}glich der Gewinnung nahezu idealer Basislinien hervorragende
Ergebnisse liefert. Alle Komponenten k\"{o}nnen mit ihren wahren
Konzentrationen ausgewertet werden. Eventuell vorhandene
Querempfindlichkeiten zu Atmosph\"{a}renkomponenten werden allerdings
nicht von vornherein eliminiert. Dieses Verfahren ist allerdings
nur f\"{u}r schmale Absorptionsbanden anwendbar.\\

\bild{htb}{isoback.wmf}{365}{410}{\bf A a \it zeigt ein mit
Spektrometer 1 gemessenes Isobuten--Einkanalspektrum mit direkt
anschlie{\ss}end gemessenem Hintergrundspektrum. Das gestrichelte
Spektrum \bf c \it ist das ideale Laborhintergrundspektrum von
Spektrometer 1 (mit Offset 2). \bf b \it zeigt ein mit
Spektrometer 2 gemessenes Hintergrundspektrum. \bf B \it zeigt die
zugeh\"{o}rigen Extinktionsspektren. \bf a \it f\"{u}hrt zu einer
Volumenanteil$\cdot$ Wegl\"{a}nge--Vorhersage von 224.1 ppm$\cdot$m
(die Verwendung des gemessenen oder idealen Hintergrundes f\"{u}hrt zu
demgleichen Ergebnis), \bf b \it (Extinktionsspektrum aus \bf A a
\it und \bf b \it von 269.4 ppm$\cdot$m. Die Intensit\"{a}ten in \bf A
\it sind in willk\"{u}rlichen Einheiten, die Extinktion in \bf B b \it
ist mit einem Offset angegeben.}

Abbildung \ref{isoback.wmf} \bf A a \rm zeigt den Spektralbereich
mit der intensivsten Bande von Isobuten. Dar\"{u}ber ist das gleich im
Anschluss gemessene Hintergrundspektrum zu sehen. Es zeigt
deutlich, dass die Isobutenbande in einem Bereich von etwa 75
cm\up{-1} vorliegt und eine Berechnung des synthetischen
Hintergrundes mit dem oben vorgestellten Algorithmus nur den
maximalen Peak um 890 cm\up{-1} erkennen w\"{u}rde und somit f\"{u}r
diesen Fall nicht einsetzbar ist. Spektrum \bf c \rm zeigt ein
Laborspektrum desselben Spektrometers. Es ist in diesem kleinen
Wellenzahlbereich von 800--975 cm\up{-1} identisch mit dem gleich
im Anschluss an das Isobutenspektrum gemessene
Hintergrundspektrum. Beide sind somit f\"{u}r die Auswertung der Bande
geeignet. Spektrum \bf b \rm hingegen ist das
Atmosph\"{a}renhintergrundspektrum von Spektrometer 2 in diesem
Bereich. Es ist deutlich zu sehen, dass es f\"{u}r die Auswertung
dieser breiten Bande nicht geeignet ist. In \bf B \rm werden die
resultierenden Banden im Extinktionsspektrum gezeigt, wenn in \bf
a \rm das gemessene Hintergrundspektrum von Spektrometer 1 und in
\bf b \rm das gemessene Hintergrundspektrum von Spektrometer 2 f\"{u}r
die Auswertung genommen werden. Die Konzentrationsbestimmung \"{u}ber
die Extinktionsspektren f\"{u}hrt zu einem Unterschied von 20\%.\\

Bei breiten Banden ist es somit im Normalfall unerl\"{a}sslich,
Hintergrundspektren, aufgenommen mit demselben Spektrometer,
vorliegen zu haben, um richtige Ergebnisse erzielen zu k\"{o}nnen. In
diesem Fall h\"{a}tte auch die Anpassung einer Geraden zwischen 850
und 925 cm\up{-1} zu einem falschen Ergebnis gef\"{u}hrt.\\




\section{\label{crosscorr}Komponentenerkennung}

W\"{a}hrend in den vorhergehenden Kapiteln die Kenntnis der
Stoffzusammensetzung als bekannt vorausgesetzt wurde, verwendet
man Methoden der Mustererkennung ("`pattern recognition"') h\"{a}ufig
dann, wenn Vorstellungen \"{u}ber die qualitative Zusammensetzung
fehlen. Eine qualitative Analyse ist dann erforderlich, wenn
beispielsweise die Basis f\"{u}r die quantitative Spektrenauswertung
bestimmt werden muss, um verl\"{a}ssliche Ergebnisse \"{u}ber das
klassische Kalibrationsverfahren zu erhalten (Kap.
\ref{clstheorie}). F\"{u}r die Bewertung kommen Abstands-- und
\"{A}hnlichkeitsma{\ss}e in Frage, die z.B. \"{u}ber Trainingssets definiert
werden. Eine Anwendung f\"{u}r die Erkennung von fl\"{u}chtigen
organischen Stoffen mittels FT--IR Spektroskopie und
Diskriminanzanalyse ist bei \cite{kroutil94} beschrieben. Auch
l\"{a}sst sich eine Spektrenbibliothekssuche f\"{u}r die qualitative
Mischungsanalyse einsetzen. \cite{lowry85} et al. untersuchten
beispielsweise verschiedene Vergleichsalgorithmen auf der Basis
von Vollspektren.\\

F\"{u}r ein Expertensystem zur automatischen Analyse von
Atmosph\"{a}renspektren mittels CLS (siehe Kap. \ref{expertensystem})
ist es unerl\"{a}sslich, die qualitative Zusammensetzung des Gemisches
zu kennen. Das Verfahren der Wahl sollte neben der zuverl\"{a}ssigen
Erkennung s\"{a}mtlicher im Gemisch enthaltener Stoffe, auch
unproblematisch jederzeit f\"{u}r zus\"{a}tzliche Substanzen erweiterbar
sein. F\"{u}r die oben angesprochenen Verfahren ist dies nicht ohne
weiteres m\"{o}glich. Der alleinige Einsatz des Moduls zur univariaten
Auswertung einer f\"{u}r jeden Stoff charakteristischen Bande
erscheint ebenso als nicht sinnvoll, weil in der \"{u}berwiegenden
Mehrzahl der F\"{a}lle Querempfindlichkeiten mit anderen Stoffen
auftreten. Aufgrund der bei den meisten Substanzen m\"{o}glichen
Querempfindlichkeiten, ist eine sichere automatische qualitative
Information \"{u}ber die Zusammensetzung des Gemisches nicht zu
erhalten.\\

Die Kreuzkorrelation (siehe z.B. \cite{ehrentreich97}) als ein
Mustererkennungsverfahren ist von diesen Querempfindlichkeiten
weit weniger beeinflusst, desweiteren k\"{o}nnen Substanzen in
geringeren Konzentrationen als bei der univariaten Analyse noch
erkannt werden. Im Gegensatz zur univariaten Analyse erh\"{a}lt man
aber \"{u}ber die Kreuzkorrelation noch keine quantitativen Aussagen
zu den einzelnen Stoffen im Gemisch. Daf\"{u}r ist die Erweiterung des
Verfahrens auf andere Substanzen schnell und unproblematisch
m\"{o}glich.\\

F\"{u}r digitalisierte Signale gilt f\"{u}r die Kreuzkorrelationsfunktion
\bf ccf \rm zweier Spektren \bf a \rm und \bf b \rm mit N
Datenpunkten:
\begin{equation}\label{eqcrosscorr}
ccf_{ab}(i\,\Delta\!\stackrel{\sim}{\nu}) = \sum_{i=-(N-1)}^{N-1}a
(\stackrel{\sim}{\nu}) \;
b(\stackrel{\sim}{\nu}+i\,\Delta\!\stackrel{\sim}{\nu})
\end{equation}
Durch die Kreuzkorrelation erh\"{a}lt man Informationen \"{u}ber die
\"{A}hnlichkeit zweier Extinktionsspektren. Sie ist das \"{u}ber den
betrachteten Wellenzahlbereich integrierte Produkt der beiden
Spektren als Funktion ihrer Wellenzahlverschiebung
untereinander.\\

F\"{u}r die Mustererkennung in FT--IR Atmosph\"{a}renspektren wird f\"{u}r
jede in der Bibliothek vorliegende Referenzsubstanz ein
charakteristischer Wellenzahlbereich herausgesucht und mit dem
Atmosph\"{a}renspektrum in diesem Bereich verglichen. Der ausgesuchte
Bereich sollte die st\"{a}rksten Absorptionsbanden der
Referenzsubstanz enthalten und m\"{o}glichst wenig H\down{2}O-- und
CO\down{2}--Querempfindlichkeit besitzen. Ist die Referenzsubstanz
im Gemisch enthalten, so besitzt die Kreuzkorrelation f\"{u}r die
Wellenzahlverschiebung $\Delta\!\stackrel{\sim}{\nu}\,=\,0$ ein
Maximum. Beide Extinktionsspektren werden so aufeinander normiert,
dass das Maximum der Kreuzkorrelationsfunktion grunds\"{a}tzlich
kleiner eins ist \cite{ehrentreich97}.\\

F\"{u}r die Praxis sind drei Fragen von Bedeutung. Welche minimale
Konzentration kann bei einer in einem Gemisch vorliegenden
Substanz noch erkannt werden, welchen Einfluss haben noch weitere
Komponenten im betrachteten Wellenzahlbereich und wieweit h\"{a}ngen
die beiden vorherigen Fragestellungen auch von der Breite der
betrachteten Banden ab? Diese Fragen k\"{o}nnen jedoch nicht eindeutig
beantwortet werden, sondern h\"{a}ngen immer vom betrachteten Fall ab,
wobei sich aber anhand von Beispielen eindeutige Trends erkennen
lassen. Als Beispielsubstanzen f\"{u}r relevante Atmosph\"{a}renschadgase
wurden Methan, Ammoniak, o--Xylol und Toluol gew\"{a}hlt. Methan und
Ammoniak zeigen im ausgew\"{a}hlten Spektralbereich schmale
Signaturen, o--Xylol besitzt grunds\"{a}tzlich nur sehr breite Banden,
wohingegen Toluol im betrachteten Bereich eine Mischung aus beiden
aufweist. Um die minimalen Nachweisgrenzen f\"{u}r diese Substanzen zu
ermitteln, wurde ein schadgasfreies Atmosph\"{a}renspektrum
herangezogen, das Rauschen in den betrachteten Bereichen bestimmt,
das Maximum in den Referenzspektren innerhalb des betrachteten
spektralen Intervalls auf 0, 2, 3 und 10 $\sigma$ (zur
Standardabweichung $\sigma$ siehe Kap. \ref{clstheorie}) dieses
Rauschens normiert und zu dem Atmosph\"{a}renspektrum addiert. Danach
wurden die zugeh\"{o}rigen Kreuzkorrelationsfunktionen bestimmt.\\

\bild{htb}{ccmeth.wmf}{365}{400}{\it Referenzspektren von Methan
und typisches Rauschen eines Atmosph\"{a}renspektrums im betrachteten
Spektralbereich plus dem jeweiligen Referenzspektrum, dessen
Maximum auf 3 $\sigma$ des RMS--Rauschens normiert wurde (oben).
Das Rauschspektrum ist mit einem Offset dargestellt. Dazugeh\"{o}rige
Kreuzkorrelationsfunktionen (unten) f\"{u}r unterschiedliche
CH\down{4}--Konzentrationen.}

\bild{htb}{ccamm.wmf}{365}{400}{\it Referenzspektren von Ammoniak
und typisches Rauschen eines Atmosph\"{a}renspektrums im betrachteten
Spektralbereich plus dem jeweiligen Referenzspektrum, dessen
Maximum auf 3 $\sigma$ des RMS--Rauschens normiert wurde (oben).
Das Rauschspektrum ist mit einem Offset dargestellt. Dazugeh\"{o}rige
Kreuzkorrelationsfunktionen (unten) f\"{u}r unterschiedliche
NH\down{3}--Konzentrationen.}

\bild{htb}{ccoxyl.wmf}{365}{400}{\it Referenzspektren von o--Xylol
und typisches Rauschen eines Atmosph\"{a}renspektrums im betrachteten
Spektralbereich plus dem jeweiligen Referenzspektrum, dessen
Maximum auf 3 $\sigma$ des Rauschens normiert wurde (oben). Das
Rauschspektrum ist mit einem Offset dargestellt. Dazugeh\"{o}rige
Kreuzkorrelationsfunktionen (unten) f\"{u}r unterschiedliche
o--Xylol--Konzentrationen.}

\bild{htb}{cctol.wmf}{365}{400}{\it Referenzspektren von Toluol
und typisches Rauschen eines Atmosph\"{a}renspektrums im betrachteten
Spektralbereich plus dem jeweiligen Referenzspektrum, dessen
Maximum auf 3 $\sigma$ des Rauschens normiert wurde (oben). Das
Rauschspektrum ist mit einem Offset dargestellt. Dazugeh\"{o}rige
Kreuzkorrelationsfunktionen (unten) f\"{u}r unterschiedliche
Toluol--Konzentrationen.}

Kriterium f\"{u}r die Detektion eines Gases ist, dass das Maximum der
Kreuzkorrelationsfunktion der beiden Spektren untereinander bei
der Punktverschiebung null liegt. F\"{u}r schmalbandige Strukturen,
wie sie Methan und Ammoniak zeigen (siehe Abb. \ref{ccmeth.wmf}
und \ref{ccamm.wmf}), sind die Kreuzkorrelationsfunktionen recht
glatt und zeigen ein eindeutiges Maximum. Schon bei
Extinktionswerten der betrachteten Stoffe, die in der
Gr\"{o}{\ss}enordnung des 2$\sigma$--Rauschens liegen, wird der Stoff
durch den Algorithmus erkannt. Eine sichere Zuordnung ist auf
jeden Fall bei Extinktionswerten \"{u}ber 3$\sigma$ des Rauschens
m\"{o}glich. Bei der recht breiten o--Xylol--Bande (Abb.
\ref{ccoxyl.wmf}) hingegen ist die Kreuzkorrelation hinsichtlich
einer Maximumsausgestaltung schlechter strukturiert und eine
eindeutige Zuordnung des Maximums erst ab etwa 5$\sigma$ des
Rauschens m\"{o}glich. Das Extinktionsspektrum der st\"{a}rksten Banden
von Toluol zeigt eine Mischung aus schmalen und breiten Banden.
Auch hier ist, wie bei Stoffen mit ausschlie{\ss}lich schmalen
Strukturen, eine eindeutige Zuordnung des Maximums in der
Kreuzkorrelationsfunktion ab einer Konzentration des Toluols,
dessen gr\"{o}{\ss}te Bande etwa 3$\sigma$ des Rauschens entspricht,
m\"{o}glich.\\

Neben den minimalen Nachweisgrenzen ist das Verhalten der
Kreuzkorrelationsfunktionen bei vorliegenden Gemischen wichtig. In
Abbildung \ref{ccat.wmf} \bf A \rm ist ein Atmosph\"{a}renspektrum,
aufgenommen bei einer Wegl\"{a}nge von 135 m, dargestellt. In dem
gezeigten Spektralbereich sind Bandenstrukturen von H\down{2}O,
N\down{2}O und CO enthalten. In \bf C \rm werden die zugeh\"{o}rigen
Kreuzkorrelationsfunktionen von H\down{2}O und N\down{2}O gezeigt.
Beide Maxima liegen eindeutig bei der Punktverschiebung null und
beide Komponenten werden somit erkannt. Das Maximum der
Kreuzkorrelationsfunktion von CO (\bf B \rm ) hingegen liegt nicht
bei der Position null, obwohl CO (siehe auch das
CO--Referenzspektrum) einen signifikanten und deutlich sichtbaren
Beitrag zum Atmosph\"{a}renspektrum liefert. Nach dem skalierten Abzug
von H\down{2}O und N\down{2}O (\bf A \rm , Res 1), wird CO durch
die Kreuzkorrelationsfunktion eindeutig detektiert. In der
Spektren\"{u}bersicht (\bf A \rm , Res 2) ist vom Res 1 auch CO noch
einmal skaliert abgezogen worden. Die restlichen Strukturen liegen
in dem unzureichenden CO--Referenzspektrum begr\"{u}ndet (siehe hierzu
auch Kap. \ref{motivation}). Auch bei solchen Strukturen l\"{a}sst
sich durch die Kreuzkorrelationsfunktion noch ein \"{U}berrest von CO
erkennen (\bf E\rm ).\\

\bild{htb}{ccat.wmf}{365}{480}{\bf A \it Atmosph\"{a}renspektrum von
H\down{2}O, N\down{2}O und CO (mit Offset). Weiterhin sind das
CO--Referenzspektrum und die Residuen (mit Offset) nach skalierten
Abzug von H\down{2}O und N\down{2}O (Res 1) und zus\"{a}tzlich auch CO
(Res 2) dargestellt. In \bf B\it, \bf C\it, \bf D \it und \bf E
\it sind die dazugeh\"{o}rigen Kreuzkorrelationsfunktionen dargestellt
(Kommentar siehe Text, KK--Kreuzkorrelation).}

Dieses Beispiel zeigt, dass es Situationen geben kann, in denen
die Kreuzkorrelationsfunktion im ersten Schritt nicht s\"{a}mtliche
Komponenten, obwohl sie signifikant vorhanden sind, herausfindet.
Abgesehen davon, dass der Wassergehalt in diesem Beispiel um den
Faktor 2 bis 3 h\"{o}her liegt als in gew\"{o}hnlichen Atmosph\"{a}renspektren
bei 135 m Wegl\"{a}nge (bei "`normalem"' Wassergehalt w\"{a}ren alle 3
Komponenten eindeutig erkannt worden), ist die Auswahl des
Spektralbereichs f\"{u}r die Komponenten nicht optimal gew\"{a}hlt. Eine
Beschr\"{a}nkung des Spektralbereichs f\"{u}r CO auf den Bereich von
2164-2180.5 cm\up{-1} ergibt f\"{u}r CO sofort eine
Kreuzkorrelationsfunktion, die ihr Maximum eindeutig bei der
Position Null besitzt.\\

\markright{\sl MATHEMATISCHE AUSWERTEVERFAHREN}

Eine sorgf\"{a}ltige Auswahl des Spektralbereichs, in dem f\"{u}r einen
jeweiligen Referenzstoff die Kreuzkorrelation durchgef\"{u}hrt wird,
ist also von \"{a}u{\ss}erster Wichtigkeit. Es ist dabei darauf zu achten,
dass die bekannten Querempfindlichkeiten zu H\down{2}O und
CO\down{2} so weit wie m\"{o}glich vermieden werden. Desweiteren
sollte ein Spektralbereich mit den gr\"{o}{\ss}ten Absorptionsbanden und
mit m\"{o}glichst viel "`Strukturen"' in den Absorptionsbanden
herausgesucht werden. Unter diesen Voraussetzungen zeigen die
bisherigen Erfahrungen, dass die automatische Komponentenerkennung
mittels Kreuzkorrelation sehr gute Ergebnisse liefert. Das
Beispiel in Abbildung \ref{ccat.wmf} zeigt, dass f\"{u}r die
Ausnahmef\"{a}lle, in denen die Kreuzkorrelation nicht alle
Komponenten erkennt und bei denen in den nach der CLS--Auswertung
verbleibenden Residuen noch Strukturen verbleiben, in einem 2.
Iterationsschritt auch die restlichen Komponenten erkannt und
ausgewertet werden k\"{o}nnen, sofern sie in der
Referenzspektrenbibliothek enthalten sind. Die Implementierung
eines neuen Referenzstoffes f\"{u}r die Komponentenerkennung mittels
Kreuzkorrelation ist unproblematisch. Es muss f\"{u}r diesen Stoff nur
ein charakteristischer Spektralbereich gefunden werden, der o.g.
Voraussetzungen erf\"{u}llt.\\


\section{\label{auswerteverfahren}Mathematische Auswerteverfahren f\"{u}r die
FTIR--Fernsondierung}

\markright{\sl MATHEMATISCHE AUSWERTEVERFAHREN}

Nachdem in den vorhergehenden Abschnitten die
spektrometerabh\"{a}ngigen und atmosph\"{a}rischen Einfl\"{u}sse, die das
Messergebnis verf\"{a}lschen k\"{o}nnen, aufgezeigt und eliminiert worden
sind, stehen nun auch die Spektren und Informationen zur
Verf\"{u}gung, die f\"{u}r eine zuverl\"{a}ssige und pr\"{a}zise Auswertung
unerl\"{a}sslich sind. Hierzu geh\"{o}ren das detektornichtlinearit\"{a}ts--
und wellenzahlkorrigierte Atmosph\"{a}reneinkanalspektrum, das
zugeh\"{o}rige Eigenstrahlungsspektrum, das gemessene oder synthetisch
generierte Hintergrundeinkanalspektrum mit Eigenstrahlungsspektrum
und die z.B. mittels Kreuzkorrelation erhaltene Information \"{u}ber
die im Gemisch enthaltenen Komponenten.\\

Bei der mathematischen Auswertung von IR--Spektren unterscheidet
man zwischen univariaten und multivariaten Methoden. Bei der
univariaten Auswertung wird nur jeweils eine charakteristische
Bande einer Komponente analysiert, bei der multivariaten
Auswertung mehrere oder ein ganzes Spektrensegment in die
Auswertung miteinbezogen. Im folgenden werden die Vorgehensweisen,
die zu erhaltenden Nachweisgrenzen und die Vor-- und Nachteile der
einzelnen Methoden dargelegt.


\subsection{\label{univariat}Univariate Auswerteverfahren}

F\"{u}r bestimmte Komponenten liegen \"{u}berlagerungsfreie
Absorptionslinien vor. Dies betrifft insbesondere kleine Molek\"{u}le,
die eine aufgel\"{o}ste Rotationsfeinstruktur in den Spektren unter
den vorgegebenen atmosph\"{a}rischen Bedingungen aufweisen. F\"{u}r
ausgew\"{a}hlte Szenarien (H\down{2}O, CO\down{2}, N\down{2}O, CO,
CH\down{4}, NH\down{3}, C\down{2}H\down{6}, SF\down{6},
C\down{6}H\down{6}, C\down{2}H\down{4} und C\down{3}H\down{6}) ist
es daher m\"{o}glich, aus den logarithmierten Einkanalspektren
verl\"{a}ssliche quantitative Informationen \"{u}ber diese Stoffe zu
erhalten. Durch die Definition von Linienmaximum und Basislinie
lassen sich ohne gro{\ss}en rechnerischen Aufwand eine Absch\"{a}tzung der
vorliegenden Konzentrationen vornehmen. Dadurch dass im
allgemeinen beide Wellenzahlpunkte eng beieinander liegen, k\"{o}nnen
Basislinienprobleme von vornherein umgangen werden. Auch bei
gr\"{o}{\ss}eren Molek\"{u}len existieren oftmals schmale
Q--Zweig--Absorptionen, die f\"{u}r solch 'univariate' Auswertungen
herangezogen werden k\"{o}nnen. Eng beieinander liegende Minimum-- und
Maximumpunkte definieren dann die zu ber\"{u}cksichtigenden
Extinktionen. Eine geringf\"{u}gige Wellenzahlverschiebung ist
unproblematisch, da z.B. das Linienmaximum in der logarithmischen
Darstellung der Einkanalspektren innerhalb bestimmter Grenzen
einwandfrei bestimmbar und der Abstand zum in die Auswertung
miteinbezogenem Minimum fest vorgegeben ist.\\

Die hiermit erhaltenen Konzentrationssch\"{a}tzungen lassen sich f\"{u}r
die Aufstellung der Referenzspektrenmatrix nutzen, die bei dem
klassischen Anpassungsverfahren CLS (Kap. \ref{clstheorie})
notwendig ist und die m\"{o}glichst nur die vorhandenen Komponenten
enthalten sollte, die einen Beitrag zum auszuwertenden
Atmosph\"{a}renspektrum liefern. Die CLS--Anpassung wird in
Extinktionsbereichen mit nichtlinearen spektralen Strukturen umso
besser, je n\"{a}her die Konzentrationen der ausgew\"{a}hlten
Referenzspektren denen in der Mischung entsprechen. Da die
Kreuzkorrelation nur eine qualitative Aussage \"{u}ber die
Zusammensetzung der Stoffmatrix, aber keine quantitativen Werte
f\"{u}r die einzelnen Stoffe liefert, k\"{o}nnen die quantitativen
Zusatzinformationen der univariaten Analyse genutzt werden, um die
Referenzspektrenmatrix f\"{u}r die CLS--Analyse aufzustellen.\\

Bei der Auswahl der Linien ist hinsichtlich atmosph\"{a}rischer
Temperaturschwankungen Sorge zu tragen, wenn nicht explizit die
Temperaturabh\"{a}ngigkeit ber\"{u}cksichtigt werden kann. Dies ist
m\"{o}glich, wenn beispielsweise von der HITRAN96--Datenbank
ausgegangen wird, in der auch Temperaturabh\"{a}ngigkeiten
dokumentiert sind. Eine Faltung mit den erforderlichen
instrumentellen Linienfunktionen ist notwendig, um die
experimentellen Daten sinnvoll auswerten zu k\"{o}nnen (siehe Abb.
\ref{cotemp.wmf}).


\subsubsection{\label{benzol}Benzolproblematik}

In einigen F\"{a}llen wie z.B. beim Benzol ist eine
"`pseudo"'--univariate Auswertung unerl\"{a}sslich. In
Atmosph\"{a}renspektren wird die st\"{a}rkste Benzolbande (Q--Zweig bei
674 cm\up{-1}) stark von der benachbarten CO\down{2}--Bande
\"{u}berlappt (die Extinktionen unterscheiden sich um ungef\"{a}hr zwei
Gr\"{o}{\ss}enordnungen). Um niedrige Nachweisgrenzen im ppb--Bereich zu
erhalten, ist es aber unerl\"{a}sslich, diese Bande auszuwerten, weil
die \"{u}brigen Schwingungsbanden im mittleren Infrarot weitaus
schw\"{a}cher ausfallen. Ein die Analyse weiter erschwerender Punkt
bei bistatischen Systemen ist, dass die Eigenstrahlung f\"{u}r die
quantitative Bestimmung mitber\"{u}cksichtigt werden muss (siehe auch
Kap. \ref{photometgen}).\\

\bild{htb}{benzausw.wmf}{365}{480}{\bf A \it Benzolatmosph\"{a}ren--,
dazugeh\"{o}riges Eigenstrahlungs-- und Kurzwegeinkanalspektrum. \bf B
\it aufeinander skaliertes und logarithmiertes Benzolatmosph\"{a}ren-
und CO\down{2}--Referenz--Einkanalspektrum. \bf C \it nach der
Subtraktion erhaltene Benzolbande bei einer spektralen Aufl\"{o}sung
von 0.2 cm\up{-1} mit der noch nicht korrigierten
Extinktionsskala. Die Intensit\"{a}t ist in willk\"{u}rlichen Einheiten
dargestellt.}


F\"{u}r die Analyse eines Benzol--Atmosph\"{a}renspektrums werden das
zugeh\"{o}rige Eigenstrahlungsspektrum, sowie ein CO\down{2}-- und ein
Kurzweg--Referenzspektrum ohne CO\down{2}--Banden ben\"{o}tigt (Abb.
\ref{benzausw.wmf} \bf A\rm ). Aufgrund der gro{\ss}en
Extinktionsunterschiede zwischen der Benzol-- und der
CO\down{2}--Bande, muss mit Hilfe eines CO\down{2}--
Referenzspektrums eine skalierte Subtraktion der CO\down{2}--Bande
vorgenommen werden. Im CO\down{2}--Referenzspektrum sollte aus
diesem Grund das ann\"{a}hernd gleiche Produkt aus
CO\down{2}--Volumenanteil mal Wegl\"{a}nge ber\"{u}cksichtigt werden wie
im auszuwertenden Atmosph\"{a}renspektrum. Dabei gilt, dass je besser
die CO\down{2}-- Konzentrationen \"{u}bereinstimmen, desto besser die
Subtraktion durchf\"{u}hrbar ist. Bei Abweichungen um mehr als 10\%
ist eine Auswertung i.a. nicht mehr m\"{o}glich. Hinzu kommt, dass bei
einer durchschnittlichen CO\down{2}--Atmosph\"{a}renkonzentration von
360 ppm ab einer optischen Wegl\"{a}nge von etwa 150 m aufgrund der
Durchabsorption der CO\down{2}--Bande eine Analyse in diesem
Spektralbereich nicht mehr m\"{o}glich ist.\\

Die skalierte CO\down{2}--Subtraktion wird \"{u}ber die
logarithmierten Einkanalspektren vorgenommen. Die
Multiplikationsfaktoren erh\"{a}lt man durch einen Least--Squares--Fit
zwischen den beiden logarithmierten Spektren in den Bereichen
671.8 -- 673.3 cm\up{-1} und 674.2 -- 676.5 cm\up{-1} (Abb.
\ref{benzausw.wmf} \bf B\rm ). Dabei ist zu beachten, dass ein
vorheriger Abzug der Eigenstrahlung von den Einkanalspektren nicht
m\"{o}glich ist, da aufgrund des dann erheblich schlechteren
Signal/Rauschverh\"{a}ltnisses eine zufriedenstellende Kompensation
der CO\down{2}--Bande nicht mehr m\"{o}glich ist. Dies f\"{u}hrt jedoch
dazu, dass die durch die Subtraktion erhaltene Extinktionsskala
mit Hilfe einer Transformationsfunktion in die korrekte Skala
umgerechnet werden muss. Zum Aufstellen dieser
Transformationsfunktion werden neben dem
Benzol--Atmosph\"{a}reneinkanalspektrum das zugeh\"{o}rige
Eigenstrahlungsspektrum und ein Kurzweg--Referenzspektrum
ben\"{o}tigt.\\

Die Nachweisgrenze f\"{u}r Benzol in Spektren, die bei einer Wegl\"{a}nge
von 100 m und einer Aufl\"{o}sung von 0.2 cm\up{-1} gemessen wurden,
liegt bei etwa 25 ppb. Mit einer schlechteren Aufl\"{o}sung von 1
cm\up{-1} erh\"{a}lt man aufgrund des Gl\"{a}ttungseffekts eine etwas
verbesserte Subtraktion der CO\down{2}--Bande, was zu einer
geringf\"{u}gig niedrigeren Nachweisgrenze f\"{u}hrt. Grunds\"{a}tzlich gilt
aber, dass die Artefakte aufgrund der Subtraktion gr\"{o}{\ss}er sind als
das Rauschen (Abb. \ref{benzausw.wmf} \bf C\rm ).\\


Bei der Analyse von Labor--Benzolspektren mit einer Aufl\"{o}sung von
0.2 cm\up{-1} im Bereich von 25 -- 400 ppb wurde bei Anwendung
dieser Auswertungstechnik ein mittlerer relativen Vorhersagefehler
der Volumenanteile von 11\% erreicht, was bei dieser aufwendigen
Art der Auswertung ein gutes Ergebnis ist. W\"{u}nschenswert w\"{a}ren
allerdings noch niedrigere Nachweisgrenzen, da Benzol carcinogen
wirkt und daher ein MAK--Wert (maximale Arbeitsplatzkonzentration)
nicht mehr angegeben wird. F\"{u}r Benzol gilt eine toxisch relevante
Konzentration (TRK--Wert) von 5 ppm, welche problemlos mit der
FT--IR Spektroskopie gemessen werden k\"{o}nnen.\\



\subsection{\label{multivariat}Multivariate Auswerteverfahren}

Bei den multivariaten Verfahren, bei denen mehrere Wellenzahlen
oder ganze Wellenzahlbereiche in die Auswertung miteinbezogen
werden, unterscheidet man zwischen der klassischen Modellbildung
(\bf C\rm lassical \bf L\rm east \bf S\rm quares, CLS) und
statistischen Modellen (z.B. unter Verwendung von \bf P\rm artial
\bf L\rm east \bf S\rm quares, PLS, Faktorzerlegungsverfahren oder
auch der Einsatz von neuronalen Netzen). Bei der atmosph\"{a}rischen
Gasanalytik mittels FT--IR haben sich aber gerade die klassischen
Verfahren aus Gr\"{u}nden, die im folgenden noch dargelegt werden,
bew\"{a}hrt, so dass nur in noch zu nennenden Ausnahmef\"{a}llen auf die
statistischen Modelle zur\"{u}ckgegriffen werden sollte. Aus diesem
Grund wird auch nur die klassische Modellbildung im folgenden
ausf\"{u}hrlich behandelt.\\

\subsubsection{\label{clstheorie}Classical Least Squares--Algorithmus}

Ein Gemischspektrum \bf a \rm setzt sich aus der Summe der
Spektren der im Gasgemisch enthaltenen n--Komponenten $\bf K\rm
^i$ (i--Spalte der Referenzspektrenmatrix \bf K\rm ) und einem
Anteil \bf e \rm zusammen, in dem Modellabweichungen und das
Rauschen ber\"{u}cksichtigt wird. Der multiplikative Faktor $c_i$
entspricht der Konzentration der jeweiligen Komponente im Gemisch
(der Einfachheit halber wird angenommen, dass f\"{u}r die
Referenzspektren Komponentenkonzentrationen von 1 vorliegen). Im
folgenden werden Matrizen mit fettgedruckten Gro{\ss}buchstaben und
Vektoren mit fettgedruckten Kleinbuchstaben bezeichnet.

\begin{equation}\label{egcls1}
  \bf a\rm = \sum^n_{i=1} \bf K\rm ^i\rm  \cdot c_i + \bf e
\end{equation}
oder in Matrixschreibweise

\begin{equation}\label{eqcls2}
  \bf a \rm = \bf K \cdot c \rm + \bf e
\end{equation}
Aufgrund des Abweichungsterms \bf e \rm kann dieses
Gleichungssystem nur n\"{a}herungsweise gel\"{o}st werden. Das Spektrum,
das nach Subtraktion der Referenzspektren mit den durch die
Anpassung erhaltenen Koeffizienten vom Atmosph\"{a}renspektrum
resultiert, wird \it Residuum \rm genannt. Gefordert ist, dass
dieses resultierende Residuumspektrum \bf r \rm mit der Methode
der kleinsten Quadrate minimiert wird, wobei man den Vektor \bf c
\rm so bestimmt, dass die euklidische L\"{a}nge von \bf r \rm minimal
wird. Die hierzu notwendige (und auch hinreichende) Bedingung

\begin{equation}\label{eqcls3}
  \nabla_c | \bf a \rm - \bf K \cdot c\rm |^2=0
\end{equation}
f\"{u}hrt auf die Normalengleichung
\begin{equation}\label{eqcls3a}
  \bf K^T \cdot K \cdot c_{opt} = K^T \cdot a
\end{equation}
Diese Gleichungen sind stets l\"{o}sbar, bei einer Matrix $\bf K$ von
vollem Rang sogar eindeutig l\"{o}sbar mit der L\"{o}sung
\begin{equation}\label{eqcls4}
 \bf c_{opt} = (K^T\cdot K)^{-1}\cdot K^T\cdot a\rm
\end{equation}
Somit ist f\"{u}r eine eindeutige L\"{o}sung eine Voraussetzung, dass die
Anzahl der betrachteten Wellenl\"{a}ngen gr\"{o}{\ss}er als die Anzahl der
Komponenten im Gemisch ist (\cite{haaland85},
\cite{saarinen91}).\\

In Kap. \ref{bereichsauswahl} werden die Wellenzahlbereiche
vorgestellt, in denen die CLS--Auswertung die besten Ergebnisse
liefert. Dort werden zwischen 100 und 5700 spektrale Wertepaare
ausgewertet, so dass das Gleichungssystem grunds\"{a}tzlich
\"{u}berbestimmt ist. Eventuell noch vorhandene Basislinien im
Atmosph\"{a}renspektrum werden dadurch mitangepasst, dass in der
Referenzspektrenmatrix je ein Vektor f\"{u}r Offset, linearen und
quadratischen Term bereitgestellt werden. Die Anpassung eines
kubischen Terms ist i.a. nicht ratsam, da dadurch schon spektrale
Strukturen mitangepasst werden k\"{o}nnen.\\

Das Rauschen s\"{a}mtlicher Spektren erzeugt, neben den Abweichungen
vom linearen Modell, den Fehler im Konzentrationsvektor. Bei der
CLS--Methode wird davon ausgegangen, dass der Abweichungsvektor
\bf e \rm ausschlie{\ss}lich das Gemischspektrum betrifft, die
Referenzspektren aber als fehlerfrei angesehen werden. Dies hat
dadurch seine Berechtigung, dass es bei Aufnahme der
Referenzspektren im Labor i.a. m\"{o}glich ist, f\"{u}r ein Spektrum \"{u}ber
eine weit h\"{o}here Anzahl von Einzelmessungen N mit dem Spektrometer
zu mitteln und damit das Signal--Rausch--Verh\"{a}ltnis um $\sqrt{N}$
zu verbessern als dies bei Routinemessungen in der offenen
Atmosph\"{a}re m\"{o}glich ist. Desweiteren wurden in dieser Arbeit
zus\"{a}tzlich noch einmal \"{u}ber je 28 Referenzspektren f\"{u}r eine
Komponente gemittelt und somit das Rauschen minimiert, wobei auch
zus\"{a}tzlich nichtlineare Terme mitber\"{u}cksichtigt werden (siehe Kap.
\ref{spekaufbereitung}). Sollen auch die unterschiedlichen
Rauschamplituden in den Referenzspektren ber\"{u}cksichtigt werden,
ist der \bf T\rm otal \bf L\rm east \bf S\rm quares--Algorithmus
(TLS) anzuwenden \cite{vanhuffel91}.\\

Beim CLS--Algorithmus ist es zudem sinnvoll, die
Referenzspektrenmatrix \bf K \rm mittels
Gram--Schmidt--Orthogonalisierung von vornherein in orthogonalen
Basisvektoren darzustellen (siehe z.B. \cite{higham96}). Dadurch
wird die m\"{o}gliche Hinzunahme von Referenzspektren erleichtert,
weil nicht die vollst\"{a}ndige Invertierung und
Matrixmultiplikationen aus Gl. \ref{eqcls4} wiederholt werden,
sondern die Rechnungen nur f\"{u}r die hinzugef\"{u}gten Komponenten
durchgef\"{u}hrt werden m\"{u}ssen. Desweiteren verhalten sich die
numerischen Computer--Algorithmen bei der Berechnung komplexer
Systeme generell stabiler, wenn orthogonale Basisfunktionen
eingesetzt werden. Bei Verwendung des Gram--Schmidt--Algorithmus
muss jedoch darauf geachtet werden, dass die Spaltenvektoren von
\bf K \rm linear unabh\"{a}ngig sind und desweiteren gen\"{u}gend
Nachkommastellen bei der Berechnung mitber\"{u}cksichtigt werden
\cite{saarinen91}, da der Algorithmus sonst numerisch instabil
werden kann. Bei dem hier betrachteten mittleren Infrarotbereich
ist die lineare Unabh\"{a}ngigkeit der Gasspektren gegeben, so dass
die Grams--Schmidt--Orthogonalisierung eingesetzt werden kann.
Ansonsten (z.B. im nahen Infrarotbereich) ist eine
Orthogonalisierung mit Householder--Matrizen numerisch stabiler
\cite{golub89}. Die Qualit\"{a}t einer Anpassung mittels CLS
beschreiben weitgehend drei Gr\"{o}{\ss}en:

\paragraph{\label{paragraphstd}Die spektrale Standardabweichung der Anpassung:}

Die Standardabweichung $\sigma$ ist definiert als die Wurzel aus
der Summe der Abweichungsquadrate einer Gr\"{o}{\ss}e geteilt durch die
Anzahl der unabh\"{a}ngigen Observablenpaare (\it Freiheitsgrade\rm ).
F\"{u}r die Beurteilung von Residuen--Spektren lautet die Gleichung
somit:
\begin{equation}\label{eqstd}
\sigma = \sqrt{ \frac{1}{n-k} \sum^{n}_{i=1}(x_{i}- x_m)^2}
\end{equation}
(n--Anzahl der Messpunkte, k--Anzahl der im Gemisch enthaltenen
Komponenten, $x_i$--Extinktion der gemessenen Gemischspektren an
der Stelle i, $x_m$--Extinktionswerte des berechneten
Modellspektrums). Das Quadrat der Standardabweichung $\sigma^2$
wird \it Varianz \rm genannt.\\


\paragraph{\label{paragraphcov}Die Kovarianzmatrix:}

Die Varianz beschreibt die Verteilung von Daten gegen\"{u}ber ihrem
arithmetischen Mittel f\"{u}r eine einzige Variable (hier die
resultierende Extinktion im Residuum). Die Verteilung von
multivariaten Daten kann mit Hilfe der Kovarianzmatrix abgesch\"{a}tzt
werden. Die Summe der Abweichungsprodukte mehrerer Variablen (hier
die Extinktionen mehrerer Komponenten) geteilt durch ihre
Freiheitsgrade ist somit die analoge Gr\"{o}{\ss}e zur Varianz:

\begin{equation}\label{eqcov}
  COV_{jk}=\frac{1}{n-k}
  \sum^n_{i=1}(x_{ij}-\overline{x}_j)(x_{ik}-\overline{x}_k),
\end{equation}
wobei $x_{ij}$ der i--te Wert der Variablen j und $x_{ik}$ der
i--te Wert der Variablen k ist.\\

Die Kovarianzmatrix \bf COV \rm ist quadratisch und symmetrisch
gegen\"{u}ber der Diagonalen, da $x_{ij}=x_{ji}$. Die Diagonalelemente
entsprechen den Varianzen der einzelnen Komponenten. Somit ergibt
sich f\"{u}r den absoluten statistischen Konzentrationsfehler einer
Komponente $\sigma_{Komp}$:

\begin{equation}\label{eqsinglst}
 \sigma_{Komp}= \sigma \sqrt{[\bf COV \rm]_{ii}^{-1}} \cdot t
\end{equation}
mit t--dem Studentfaktor f\"{u}r 95\%--ige Sicherheit.


\paragraph{\label{paragraphconc}Die Konditionszahl:}

Ein weiterer wichtiger Punkt ist zum Auffinden der optimalen
Segmente zur CLS--Auswertung zu beachten. Zur Bildung der
Least--Squares--L\"{o}sung in Gl. \ref{eqcls4} muss die Matrix der
Referenzspektren ($\bf K^T \cdot K$) invertiert werden. Die
Inverse einer Matrix $\bf A$ existiert genau dann, wenn $\rm det
\bf A \rm \neq 0$:

\begin{equation}\label{eqinv}
  \bf A^{-1} \rm = \frac{1}{\text{det} \bf A} \quad \bf A^*
\end{equation}
Matrizen, die invertierbar sind, werden \it regul\"{a}r \rm genannt.
F\"{u}r regul\"{a}re Matrizen gilt, dass Rang (\bf A\rm )= n (Anzahl der
Spaltenvektoren) ist. Der Rang einer Matrix ist gleich der
Maximalzahl der \it linear unabh\"{a}ngigen \rm Spaltenvektoren. Ist
Rang (\bf A\rm )< n, so wird die Matrix \it singul\"{a}r \rm
genannt.\\

Die Least-Squares--L\"{o}sung in Gl. \ref{eqcls4} erfordert somit die
Aufstellung einer regul\"{a}ren Matrix der Referenzspektren. Ob eine
Matrix regul\"{a}r ist, kann anhand der Konditionszahl \"{u}berpr\"{u}ft
werden. Die Konditionszahl ist definiert als der Quotient aus dem
gr\"{o}{\ss}ten Eigenwert $\lambda_{max}$ und dem kleinsten Eigenwert
$\lambda_{min}$ einer Matrix:

\begin{equation}\label{eqcond}
  COND=\frac{\lambda_{max}}{\lambda_{min}}
\end{equation}
Dabei ist folgendes zu beachten: Die Matrix $\bf K^T \cdot K$ ist
symmetrisch und positiv semidefinit; falls \bf K \rm vollen Rang
hat, liegt sogar eine positiv definite Matrix vor und nur in
diesem Fall ist es zul\"{a}ssig, die Kondition der Matrix durch Formel
\ref{eqcond} anzugeben. Es ist anzumerken, dass verrauschte Daten
meistens automatisch positiv definite Matrizen liefern, wobei die
Konditionszahlen dann allerdings sehr hoch sind.\\

F\"{u}r die Konditionszahl der Referenzspektrenmatrix folgt daraus,
dass bei querempfindlichkeitsfreien und normierten
Referenzspektren die Eigenwerte gleich und somit die
Konditionszahl 1 ist. Treten Querempfindlichkeiten zwischen den
Spektren auf, wird der Quotient zwischen gr\"{o}{\ss}tem und kleinstem
Eigenwert gr\"{o}{\ss}er. Sind zwei Spektren linear abh\"{a}ngig, so ist der
kleinste Eigenwert 0, die Konditionszahl ist $\infty$. Welche
Konditionszahlen f\"{u}r die CLS--Analyse noch akzeptiert werden
k\"{o}nnen, wird in Kap. \ref{bereichsauswahl} diskutiert.\\

\paragraph{\label{paragraphweighted}Gewichtete CLS--Regression:}
Die Abweichungen vom linearen Modell, welche in Gl. \ref{egcls1}
im Vektor \bf e \rm auch ber\"{u}cksichtigt worden sind, sind f\"{u}r
Messungen, denen von vornherein ein hoher systematischer Fehler
innewohnt, weniger bedeutsam als f\"{u}r Messungen mit kleinem
systematischen Fehler. Hinzu kommt, dass die Auswertung der
Spektren in der Extinktionsdom\"{a}ne erfolgt. Durch die
Logarithmierung der Transmissionsspektren wird dadurch das
Rauschen bei hohen Extinktionen \"{u}berbetont. Um diese Effekte zu
kompensieren, wird ein Gewichtsfaktor eingef\"{u}hrt, der aus o.g.
Gr\"{u}nden umgekehrt proportional zur Varianz $\sigma_{ii}^2$ der
spektralen Einzelkomponenten (Diagonalelemente der
Kovarianzmatrix) ist. Die Gewichtsmatrix lautet:

\begin{equation}\label{eqclsweight1}
 \bf W\rm =
\begin{pmatrix}
  1/\sigma_{11}^2 & & & 0\\
  & 1/\sigma_{22}^2 & & \\
  & & \ddots & \\
  0 & & & 1/\sigma_{nn}^2\\
\end{pmatrix}
\end{equation}
mit $\bf U=W^T \cdot W$ folgt dann f\"{u}r den Least Squares Sch\"{a}tzer
$\bf c_{opt}$ aus Gl. \ref{eqcls4} \cite{massart88}:

\begin{equation}\label{eqclsweight2}
  \bf c_{opt} \rm = (\bf K^T \cdot U \cdot K\rm )^{-1} \cdot (\bf
  K^T \cdot U \cdot a\rm )
\end{equation}
\\


\subsubsection{\label{h2oco2quer}Querempfindlichkeit mit
atmosph\"{a}rischen Komponenten} \markright{\sl QUEREMPFINDLICHKEIT
MIT ATMOSPH. KOMPONENTEN}

F\"{u}r die optimale Auswahl der Wellenzahlbereiche, in denen eine
CLS--Analyse sinnvoll ist, ist es wichtig, den Einfluss der in
Atmosph\"{a}renspektren immer vorhandenen Komponenten H\down{2}O und
CO\down{2} zu kennen. Beide Gase besitzen in weiten Bereichen des
mittleren Infrarot sehr starke Absorptionen, z.T. gibt es auch
Bereiche, in denen die Transmission extrem kleine Werte annimmt
und die von vornherein nicht f\"{u}r die Auswertung zur Verf\"{u}gung
stehen. Diese starken Absorptionen gehen mit gro{\ss}en nichtlinearen
Abh\"{a}ngigkeiten einher, die bei der Auswertung nur schlecht
modelliert werden k\"{o}nnen. Eine Option, diese Modellierung zu
umgehen, besteht nat\"{u}rlich darin, ein Hintergrundspektrum
m\"{o}glichst unter gleichen spektroskopischen (hinsichtlich der
optischen Wegl\"{a}nge) und meteorologischen Bedingungen aufzunehmen,
so dass von vornherein eine gute Kompensation der atmosph\"{a}rischen
Absorptionen in Gebieten, in denen die Transmission nicht gegen
null geht, gegeben ist. Wie im Kapitel \ref{hintergrund} aber
schon aufgezeigt, gibt es Messsituationen, in denen dies nicht
m\"{o}glich ist, wo aber ein synthetisch berechnetes oder ein im Labor
gemessenes Hintergrundspektrum helfen. Die H\down{2}O-- und
CO\down{2}--Banden werden bei der Berechnung des
Transmissionsspektrums dabei aber nicht kompensiert.\\

 \tabelle{htb}{1.2}{clsh2oreferenz}{{|c c c c c c c c|}\hline
 & & \multicolumn{2}{c}{\underline{\small \bf Konzentration}} & \multicolumn{2}{c}{\underline{\small \bf nom. Aufl. 0.1 cm\up{-1}}}
 & \multicolumn{2}{c|}{\underline{\small \bf nom. Aufl. 1 cm\up{-1}}} \\
 & & \small wahre & \small Referenz & \small Iterat. 1\up{\star} & \small Iterat. 2 &
 \small  Iterat. 1\up{\star} & \small  Iterat. 4\\
 \multicolumn{2}{|c}{\raisebox{3.7ex}[-1.5ex]{\underline{\small \bf
 Komponente}}} & \small  [ppm$\cdot$m] & \small  [ppm$\cdot$m] & \small [\%] & \small [\%] &
 \small  [\%] & \small  [\%]\\
 \hline
 &\small  Gem. 1 & \small  600 & \small 101 & \small 0.5(-1.4) &\small  -1.2 & \small 11.7(-6.0) & \small -3.4\\
 \raisebox{1.5ex}[-1.5ex]{\small \bf CH\down{4}} &\small  Gem. 2 & \small 400 &  &\small  -0.7(-0.8) &\small  -0.7 & \small 11.5(-3.5) & \small -0.9\\
 \hline
 &\small Gem. 1 & \small 500 000 & \small 50 000 & \small -27.5(-28.4) & \small -3.4 & \small -232(-234) & \small -13.1\\
 \raisebox{1.5ex}[-1.5ex]{\small \bf H\down{2}O} &\small  Gem. 2 & \small 300 000 &  & \small -18.5(-18.7) & \small -2.2 & \small -146(-145) & \small -8.1\\
 \hline
 &\small  Gem. 1 & \small 800 & \small 101 &\small  -6.9(-7.2) & \small -7.0 &\small  -22(-25.7) & \small -3.9\up{\dagger}\\
 \raisebox{1.5ex}[-1.5ex]{\small \bf NH\down{3}} &\small  Gem. 2 & \small 400 &  &\small  -4.5(-4.9) &\small  -4.8 & \small -7.8(-11.7) & \small -1.0\up{\dagger}\\
 \hline
 &\small  Gem. 1 & \small 800 & \small 101 & \small 0.5(-0.5) & \small -0.2 & \small 5.4(-2.7) & \small -0.6\\
 \raisebox{1.5ex}[-1.5ex]{\small \bf SO\down{2}} &\small  Gem. 2 & \small 400 &  & \small 0.8(-0.2) & \small 0.1 & \small -7.1(-1.1) &\small  0.0\\
 \hline
 &\small  Gem. 1 &  &  & \small 3.7$\cdot$10\up{-2} & \small 6.8$\cdot$10\up{-3} &\small  6.9$\cdot$10\up{-2} &\small  9.1$\cdot$10\up{-3}\\
 \raisebox{1.5ex}[-1.5ex]{\small \bf STD} &\small  Gem. 2 &  &  &\small  2.5$\cdot$10\up{-2} & \small 3.8$\cdot$10\up{-3} & \small 4.8$\cdot$10\up{-2} &\small  5.5$\cdot$10\up{-3}\\
\cline{1-8}
 \multicolumn{1}{r}{\small \up{\star}} & \multicolumn{7}{l}{\small in Klammern die Resultate f\"{u}r die Anpassung eines
 Einzelkomponenten--}\\
 \multicolumn{1}{c}{} &\multicolumn{7}{l}{\raisebox{1.5ex}[-1.5ex]{\small spektrums mit der gegebenen
 Gemischkonzentration}}\\
 \multicolumn{1}{r}{\raisebox{1.5ex}[-1.5ex]{\small \up{\dagger}}} & \multicolumn{7}{l}{\raisebox{1.5ex}[-1.5ex]{\small in
 diesen F\"{a}llen wurde auch das NH\down{3}--Referenzspektrum
 mitangepasst}}
 }{Iterative Konzentrationsvorhersage mittels CLS und konzentrationsangepassten H\down{2}O--Referenzspektren
 (ausgewerteter Bereich: 500-2000 cm\up{-1}, Extinktionslimit 1.0, Dreiecksapodisation).}

Welchen Einfluss vor allem H\down{2}O auf die
Auswertungsergebnisse haben kann, zeigt Tabelle
\ref{clsh2oreferenz}. Hier wurden zwei Gemische mit der
HITRAN96--Datenbank und typischen H\down{2}O--Konzentrationen
berechnet und mittels CLS im Bereich von 500--2000 cm\up{-1} f\"{u}r
0.2 und 1 cm\up{-1} Aufl\"{o}sung angepasst. Die nichtlinearen
spektralen Abh\"{a}ngigkeiten wurden durch schrittweise Neuberechnung
des Wasserdampfreferenzspektrums mit der Konzentration der
vorhergehenden Iteration angepasst. Das so dem jeweiligen
Atmosph\"{a}renspektrum n\"{a}herkommende Wasserdampfspektrum wurde
solange im n\"{a}chsten Anpassungszyklus eingesetzt, bis keine weitere
Reduzierung der spektralen Residuen erreicht werden konnte. Hier
zeigt sich deutlich der Vorteil einer besseren spektralen
Aufl\"{o}sung, da aufgrund der geringeren nichtlinearen Anteile im
Spektrum und der besseren Trennung der Komponenten untereinander
bereits nach einer Iteration zufriedenstellende Ergebnisse erzielt
werden konnten. Es zeigt sich aber auch, dass die
Wellenzahlbereiche mit den nichtlinearen Anteilen von H\down{2}O,
und das gilt auch f\"{u}r CO\down{2}, f\"{u}r die normale CLS--Auswertung
nach M\"{o}glichkeit zu meiden sind. Dem steht gegen\"{u}ber, dass die
Strukturen vieler Komponenten von H\down{2}O-- und
CO\down{2}--Banden \"{u}berlagert sind.\\

\bild{h}{h2oco21.wmf}{400}{390}{Anpassung von CO\down{2} (\bf a\it
, links) und H\down{2}O (\bf a\it , rechts) in einem
Atmosph\"{a}renspektrum (Pfadl\"{a}nge: 135 m), wobei die maximalen
Extinktionen von 0.35 im unteren Bereich liegen. Die
Extinktionsspektren \bf b \it und \bf c \it wurden mit gemessenen
Hintergrundspektren (1 h bzw. 5 h sp\"{a}ter aufgenommen), die
Extinktionsspektren \bf d \it und \bf e \it mit synthetischen
Hintergrundspektren berechnet. Die Anpassung der CO\down{2} und
H\down{2}O--Banden in \bf d \it und \bf e \it erfolgte mit
Laborspektren (\bf d\it , 44 577 ppm$\cdot$m CO\down{2} und 1 162
055 ppm$\cdot$m H\down{2}O) bzw. HITRAN96--Spektren (\bf e\it , 44
550 ppm$\cdot$m CO\down{2} und 675 000 ppm$\cdot$m H\down{2}O).
Die jeweiligen spektralen Standardabweichungen (STD) der
resultierenden Residuen sind in der Abbildung angegeben. Bis auf
die Spektren \bf e \it sind die Spektren mit einem Offset
dargestellt.}


\bild{h}{h2oco22.wmf}{400}{390}{Anpassung von H\down{2}O (\bf a\it
, links) und CO\down{2} (\bf a\it , rechts) in einem
Atmosph\"{a}renspektrum (Pfadl\"{a}nge: 135 m), wobei die maximalen
Extinktionen von 1.5 im oberen Bereich liegen. Die
Extinktionsspektren \bf b \it und \bf c \it wurden mit gemessenen
Hintergrundspektren (1 h bzw. 5 h sp\"{a}ter aufgenommen), die
Extinktionsspektren \bf d \it und \bf e \it mit synthetischen
Hintergrundspektren berechnet. Die Anpassung der H\down{2}O und
CO\down{2}--Banden in \bf d \it und \bf e \it erfolgte mit
Laborspektren (1 162 055 ppm$\cdot$m H\down{2}O und \bf d\it , 44
577 ppm$\cdot$m CO\down{2}) bzw. HITRAN96--Spektren (\bf e\it ,
675 000 ppm$\cdot$m H\down{2}O und 44 550 ppm$\cdot$m CO\down{2}).
Die jeweiligen spektralen Standardabweichungen (STD) der
resultierenden Residuen sind in der Abbildung angegeben. Bis auf
die Spektren \bf e \it sind die Spektren mit einem Offset
dargestellt.}

Die Abbildungen \ref{h2oco21.wmf} und \ref{h2oco22.wmf} zeigen
dasselbe Atmosph\"{a}renspektrum in unterschiedlichen spektralen
Bereichen, in denen entweder H\down{2}O oder CO\down{2} spektrale
Strukturen aufweisen (jeweils die Spektren \bf a\rm ). Diese
Strukturen werden durch vier verschiedene Arten kompensiert. Als
Ma{\ss} f\"{u}r die G\"{u}te der Kompensation gilt die Standardabweichung des
resultierenden Residuums. In den Spektren \bf b \rm und \bf c \rm
werden die Strukturen durch 5 h bzw. 1 h nach dem
Atmosph\"{a}renspektrum gemessene Hintergrundspektren angepasst. Im
Spektrum \bf d \rm wird zu dieser Kompensation ein H\down{2}O--
und CO\down{2}--Laborspektrum, in \bf e \rm ein f\"{u}r die
Konzentration, Temperatur und Druck berechnetes HITRAN96--Spektrum
verwandt (jeweilige Konzentrationen siehe Bildunterschrift). In
Abbildung \ref{h2oco21.wmf} sind Bereiche mit relativ kleinen
Banden gew\"{a}hlt, in Abbildung \ref{h2oco22.wmf} Bereiche mit Banden
mittlerer Extinktion.\\

Es zeigt sich deutlich, dass grunds\"{a}tzlich die Anpassung mittels
gemessenen Hintergrundspektren besser ist. Wie gut in diesem Fall
kompensiert werden kann, h\"{a}ngt nat\"{u}rlich von den Schwankungen der
H\down{2}O-- und CO\down{2}--Konzentrationen in der Zeit zwischen
Aufnahme des Atmosph\"{a}ren-- und des Hintergrundspektrums ab. In dem
gezeigten Beispiel sind die Schwankungen nur gering. H\down{2}O
konnte besser angepasst werden als CO\down{2}, die Anpassung von
H\down{2}O verschlechterte sich zu den Bereichen mit gr\"{o}{\ss}er
Extinktion um den Faktor 2, bei CO\down{2} um den Faktor 6. Bei
der Anpassung mit Labor-- oder HITRAN96--Spektren sind diese
Anpassungen deutlich schlechter (Faktor 17 bzw. 30).\\

F\"{u}r die Spektrenbereichsauswahl ist es somit in jedem Fall
vorteilhaft, von vornherein Bereiche mit H\down{2}O-- und
CO\down{2}--Extinktionen gr\"{o}{\ss}er als 0.8 zu meiden, sofern sie
nicht f\"{u}r die Analyse unbedingt erforderlich sind, weil eine zu
analysierende Komponente nur in den Spektralbereichen Strukturen
besitzt. Lassen es die \"{o}rtlichen Gegebenheiten zu, sollte ein
Hintergrundspektrum gemessen werden, wenn sichergestellt ist, dass
die zu analysierenden Zielkomponenten nicht im Spektrum enthalten
sind. In diesem Fall ist es m\"{o}glich, abh\"{a}ngig von den H\down{2}O--
und CO\down{2}--Schwankungen zwischen der Messung des
Atmosph\"{a}ren-- und des Hintergrundspektrums, ein wenig gr\"{o}{\ss}ere
Spektralbereiche zur Auswertung als bei der Verwendung von
synthetisch berechneten Hintergrundspektren heranzuziehen.\\



\subsubsection{\label{bereichsauswahl}Spektrenbereichsauswahl zur
CLS--Auswertung} \markright{SPEKTRENBEREICHSAUSWAHL ZUR
CLS--AUSWERTUNG}

Die im vorherigen Abschnitt gewonnenen Erkenntnisse \"{u}ber die
H\down{2}O-- und CO\down{2}--\-Quer\-em\-pfind\-lich\-keit\-en
lassen nur noch ganz bestimmte spektrale Fenster im mittleren
Infrarotbereich zu, in denen Atmosph\"{a}renspektren ausgewertet
werden k\"{o}nnen. Um nun die Frage zu kl\"{a}ren, ob es sinnvoller ist,
den gesamten Fensterbereich auszuwerten oder diesen zu
unterteilen, um m\"{o}gliche Querempfindlichkeiten mit anderen Stoffen
zu minimieren, wird das in Abbildung \ref{spekseg4.wmf} gezeigte
Gemisch vorgegeben.\\

\bild{htb}{spekseg4.wmf}{400}{360}{\it Gemischspektrum f\"{u}r die
Ermittlung des optimalen Wellenzahlbereichs der einzelnen
Komponenten im Segment 4 (2630--3003 cm\up{-1}). Das
Gemischspektrum setzt sich aus den oben mit Offset dargestellten
Einzelkomponenten mit den angegebenen Volumenanteilen f\"{u}r eine
Wegl\"{a}nge von 100 m und einem f\"{u}r 100 m charakteristischen
Transmissionsrauschen von 8.8$\cdot$10\up{-4} zusammen. Die
Darstellung der Einzelkomponenten erfolgt f\"{u}r eine Wegl\"{a}nge von
100 m. Bei einigen Komponenten wurde die spektrale Struktur zur
besseren Darstellung mit dem angegebenen Faktor multipliziert. Die
Spektren stammen aus der QASoft-Spektrenbibliothek.}

\tabelle{htb}{1.2}{seg4iteration}{{|l c c c c c|}\hline
\underline{\small \bf Stoff} & \underline{\small \bf wahre Konz.}
& \underline{\small \bf ber. Konz.}  & \underline{\small \bf stat.
Fehl.} & \underline{\small \bf syst. Fehl.} & \underline{\small
\bf opt. Bereich}\\ & \bf [ppm] & \bf [ppm] & $\bf \pm$ \bf [\%] &
\bf [\%] & \bf [cm\up{-1}]\\ \hline

Wasser & 5000.00 & 4977.96 & 0.13 & -0.44 & 2633--3003\\

Formald. & 1.7000 & 1.6975 & 0.15 & -0.15 & 2633--2923\\

Acetald. & 1.0000 & 0.9991 & 0.26 & -0.09 & 2638--3003\\

Methan & 3.8000 & 3.7863 & 0.13 & -0.36 & 2653--3003\\

Ethan & 2.6000 & 2.5936 & 0.15 & -0.25 & 2633--3003\\

Methanol & 1.4000 & 1.4050 & 1.07 & +0.36 & 2668--3003\\

Ethanol & 3.2000 & 3.1993 & 0.26 & -0.02 & 2643--3003\\

Butan & 1.5000 & 1.4944 & 0.52 & -0.37 & 2643--3003\\

Pentan & 1.5000 & 1.4954 & 0.39 & -0.31 & 2633--3003\\ \hline

}{Auswertung des Gemischspektrums aus Abb. \ref{spekseg4.wmf}
mittels CLS. Hierzu wurde das Wellenzahlintervall 2993--3003
cm\up{-1} als Startintervall genommen und die untere Grenze 2993
cm\up{-1} in 5 cm\up{-1}--Intervallen bis 2633 cm\up{-1}
verschoben. Das n\"{a}chste Startintervall war 2983--2993 cm\up{-1}
usw. Insgesamt wurden somit 1369 Intervalle durchlaufen.
Dargestellt sind die errechneten Volumenanteile der einzelnen
Komponenten in dem Intervall, in dem sie die kleinste statistische
Abweichung aufzeigen.}

Die Spektren f\"{u}r das Gemisch wurden der
QASoft--Ref\-er\-enz\-spek\-tren\-bib\-lio\-thek entnommen. Um f\"{u}r
die Auswertung mittels CLS ein vollst\"{a}ndig bekanntes
Gemischspektrum zu bekommen, wurden sie mit praxisrelevanten
Konzentrationen f\"{u}r ein Spektrum, gemessen bei einer Wegl\"{a}nge von
100 m, im Bereich von 2630--3003 cm\up{-1} aufeinander addiert und
zus\"{a}tzlich in der Transmissionsdom\"{a}ne mit einem entsprechenden f\"{u}r
ein K300 gemessenes Atmosph\"{a}renspektrum typischen Rauschen belegt.
Der betrachtete Wellenzahlbereich wird an der oberen und unteren
Grenze von st\"{a}rkeren Wasserabsorptionen begrenzt. Die Spektren
weisen teilweise sehr \"{a}hnliche Strukturen auf, es sind aber sowohl
breite als auch sehr schmale Banden zu sehen. Somit wird eine
Vielzahl der m\"{o}glichen vorkommenden Szenarien abgedeckt.\\

Um nun s\"{a}mtliche in diesem Bereich m\"{o}gliche Intervalle zu
ber\"{u}cksichtigen, wurde bei der Auswertung iterativ vorgegangen.
Als Startsegment wurde das Intervall von 2993--3003 cm\up{-1}
herangezogen und mittels CLS ausgewertet. Die untere Grenze wurde
dann jeweils um 5 cm\up{-1} iterativ bis 2633 cm\up{-1} nach unten
verschoben und jeweils ausgewertet. Anschlie{\ss}end wurde mit dem
Startsegment von 2983--2993 cm\up{-1} genauso verfahren, bis
s\"{a}mtliche M\"{o}glichkeiten der Intervallanalyse im Wellenzahlbereich
von 2630--3003 cm\up{-1} abgedeckt waren (insgesamt wurden 1369
Intervalle ausgewertet). Tabelle \ref{seg4iteration} zeigt die
Ergebnisse f\"{u}r das Intervall, in dem die jeweilige Komponente den
kleinsten statistischen Fehler aufzeigt. Als Referenzspektren bei
der Auswertung wurden dieselben Spektren genommen, die auch f\"{u}r
die Addition des Gemischspektrums verwendet wurden. Der
systematische Fehler resultiert daher zum Teil aus numerischen
N\"{a}herungen bei der Berechnung der optimalen Konzentration, der
statistische aus dem im Gemischspektrum enthaltenen Rauschen.
Grunds\"{a}tzlich ist zu erkennen, dass bei fast allen Spektren das
gr\"{o}{\ss}tm\"{o}gliche Intervall zum besten Ergebnis f\"{u}hrt, obwohl viele
Spektren im unteren Bereich keine Signaturen mehr aufweisen. Dies
ist vor allem darauf zur\"{u}ckzuf\"{u}hren, dass die Selektivit\"{a}t der
Spektren im Vektorraum mit zunehmendem Intervall auch steigt.\\

\bild{htb}{seg4conc.wmf}{305}{405}{\it Iterative Auswertung dreier
Beispielkomponenten des Gemischspektrums aus Abb.
\ref{spekseg4.wmf}. Hierzu wurde das oberste 10
cm\up{-1}--Wellenzahlintervall (durchgezogene und gestrichelte
senkrechte Linie auf der rechten Seite dieser Abbildung) des
jeweiligen optimalen Auswertungsbereichs (siehe Tab.
\ref{seg4iteration}) als Startintervall genommen und die untere
Grenze (gestrichelte senkrechte Linie) in 5 cm\up{-1}--Schritten
bis 2633 cm\up{-1} durchiteriert. Die dabei errechneten
Volumenanteile (schwarze Punkte) und die dazugeh\"{o}rigen
statistischen Fehler (gestrichelte Linien), sowie die Ergebnisse
f\"{u}r die Auswertung im gesamten Segment 4 (2630-3003 cm\up{-1},
schwarze Rauten) werden ebenso wie der Verlauf der spektralen
Standardabweichung (nicht gepunktete Linie entspricht den
Ergebnissen f\"{u}r die gewichtete CLS--Analyse) und der
Konditionszahl im gesamten Bereich gezeigt.}

In Kapitel \ref{clstheorie} wurden die Standardabweichung, die
statistische Unsicherheit der Konzentrationssch\"{a}tzung der
Einzelkomponenten und die Konditionszahlen als Gr\"{o}{\ss}en f\"{u}r die
Bewertung der Qualit\"{a}t einer CLS--Anpassung vorgestellt. Abbildung
\ref{seg4conc.wmf} zeigt den Verlauf dieser Gr\"{o}{\ss}en mit zunehmender
Intervallgr\"{o}{\ss}e. Sowohl die spektrale Standardabweichung wie die
Konditionszahl nehmen mit zunehmender Intervallgr\"{o}{\ss}e ab. Die
Standardabweichung ist ein Ma{\ss} f\"{u}r die verbleibenden spektralen
Strukturen im resultierenden Residuum. Die nichtgepunktete Linie
zeigt die Ergebnisse f\"{u}r die gewichtete Analyse. Das Rauschen im
Transmissionsspektrum ist in diesem Beispiel \"{u}ber dem gesamten
ausgewerteten Intervall gleichverteilt. Die gewichtete Analyse
ber\"{u}cksichtigt die aus der Logarithmierung der
Transmissionsspektren resultierenden h\"{o}heren Rauschwerte bei
gro{\ss}en Extinktionen in der Weise, dass diese Bereiche weniger
gewichtet in die Analyse eingehen. Hier im idealen Fall, wo die
Referenzspektren mit den Konzentrationen der Komponenten des
Gemischspektrums exakt \"{u}bereinstimmen und keine nichtlinearen
Modellfehler in die Analyse hineinspielen, folgt der Verlauf der
verbleibenden Standardabweichung den gegl\"{a}tteten Extinktionswerten
des betrachteten Intervalls im Gemischspektrum. Die auftretende
Standardabweichung ist in diesem Fall somit zum gro{\ss}en Teil auf
das h\"{o}here Extinktionsrauschen im Intervall von etwa 2800--3003
cm\up{-1} zur\"{u}ckzuf\"{u}hren. Bei einer Analyse des gesamten
Intervalls wird sie minimal.\\

Die Konditionszahl, die ein Ma{\ss} f\"{u}r die lineare Unabh\"{a}ngigkeit der
Referenzspektren untereinander ist, nimmt mit zunehmender
Intervallgr\"{o}{\ss}e ab. D.h. je gr\"{o}{\ss}er das Intervall gew\"{a}hlt wird,
desto stabiler sind die numerischen Algorithmen zur Berechnung der
optimalen Konzentration. Konditionszahlen um die 40 sind dabei
nach eigenen Erfahrungen noch tolerabel.\\

Die gestrichelten Linien bei der Konzentrationsanalyse der hier
exemplarisch gezeigten Gase geben die statistischen
Fehlerschranken der Analyse wieder. Diese werden mit zunehmendem
Intervall kleiner. Die Rauten am linken Bildrand zeigen das
Auswertungsergebnis f\"{u}r das gesamte Intervall von 2630--3003
cm\up{-1}. Es unterscheidet sich nicht signifikant von den hier
dargestellten optimalen Intervallen, was unter Beweis stellt, dass
die Auswertung im gesamten Bereich, den die \it atmosph\"{a}rischen
Fenster \rm (also Wellenzahlbereiche, die nicht von starken
H\down{2}O-- oder CO\down{2}--Banden \"{u}berlagert sind) zulassen,
nicht nur zul\"{a}ssig, sondern zu fordern ist.\\

\tabelle{htb}{1.2}{segmente}{{|c|c|l|}\hline  \bf Segment &
\rule[-0.25cm]{0mm}{0.85cm} \bf Spektralbereich
&\parbox[t]{8.5cm}{\bf auszuwertende Stoffe im
Spektralbereich\\[-0.18cm]}\\ \hline

1 & \rule[-0.25cm]{0mm}{0.85cm}671--680 cm\up{-1} &
\parbox[t]{8.5cm}{Benzol (Querempfindlichkeit CO\down{2},
H\down{2}O)\\[-0.18cm]} \\ \hline


2 & \rule[-0.25cm]{0mm}{0.85cm}723--1242 cm\up{-1} &
\parbox[t]{8.5cm}{H\down{2}O, CO\down{2}, O\down{3}, N\down{2}O,
NH\down{3}, Methan, Ethan, Propan, Butan, Pentan, Hexan, Ethen,
Propen, Isobuten, Acetylen, 1,3--Butadien, Cyclohexan, Ethanol,
Phenol, Toluol, m--Xylol, o--Xylol, Aceton, 2--Butanon,
Dichlormethan, Trichlorethylen, Tetrachlorethylen, Acetaldehyd,
Formaldehyd, Ameisens\"{a}ure, Essigs\"{a}ure, Schwefelhexafluorid
\\[-0.18cm]}\\ \hline

3 & \rule[-0.25cm]{0mm}{0.85cm}2091--2224 cm\up{-1} &
\parbox[t]{8.5cm}{H\down{2}O, CO\down{2}, N\down{2}O, CO\\[-0.18cm]}
\\ \hline

4 & \rule[-0.25cm]{0mm}{0.85cm}2630-3003 cm\up{-1} &
\parbox[t]{8.5cm}{H\down{2}O, Methan, Ethan,
Propan, Butan, Pentan, Hexan, Ethen, Propen, Isobuten,
1,3--Butadien, Cyclohexan, Methanol, Ethanol, Dichlormethan,
Toluol, m--Xylol, o--Xylol, p--Xylol, Aceton, 2--Butanon,
Acetaldehyd, Formaldehyd\\[-0.18cm]}\\ \hline

5 & \rule[-0.25cm]{0mm}{0.85cm}4130--4300 cm\up{-1} &
\parbox[t]{8.5cm}{H\down{2}O\\[-0.18cm]} \\ \hline

6 & \rule[-0.25cm]{0mm}{0.85cm}732.5--740 cm\up{-1} &
\parbox[t]{8.5cm}{CO\down{2} (Querempfindlichkeit H\down{2}O),
ggf. mit Ergebnis aus Segment 2 vergleichen\\[-0.18cm]}
\\ \hline}{Wahl von 6 Wellenzahlsegmenten zur optimalen Auswertung von
34 f\"{u}r die Atmosph\"{a}renspektroskopie relevanten Komponenten.}

\bild{htb}{no2.wmf}{400}{470}{\bf A \it NO\down{2}--Spektrum mit
0.2 cm\up{-1} (unten) und 1 cm\up{-1}--Aufl\"{o}sung (mit Offset), \bf
B \it NO\down{2}-- und H\down{2}O--Spektrum (gestrichelt) mit 0.2
cm\up{-1}--Aufl\"{o}sung, \bf C \it NO\down{2}-- und
H\down{2}O--Spektrum (gestrichelt) mit 1 cm\up{-1}--Aufl\"{o}sung. Die
Spektren wurden berechnet mit der GEISA--Datenbank,
NO\down{2}--173 ppm$\cdot$m, H\down{2}O-- 1162000 ppm$\cdot$m.}

\tabelle{htb}{1}{nachweisgr}{{|l|c|c|c|c|c|c|}\hline &
\multicolumn{2}{c|}{\bf \small CLS--NWG [ppb]} &  &
\multicolumn{2}{c|}{\bf \small univ. NWG [ppb]} & \bf \small
univariat\\ \raisebox{1.5ex}[-1.5ex]{\bf \small Stoff} & \bf 0.2
cm\up{-1} & \small \bf 1 cm\up{-1} & \raisebox{1.5ex}[-1.5ex]{\bf
\small Segnr.} & \small \bf 0.2 cm\up{-1} & \small \bf 1 cm\up{-1}
& \small \bf [cm\up{-1}]\\ \hline

\small N\down{2}O & 1 & 1 & 3 & 3 & 2 & 2209.6\\

\small CO & 3 & 1 & 3 & 9 & 9  & 2111.5\\

\small NH\down{3} & 2 & 1  & 2 & 5 & 5  & 1046.5\\

\small Methan & 4 & 5  & 4 & 14 & 17 & 2979.0\\

\small Ethan & 1 & 2 & 4 & 5 & 12 & 2993.5\\

\small Propan & 1 & 1  & 4 & 7 & 5  & 2967.5\\

\small Butan & 1 & 1 & 4 & 7 & 6 & 2966.2\\

\small Pentan & 1 & 1  & 4 & 7 & 5  & 2965.4\\

\small Hexan & 1 & 1 & 4 & 8 & 2 & 2964.4\\

\small Ethen & 1 & 1 & 2 & 3 & 3 & 949.4\\

\small Propen & 1 & 1 & 2 & 6 & 5 & 912.6\\

\small Isobuten & 1 & 1 & 2 & 4 & 5 & 890.4\\

\small Acetylen & 2 & 3 & 2 & 4 & 5 & 729.4\\

\small 1,3--Butadien & 2 & 1 & 2 & 6 & 4 & 908.1\\

\small Cyclohexan & 1 & 1 & 4 & 7 & 2 & 2933.1\\

\small Methanol & 1 & 1 & 2 & 4 & 4 & 1033.4\\

\small Ethanol & 1 & 1 & 2 & 6 & 7 & 1066.4\\

\small Phenol & 1 & 1 & 2 & 5 & 12 & 752.1\\

\small Toluol & 1 & 4 & 4 & 10 & 40 & 2938.4\\

\small m--Xylol & 1 & 2 & 2 & 6 & 8 & 768.7\\

\small o--Xylol & 1 & 1 & 2 & 4 & 3 & 741.5\\

\small p--Xylol & 1 & 2 & 2 & 3 & 12 & 795.5\\

\small Aceton & 1 & 1 & 2 & 7 & 9 & 1217.6\\

\small 2--Butanon & 1 & 2 & 4 & 4 & 20 & 2992.5\\

\small Dichlormethan & 1 & 1 & 2 & 6 & 4 & 749.4\\

\small Trichlorethylen & 1 & 1 & 2 & 5 & 6 & 782.9\\

\small Tetrachlorethylen & 1 & 1 & 2 & 4 & 2 & 916.2\\

\small Acetaldehyd & 1 & 2 & 4 & 19 & 24 & 2739.5\\

\small Formaldehyd & 1 & 2 & 4 & 10 & 16 & 2778.4\\

\small Ameisens\"{a}ure & 1 & 1 & 2 & 6 & 5 & 1104.9\\

\small Essigs\"{a}ure & 1 & 1 & 2 & 15 & 5 & 1183.9\\

\small Schwefelhexafluorid & 1 & 1 & 2 & 4 & 2 & 947.8\\ \hline

\small Benzol & \multicolumn{5}{c|}{bes. Verfahren: 25 ppb (0.2
cm\up{-1}), 20 ppb (1 cm\up{-1})} & 673.9\\
\hline}{Nachweisgrenzen NWG relevanter Schadgase f\"{u}r
Atmosph\"{a}renspektren, aufgenommen bei 100 m Wegl\"{a}nge.}

In Tabelle \ref{segmente} sind diese Wellenzahlbereiche
aufgef\"{u}hrt, mit 34 f\"{u}r die Atmosph\"{a}renspektroskopie relevanten
Komponenten. Viele Stoffe haben in mehreren Segmenten Strukturen.
Zur Auswertung ist dann das Segment zu w\"{a}hlen, in dem der
statistische Fehler f\"{u}r diese Komponente am geringsten ist.
Weitere Verbesserungen k\"{o}nnen u.U. dadurch erzielt werden, dass
die Vektoren s\"{a}mtlicher Segmente mit den Strukturen eines Stoffes
zusammengef\"{u}gt werden und dann die CLS--Auswertung erfolgt.\\

F\"{u}r einige Komponenten wie die NO\down{x} oder SO\down{2}, deren
intensivsten Strukturen s\"{a}mtlich von starken Wasserbanden
\"{u}berlagert sind, sind gesonderte Segmente zu w\"{a}hlen. Die hier
untersuchten und betriebenen Verbesserungen in der photometrischen
Genauigkeit insbesondere f\"{u}r starke Absorptionslinien er\"{o}ffnen
speziell f\"{u}r die Erweiterung der spektralen Bereiche neue
M\"{o}glichkeiten f\"{u}r analytische Anwendungen, die sonst aufgrund der
starken atmosph\"{a}rischen Absorptionen nur eingeschr\"{a}nkt nutzbar
w\"{a}ren. In Abbildung \ref{no2.wmf} ist beispielhaft gezeigt, dass
nur unter Verwendung einer hohen spektralen Aufl\"{o}sung die
Auswertung von NO\down{2}--Absorptionssignaturen Sinn macht.\\


\paragraph{\label{nwg}Nachweisgrenzen:}

In Tabelle \ref{nachweisgr} sind die Nachweisgrenzen relevanter
Schadgase zusammengestellt, so wie sie nach den VDI--Richtlinien
\cite{vdi98} zu bestimmen sind. Dabei wurden in diesem Fall 22
direkt hintereinander aufgenommene Atmosph\"{a}renspektren genommen
und das jeweils folgende Spektrum zur Berechnung des
Extinktionsspektrums verwandt. Diese resultierenden 21
Extinktionsspektren, die nur noch statistisches Rauschen
enthalten, werden dann in den oben beschriebenen Segmenten nach
der jeweiligen Zielkomponente, deren Nachweisgrenze bestimmt
werden soll, ausgewertet (bei der univariaten Auswertung wurde ein
Intervall von $\pm$5 cm\up{-1} um die st\"{a}rkste Absorptionsbande
vorgegeben). Es ergeben sich sowohl negative wie auch positive
Konzentrationsvorhersagen, deren Standardabweichung $\sigma$
bestimmt wird, so dass sich die Nachweisgrenze folgenderma{\ss}en
ergibt:

\begin{equation}\label{eqnwg}
  NWG = t_{n-1;0.95} \cdot \sigma
\end{equation}
\begin{tabbing}
mit \quad \=-- \=test \=\kill

mit \> t \> -- \> Studentfaktor f\"{u}r eine statistische Sicherheit
von 95 \%\\

\> $\sigma$ \> -- \>Standardabweichung\\

\> n \> -- \> Anzahl der gemittelten Rauschspektren\\
\end{tabbing}

Es zeigt sich, dass die multivariaten Verfahren aufgrund des
verbesserten Sig\-nal/\-Rausch\-ver\-h\"{a}lt\-nisses grunds\"{a}tzlich
bessere Nachweisgrenzen besitzen als dies bei der univariaten
Auswertung der Fall ist. Dennoch sind die erhaltenen Werte in
dieser Tabelle nur mit Einschr\"{a}nkungen praxisrelevant, da m\"{o}gliche
Querempfindlichkeiten und schlechtere justage-- oder
wetterbedingte Signal/Rauschverh\"{a}ltnisse nicht ber\"{u}cksichtigt
werden. Da diese St\"{o}rungen aber sehr von der einzelnen
Messsituation abh\"{a}ngen, ist eine generellere Angabe der
Nachweisgrenze nicht m\"{o}glich.\\


\subsubsection{\label{alternativen}Alternativen zur
CLS--Auswertung} \markright{ALTERNATIVEN ZUR CLS--AUSWERTUNG}

Das in den vorhergehenden Abschnitten beschriebene CLS--Verfahren
(in der IR--\-Spek\-tros\-ko\-pie h\"{a}ufig auch "`\bf K\rm
--Matrix"' Methode genannt) erfordert, dass s\"{a}mtliche Komponenten,
die Beitr\"{a}ge zum zu analysierenden Gemischspektrums liefern,
bekannt sind, um eine verl\"{a}ssliche quantitative Analyse
sicherzustellen. Ist dies nicht der Fall, ist es in einfachen
F\"{a}llen unter Zuhilfenahme des Residuums m\"{o}glich, die
Spektralbereiche, in denen die unbekannte Komponente Strukturen
hat, auszublenden, die resultierenden Spektralbereiche zu einem
Vektor zusammenzufassen und mittels CLS auszuwerten.\\

Verfeinerte Methoden wie das \it Kalman--Filter \rm \cite{rutan84}
verwenden zur iterativen optimalen Auswahl der Spektralbereiche
der bekannten Komponenten die Kovarianzmatrix und das spektrale
Rauschen. Damit dieses Filter aber ad\"{a}quate Ergebnisse liefern
kann, m\"{u}ssen die im Algorithmus gew\"{a}hlten Startwerte f\"{u}r die
Kovarianzmatrix und das Rauschen sehr sorgf\"{a}ltig ausgew\"{a}hlt
werden. Eine Anwendung im Bereich der Luftanalytik, speziell
ausgerichtet auf die Bestimmung von L\"{o}sungsmitteld\"{a}mpfen, wurde
k\"{u}rzlich beschrieben \cite{gu98}. Die Ergebnisse sind, wie zu
erwarten, \"{a}quivalent zu denen des CLS--Algorithmus.\\

Ein anderes m\"{o}gliches Vorgehen besteht darin, Bande f\"{u}r Bande
mittels CLS auszuwerten, und die einzelnen Ergebnisse mit einer
statistisch leistungsf\"{a}higen Methode zusammenzufassen (siehe z.B.
\cite{haaland80}).\\

Angemerkt sei, dass die vom Lambert--Beer'schen Gesetz geforderte
lineare Abh\"{a}ngigkeit jedoch nicht immer gegeben ist, insbesondere
wenn beispielsweise die spektrale Aufl\"{o}sung im Vergleich zur
Halbwertsbreite der Absorptionsbanden nicht geeignet gew\"{a}hlt wurde
(siehe Kap. \ref{photometgen}). Stehen keine Referenzspektren zur
Verf\"{u}gung, die diese Nichtlinearit\"{a}ten mitber\"{u}cksichtigen (siehe
Kap. \ref{spekaufbereitung}), m\"{u}ssen bei der Auswertung dann
Modellerweiterungen vorgenommen werden, bei denen nichtlineare
Gleichungssysteme zu l\"{o}sen sind (siehe z.B. \cite{haaland87},
\cite{wuelbern98}).\\

Neben der direkten Least Squares--Methode CLS, d.h. als
mathematisches Modell wird das Lambert Beer'sche Gesetz
angenommen, ist es auch m\"{o}glich, das inverse Lambert Beer'sche
Gesetz (die Konzentration ist eine Funktion des zu analysierenden
Extinktionsspektrums) als Ausgangspunkt der Anpassung zu nehmen.
In diesem Fall wird vorausgesetzt, dass der Modellfehler in den
Konzentrationen liegt. Diese \bf I\rm nversive \bf L\rm east \bf
S\rm quares (ILS, oder auch \bf P\rm --Matrix)--Methode hat den
Vorteil, dass nicht alle Komponenten des Gemisches bekannt sein
m\"{u}ssen. Als Referenzspektren werden in diesem Fall im
Kalibrationsschritt Gemischspektren verwendet. Aufgrund der
vorzunehmenden Matrixinversion und der daraus resultierenden
Forderung von quadratischen Matrizen, kann die Anzahl der im
Spektrum ber\"{u}cksichtigten Punkte allerdings nicht gr\"{o}{\ss}er sein als
die Anzahl der verwendeten Gemischreferenzspektren, was wiederum
die Zahl der zur Verf\"{u}gung stehenden Punkte stark einschr\"{a}nkt.\\

Neben dem beschriebenen ILS werden Verfahren, die die Rolle der
abh\"{a}ngigen und unabh\"{a}ngigen Variablen im Lambert--Beer'schen
Gesetz vertauschen, statistische Verfahren genannt. PLS (\bf P\rm
artial \bf L\rm east \bf S\rm quares) geh\"{o}rt ebenfalls zu den
statistischen Verfahren, hat aber gegen\"{u}ber ILS den Vorteil, dass
dieselben breiten Spektralbereiche wie beim CLS genutzt werden
k\"{o}nnen, gleichzeitig aber nicht alle Komponenten des zu
analysierenden Gemisches bekannt sein m\"{u}ssen. PLS geh\"{o}rt zu den
Faktormethoden, d.h. die Kalibrationsmatrix wird hierbei in ein
neues Koordinatensystem transformiert, in dem die Richtungen
maximaler Varianz (PCR) die Achsen bilden oder die Kovarianz
zwischen Spektren-- und Konzentrationsvektor (PLS) den ersten
Vektor des orthogonalen Basisvektorsystems definiert und iterativ
die weiteren berechnet werden. Diese Matrix wird nun
rangreduziert, so dass nur eine reduzierte Anzahl von
Hauptkomponenten betrachtet werden. Die neuen Basisvektoren werden
bei dieser Methode \it loadings \rm und die neuen Intensit\"{a}ten \it
scores \rm genannt.\\

Als Referenzspektren dienen beim PLS i.a. Gemischspektren, die
allerdings die zu erwartende Varianz der
Konzentrationsunterschiede in den zu analysierenden Spektren voll
abdecken m\"{u}ssen. Extrapolationen sind nur beschr\"{a}nkt m\"{o}glich.
Daraus folgt ein sehr hoher Kalibrieraufwand, auch wenn bei
Gasspektren i.a. vorausgesetzt werden kann, dass keine
Wechselwirkungen der Komponenten untereinander stattfinden und
somit die Gemischspektren additiv aus Spektren der reinen
Komponenten berechnet werden k\"{o}nnen. Probleme kann es aber hier
dann auch geben, wenn die Extinktionswerte in den nichtlinearen
Bereich kommen.\\

Vorgenannte, nicht im Detail besprochene Methoden, werden
ausf\"{u}hrlich in Lehrb\"{u}chern der Chemometrie besprochen.
Beispielhaft seien \cite{massart88}, \cite{martens91} und als
deutsches Lehrbuch \cite{henrion95} genannt. Eine vergleichende
Abhandlung findet sich auch in \cite{haaland88}.\\

In der Literatur finden sich mehrere Beispiele f\"{u}r die Anwendung
von neuronalen Netzen bei der Strukturaufkl\"{a}rung von Molek\"{u}len aus
ihrem IR Spektrum (siehe z.B. \cite{gasteiger97}). Auch f\"{u}r die
Komponentenerkennung von FTIR--Gasspektren wurden neuronale Netze
eingesetzt \cite{yang98}. Hierbei muss allerdings angemerkt
werden, dass mit einer sehr schlechten Aufl\"{o}sung von 8 cm\up{-1}
gemessen wurde, so dass dort eine Trennung der dort gemessenen
Alkohole weitaus schwieriger ist als mit hochaufl\"{o}senden
Spektrometern mit 0.2 cm\up{-1} Aufl\"{o}sung.\\

F\"{u}r die quantitative Analyse von IR--Atmosph\"{a}renspektren mittels
neuronalen Netzen ist bislang allerdings nur eine Arbeit
erschienen (\cite{clerbaux95}), worin vertikale H\"{o}henprofile von
Konzentrationen atmosph\"{a}rischer Gase unter Verwendung von
IR--Spektren mit Hilfe von neuronalen Netzen berechnet werden. Der
sehr hohe Trainingsaufwand der Netze und fehlende Werkzeuge zur
Kontrolle der Qualitit\"{a}t der Ergebnisse lassen die neuronalen
Netze allerdings als nicht ad\"{a}quate Methode f\"{u}r die quantitative
Auswertung von IR--Spektren erscheinen.\\



\subsubsection{\label{vornachteilmulti}Vor- und Nachteile der
multivariaten Verfahren} \markright{VOR-- UND NACHTEILE DER
MULTIVARIATEN VERFAHREN}

S\"{a}mtliche multivariate Verfahren bieten gegen\"{u}ber dem univariaten
Verfahren den gro{\ss}en Vorteil, dass sie die Querempfindlichkeiten,
gegeben durch die \"{U}berlagerung von Absorptionsbanden der
verschiedenen Komponenten, ber\"{u}cksichtigen k\"{o}nnen und, wie oben
bereits angemerkt, verbesserte Nachweisgrenzen gegen\"{u}ber
univariaten Auswertungen mittels einzelner Absorptionsbanden
garantieren. So verbessert sich das Signal/Rauschverh\"{a}ltnis
grunds\"{a}tzlich mit $\sqrt{N}$ zur Anzahl der in die Auswertung
miteinbezogenen Punkte. Aufgrund der Mittelung \"{u}ber einen gro{\ss}en
Wellenzahlbereich, wirken sich Temperatureffekte nicht in dem Ma{\ss}e
auf die Qualit\"{a}t der Analyse aus wie bei der univariaten
Auswertung. Dennoch hat die pseudo--univariate Auswertung dort
ihre Bedeutung, wo eine multivariate Auswertung nicht m\"{o}glich ist.
Dies ist z.B. beim Benzol der Fall, eine \"{a}hnliche Vorgehensweise
sollte aber auch bei Toluol und den Xylolen bei der Auswertung im
Fingerprintbereich angestrebt werden. Unterst\"{u}tzend kann sie bei
den Molek\"{u}len, bei denen eine univariate Auswertung
querempfindlichkeitsfrei m\"{o}glich ist, neben der Kreuzkorrelation
bei der Komponentenerkennung eingesetzt werden, und eventuelle
systematische Fehler bei der multivariaten Analyse aufzeigen.
Desweiteren kann mit Hilfe der dabei erhaltenen
Konzentrationswerte die Referenzspektrenmatrix f\"{u}r die
CLS--Auswertung f\"{u}r diese Komponenten mit Spektren, deren
Konzentration nahe an der Gemischkonzentration liegt, aufgestellt
werden.\\

Methode der Wahl bei den multivariaten Verfahren ist die
klassische CLS--Methode. Der sehr gro{\ss}e Vorteil ist, dass die
CLS--Methode ohne gro{\ss}en rechnerischen Aufwand auch Sch\"{a}tzungen
f\"{u}r die Fehler auf der Grundlage der Regressionsresiduen und der
Inversen der Kovarianzmatrix liefert, so dass Aussagen zum
Vertrauensbereich der Messung gemacht werden k\"{o}nnen. Informationen
\"{u}ber Modellfehler (Nichtlinearit\"{a}ten, Interferenzen mit
unbekannten Komponenten und instrumentelle Messfehler) sind leicht
zug\"{a}nglich und f\"{u}r den Spektroskopiker verst\"{a}ndlich. Diese
unmittelbare Qualit\"{a}tskontrolle ist ein Werkzeug, dessen Bedeutung
sehr hoch einzusch\"{a}tzen ist. Probleme k\"{o}nnen bei schlecht
konditionierten Systemen auftreten, insbesondere wenn gro{\ss}e
\"{A}hnlichkeiten bei den Komponentenspektren bestehen. M\"{o}glichkeiten
zur L\"{o}sung der linearen Gleichungssysteme bestehen darin,
pseudoinverse Matrizen zu definieren, die mit heutigen
mathematischen Softwarepaketen berechnet werden k\"{o}nnen, in denen
leistungsf\"{a}hige Algorithmen zur Matrizeninvertierung zur Verf\"{u}gung
stehen \cite{golub89}.\\

Der Nachteil der CLS--Methode, dass s\"{a}mtliche Einzelkomponenten
des Gemischspektrums bekannt sein m\"{u}ssen, ist nicht sehr
gravierend. Die kommerziell erh\"{a}ltliche Datenbank QASoft (Kap.
\ref{qasoft}) enth\"{a}lt momentan IR--Spektren von 248 Gasen in den
Aufl\"{o}sungen von 0.125, 0.25, 0.5 und 1 cm\up{-1}, weitere sind in
anderen Datenbanken zu bekommen. Dass Abstriche bei der
quantitativen Genauigkeit dieser Spektren gemacht werden m\"{u}ssen,
wird an anderer Stelle diskutiert (Kap. \ref{bibliotheken}). In
dieser Arbeit wurde ein Labor zur Messung qualitativ sehr
hochwertiger IR--Spektren aufgebaut. Es ist mittlerweile in 1 1/2
Tagen m\"{o}glich, ein Gas zu vermessen und dabei auch die
nichtlinearen Anteile zu ber\"{u}cksichtigen (Kap.
\ref{kalibriergas}). Aus diesem Grund erfolgt in dieser Arbeit die
quantitative Analyse der FTIR--Atmosph\"{a}renspektren ausschlie{\ss}lich
mit der CLS--Methode und die multivariaten statistischen Methoden
werden nicht ausf\"{u}hrlicher behandelt.\\

In den Ausnahmef\"{a}llen, dass Spektren von im Gemisch enthaltenen
Einzelkomponenten nicht bekannt sind und das Herausschneiden
bestimmter Bereiche, das Kalman--Filter und die iterative
CLS--Auswertung einzelner Banden nicht zum Erfolg f\"{u}hren, kann man
sich der statistischen Modelle bedienen. Diese bringen jedoch, wie
oben beschrieben, f\"{u}r die Kalibrationsphase einen erheblichen
Aufwand mit sich, wenn die zu erwartende spektrale Varianz sehr
hoch ausf\"{a}llt, da diese vollst\"{a}ndig abgedeckt werden muss. Ein
gro{\ss}er Vorteil der statistischen Verfahren und eine \"{U}berlegenheit
zum CLS--Verfahren liegt dann vor, wenn starke
Querempfindlichkeiten, wie sie in festen und fl\"{u}ssigen Proben
gegeben sind, auftreten. Sie liefern dann verl\"{a}ssliche Ergebnisse
auch bei gro{\ss}en Konzentrationsunterschieden der einzelnen
Komponenten untereinander. Starke \"{U}berlagerungen sind in der
Gasanalytik z.B. bei gleichzeitiger Messung verschiedener
Kohlenwasserstoffe zu erwarten. Doch auch in einem solchen Fall
konnte die Anwendung des klassischen CLS--Verfahrens hervorragende
Ergebnisse liefern \cite{jaakola97}.\\

Applikationen mit statistischen Kalibrationsverfahren f\"{u}r die
Luftanalytik finden sich erst seit kurzem in der Literatur, wobei
haupts\"{a}chlich FT--Spektrometer mit geringer Aufl\"{o}sung zum Einsatz
kommen. So wurden von \cite{johansen97} PLS--Kalibrationen f\"{u}r die
Raumluftanalytik vorgestellt; die Liste der Substanzen umfasst
insgesamt 23 verschiedene Stoffe, f\"{u}r die jeweils
substanzspezifisch optimale spektrale Intervalle ermittelt wurden.
Der Kalibrieraufwand mit Reinstoffspektren ist enorm, wobei
insbesondere jeweils die extrem hohe Anzahl von PLS--Faktoren
anzumerken ist. Wellenzahlbereiche mit hohen Extinktionswerten,
z.B. gr\"{o}{\ss}er 1.0 f\"{u}r die Referenzspektren bei den maximal zu
erwartenden Konzentrationen, wurden von vornherein ausgeschlossen,
um den Einfluss von Nichtlinearit\"{a}ten in Abh\"{a}ngigkeit von der
Konzentration zu verringern. Die Basisliniendrift (konstante
Verschiebungen, bzw. lineare Basislinenfunktionen mit
unterschiedlicher Steigung) kann bei der Kalibrierung wie auch
beim CLS--Verfahren ebenfalls ber\"{u}cksichtigt und einkalibriert
werden. Grunds\"{a}tzlich muss festgehalten werden, dass wenn
s\"{a}mtliche Komponenten im Gemisch bekannt sind und keine
Modellfehler vorliegen, die statistischen Methoden der klassischen
nicht \"{u}berlegen sein k\"{o}nnen.

\cleardoublepage
