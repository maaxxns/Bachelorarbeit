\begin{appendix}

\markboth{ANHANG A}{ANHANG A}
\lhead[\fancyplain{}{\sl\thepage}]{\fancyplain{}{\sl\rightmark}}
\rhead[\fancyplain{}{\sl\leftmark}]{\fancyplain{}{\sl\thepage}}

\chapter*{\label{anhanga}Anhang A}
%\contentsline{chapter}{\numberline{\thechapter}Anhang A}{\thepage}
\addcontentsline{toc}{chapter}{Anhang A}

\newpage


\bildlinksanhang{htb}{ammonpg.wmf}{340}{570}{205}{90}{-35}{\underline{Prüfgas Ammon--}\
\underline{iak:}\\

\bf A \it univariate Auswertung bei 1046.5 cm\up{-1}. Die Dreiecke stellen die Messreihe 1, die Kreise
die Messreihe 2 dar (siehe auch Kap. \ref{messablauf} und \ref{pruefgasauswertung}).
Prüfgaskonzentration: 194 ppm (Messer--Griesheim). Die offenen Kreise zeigen die Ergebnisse der
Qualitätsüberprüfung nach einer Messreihe bzw. den Referenzwert der QASoft--Spektrenbibliothek.\\

\bf B a \it -- linearer, \bf b \it -- quadratischer und \bf c \it -- kubischer Term für 203 ppm$\cdot$m (siehe Kap.
\ref{pruefgasauswertung}).\\

\bf C \it Residuum: be\-rech\-ne\-tes -- ge\-mess\-en\-es Spektrum für 203 ppm$\cdot$m.}



\bildlinksanhang{htb}{acetalpg.wmf}{340}{570}{205}{90}{-35}{\underline{Prüfgas Acet\-al\-de--}\
\underline{hyd:}\\

\bf A \it univariate Auswertung bei 1762 cm\up{-1}. Die Dreiecke stellen die Messreihe 1, die Kreise
die Messreihe 2 dar (siehe auch Kap. \ref{messablauf} und \ref{pruefgasauswertung}).
Prüfgaskonzentration: 422 ppm (Messer--Griesheim). Die offenen Kreise zeigen die Ergebnisse der
Qualitätsüberprüfung nach einer Messreihe.\\

\bf B a \it -- linearer, \bf b \it -- quadratischer und \bf c \it -- kubischer Term für 266 ppm$\cdot$m (siehe Kap.
\ref{pruefgasauswertung}).\\

\bf C \it Residuum: be\-rech\-ne\-tes -- ge\-mess\-en\-es Spektrum für 266 ppm$\cdot$m.}



\bildlinksanhang{htb}{n2opg.wmf}{340}{570}{205}{90}{-35}{\underline{Prüfgas Di\-stick--}\
\underline{stoffoxid:}\\

\bf A \it univariate Auswertung bei 2234.9 cm\up{-1}. Die Dreiecke stellen die Messreihe 1, die Kreise
die Messreihe 2 dar (siehe auch Kap. \ref{messablauf} und \ref{pruefgasauswertung}).
Prüfgaskonzentration: 206 ppm (Messer--Griesheim). Die offenen Kreise zeigen die Ergebnisse der
Qualitätsüberprüfung nach einer Messreihe bzw. den Referenzwert der QASoft--Spektrenbibliothek.\\

\bf B a \it -- linearer, \bf b \it -- quadratischer und \bf c \it -- kubischer Term für 215 ppm$\cdot$m (siehe Kap.
\ref{pruefgasauswertung}).\\

\bf C \it Residuum: be\-rech\-ne\-tes -- ge\-mess\-en\-es Spektrum für 215 ppm$\cdot$m.}




\bildlinksanhang{htb}{ethenpg.wmf}{340}{570}{205}{90}{-35}{\underline{Prüfgas Ethen:}\\

\bf A \it univariate Auswertung bei 949.5 cm\up{-1}. Die Dreiecke stellen die Messreihe 1, die Kreise
die Messreihe 2 dar (siehe auch Kap. \ref{messablauf} und \ref{pruefgasauswertung}).
Prüfgaskonzentration: 291 ppm (Messer--Griesheim). Die offenen Kreise zeigen die Ergebnisse der
Qualitätsüberprüfung nach einer Messreihe bzw. den Referenzwert der QASoft--Spektrenbibliothek.\\

\bf B a \it -- linearer, \bf b \it -- quadratischer und \bf c \it -- kubischer Term für 243 ppm$\cdot$m (siehe Kap.
\ref{pruefgasauswertung}).\\

\bf C \it Residuum: be\-rech\-ne\-tes -- ge\-mess\-en\-es Spektrum für 243 ppm$\cdot$m.}




\bildlinksanhang{htb}{isobutpg.wmf}{340}{570}{205}{90}{-35}{\underline{Prüfgas Iso\-but\-en:}\\

\bf A \it univariate Auswertung bei 890.5 cm\up{-1}. Die Dreiecke stellen die Messreihe 1, die Kreise
die Messreihe 2 dar (siehe auch Kap. \ref{messablauf} und \ref{pruefgasauswertung}).
Prüfgaskonzentration: 302 ppm (Messer--Griesheim). Die offenen Kreise zeigen die Ergebnisse der
Qualitätsüberprüfung nach einer Messreihe bzw. den Referenzwert der QASoft--Spektrenbibliothek.\\

\bf B a \it -- linearer, \bf b \it -- quadratischer und \bf c \it -- kubischer Term für 316 ppm$\cdot$m (siehe Kap.
\ref{pruefgasauswertung}).\\

\bf C \it Residuum: be\-rech\-ne\-tes -- ge\-mess\-en\-es Spektrum für 316 ppm$\cdot$m.}



\bildlinksanhang{htb}{copg.wmf}{340}{570}{205}{90}{-35}{\underline{Prüfgas Kohl\-en--}\
\underline{stoffmonoxid:}\\

\bf A \it univariate Auswertung bei 2169.3 cm\up{-1}. Die Dreiecke stellen die Messreihe 1, die Kreise
die Messreihe 2 dar (siehe auch Kap. \ref{messablauf} und \ref{pruefgasauswertung}).
Prüfgaskonzentration: 207 ppm (Messer--Griesheim). Die offenen Kreise zeigen die Ergebnisse der
Qualitätsüberprüfung nach einer Messreihe bzw. den Referenzwert der QASoft--Spektrenbibliothek.\\

\bf B a \it -- linearer, \bf b \it -- quadratischer und \bf c \it -- kubischer Term für 131 ppm$\cdot$m (siehe Kap.
\ref{pruefgasauswertung}).\\

\bf C \it Residuum: be\-rech\-ne\-tes -- ge\-mess\-en\-es Spektrum für 131 ppm$\cdot$m.}



\bildlinksanhang{htb}{co05pg.wmf}{340}{570}{205}{90}{-35}{\underline{Prüfgas Kohl\-en--}\
\underline{stoffmonoxid,}\\ \underline{(0.5 cm\up{-1} Auf--}\\ \underline{l�sung):}\\

\bf A \it univariate Auswertung bei 2169.3 cm\up{-1}. Die Dreiecke stellen die Messreihe 1, die Kreise
die Messreihe 2 dar (siehe auch Kap. \ref{messablauf} und \ref{pruefgasauswertung}).
Prüfgaskonzentration: 207 ppm (Messer--Griesheim). Die offenen Kreise zeigen die Ergebnisse der
Qualitätsüberprüfung nach einer Messreihe. Die Kreuze zeigen noch einmal die Ergebnisse der univariaten
Auswertung der Bande bei 2169.3 cm\up{-1} für 0.2 cm\up{-1} aus Abbildung \ref{copg.wmf}.\\

\bf B \it und \bf C \it siehe Abbildung CO, 0.2 cm\up{-1} Aufl�sung.}


\bildlinksanhang{htb}{methanpg.wmf}{340}{570}{205}{90}{-35}{\underline{Prüfgas Methan:}\\

\bf A \it univariate Auswertung bei 3086 cm\up{-1}. Die Dreiecke stellen die Messreihe 1, die Kreise
die Messreihe 2 dar (siehe auch Kap. \ref{messablauf} und \ref{pruefgasauswertung}).
Prüfgaskonzentration: 315 ppm (Messer--Griesheim). Die offenen Kreise zeigen die Ergebnisse der
Qualitätsüberprüfung nach einer Messreihe bzw. den Referenzwert der QASoft--Spektrenbibliothek.\\

\bf B a \it -- linearer, \bf b \it -- quadratischer und \bf c \it -- kubischer Term für 261 ppm$\cdot$m (siehe Kap.
\ref{pruefgasauswertung}).\\

\bf C \it Residuum: be\-rech\-ne\-tes -- ge\-mess\-en\-es Spektrum für 261 ppm$\cdot$m.}



\bildlinksanhang{htb}{methnlpg.wmf}{340}{570}{205}{90}{-35}{\underline{Prüfgas Metha--}\
\underline{nol:}\\

\bf A \it univariate Auswertung bei 1033.5 cm\up{-1}. Die Dreiecke stellen die Messreihe 1, die Kreise
die Messreihe 2 dar (siehe auch Kap. \ref{messablauf} und \ref{pruefgasauswertung}).
Prüfgaskonzentration: 299 ppm (Messer--Griesheim). Die offenen Kreise zeigen die Ergebnisse der
Qualitätsüberprüfung nach einer Messreihe bzw. den Referenzwert der QASoft--Spektrenbibliothek.\\

\bf B a \it -- linearer, \bf b \it -- quadratischer und \bf c \it -- kubischer Term für 315 ppm$\cdot$m (siehe Kap.
\ref{pruefgasauswertung}).\\

\bf C \it Residuum: be\-rech\-ne\-tes -- ge\-mess\-en\-es Spektrum für 315 ppm$\cdot$m.}



\bildlinksanhang{htb}{propenpg.wmf}{340}{570}{205}{90}{-35}{\underline{Prüfgas Propen:}\\

\bf A \it univariate Auswertung bei 912.5 cm\up{-1}. Die Dreiecke stellen die Messreihe 1, die Kreise
die Messreihe 2 dar (siehe auch Kap. \ref{messablauf} und \ref{pruefgasauswertung}).
Prüfgaskonzentration: 300 ppm (Messer--Griesheim). Die offenen Kreise zeigen die Ergebnisse der
Qualitätsüberprüfung nach einer Messreihe bzw. den Referenzwert der QASoft--Spektrenbibliothek.\\

\bf B a \it -- linearer, \bf b \it -- quadratischer und \bf c \it -- kubischer Term für 315 ppm$\cdot$m (siehe Kap.
\ref{pruefgasauswertung}).\\

\bf C \it Residuum: be\-rech\-ne\-tes -- ge\-mess\-en\-es Spektrum für 315 ppm$\cdot$m.}




\bildlinksanhang{htb}{pxylpg.wmf}{340}{570}{205}{90}{-35}{\underline{Prüfgas p--\-Xyl\-ol:}\\

\bf A \it univariate Auswertung bei 795.5 cm\up{-1}. Die Dreiecke stellen die Messreihe 1, die Kreise
die Messreihe 2 dar (siehe auch Kap. \ref{messablauf} und \ref{pruefgasauswertung}).
Prüfgaskonzentration: 63.8 ppm (Messer--Griesheim). Die offenen Kreise zeigen die Ergebnisse der
Qualitätsüberprüfung nach einer Messreihe bzw. den Referenzwert der QASoft--Spektrenbibliothek.\\

\bf B a \it -- linearer, \bf b \it -- quadratischer und \bf c \it -- kubischer Term für 236 ppm$\cdot$m (siehe Kap.
\ref{pruefgasauswertung}).\\

\bf C \it Residuum: be\-rech\-ne\-tes -- ge\-mess\-en\-es Spektrum für 236 ppm$\cdot$m.}




\bildlinksanhang{htb}{so2pg.wmf}{340}{570}{205}{90}{-35}{\underline{Prüfgas Schwe\-fel--}\
\underline{dioxid:}\\


\bf A \it univariate Auswertung bei 1347 cm\up{-1}. Die Dreiecke stellen die Werte der Messreihe 1
dar.
Prüfgaskonzentration: 21.2 ppm (Messer--Griesheim). Die offenen Kreise zeigen das Ergebnis der
Qualitätsüberprüfung nach der Messreihe bzw. den Referenzwert der QASoft--Spektrenbibliothek.\\

\bf B a \it -- linearer, \bf b \it -- quadratischer und \bf c \it -- kubischer Term für 147 ppm$\cdot$m (siehe Kap.
\ref{pruefgasauswertung}).\\

\bf C \it Residuum: be\-rech\-ne\-tes -- ge\-mess\-en\-es Spektrum für 147 ppm$\cdot$m.}




\bildlinksanhang{htb}{no2pg.wmf}{340}{570}{205}{90}{-35}{\underline{Prüfgas Stick--}\
\underline{stoffdioxid:}\\

\bf A \it univariate Auswertung bei 1630.3 cm\up{-1}. Die Dreiecke stellen die Messreihe 1, die Kreise
die Messreihe 2 dar (siehe auch Kap. \ref{messablauf} und \ref{pruefgasauswertung}).
Prüfgaskonzentration: 204 ppm (Messer--Griesheim). Die offenen Kreise zeigen die Ergebnisse der
Qualitätsüberprüfung nach einer Messreihe bzw. den Referenzwert der QASoft--Spektrenbibliothek.\\

\bf B a \it -- linearer, \bf b \it -- quadratischer und \bf c \it -- kubischer Term für 216 ppm$\cdot$m (siehe Kap.
\ref{pruefgasauswertung}).\\

\bf C \it Residuum: be\-rech\-ne\-tes -- ge\-mess\-en\-es Spektrum für 216 ppm$\cdot$m.}



\bildlinksanhang{htb}{toluolpg.wmf}{340}{570}{205}{90}{-35}{\underline{Prüfgas Toluol:}\\

\bf A \it univariate Auswertung bei 694 cm\up{-1}. Die Dreiecke stellen die Messreihe 1, die Kreise
die Messreihe 2 dar (siehe auch Kap. \ref{messablauf} und \ref{pruefgasauswertung}).
Prüfgaskonzentration: 122 ppm (Messer--Griesheim). Die offenen Kreise zeigen die Ergebnisse der
Qualitätsüberprüfung nach einer Messreihe bzw. den Referenzwert der QASoft--Spektrenbibliothek.\\

\bf B a \it -- linearer, \bf b \it -- quadratischer und \bf c \it -- kubischer Term für 445 ppm$\cdot$m (siehe Kap.
\ref{pruefgasauswertung}).\\

\bf C \it Residuum: be\-rech\-ne\-tes -- ge\-mess\-en\-es Spektrum für 445 ppm$\cdot$m.}




\bildlinksanhang{htb}{benz1pg.wmf}{340}{570}{205}{90}{-35}{\underline{Prüfgas Benzol:}\\

univariate Auswertung \bf A \it bei 674 cm\up{-1} und \bf B \it bei 3047 cm\up{-1}.
Die Dreiecke stellen die Messreihe 1, die Kreise
die Messreihe 2 dar (siehe auch Kap. \ref{messablauf} und \ref{pruefgasauswertung}).
Prüfgaskonzentration: 296 ppm (Messer--Griesheim). Die offenen Kreise zeigen die Ergebnisse der
Qualitätsüberprüfung nach einer Messreihe bzw. den Referenzwert der QASoft--Spektrenbibliothek.\\

\bf C a \it -- linearer, \bf b \it -- quadratischer und \bf c \it -- kubischer Term für 62 ppm$\cdot$m (siehe Kap.
\ref{pruefgasauswertung}).\\

\bf D \it Residuum: be\-rech\-ne\-tes -- ge\-mess\-en\-es Spektrum für 62 ppm$\cdot$m.}


\bildlinksohnetext{htb}{benz2pg.wmf}{340}{570}{205}{90}{-35}


\end{appendix}

\cleardoublepage
