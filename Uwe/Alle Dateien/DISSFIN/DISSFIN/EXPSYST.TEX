\chapter{\label{expertensystem}Erstellung eines Expertensystems}

Um Routinemessungen h\"{o}chst zuverl\"{a}ssig auch ohne
hochqualifiziertes Fachpersonal durchf\"{u}hren zu k\"{o}nnen, muss eine
Software vorhanden sein, die das bekannte Expertenwissen ohne
Benutzerinteraktion zur Spektrenauswertung anwendet. Aufgabe in
dieser Arbeit war es, die gewonnenen Erkenntnisse zur Verbesserung
der univariaten und multivariaten Analyse gleichzeitig auch zur
Entwicklung eines Expertensystems zu nutzen.\\

Diese Software wurde mit Hilfe des Mathematikprogramms MATLAB 5.2
(\bf Mat\rm rix \bf Lab\rm oratory, The Mathworks Inc., South
Natick, MA) erstellt, welches f\"{u}r numerische Matrizenoperationen
hervorragend geeignet ist. Ein Compiler erlaubt die schnelle
Umsetzung der an die Computersprache C angelehnten
MATLAB-Programmiersprache in C--Quellcode. Die MATLAB-Funktionen
zur numerischen Matrizenberechnung stehen in einer
C--Math--Library zur Verf\"{u}gung. Somit war es schnell und
unproblematisch m\"{o}glich, Teile des MATLAB-Programms zur off--line
Auswertung in die GRAMS--Spektrometersoftware des
K300--Spektrometers, mit der gleichzeitig eine on--line Analyse
m\"{o}glich ist, zu \"{u}bernehmen.\\

Das Programm umfasst mittlerweile 88 Unterfunktionen (in der
MATLAB-Sprache "`m-files"') mit insgesamt 10100 Zeilen
Programmcode. Ziel der Software ist es, s\"{a}mtliche Schritte zu
einer hochwertigen quantitativen Analyse von Atmosph\"{a}renspektren
vorzunehmen, wenn gewollt ohne Benutzerinteraktion.\\

Zu Anfang des Programmcodes k\"{o}nnen in einer Eingabemaske s\"{a}mtliche
zur Auswertung ben\"{o}tigten Parameter gesetzt werden. Es kann auch
der nicht--automatische Mode gestartet werden, in dem diese
Parameter (z.B. Filenamen, Pfadl\"{a}nge, Auswertungsart) vor jeder
Auswertung abgefragt werden.\\

Das Programm gliedert sich in die vier Hauptteile:
Spektrenvorbearbeitung, Auswahl der Referenzspektren, Auswertung
und Darstellung, und Ermitteln und Abspeichern der besten
statistischen Ergebnisse (siehe Abb. \ref{softw1.wmf}).\\

\bild{htb}{softw1.wmf}{400}{300}{\it \"{U}bersichtsdiagramm zur
CLS-Auswertungssoftware.}


\section{\label{spektrenvorbearb}Spektrenvorbearbeitung}

Es hat sich gezeigt, dass mit den verwendeten MCT-Detektoren f\"{u}r
eine hochwertige Analyse eine Nichtlinearit\"{a}tskorrektur
unerl\"{a}sslich ist. Dazu m\"{u}ssen die Interferogramme der Spektren
vorliegen. In der Spektrometersoftware des K300--Spektrometers ist
die Abspeicherung der Interferogramme mittlerweile vorgesehen.
Desweiteren findet nach der Fouriertransformation in der
Spektrendom\"{a}ne eine Anpassung der Sensitivit\"{a}t des Detektors
statt, indem im auszuwertenden Einkanalspektrum der Q--Zweig der
durchabsorbierenden CO\down{2}--Bande bei 667 cm\up{-1} auf das
Niveau des dazugeh\"{o}rigen Eigenstrahlspektrums gebracht wird. Diese
Korrekturen liegen in der Gr\"{o}{\ss}enordnung von 1 \%. Da die
Nichtlinearit\"{a}tskorrektur recht zeitaufwendig ist, empfiehlt es
sich, diese Vorbearbeitung der Spektren im vorhinein zu rechnen
und die korrigierten Spektren abzuspeichern. Anschlie{\ss}end findet
der f\"{u}r bistatische Spektrometersysteme notwendige
Eigenstrahlungsabzug statt.\\

Vor der weiteren Bearbeitung der Spektren ist eine
Qualit\"{a}ts\"{u}berpr\"{u}fung notwendig. Grunds\"{a}tzlich wird mit dem
Programm das MATLAB eigene bin\"{a}re mat--Format wie auch g\"{a}ngige
ASCII--Files ohne Header unterst\"{u}tzt. Diese k\"{o}nnen sowohl nur als
Einkanalspektren als auch als Einkanalspektren mit
Wellenzahlachse, wie auch als bereits vorher bearbeitete
Extinktionsspektren vorliegen. Da die Referenzspektren f\"{u}r die
multivariate und Absorptionskoeffizienten f\"{u}r die univariate
Auswertung z.Z. nur f\"{u}r 0.2 cm\up{-1}--Aufl\"{o}sung und
Dreiecksapodisation vorliegen, sollten auch nur solche Spektren
ausgewertet werden. Da die in dieser Arbeit gemessenen Pr\"{u}fgase
aber als Interferogramme vorliegen, ist es in Zukunft auch
grunds\"{a}tzlich m\"{o}glich, Spektren mit anderen Apodisationsfunktionen
und schlechteren Aufl\"{o}sungen auszuwerten. Als Punktabstand sind
die f\"{u}r das K300--Spektrometer charakteristischen 0.0904 cm\up{-1}
notwendig. Bei Einkanalspektren wird weiterhin bei den
Wellenzahlen 1098 und 2450 cm\up{-1} ein Signalh\"{o}hentest
vorgenommen, um eventuell schadhafte Spektren auszusortieren.
Diese Ergebnisse der Qualit\"{a}ts\"{u}berpr\"{u}fung werden wie alle
folgenden Schritte bei der Auswertung der Spektren in einem
"`filename"'.log--File dokumentiert.\\

Grunds\"{a}tzlich ist es m\"{o}glich, dass zwischen den auszuwertenden
Spektren und den Referenzspektren ein Wellenzahlshift auftritt.
Diese Feinabstimmung ist vor allem f\"{u}r die pr\"{a}zise Auswertung sehr
schmaler Rotations--/Schwingungsbanden von kleinen Molek\"{u}len
notwendig (siehe auch Kap. \ref{wellenzahlstab} und Tab.
\ref{wellenzahlgenauigkeit}). Um diesen zu detektieren, werden die
immer vorhandenen querempfindlichkeitsfreien Wasserbanden bei
784.46, 1014.48, 1149.47, 1187.02, 3969,08 und 4181.48 cm\up{-1}
bestimmt. Dies geschieht durch eine Schwerpunktsbestimmung dieser
Banden, wobei die Abtastung der einzelnen Linien durch einen
Zerofill 8 in der Fourierdom\"{a}ne noch verbessert wird (siehe auch
Abb. \ref{h2ouni.wmf} zur univariaten Auswertung). Aus dem
Quotienten der ermittelten Verschiebung durch die genaue
Wellenzahl wird ein relativer Wert bestimmt, der \"{u}ber s\"{a}mtliche
o.g. Banden gemittelt wird. Mit diesem ist es nun m\"{o}glich, f\"{u}r
jeden Wellenzahlbereich im Spektrum die entsprechende
Wellenzahlverschiebung zu eliminieren. F\"{u}r die exakte Position der
o.g. Banden liegen sowohl Literaturwerte wie auch die
HITRAN96--Werte vor. Solange aber die meisten Referenzspektren
noch der QASoft--Datenbank entnommen werden, sind die Werte des
H\down{2}O--Spektrums aus dieser Datenbank relevant. Leider hat
sich gezeigt, dass es auch kleine Unterschiede in der
Wellenzahlverschiebung bei verschiedenen QASoft--Referenzspektren
untereinander gibt. Die Auswirkungen bei der Auswertung sind
allerdings nur sehr gering. Grunds\"{a}tzliches Vorgehen muss es in
Zukunft sein, selbstgemessene Pr\"{u}fgasspektren auf die
Literaturwerte anzupassen.\\

Da die Benzolauswertung aufwendiger als bei anderen Stoffen ist
und zus\"{a}tzliche Spektren bereit stehen m\"{u}ssen (siehe Kap.
\ref{benzol}), macht es Sinn, im vorhinein durch eine vereinfachte
Prozedur qualitativ abzukl\"{a}ren, ob Benzol im Gemisch enthalten
ist. Optional kann dieser Test durchgef\"{u}hrt werden und der
relevante Wellenzahlbereich nach CO\down{2}--Abzug, wenn
gew\"{u}nscht, auch graphisch dargestellt werden.\\


\section{\label{referenzauswahl}Auswahl der Referenzspektren}

F\"{u}r eine CLS-Analyse werden von s\"{a}mtlichen in der Stoffmatrix
enthaltenen Gasen, die in den auszuwertenden Bereichen
signifikante Strukturen besitzen, Referenzspektren ben\"{o}tigt. Wird
nur eine univariate Analyse des Gemisches durchgef\"{u}hrt, kann auf
diese Auswahl verzichtet werden.\\

Ist die Zusammensetzung der Stoffmatrix bekannt, k\"{o}nnen in dem
Auswertungsprogramm auf MATLAB--Basis die einzuladenden
Referenzspektren festgelegt werden. F\"{u}r die Gase, die im Rahmen
dieser Arbeit vermessen wurden, wird f\"{u}r eine mittlere
Konzentration das Referenzspektrum aus dem linearen, dem
quadratischen und dem kubischen Anteil gerechnet (siehe Kap.
\ref{pruefgasauswertung}). Zeigen sich nach der anschlie{\ss}enden
Auswertung im Residuum noch Strukturen aufgrund vorhandener
Nichtlinearit\"{a}ten, werden in einem zweiten Iterationsschritt die
betroffenen Referenzspektren mit der im ersten Iterationsschritt
erhaltenen Konzentration gerechnet. Im allgemeinen Fall werden
somit die Nichtlinearit\"{a}ten bestm\"{o}glich ber\"{u}cksichtigt. Weitere
Iterationsschritte sind zus\"{a}tzlich jedoch problemlos m\"{o}glich.\\

Grunds\"{a}tzlich sollte die Anzahl der Referenzspektren der Anzahl
der Gase in der Stoffmatrix entsprechen. Weniger Referenzspektren
f\"{u}hren, sofern sie Querempfindlichkeiten zu den anderen Gasen
besitzen, offensichtlich zu systematisch falschen Ergebnissen der
analysierten Gase. Mehr Referenzspektren bergen die Gefahr, dass
Strukturen im Gasgemisch von Referenzspektren mit angepasst
werden, die in der Stoffmatrix nicht enthalten sind. Dies ist vor
allem oft bei sehr breiten Banden von bestimmten Stoffen der
Fall.\\

F\"{u}r unbekannte Stoffmatrizes werden daher zwei leistungsstarke
Hilfsprogramme zur Verf\"{u}gung gestellt. F\"{u}r den Spektroskopiker
wird zus\"{a}tzlich ein Paket angeboten, in dem f\"{u}r jeden in der
Referenzdatenbank enthaltenen Stoff der Bereich mit dessen gr\"{o}{\ss}ter
Signatur (vorausgesetzt es gibt keine schwerwiegende H\down{2}O-
oder CO\down{2}-Querempfindlichkeit) herausgesucht wird, und
Gemisch- und jeweiliges Referenzspektrum normiert \"{u}bereinander
dargestellt werden, so dass eine visuelle Auswahl der einzelnen
Referenzspektren m\"{o}glich ist.\\

Alternativ kann eine Kreuzkorrelation durchgef\"{u}hrt werden (siehe
Kap. \ref{crosscorr}), die f\"{u}r \"{u}bliche Atmosph\"{a}ren- und
Konzentrationsunterschiede sehr gute Ergebnisse liefert. Da diese
Kreuzkorrelation recht zeitintensiv ist (ca. 2 min (P200) f\"{u}r die
implementierten 36 Stoffe), sollte dann nat\"{u}rlich f\"{u}r folgende
Spektren einer Messreihe \"{u}berlegt werden, ob die Stoffmatrix sich
ge\"{a}ndert hat oder die Referenzspektren nun vorgegeben werden
k\"{o}nnen.\\


\section{\label{darstellung}Auswertung und Darstellung}

Zur Auswertung der Spektren stehen 4 Module zur Auswahl. Diese
k\"{o}nnen einzeln wie auch nebeneinander benutzt werden. Die
univariate Auswertung in der MATLAB-Software ist f\"{u}r die 11 Gase
H\down{2}O, CO\down{2}, N\down{2}O, CO, CH\down{4}, NH\down{3},
C\down{2}H\down{6}, SF\down{6}, C\down{6}H\down{6},
C\down{2}H\down{4} und C\down{3}H\down{6} m\"{o}glich. Zur Auswertung
werden f\"{u}r jedes Gas mehrere Banden in den logarithmierten
Einkanalspektren herangezogen. Diese Banden sind so ausgesucht
worden, dass sie querempfindlichkeitsfrei sind. Ist das nicht der
Fall, so werden verfeinerte Strategien verfolgt, die diese
Querempfindlichkeiten ber\"{u}cksichtigen (siehe auch Kap.
\ref{univariat}). Warnhinweise im log-File machen zudem auf die
Problematik aufmerksam. Die Abtastung wird wie bei der
Wellenzahlkorrektur durch einen Zerofill 8 in der Fourierdom\"{a}ne
verbessert, die Basislinie l\"{a}sst sich gut durch einen linearen Fit
in den Minima der Bandenflanken anpassen. Der
Auswertungsalgorithmus liefert den Schwerpunkt, das Maximum und
die Halbwertsbreite der Bande. Die Auswertung der einzelnen Banden
kann auch graphisch dargestellt und ausgedruckt werden (siehe Abb.
\ref{h2ouni.wmf}). Die Absorptionskoeffizienten wurden aus den
Referenzspektren der QASoft--Datenbank gewonnen. Diese gelten
strenggenommen nur f\"{u}r die Temperatur (meist 25�C) bei Aufnahme
der Spektren und Abweichungen von diesen bei Feldmessungen k\"{o}nnen
im Einzelfall zu signifikanten systematischen Fehlern f\"{u}hren
(siehe auch Kap. \ref{austempdr}). Da aber das multivariate
Analysenkonzept, wie schon ausf\"{u}hrlich beschrieben, deutliche
Vorteile gegen\"{u}ber der univariaten Auswertung hat, wurden keine
weiteren Anstrengungen unternommen, die Absorptionskoeffizienten
auch temperaturabh\"{a}ngig zu bestimmen, abgesehen von den
vorgestellten Studien basierend auf den HITRAN--Rechnungen.\\

\bildlinks{htb}{h2ouni.wmf}{300}{220}{75}{85}{-30}{\it Beispiel
einer graphischen Darstellung bei der univariaten Auswertung. Die
Abtastung der Bande wurde durch ein Zerofill 8 im Interferogramm
verbessert.}

Die univariate Auswertung bietet vor allem dann Vorteile, wenn
Komponenten aus der Stoffmatrix nicht als Referenzspektren
vorliegen. Dann k\"{o}nnen ggf. bekannte Komponenten univariat
bestimmt werden. Ansonsten ist in der Praxis eine parallele
Analyse zum Herausfinden systematischer Fehler sinnvoll. Da Benzol
mittels CLS nicht bestimmt werden kann, ist dort die univariate
Analyse unerl\"{a}sslich. Bei der Benzolanalyse wird zus\"{a}tzlich ein
Kurzweg-Referenzspektrum und ein Hintergrundspektrum mit ann\"{a}hernd
gleicher CO\down{2}--Konzentration wie im zu analysierenden
Spektrum ben\"{o}tigt. Dabei ist darauf zu achten, dass die
CO\down{2}--Konzentrationen m\"{o}glichst genau \"{u}bereinstimmen. Bei
Abweichungen um mehr als 10\% ist eine Auswertung i.a. nicht mehr
m\"{o}glich. Der genaue Algorithmus wurde in Kap. \ref{benzol}
beschrieben.\\

Die multivariate CLS-Auswertung erfordert ein Hintergrundspektrum
zur Extinktionsspektrenberechnung. Liegt kein gemessenes
Hintergrundspektrum vor, k\"{o}nnen optional auch ein "`ideales"'
Laborhintergrundspektrum oder ein berechneter Hintergrund benutzt
werden. Ausf\"{u}hrlich wurde diese Problematik in Kap.
\ref{hintergrund} diskutiert.\\

\bildlinks{htb}{sf6temp.wmf}{275}{340}{115}{75}{-35}{\\ \bf A \it
Laborspektrum SF\down{6} (T=25�C, 1.4 ppm $\cdot$ m, opt.
Pfadl\"{a}nge 6.9 m, mit Offset) und darauf skaliertes Feldspektrum
(T=7.3�C, opt. Pfadl\"{a}nge 119 m).\\

\bf B \it Spektrale Residuen eines Feld- und Laborspektrums
(Temperaturen siehe \bf A\it ) nach der CLS--Auswertung mit dem
SF\down{6}--Referenzspektrum (T=25�C, QSoft). Die Residuen des
Feldspektrums sind mit einem Offset dargestellt.}

Es stehen z.Z. drei M\"{o}glichkeiten zur CLS--Auswertung mit
mittlerweile 36 Stoffen zur Verf\"{u}gung, die auch nebeneinander
genutzt werden k\"{o}nnen. Ihnen gemeinsam ist, dass sie auf denselben
CLS--Auswertungs--Kernel zur\"{u}ckgreifen, der zus\"{a}tzlich zur
eigentlichen Konzentrationsbestimmung mittels Least--Squares noch
eine Gram--Schmidt--Orthogonalisierung der Referenzspektrenmatrix
vornimmt. Neben einer verbesserten numerischen Stabilit\"{a}t (vor
allem bei der Invertierung der Matrix), k\"{o}nnen so auch in Zukunft
Spektren zur Referenzspektrenmatrix hinzugef\"{u}gt werden, ohne dass
die Berechnungen (Matrixinvertierung) f\"{u}r das gesamte System noch
einmal begonnen werden m\"{u}ssen. Vor der CLS-Analyse werden die
Bereiche, die Extinktionen \"{u}ber ein vom Benutzer zu setzendes
Limit (empfohlen: Extinktion 1) \"{u}berschreiten, eliminiert, um
eventuell noch vorhandene Nichtlinearit\"{a}ten (vor allem bei nicht
selbst gemessenen Referenzspektren) nicht in die Auswertung
miteinbeziehen zu m\"{u}ssen.\\

Aufgrund der Logarithmierung der Transmissionspektren ist das
Rauschen in den Extinktionsmaxima gr\"{o}{\ss}er als in den Flanken. Dies
kann mit einer gewichteten Analyse ber\"{u}cksichtigt werden. Die
Temperaturproblematik besteht auch bei der CLS-Analyse, i.a.
werden solche Effekte aber bei Rotationsfeinstruktur \"{u}ber den
gesamten Bereich oft gemittelt (siehe Abb. \ref{cotemp.wmf}).
Abbildung \ref{sf6temp.wmf} zeigt die Auswirkungen f\"{u}r
Atmosph\"{a}renspektren mit SF\down{6} bei den unterschiedlichen
Temperaturen 7.3�C und 25.0�C. Beide Spektren wurden mit einem
Laborspektrum, bei 25�C aufgenommen, angepasst. Im Residuum f\"{u}r
das bei 7.3�C erhaltene Spektrum ist noch eine deutliche Struktur
zu sehen, die aufgrund der Temperaturdifferenz zum
Referenzspektrum nicht mit angepasst werden konnte. Mit einer
spektralen Verschiebung von 0.07 cm\up{-1} konnte die Auswertung
noch verbessert werden (siehe Residuum mittleres Spektrum in Abb.
\ref{sf6temp.wmf} \bf B\rm ).
\\


\subsection{\label{segmentierung}Segmentierung von
Spektralbereichen}

Unterschiede in den verschiedenen CLS--Modulen bestehen in der
Bereichsauswahl und einer weiteren Selektierung der
Referenzspektrenmatrix. Der Experte kann einen Bereich vorgeben,
in dem er das Spektrum ausgewertet haben m\"{o}chte. S\"{a}mtliche bei der
Auswahl der Referenzspektren festgelegten Spektren werden dann
auch in die Auswertung miteinbezogen.\\

\bild{htb}{seg3.wmf}{450}{280}{\it Beispiel der
Bildschirmdarstellung des Segments Nr.3 bei der CLS--Auswertung.
Sowohl die Referenz-- als auch das auszuwertende Gemischspektrum
werden gezeigt.}

Die anderen beiden Auswertungsmodi geben die auszuwertenden
Bereiche automatisch vor (zur Bereichsauswahl siehe auch Kap.
\ref{bereichsauswahl}). So kann eine Analyse in 6 Segmenten
vorgenommen werden. Segment 1 ist dabei nur f\"{u}r die Benzolanalyse
bestimmt, Segment 5 f\"{u}r H\down{2}O und Segment 6 f\"{u}r CO\down{2}.
In diesen Segmenten werden dann auch nur die Referenzspektren zur
Analyse herangezogen, die in diesem Bereich Signaturen
aufweisen.\\

Nochmal reduziert werden kann die Referenzmatrix im
gasspezifischen Analysemodus. Das ist f\"{u}r die F\"{a}lle wichtig, in
denen einige Komponenten des Gasgemisches nicht als Referenzgas
vorliegen. Dort werden f\"{u}r jedes Gas wenige Bereiche mit den
h\"{o}chsten Extinktionen vorgegeben, die im Normalfall deutlich
kleiner sind als die Bereiche im Segmentauswertungsmodus. Folglich
reduziert sich dort auch die Zahl der m\"{o}glichen
Querempfindlichkeiten und somit die Referenzspektrenmatrix. Dies
hat vor allem f\"{u}r den Fall unbekannter Komponenten im Gasgemisch
den Vorteil, dass sie in einigen Bereichen nicht mehr mitangepasst
werden m\"{u}ssen. Auf der anderen Seite stehen aber auch weniger
Datenpunkte zur Auswertung zur Verf\"{u}gung. Es ist zus\"{a}tzlich
m\"{o}glich, diesen Bereich in iterativen Schritten zu vergr\"{o}{\ss}ern. Das
Segment mit dem kleinsten statistischen Fehler wird nach der
Analyse als Ergebnis ausgegeben.\\

Grunds\"{a}tzlich wird bei allen drei CLS--Auswertungsmodi die
Basislinie durch einen Offset, einen linearen und einen
quadratischen Term mitangepasst. Die Anpassung eines zus\"{a}tzlichen
kubischen Terms ist auch m\"{o}glich, wobei von dieser Funktion jedoch
sparsam Gebrauch gemacht werden sollte. Es ist m\"{o}glich, dass
dadurch Strukturen des Spektrums als Basislinie mit angepasst
werden. Jeder Schritt der Auswertung kann sofort w\"{a}hrend der
Auswertung auf dem Bildschirm verfolgt werden (siehe Abb.
\ref{seg3.wmf}).\\

\bild{htb}{rescls.wmf}{450}{350}{\it Darstellung aller
Segment--Residuen nach der CLS--Auswertung.}

Das Gemischspektrum sowie s\"{a}mtliche Referenzspektren des zu
analysierenden Segmentes werden nebeneinander auf dem Bildschirm
dargestellt. Die aufgrund zu hoher Extinktion herausgeschnittenen
Bereiche werden durch einen farbigen Balken gekennzeichnet. Das
Residuum gibt sofort Hinweis auf noch nicht ber\"{u}cksichtigte
Komponenten. S\"{a}mtliche Residuen der gasspezifischen Auswertung wie
auch der Segmentauswertung werden zum Schluss noch einmal
nebeneinander zusammengestellt (siehe Abb. \ref{rescls.wmf}). Sie
k\"{o}nnen abgespeichert oder mit einem Kommentar versehen gedruckt
werden. Abbildung \ref{rescls.wmf} zeigt sehr sch\"{o}n die
zur\"{u}ckbleibenden Spikes aufgrund unzureichender
Wasserkompensation. \"{U}ber die graphische Darstellung hinaus wird
die Auswertung jedes einzelnen Segmentes nat\"{u}rlich mit erhaltener
Konzentration f\"{u}r die einzelnen Stoffe und Anteile der
Basislinienanpassung, Einzelstandardabweichungen, Anzahl der
ausgewerteten und herausgeschnittenen Punkte, Konditionszahl der
Referenzspektrenmatrix, Standardabweichung des Residuums und Gr\"{o}{\ss}e
der vorgenommenen Wellenzahlverschiebung im log-File
dokumentiert.\\

Nach einem Auswertungsdurchlauf k\"{o}nnen weitere Iterationen folgen.
So sollte bei signifikanten Strukturen im Residuum die
Kreuzkorrelationsfunktion in diesem Bereich angewandt werden, um
eventuell vorhandene, vorher noch nicht erkannte Komponenten in
die Auswertung iterativ miteinzubeziehen (zu dieser Problematik
siehe auch Kap. \ref{crosscorr}). Ebenfalls ist es sinnvoll, nach
einer ersten Voranalyse der Stoffe, Referenzspektren zu w\"{a}hlen,
die m\"{o}glichst nah an der Konzentration der Komponente im
Gasgemisch heranreicht. Mit dem in dieser Arbeit verfolgten
Konzept der Abspeicherung der Parameter der Pr\"{u}fgase, um auch
nichtlineare Terme mitber\"{u}cksichtigen zu k\"{o}nnen (siehe auch Kap.
\ref{kalibriergas}), k\"{o}nnen diese Referenzspektren f\"{u}r die 2.
Iteration sehr schnell bereitgestellt werden.\\


\section{\label{abspeichern}Ermitteln der besten statistischen Ergebnisse}


\bild{htb}{ergfile.wmf}{399}{564}{\it Beispiel eines erg--Files
mit den Auswertungsergebnissen, die den geringsten statistischen
Fehler aufweisen}


Um aus dieser Flut von Daten sofort eine \"{U}bersicht zu erhalten,
wird ein Ergebnis--File ("`spektrenname"'.erg) angelegt (siehe
Abb. \ref{ergfile.wmf}). Dort sind neben den ermittelten
Konzentrationen der einzelnen Gase, deren 2$\sigma$--Fehler und
dem Segment, in dem sie ausgewertet wurden, auch s\"{a}mtliche Daten
der Qualiti\"{a}ts\"{u}berwachung enthalten. So kommen zu den schon in
Kap. \ref{spektrenvorbearb} erw\"{a}hnten Punkten, die
Standardabweichungen der Residuen der einzelnen Segmente und die
Konditionszahlen der Referenzmatrix in diesen Bereichen und die
Bestimmung des Signal/Rauschverh\"{a}ltnisses in den drei Bereichen
vor, wie sie die \cite{to1697} vorschl\"{a}gt. Ebenfalls werden
Warnhinweise gegeben, wenn z.B. die Nichtlinearit\"{a}ts- oder
Wellenzahlkorrektur nicht durchgef\"{u}hrt wurde.\\

Werden Messreihen ausgewertet, so ist es sinnvoll, die
Konzentrationen der einzelnen Komponenten im zeitlichen Verlauf
als \"{U}bersicht zu haben. Das Programm legt optional ASCII--Files
mit den Konzentrationen der einzelnen ausgewerteten Gase und deren
2$\sigma$--Fehler, zus\"{a}tzlich \"{u}ber das Signal/Rauschverh\"{a}ltnis in
den Spektren an. Diese k\"{o}nnen dann problemlos in
Spektrenbearbeitungsprogramme wie \it Excel \rm oder \it Origin
\rm eingeladen werden.\\

Aufbau und Struktur des auf MATLAB basierenden
Auswertungsprogrammes erlauben eine schnelle Modifizierung und
Erweiterung und somit eine schnelle Implementierung neuer
Erkenntnisse in das vorliegende Expertensystem.

\cleardoublepage
