\thispagestyle{empty} %\textwidth 150mm \textheight 295mm
%\topmargin -17mm
%\headheight -8mm \oddsidemargin 10mm
%\evensidemargin 10mm
%\renewcommand{\baselinestretch}{1.1}
%\parindent 0mm

%\unitlength 1mm
\vspace*{5mm}
\begin{center}
{\LARGE\bf {\underline{Lebenslauf}}} \end{center}\vspace*{5mm}

%\begin{picture}(0,0)
%\put(49,-21){\line(0,-1){67}}
%\put(49,-107){\line(0,-1){38}}
%\put(49,-164){\line(0,-1){22}}
%\put(49,-204){\line(0,-1){35}}
%\end{picture}

%\begin{picture}(0,0)
%\put(-5,0){\line(0,-1){265}}
%\put(115,0){\line(1,0){40}} \put(115,0){\line(0,-1){54}}
%\put(115,-54){\line(1,0){40}} \put(155,0){\line(0,-1){54}}
%\end{picture}

\begin{tabbing}
\rm Staatsangeh\"{o}rigkeit: \hspace{2.3cm}\= Uwe
M\"{u}ller\hspace{17mm}\=\kill \\ {\large\bf Pers\"{o}nliche Daten}\\[5mm]
Name, Vorname: \> \sl M\"{u}ller, Uwe\\[3mm] \rm Stra{\ss}e: \> \sl
Chemnitzer Str. 33\\[3mm] \rm PLZ/Wohnort: \> \sl 44139
Dortmund\\[3mm] \rm Geburtsdatum: \> \sl 03. November 1968\\[3mm]
\rm Geburtsort: \> \sl H\"{o}xter\\[3mm] \rm Familienstand: \> \sl
ledig\\[3mm] \rm Staatsangeh\"{o}rigkeit: \> \sl deutsch\\[5ex]

{\large \bf Schulausbildung}\\[5mm] \rm 1975 -- 1979 \> \sl Besuch
der kath. Grundschule Bredenborn\\[3mm] \rm 1979 -- 1988 \> \sl
Besuch des Gymnasiums St. Xaver in Bad Driburg\\[3mm] \rm Juni
1988 \> \sl Erwerb der allgemeinen Hochschulreife mit den\\ \> \sl
Pr\"{u}fungsf\"{a}chern Physik, Englisch, Sozialwissen--\\ \> \sl schaften
und Geschichte.
\\[5ex]

{\large \bf Wehrdienst}\\[5mm] \rm Juli 1988 -- September 1989 \>
\sl 15--monatiger Grundwehrdienst als LKW-- und\\ \> \sl
Raupenfahrer beim Pionierbataillon 7 in H\"{o}xter\\[5ex]

{\large \bf Studium}\\[5mm] \rm Oktober 1989 \> \sl Beginn des
Physikstudiums an der Universit\"{a}t--GH\\ \> \sl Paderborn\\[3mm]
\rm M\"{a}rz 1992 \> \sl Erlangung des Vordiploms in Physik.\\[5ex]
\end{tabbing}
%\clearpage

\thispagestyle{empty} %\textwidth 150mm \textheight 295mm
%\topmargin -17mm
%\headheight -8mm \oddsidemargin 10mm
%\evensidemargin 10mm
%\renewcommand{\baselinestretch}{1.1}
%\parindent 0mm

%\unitlength 1mm
%\begin{picture}(0,0)
%\put(49,-10){\line(0,-1){54}}
%\put(49,-86){\line(0,-1){58}}
%\put(49,-195){\line(0,-1){29}}
%\end{picture}

\begin{tabbing}
\rm Staatsangeh\"{o}rigkeit: \hspace{2.3cm}\= Uwe
M\"{u}ller\hspace{17mm}\=\kill \\

\rm Juni 1994: \> \sl Beginn der einj\"{a}hrigen Diplomarbeit in der
Arbeits--\\ \> \sl gruppe von Prof. Dr. W. von der Osten zum
Thema\\ \> \sc Quantum--Beat--Spektroskopie an Exziton--\\ \> \sc
zust\"{a}nden in Halbleitern\\[5ex]

\rm Juni 1995: \> \sl Erlangung des akademischen Grades
Diplom--Physi--\\ \> \sl ker\\[5ex]

{\large \bf Wissenschaftlicher Werdegang}\\[5mm] \rm
Juli--November 1995: \> \sl Wissenschaftlicher Mitarbeiter in der
Gruppe von\\ \> \sl Prof. Dr. W. von der Osten\\[5ex]

16. November 1995: \> \sl Anstellung als wissenschaftlicher
Mitarbeiter am\\ \> \sl Institut f\"{u}r Spektrochemie und Angewandte
Spektros--\\ \> \sl kopie in Dortmund mit angestrebter Promotion
im\\ \> \sl Rahmen des BMBF--Projektes \sc Multivariate Ver--\\ \>
\sc fahren\\[5ex]

17. November 1999: \> \sl Abgabe der Dissertation \sc Entwicklung
optimaler\\ \> \sc Mess-- und Auswertestrategien f\"{u}r die FTIR--\\
\> \sc spektrometrische Atmosph\"{a}renanalytik \sl an der\\ \> \sl
Fakult\"{a}t I--Allgemeine und Angewandte Natur--\\ \> \sl
wissenschaften der Universit\"{a}t Hohenheim\\[10ex]

\end{tabbing}

\noindent\rm Dortmund, 17. November 1999
