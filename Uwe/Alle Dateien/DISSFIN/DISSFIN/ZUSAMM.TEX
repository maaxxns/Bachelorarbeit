\chapter{\label{zusammenfassung}Zusammenfassung}

Basierend auf den in der Vergangenheit f\"{u}r den Laboreinsatz
erarbeiteten multivariaten Analyseverfahren, wurden im Rahmen
dieser Arbeit umfangreiche Verbesserungen in der Analytik f\"{u}r
Anwendungen bei der Fernsondierung von Schadstoffen mittels
FTIR--Spektroskopie in der offenen Atmosph\"{a}re entwickelt. Dies ist
notwendig, um zuk\"{u}nftigen Anforderungen an die zunehmend
komplexeren Aufgabenstellungen bei Atmosph\"{a}renmessungen gerecht zu
werden. Durch den Einsatz von multivariaten Verfahren konnten
generell f\"{u}r die weitaus meisten Komponenten die Nachweisgrenzen
gegen\"{u}ber univariaten Auswertemethoden verbessert werden.
Praxisnahe Strategien f\"{u}r die Erzeugung von Hintergrundspektren
erlauben genauere und h\"{a}ufig gegen\"{u}ber der Auswertung mit einem
gemessenen Hintergrundspektrum Analysenergebnisse mit geringeren
Unsicherheiten. Der Verbesserung der photometrischen Genauigkeit
in den Spektren wurde ein erheblicher Teil der Arbeit gewidmet, da
der sonst nutzbare Extinktionsbereich eingeschr\"{a}nkt ist. Da zudem
die verwendeten Auswerteverfahren erfordern, s\"{a}mtliche zum zu
analysierenden Atmosph\"{a}renspektrum beitragenden Komponenten zu
kennen, ist eine qualitative Analyse unerl\"{a}sslich. Hier wurden
erfolgreich Strategien entwickelt, um im Bereich der
Atmosph\"{a}renspektroskopie eine zuverl\"{a}ssige quantitative
Komponentenbestimmung zu erm\"{o}glichen.\\

Zur Steigerung der Verl\"{a}sslichkeit der quantitativen Auswertungen
wurden erhebliche Anstrengungen unternommen, um qualitativ
hochwertige Referenzspektren von in der Atmosph\"{a}re vorkommenden
Schadstoffen, aber auch von Wasser und Kohlendioxid, in der von
uns geforderten spektralen Aufl\"{o}sung von 0.2 cm\up{-1}
bereitzustellen. Die in dieser Arbeit vorgeschlagenen Strategien
mit speziellen Spektrenvorbearbeitungsschritten erlauben die
Berechnung von Absorptionskoeffizienten in einem gro{\ss}en
Konzentrationsbereich, da systematisch auch die Abweichungen vom
linearen Lambert-Beerschen Gesetz untersucht und dokumentiert
wurden. Da Interferogramme abgespeichert wurden, k\"{o}nnen die
vorliegenden Daten ebenfalls dazu verwendet werden, um niedrigere
spektrale Aufl\"{o}sungen zu realisieren und andere, m\"{o}glicherweise
besser geeignete Apodisationsfunktionen, zu ber\"{u}cksichtigen.
Dieser Teil der Auswertung konnte automatisiert werden.\\

Ein entscheidender Fortschritt wurde dadurch erzielt, dass
s\"{a}mtliche dieser Erkenntnisse in ein Expertensystem zur
Spektrenvorbearbeitung und --auswertung umgesetzt wurden, die es
auch dem Nicht--Spektroskopiker erlauben, Analysen von
Atmosph\"{a}renspektren vorzunehmen. Der Spektroskopiker hat jederzeit
die M\"{o}glichkeit, s\"{a}mtliche Schritte der Analyse nachzuvollziehen
und zu begutachten. Damit steht eine Software zur Verf\"{u}gung, die
Analysenergebnisse nicht nur in einem Blackbox--Verfahren liefert,
sondern auch \"{u}ber automatisierte Grafikabfragen transparente und
\"{u}berpr\"{u}fbare Resultate bereitstellt. Weitere Verbesserungen der
Analysensoftware k\"{o}nnen in Zukunft vor allem durch eine weitere
Verbesserung der Komponentenerkennung in Problemf\"{a}llen durch
Kombination mit anderen Verfahren erreicht werden.\\

Weiterer Forschungsbedarf f\"{u}r die Auswertung der in der offenen
Atmosph\"{a}re gemessenen FTIR--Spektren besteht vor allem in einer
verbesserten Kompensation von Absorptionsbanden von H\down{2}O und
CO\down{2}. Eine optimiertere Anpassung der
konzentrationsabh\"{a}ngigen Extinktionsnichtlinearit\"{a}ten dieser
Komponenten w\"{u}rde die statistischen Fehler der CLS--Analyse in den
schon bestehenden Segmenten weiter verringern und erweiterte
Auswertungsbereiche vor allem f\"{u}r die NO\down{x} und SO\down{2}
erm\"{o}glichen.\\

Der Bedarf an Emissions\"{u}berwachung von diffusen Quellen und an
analytischen Messmethoden f\"{u}r die relevanten Klimagase wird in
Zukunft steigen. Optische Methoden sind f\"{u}r diese Aufgabe
pr\"{a}destiniert. Die FTIR--Spektroskopie hat gegen\"{u}ber anderen
optischen Methoden den Vorteil, dass mit ihr simultan eine
Vielzahl von Gasen gemessen werden k\"{o}nnen. Die Systeme sind
mittlerweile einfach in der Handhabung geworden und k\"{o}nnen f\"{u}r
spektrale Aufl\"{o}sungen von 0.5 cm\up{-1} und schlechter oft auch
einen stabilen Dauerbetrieb gew\"{a}hrleisten. Verbesserungen der
Spektrometerauswertungssoftware, so wie sie in dieser Arbeit
vorgeschlagen werden, sind hingegen unerl\"{a}sslich, um den hohen
Qualit\"{a}tsanspr\"{u}chen in der quantitativen Atmosph\"{a}renanalytik
nachkommen und die Bedienung auch durch technisch geschultes
Personal gew\"{a}hrleisten zu k\"{o}nnen. F\"{u}r die \"{U}berwachung spezieller
Substanzen wie Benzol, Ozon, die NO\down{x} oder SO\down{2}, sowie
generell bei geforderten Nachweisgrenzen besser als im unteren
ppb--Bereich kann die FTIR--Spektroskopie mit speziellen optischen
Methoden wie DOAS oder dem Einsatz von Diodenlasern nicht immer
konkurrieren.

\cleardoublepage
