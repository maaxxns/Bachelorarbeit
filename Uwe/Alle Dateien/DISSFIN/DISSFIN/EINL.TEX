\chapter{\label{einleitung}Einleitung}


Insbesondere zur Charakterisierung diffus emittierender
Fl\"{a}chenquellen wie z.B Deponien, Industrieanlagen, Kl\"{a}ranlagen,
landwirtschaftlich genutzte Fl\"{a}chen, Verkehrsstrecken oder auch
zur \"{U}berwachung von industriellen Prozessen oder bei
Umweltunf\"{a}llen werden zuverl\"{a}ssige, kosteng\"{u}nstige,
automatisierbare und somit nicht personalintensive analytische
Methoden gesucht. Seit einigen Jahren wird verst\"{a}rkt die \bf F\rm
ourier \bf T\rm ransform \bf I\rm nfrarot \bf S\rm pektroskopie
(FTIR--Spektroskopie) auch als Fernmessverfahren f\"{u}r die
Atmosph\"{a}renspektroskopie eingesetzt. Dabei kann anhand der durch
Luftverunreinigungen bedingten Absorption von IR--Strahlung auf
dem Weg zwischen einer k\"{u}nstlichen IR--Strahlungsquelle und einem
FTIR--Spektrometer der integrale Volumenanteil \"{u}ber die
Messstrecke bestimmt werden (open--path FTIR).\\

Die Ergebnisse punktf\"{o}rmig messender Verfahren, die in der
Vergangenheit zur Verf\"{u}gung standen, sind oft nicht repr\"{a}sentativ
f\"{u}r die Emissionen einer Fl\"{a}chenquelle. Die ermittelten
Analyseergebnisse der im Feld genommenen und dann im Labor
ausgewerteten Proben geben meist nur einen kurzen zeitlichen
Ausschnitt der jeweiligen Messsituation wieder. Die
FTIR--Spektroskopie hingegen erlaubt die kontinuierliche
\"{U}berwachung von Emissionen in kleinen Zeitfenstern mit
gleichzeitiger online--Auswertung der Messergebnisse. Aufgrund der
ber\"{u}hrungslosen optischen Messung kann auch in Bereichen gemessen
werden, die f\"{u}r Personen nicht zug\"{a}nglich sind. Weiterhin ist
weder eine Probenaufbereitung erforderlich, noch ist eine
Kontamination des verwendeten Messsystems gegeben. Der Aufbau
eines "`optischen Zaunes"' um eine Anlage erlaubt zudem eine
kontinuierliche Kontrolle \"{u}ber den Gesamtemissionsaustrag dieser
Anlage.\\

Der Vorteil der FTIR--Spektroskopie gegen\"{u}ber anderen optischen
Langwegmessverfahren, wie z.B. der Verwendung von Diodenlasern
oder dem DOAS (\bf D\rm ifferentielle \bf o\rm ptische \bf A\rm
b\-sorp\-ti\-ons \bf S\rm pektroskopie)--Verfahren, beruht vor
allem darin, dass weitaus mehr Komponenten simultan in demselben
Probenvolumen detektiert werden k\"{o}nnen. Die Clean Air Act
Amendments der USA von 1990 fordern z.B. von der Vielstoffanalytik
die \"{U}berwachung von 189 Luftschadstoffen, was mit der
IR--Spektroskopie m\"{o}glich ist. Auf der anderen Seite sind die
Nachweisgrenzen der Stoffe, die mittels Diodenlaser oder DOAS
bestimmt werden k\"{o}nnen, meist besser als mit FT--spektrometrischen
Methoden.\\

F\"{u}r die Akzeptanz der FTIR--Methode f\"{u}r Atmosph\"{a}renmessungen und
der damit gewonnenen Analysenergebnisse ist es wichtig, dass
wissenschaftlich anerkannte Vorschriften zur Planung, Durchf\"{u}hrung
und Dokumentation von Messungen bestehen. In den USA hat die EPA
(\bf E\rm nvironmental \bf P\rm rotection \bf A\rm gency) hierzu
im Jahr 1997 die \it Compendium Method TO--16 \rm herausgegeben
\cite{to1697}, in Deutschland werden voraussichtlich im Herbst
1999 die VDI--Richtlinien f\"{u}r \it Fernmessverfahren -- Messungen
in der bodennahen Atmosph\"{a}re nach dem FTIR--Prinzip \rm vorliegen
(Entwurf: \cite{vdi98}). Die TO--16 sieht die letztendliche
Bedeutung, die der FTIR--Spektroskopie in der offenen Atmosph\"{a}re
zukommt, wie folgt:\\

\it "`The ultimate significance of remote sensing with FT--IR
systems is a matter of cost--effectiveness and of technological
advances. Technological advances are required in at least two
important areas:

\begin{enumerate}

\item the improvement in the characteristics of the
instrumentation itself

\item the development of 'intelligent' software "' \rm

\end{enumerate}

\rm W\"{a}hrend die Spektrometerhersteller mittlerweile sehr robuste,
stabil laufende und weitgehend auch kompakte Systeme auf den Markt
gebracht und somit der ersten Forderung aus der TO--16
nachgekommen sind, wurden nur \"{a}u{\ss}erst wenige Anstrengungen auf die
Verbesserung der quantitativen Auswertung der Spektren verwendet.
Obwohl in der Analysensoftware die gerade bei
Querempfindlichkeiten mit anderen Komponenten problematische
univariate Auswertung von Absorptionsbanden durchgehend durch
multivariate Auswertungsverfahren ersetzt wurde, gibt es noch eine
Reihe von Defiziten, die es zu beheben gilt, um eine optimale
quantitative Auswertung der Spektren zu gew\"{a}hrleisten.\\

Die Defizite k\"{o}nnen in zwei Bereiche unterteilt werden. Der eine
umfasst die Vorbearbeitung der gemessenen Interferogramme und
Spektren und die Qualit\"{a}t der bereitgestellten Referenzdaten, der
zweite Bereich die mathematische Auswertung selbst.\\

F\"{u}r eine optimale mathematische Auswertung ist es notwendig, dass
bei der Spektrenberechnung systematische Fehler reduziert werden.
Hierzu geh\"{o}rt die Korrektur von Auswirkungen nichtlinearen
Detektorverhaltens, welche bei der Aufnahme der prim\"{a}ren
Information, n\"{a}mlich der Interferogramme, bei hohen
Strahlungsleistungen auftreten und z.T. zu erheblichen
photometrischen Fehlern f\"{u}hren k\"{o}nnen.Weiterhin bietet die
FT--Spektroskopie an sich eine hohe Reproduzierbarkeit bzgl. der
Wellenzahlachse, doch bei der \"{U}bertragung von Spektren auf andere
Spektrometer m\"{u}ssen Abweichungen in der Wellenzahlskala unbedingt
ber\"{u}cksichtigt werden. In dieser Arbeit werden die Auswirkungen
solcher Abweichungen quantifiziert und eine effiziente
automatische Korrekturstrategie vorgestellt.\\

F\"{u}r die multivariate Auswertung wird zudem ein Hintergrundspektrum
ben\"{o}tigt, um Extinktionsspektren berechnen zu k\"{o}nnen. In vielen
F\"{a}llen k\"{o}nnen unter bestimmten Randbedingungen solche
experimentell gemessen werden, doch auch f\"{u}r andere Situationen,
in denen dies nicht m\"{o}glich ist, werden leistungsf\"{a}hige und
praktikable Algorithmen zur Berechnung dieses Hintergrundspektrums
gebraucht, die in dieser Arbeit entwickelt und getestet wurden.\\

Die f\"{u}r die Auswertung erforderlichen Referenzspektren werden
vorwiegend aus kommerziell erh\"{a}ltlichen oder von offiziellen
Stellen (NIST/EPA) herausgegebenen Datenbanken entnommen.
Abgesehen von der Tatsache, dass die angegebenen Extinktionswerte
in den Spektren Unsicherheiten bis zu $\pm$ 10 \% aufweisen
k\"{o}nnen, ber\"{u}cksichtigen diese Datenbanken nicht die nichtlinearen
funktionalen Abh\"{a}ngigkeiten, die durch die Faltung der wahren
Absorptionsbanden mit der instrumentellen Spektrometerfunktion
existieren. Im Rahmen dieser Arbeit wurde eine Bibliothek von
qualitativ hochwertigen Spektren von insgesamt 16 f\"{u}r die
Atmosph\"{a}renspektroskopie wichtigen Gasen erstellt. Es wird ein
Konzept beschrieben, mit dem eine generelle Spektrenberechnung
unter Ber\"{u}cksichtigung nichtlinearer Extinkionsabh\"{a}ngigkeiten
erfolgt, die Abweichungen vom Lambert--Beer'schen Gesetz
einbeziehen.\\

Der zweite Bereich der angesprochenen Defizite betrifft die
multivariate Auswertung selbst. Eine in der
Atmosph\"{a}renspektroskopie erfolgreich verfolgte Strategie geht von
der Modellierung der gemessenen Spektren \"{u}ber eine Anpassung mit
bekannten Komponentenspektren anhand einer Minimierung der
Fehlerquadrate aus (Ausgleichsrechnung nach der Methode der
kleinsten Quadrate, in der Literatur vielfach als klassisches
Least Squares--Verfahren, engl. CLS, bezeichnet). Unabh\"{a}ngig davon
werden seit kurzem auch statistische Kalibrierverfahren
eingesetzt, deren Diagnostikm\"{o}glichkeiten bzgl.
Modellbildungsdefiziten gegen\"{u}ber dem vorherigen Verfahren jedoch
eingeschr\"{a}nkt sind. Unter Verwendung des bekannten
CLS--Algorithmus (siehe z.B. \cite{saarinen91}), wurden in dieser
Arbeit eine Reihe von Verbesserungen vorgenommen, insbesondere
hinsichtlich einer Spektrenbereichsauswahl zur Optimierung der
Selektivit\"{a}t des Auswerteverfahrens oder zur iterativen
Ber\"{u}cksichtigung der von der Konzentration abh\"{a}ngigen
nichtlinearen Extinktionssignale. Erfolgreiche Strategien zur
Verbesserungen der Analytik hierzu werden aufgezeigt.
Dar\"{u}berhinaus fordert das CLS--Verfahren die Kenntnis s\"{a}mtlicher
im Spektrum vorhandener Komponenten. Mit Hilfe der
Kreuzkorrelation wird hier eine Methode vorgestellt, die eine
sichere qualitative Komponentenanalyse erm\"{o}glicht. Desweiteren ist
die Auswertung von Benzolspektren aufgrund starker atmosph\"{a}rischer
Untergrundabsorptionen aufwendiger als bei anderen Gasen und
erfordert dementsprechend komplexere Vorgehensweisen. Ein solches
Verfahren wird beschrieben, und es werden in dieser Arbeit
erstmals verl\"{a}ssliche Daten f\"{u}r die Bestimmung der Nachweisgrenze
bei Atmosph\"{a}renmessungen mit FTIR--Spektrometern geliefert.\\

Kommerziell erh\"{a}ltliche Software bietet im allgemeinen nur den
CLS--Algorithmus an, wobei Kom\-po\-nen\-ten-- und Segmentauswahl
vom erfahrenen Spektroskopiker selbst vorgenommen werden m\"{u}ssen.
Abgesehen davon, dass dies zum Teil sehr zeitaufwendig ist, ist es
f\"{u}r die Akzeptanz der Methode unerl\"{a}sslich, die Auswertung soweit
zu automatisieren, dass auch Nicht--Spektroskopiker die gemessenen
Atmosph\"{a}renspektren auswerten k\"{o}nnen. Im Rahmen dieser Arbeit ist
erstmals ein Expertensystem entstanden, welches s\"{a}mtliche der oben
genannten Problematiken zur Spektrenvorbearbeitung und
--auswertung ber\"{u}cksichtigt und eine automatische Analyse von
umfangreichen Messreihen erlaubt, wobei eine neue Qualit\"{a}t in der
Zuverl\"{a}ssigkeit der Analysenergebnisse angestrebt wurde.\\

Die Spektren wurden mit einem hochaufl\"{o}senden K300--Spektrometer
der Firma Kayser--Threde (M\"{u}nchen) aufgenommen. Vor-- und
Nachteile gegeben\"{u}ber \"{u}blicherweise routinem\"{a}{\ss}ig eingesetzten,
doch schlechter aufl\"{o}senden Spektrometern werden an verschiedenen
Stellen dieser Arbeit diskutiert. Dar\"{u}berhinaus wurden die
vorgestellten Strategien in zwei Messkampagnen \"{u}berpr\"{u}ft und die
Verl\"{a}sslichkeit der Ergebnisse im Vergleich mit einem zweiten
Spektrometer gezeigt. Die erreichten Forschritte in der
Spektrenvorbearbeitung und Standardisierung der Auswertung, sowie
bei der Erstellung des Expertensystems zur qualitativen und
quantitativen Auswertung von Atmosph\"{a}renspektren wie sie bei der
Erfassung diffuser Emissionen aus anthropogener Aktivit\"{a}t erhalten
werden, sollen helfen, die Akzeptanz der FTIR--Spektroskopie in
der Industrie und bei \"{U}berwachungsbeh\"{o}rden zu erh\"{o}hen.\\

\cleardoublepage
