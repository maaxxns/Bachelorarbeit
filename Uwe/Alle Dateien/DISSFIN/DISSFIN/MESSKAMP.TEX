\chapter{\label{messkampagnen}Messkampagnen}

\section{\label{raffinerie}Messkampagne an einer petrochemischen
Anlage}

\subsection{\label{raffziel}Zielsetzung der Messkampagne}

Vom 12.-16.5.1997 fand die erste Messkampagne im Rahmen des
Projektes 'Multivariate Verfahren' an einer petrochemischen Anlage
statt. Beteiligt waren das Institut für Spektrochemie und
Angewandte Spektroskopie (ISAS), das Landesumweltamt NRW (LUA),
die Firma Kayser-Threde (KT) und die Fachhochschule Düsseldorf
(FH). Es standen damit vier K300--Spektrometer mit 0.2 cm\up{-1}
und zusätzlich ein ETG--Spektrometer mit 1 cm\up{-1} Auflösung von
der FH zur Verfügung. Die gesamte Messplanung und die
Kommunikation mit dem Betreiber lag beim Landesumweltamt.\\

Der Stand des wissenschaftlichen Fortschritts zum damaligen
Zeitpunkt war, dass in der Spektrometersoftware erstmals das im
Rahmen dieser Arbeit entwickelte CLS--Modul genutzt wurde. Die
Module für die synthetische Hintergrundberechnung und die
Nichtlinearitätskorrektur standen zum damaligen Zeitpunkt noch
nicht zur Verfügung. Aus diesem Grunde wurden damals auch nur die
Spektren, nicht aber die Interferogramme aufgezeichnet, so dass
eine nachträgliche Korrektur nicht möglich war.\\

\bild{htb}{olefin.wmf}{420}{260}{\it Grobskizze der Olefinanlage.}

\underline{Ziele dieser ersten Messkampagne waren:}\\

\begin{itemize}
  \item der erstmalige Test des CLS--Analysemoduls im Feld
  \item Vergleich der CLS--Ergebnisse und statistischen
  Vertrauensbereiche mit den Ergebnissen einer univariaten
  Auswertung
  \item Ermittlung von analytischen Diskrepanzen bei der Auswertung von
  realen Atmosphärenspektren und deren Beseitigung im Verlauf des
  weiteren Projektes
  \item der Nachweis, dass diffuse Emissionen aus industriellen
  Anlagen mittels FTIR--Fernsondierung geeignet vermessen werden
  können
  \item die Generierung von Datenmaterial zur Untersuchung von
  Verfahrenskenngrö{\ss}en wie z.B. Nachweisgrenzen und
  Vergleichbarkeit von Atmosphärenspektren
\end{itemize}

für die Fragestellungen dieser Arbeit waren vor allem die ersten
drei Punkte von Wichtigkeit, so dass im folgenden auch nur darauf
eingegangen wird. für die FH Düsseldorf stand hingegen die
Generierung von Datenmaterial im Vordergrund, das Landesumweltamt
hatte als Behörde, die wie auch der Spektrometerhersteller
Kayser--Threde die FTIR--Fernsondierung in der alltäglichen
Analyse von diffusen Emissionen einsetzen möchte, an allen Punkten
Interesse.\\

\subsection{\label{raffaufbau}Aufbau der Olefinanlage }

Abbildung \ref{olefin.wmf} zeigt die Grobskizze der Olefinanlage.
Sie hat eine Ausdehnung von ca. 180 m x 250 m, wobei das
umliegende Gelände, in dem die Messungen durchgeführt wurden, frei
zugänglich ist und nicht im explosionsgefährdeten Bereich liegt.
Auf einem wenig befahrenen landwirtschaftlichen Weg, etwa 0.5 km
von der Anlage entfernt und entgegen der Windrichtung liegend, war
es möglich, die Messung von Hintergrundspektren vorzunehmen.\\

Die Olefinanlage (Olefine -- andere Bezeichnung für Alkene) stellt
nur einen kleinen Teil der gesamten Raffinerie dar. 440 000 t/a
Ethen werden dort nach Angaben des Betreibers hergestellt. Als
Rohstoffe für den Herstellungsprozess dienen Benzin, Flüssiggas
oder hydriertes Vakuuml, die nach Zugabe von Dampf thermisch
gespalten werden (\it steam cracking\rm ). Die anschließende
Trennung des Spaltproduktes in die einzelnen Bestandteile erfolgt
durch verschiedene Wäschen sowie Tieftemperaturzerlegung. Das in
der Anlage hergestellte Ethen, sowie das mit anfallende Propen
werden direkt zu Rohstoffen für die kunststoffverarbeitende
Industrie weiterverarbeitet.\\

Trotz der modernen Bauweise stellt eine solche Anlage aufgrund
ihrer großen räumlichen Ausdehnung und ihres komplexen
Rohrleitungsnetzes grundsätzlich eine diffuse Emissionsquelle dar.
Kontrollen vom Betreiber finden durch Gasprobenahme an bestimmten
Orten und anschließender Analyse mittels
Gaschromatographie--/IR--Messungen im Labor statt.

\subsection{\label{raffergebnisse}Ergebnisse bzgl. univariater und
multivariater Auswertung}

Um die Messzeit effizient zu nutzen, wurde der erste Tag der
Messwoche ausschließlich darauf verwandt, auf dem wenig befahrenen
Feldweg Hintergrundspektren unter Berücksichtigung der
verschiedenen möglichen Messweglängen in der Raffinerie
aufzunehmen. Ein Teil des zweiten Tages wurde zudem dazu genutzt,
durch Parallelmessungen aller Spektrometer nebeneinander, die
Vergleichbarkeit der Ergebnisse abzusichern.\\

\bild{h}{veba1.wmf}{410}{300}{\it CLS-Auswertung eines
Atmosphärenspektrums, aufgenommen an der Olefinanlage auf der
ISAS--Messstrecke bei einer Weglänge von 190 m. Das Atmosphären--
und die Referenzspektren sind mit einem Offset dargestellt.}

\bild{htb}{vebauni.wmf}{410}{300}{\it Univariate Auswertung von
Propen. Bis auf das untere sind alle Spektren mit einem Offset
dargestellt.}

Bei der Auswertung mittels des CLS--Moduls orientierte sich die
Wahl des Hintergrundspektrums weniger an gleichen Messstrecken,
sondern vielmehr nach optimaler Kompensation der Wasserspektren.
Aufgrund dieser Kompensation mittels gemessenem
Hintergrundspektrums war es aber mit der CLS--Analyse nicht
möglich, absolute Werte der Atmosphärenkomponenten H\down{2}O,
CO\down{2}, N\down{2}O, CO und CH\down{4} zu erhalten. Außer
diesen üblichen Spurenstoffen konnten die Komponenten Ethen,
Propen, Ethan, Isobuten, 1.3--Butadien, Ammoniak und Methanol
nachgewiesen werden.\\

Abbildung \ref{veba1.wmf} zeigt einen Ausschnitt des
Atmosphärenspektrums, darüber die Referenzspektren der einzelnen
Komponenten, die in diesem Bereich spektrale Signaturen haben, und
unten das Residuum nach der Anpassung. Es ist gut zu erkennen,
dass die multivariate Anpassung mittels CLS problemlos möglich
ist. Abbildung \ref{vebauni.wmf} zeigt auf der anderen Seite, dass
die univariate Auswertung (hier das Beispiel Propen) mit gewissen
Problemen verbunden war. Im oberen Teil sind gestrichelt die
Referenzspektren der Stoffe zu sehen, die in diesem Bereich
Signaturen aufweisen, die geringe Konzentration von 1.3--Butadien
führt hier nur zu sehr kleinen spektralen Strukturen. Im unteren
Teil ist mit der durchgezogenen Linie die Addition der
Referenzspektren mit den durch die CLS--Analyse erhaltenen
Konzentrationen zu sehen. Man sieht deutlich, dass bei einer
univariaten Auswertung von Propen bei der 912.6 cm\up{-1}--Bande
es zu Querempfindlichkeiten vor allem mit dem Wasser kommen würde.
Die Bande bei 912.24 cm\up{-1} dagegen ist hinlänglich
querempfindlichkeitsfrei. Beim logarithmierten Einkanalspektrum
muss jedoch sehr genau darauf geachtet werden, an welchen Stellen
die Basislinie bestimmt wird (schmale Pfeile außen). Desweiteren
ist die Abtastung des Spektrums mittels Zerofill in der
Fourierdomäne zu verbessern.\\

\bild{htb}{veba2.wmf}{410}{300}{\it Konzentrationsprofile dreier
Komponenten, aufgenommen über 6 Stunden im Vergleich zu den
univariaten Ergebnissen.}

\bild{htb}{veba3.wmf}{410}{300}{\it Gegenüberstellung der CLS--
und univariaten Ergebnisse für Propen mit den dazugehörigen
Fehlerintervallen (95 \% Vertrauensbereich).}

In Abbildung \ref{veba2.wmf} ist für drei Komponenten beispielhaft
der Vergleich zwischen den univariaten und multivariaten
Ergebnissen über einen Tagesverlauf von 6 h dargestellt. Es sind
leichte Unterschiede in den Ergebnissen zu erkennen, wobei beim
Ethen die univariaten Ergebnisse leicht niedriger, beim Propen
leicht höher als die CLS--Ergebnisse liegen. Abbildung
\ref{veba3.wmf} zeigt für Propen beispielhaft, den statistischen
Vertrauensbereich der Ergebnisse. Dieser ist bei der CLS--Methode
doch deutlich besser als bei der univariaten Auswertung. Hinzu
kommen die oben geschilderten Probleme der möglichen
Querempfindlichkeiten mit anderen Komponenten bei der
Propenauswertung, was zu zusätzlichen systematischen Fehlern
führen kann.\\

\subsection{\label{raffbewert}Bewertung der Messkampagne}

Die Messkampagne hat deutlich gezeigt, dass das FTIR--Verfahren
sehr gut geeignet ist, um die diffusen Emissionen der Olefinanlage
zu charakterisieren. Vergleiche der damit erhaltenen Werte mit den
Werten aus den Punktmessungen des Betreibers führen teilweise zu
1--1.5 Größenordnungen in den Abweichungen, was daran liegt, dass
das FTIR--Verfahren als integrierendes Messverfahren unabhängig
von der jeweiligen Stelle der Gasprobenahme der Punktmessungen
ist. Leider sind die Ergebnisse der Punktmessungen vom Betreiber
nicht freigegeben und können somit hier nicht gezeigt werden.
Grundsätzlich ist es auch kein Problem, in explosionsgeschützten
Bereichen mittels FTIR zu arbeiten. In diesem Fall werden
Retroreflektoren eingesetzt und der Globar neben dem Spektrometer
positioniert.\\

Die Auswertung der Atmosphärenspektren hat die überlegenheit der
CLS--Analyse gegenüber des bis dahin gebräuchlichen univariaten
Auswertungsverfahrens gezeigt. Gleichzeitig wurde bei Auswertung
dieser Messkampagne aber auch deutlich, wie wichtig eine
Nichtlinearitätskorrektur ist. Der Abzug der
Eigenstrahlungsspektren führte oft zu negativen Werten. Die
Multiplikation der Atmosphäreneinkanalspektren mit einem Faktor,
um keine negativen Werte bei der Subtraktion zu erhalten, konnte
nur als vorläufig und unzureichend angesehen werden. Die Aufnahme
von Hintergrundspektren gestaltete sich als schwierig und konnte
aufgrund der örtlich Gegebenheiten nur am ersten Tag
durchgeführt werden. Eine synthetische Hintergrundberechnung
schien aus diesem Grund sehr wünschenswert. Diese Punkte wurden
dann im Verlauf des Projektes intensiv verfolgt (siehe Kap.
\ref{photometgen} und \ref{hintergrund}).\\

\newpage
\markright{\sl MESSKAMPAGNE AN EINER MBA}

\section{\label{mba}Messkampagne an einer mechanisch--biologischen
Abbauanlage}\markright{\sl MESSKAMPAGNE AN EINER MBA}

\subsection{\label{zielmba}Zielsetzung der Messkampagne}

Vom 16.-19.11.1998 fand eine weitere Messkampagne an einer
mechanisch-biologischen Abfallbehandlungsanlage (MBA) im Land
Brandenburg statt. Teilnehmer waren das ISAS, KT und das LUA,
welches wiederum den Kontakt mit dem Betreiber herstellte und die
Messplanung übernahm. Aufgrund eines Defektes standen während der
Messkampagne aber nur zwei Spektrometer zur Verfügung. Weiterhin
nahmen an zwei Tagen auch Mitarbeiter des ISAS--Berlin mit einem
DOAS--Gerät an der Messung teil. Die Daten dieser Auswertung
liegen aber noch nicht vor. Die Firma RUK Umweltanalytik in Trier
führte die Punktmessungen durch und überwachte die Anlage über
einen Zeitraum von mehreren Monaten. Die FTIR--Messungen stellten
deshalb nur einen kleinen Teil des Messprogramms dar und waren
zeitlich auf eine Woche begrenzt.\\

Das Interesse an der mechanisch--biologischen Restabfallbehandlung
als Alternative zur Müllverbrennung ist in den letzten Jahren
stetig gestiegen. Diese Art der Abfallbehandlung scheint vor allem
für flächenmäßig nur dünn besiedelte Gebiete sinnvoll, um weite
Transportwege zur nächsten Müllverbrennungsanlage zu vermeiden.
Weiterhin fehlt es aber noch immer an genügendem Datenmaterial, um
die Emissionen der ungeregelten Rotteverfahren im Freilandbetrieb
zu charakterisieren und die Werte mit der TA Luft zu vergleichen.
Erste Veröffentlichungen \cite{turk97}, \cite{cuhls98} stellen
zwar schon die mit Hilfe von Punktmessverfahren gewonnenen
Volumenfrachten einer ganzen Reihe von Luftschadstoffen aus
Rottemieten zusammen, Aussagen über die Repräsentanz der Werte und
die Größe der anschließenden Verdünnung und somit die Belastung
der unmittelbaren Nachbarschaft der MBA sind aber nur schwer
möglich. Die FTIR--Spektroskopie kann hier als integrales
Messsystem in Verbindung mit den Punktmessverfahren von Nutzen
sein. Aufgrund der zu erwartenden Verdünnung in der Luft werden
voraussichtlich nur wenige Schadstoffe mittels FTIR gemessen
werden können. Anhand von Leitsubstanzen sind aber unter Einsatz
von zwei oder mehr Spektrometern in Verbindung mit den
Punktmessungen und den meteorologischen Daten detaillierte
Aussagen über den Emissionszustand der Rottemieten möglich.\\

\underline{Ziele dieser zweiten Messkampagne waren:}\\

\begin{itemize}
  \item Test des verfeinerten CLS--Analysemoduls im Feld
  \item Vergleich der CLS--Anpassung der Feldspektren mit eigenen
  und fremden Referenzspektren
  \item Test der entwickelten Verfahren zur Hintergrundproblematik
  an Feldspektren
  \item überprüfung des Einflusses eines Temperaturunterschiedes
  von 20 \degree C und mehr zwischen Feld-- und Referenzspektren
  \item Der Nachweis, dass der Emissionsverlauf einer Rottemiete
  anhand von Leitsubstanzen mittels FTIR und in Verbindung mit
  Punktmessverfahren und meteorologischen Daten geeignet
  charakterisiert werden kann
\end{itemize}


\subsection{\label{aufbaumba}Aufbau der Anlage und Funktionsweise der Kompostierung}

\bild{htb}{mbanauen.wmf}{425}{425}{\it Aufbau der
mechanisch--biologischen Abfallbehandlungsanlage. Die Zahlen geben
die verschiedenen Messstrecken wieder. Index S kennzeichnet dabei
den Standort des Spektrometers, G den des Globars. W ist der
Standort der Wetterstation.}

Nach Angaben des Betreibers leben im Einzugsgebiet der
mechanisch--biologischen Abfallbehandlungsanlage etwa 80 000
Einwohner, wobei dort im Jahr etwa 22 500 t Haus-- und Sperrmüll
behandelt werden. Eine Bioabfallsammlung wird in diesem
Einzugsgebiet nicht durchgeführt.\\

Der übersichtsplan Abbildung \ref{mbanauen.wmf} zeigt den Aufbau
der mechanisch--biologischen Abfallbehandlungsanlage. Im
Messzeitraum waren dort die Mieten I-III mit den Abmessungen von
jeweils 22 m x 134 m mit einer Höhe von ca. 2.5 m eingerichtet.
Mit dem Aufbau der Mieten wurde im Feb. 1998 (Miete I), Anfang Mai
1998 (Miete II) und Ende Juli 1998 (Miete III) begonnen. Der
Aufbau der Miete IV begann Anfang November 1998, die Länge während
des Messzeitraumes betrug 85 m, bei gleicher Breite und Höhe wie
bei den anderen Mieten.\\

Am Kopfende der Mieten III und IV waren Baugerüste aufgestellt,
auf denen die Spektrometer mit Hilfe von Gabelstaplern gebracht
werden konnten. Die Messstrecken lagen damit ca. 0.5 m über der
Mietenoberfläche. Die Globare wurden an das gegenüberliegende Ende
der Miete verbracht. Die Zahlen 1-7 kennzeichnen die 7
verschiedenen Messstrecken, die während der Messkampagne
realisiert wurden. Der Index S kennzeichnet dabei den
Spektrometerstandort, der Index G den des Globars. Am Standort W
der Wetterstation wurden die Temperatur--, Luftdruck--,
Luftfeuchte--, Windstärke-- und Windrichtungsdaten aufgenommen.\\

Der angelieferte Hausmüll wurde in der Maschinenhalle geschreddert
und das dadurch erhaltene Material zusammen mit organischem
Kompost und Erde auf die asphaltierte Rottefläche aufgebracht. Zur
Einstellung eines für die Mikrobiologie optimalen Feuchtegehaltes
und zur Schaffung einer lockeren, porenreichen Struktur für die
Optimierung der Sauerstoffversorgung werden die Mieten mittels
Teilkreisberegner bewässert. Entwässerungsrinnen sorgen für einen
Ablauf von Regenwasser und während der Rottephase entstehender
Flüssigkeit. Durch das Anlagenpersonal werden prozessbegleitende
Messungen der Temperatur und des Wassergehaltes durchgeführt. Die
Intensivrottephase findet in den ersten 6-8 Wochen statt. Der
Austrag von Gasen erreicht nach ca. 2-4 Wochen (abhängig von der
Art der Gase) ein Maximum und fällt dann exponentiell ab. Nach 2-3
Monaten findet von einem niedrigen Niveau nur noch ein leicht
linearer Abfall der Gasemissionen statt bis nach ca. 6 Monaten
keine Emissionen mehr festgestellt werden \cite{turk97}.\\

Während der Verrottung zersetzt sich der organische Anteil des
geschredderten Hausmülls. Dieser Kompost wird zusammen mit Erde
bei folgenden Mietenaufbauten dem neuen Hausmüll hinzugegeben. Der
nicht--organische Anteil wird nach Abbau einer Miete
(üblicherweise nach ca. 8 Monaten) der normalen Hausmülldeponie
zugeführt.\\

\noindent Der Betreiber der Anlage sieht folgende Vorteile des
gewählten Verfahrens:

\begin{itemize}

\item geringer technischer Aufwand und niedrige Kosten
\item gute Anpassungsmöglichkeit an sich ändernde Abfallmengen und
--zu\-samm\-en\-setz\-ung\-en
\item modulare Ergänzungsmöglichkeiten, wie z.B. Intensivrotte und
Ausschleusung der heizverwertreichen Fraktion
\item optionale Nachrüstung von weiter entwickelten
Verfahrensstufen
\item eine Müllvolumenverringerung von ca. 20 \% wird durch die
Verrottung erwartet

\end{itemize}


\subsection{\label{auswmba}Auswertung der Messergebnisse}

für die Messkampagne von Interesse waren die Emissionen der $3
\frac{1}{2}$ Monate alten Miete III und der etwa 2 Wochen alten
Miete IV, die während der Messkampagne noch im Aufbau begriffen
war. Grundsätzliches Problem der Messkampagne war es, dass
aufgrund der örtlich Gegebenheiten keine Hintergrundspektren
aufgenommen werden konnten, die frei von den gesuchten Substanzen
waren. Dies wurde an mehreren Stellen versucht, die Ergebnisse
waren aber nicht zufriedenstellend. Aus diesem Grund wurden
sämtliche Spektren grundsätzlich mit dem in Kap. \ref{hintergrund}
vorgestellten Verfahren mit dem synthetischen Hintergrund
ausgewertet. Die noch resultierenden Basislinien konnten
problemlos mit angepasst werden. Ein weiterer Vorteil dieser
Vorgehensweise war, dass absolute Konzentrationsvorhersagen der
natürlichen Atmosphärenkomponenten gegeben werden konnten.\\


\bild{htb}{schw17al.wmf}{340}{480}{\it Tagesgang vom 17.11.98 der
atmosphärischen Komponenten H\down{2}O, CO\down{2}, N\down{2}O und
CO auf der Messstrecke 1. Als Leitsubstanzen für die Erfassung der
Emissionen aus der MBA wurden die Stoffe CH\down{4} und NH\down{3}
ausgewählt. Windrichtung und --geschwindigkeit sind eine mögliche
Ursache für Schwankungen in den Konzentrationen (Kreise schwarz --
Messwerte ISAS, Dreiecke grau -- Messwerte LUA).}


\subsubsection{\label{miete3} Miete III}

\bildlinks{htb}{con2o.wmf}{275}{340}{120}{75}{-35}{\it
Statistische und systematische Fehler bei der Auswertung von
N\down{2}O und CO. Die durchgezogenen Linien zeigen jeweils die
Mittelwerte, die gestrichelten die dazugehörigen statistischen
Vertrauensbereiche. Abhängig von der Qualität der Referenzspektren
und der Temperatur, bei der sie aufgenommen wurden, treten noch
systematische Fehler auf (siehe auch Text).}

An der Miete III wurden auf der Messstrecke 1 (siehe Abb.
\ref{mbanauen.wmf}) Parallelmessungen mit beiden
K300--Spektrometern vorgenommen. Zwischen den Spektrometern
befanden sich die Abzugsrohre der Miete III. Beide Spektrometer
wurden auf nur einen Globarstrahler ausgerichtet, der Messweg
betrug dabei 135 m.\\

Abbildung \ref{schw17al.wmf} zeigt den Tagesgang der
atmosphärischen Komponenten H\down{2}O, CO\down{2}, N\down{2}O und
CO. Es herrschte ein starker böiger Wind mit 4-5 m/s
Geschwindigkeit bei Temperaturen um den Gefrierpunkt. Beide
Spektrometer zeigen eine sehr gute übereinstimmung in der
Konzentrationsbestimmung der einzelnen Komponenten. Der
H\down{2}O-- und der CO\down{2}--Gehalt entsprechen dabei den zu
erwartenden Hintergrundkonzentrationen, der N\down{2}O--Gehalt
liegt mit etwa 355-360 ppb etwa 20 ppb darüber, was bei diesem
Abbauprozess auch erwartet wird. Da die CO--Werte in der
Atmosphäre von 0.1-1 ppm schwanken können, ist eine Aussage
darüber, wieviel von der Miete emittiert wurde, in dieser
Messanordnung nicht möglich. Windgeschwindigkeit und Windrichtung
geben Hinweise auf die zu erwartende Verdünnung und Beiträge
anderer Mieten zu den gemessenen Konzentrationen. Die Windrichtung
wird vereinbarungsgemäß in Grad von Norden aus im Uhrzeigersinn
angegeben. Eine Windrichtung von z.B. 270\degree zeigt Westwind an.\\

Zur Beschreibung des Mietenzustandes sind vor allem die
Konzentrationen der Leitsubstanzen CH\down{4} und NH\down{3} von
Interesse. Beide Spektrometer haben über den gesamten Messzeitraum
mit einer sehr guten übereinstimmung Methanwerte von
durchschnittlich 3.1 ppm registriert, ein Wert der nur etwa 1 ppm
über der natürlichen Hintergrundkonzentration von Methan in der
Außenluft liegt. Da beide Spektrometer gleiche
Methankonzentrationen messen, konnte eine Methanemission aus den
Abzugsrohren der Miete nicht festgestellt werden. Festgehalten
werden muss jedoch, dass bei Windgeschwindigkeiten von 4-5 m/s
eine schnelle Verdünnung der Gase in der Umgebungsluft zu erwarten
ist. Da die Messwerte bezogen auf den Mittelwert nur geringe
Schwankungen aufweisen, ist davon auszugehen, dass bereits eine
homogene Vermischung mit der Umgebungsluft stattgefunden hat und
die Miete III nicht die einzige Quelle dieser zusätzlichen
Methanemissionen ist. Bei Windrichtungen um die 320* (NW) sind
somit auch Eintr*ge aus den Mieten I und II und der nahegelegenen
M*lldeponie nicht ausgeschlossen.\\

Die Ammoniakkonzentrationen bieten hingegen ein etwas anderes
Bild. Beide Spektrometer zeigen deutlich erh*hte Werte gegenüber
der Hintergrundkonzentration (2-5 ppb) an. Das ISAS--Spektrometer,
welches in Windrichtung von den Abzugsrohren stand, zeigt mit
einem Mittelwert von 180 ppb eine ca. 20 ppb höhere Konzentration
an als das LUA--Spektrometer. Da die Konzentrationsschwankungen
von beiden Geräten gleicherma*en erfasst werden, deutet dies auf
Ammoniakemissionen aus der Miete hin.\\

Messungen an der Messstrecke 3 zwischen den Mieten III und IV
ergaben durchschnittliche Konzentrationen von 4.6 ppm Methan und
249 ppb Ammoniak. Diese Konzentrationserh*hung ist wahrscheinlich
darauf zur*ckzuführen, dass die aus der Seitenfl*che der Miete III
austretenden Emissionen aufgrund der geringeren
Windgeschwindigkeit im Weg zwischen den Mieten III und IV nicht so
schnell verdünnt werden. Weiterhin sind die bei diesem Tagesgang
gewonnen Daten auch sehr gut geeignet, Probleme bei der Auswertung
von atmosphärischen IR--Spektren aufzuzeigen.\\

Abbildung \ref{con2o.wmf} zeigt die Auswertung von N\down{2}O und
CO, die sehr schmale Banden besitzen, im Bereich von 2163-2220
cm\up{-1}. Schon geringe Abweichungen in der Wellenzahlachse
zwischen Atmosphären-- und Referenzspektren können zu
signifikanten systematischen Fehlern führen. Als Referenzspektren
wurden jeweils die nicht ver*nderten Spektren der
QASoft--Datenbank, ein für 1 *C und ein für 20 *C berechnetes
HITRAN96--Spektrum genommen. Die HITRAN96--Spektren wurden bei 0.2
cm\up{-1} Auflösung dreiecksapodisiert und um 0.03 cm\up{-1}
verschoben, um optimal mit den gemessenen Atmosphärenspektren in
der Wellenzahlachse übereinzustimmen. Die Atmosphärenspektren
wurden bei einer Temperatur von 1 *C aufgenommen. Beim CO ist
deutlich zu sehen, dass ein systematischer Fehler von 50 ppb
zwischen der CLS--Anpassung der Atmosphärenspektren mit HITRAN96--
und unver*nderten QASoft--Spektren auftritt.\\

\bildlinks{htb}{rescon2o.wmf}{275}{340}{120}{75}{-35}{\it Residuum
nach CLS--Anpassung eines MBA--Deponiespektrums von CO, N\down{2}O
und H\down{2}O mit \bf A \it  QASoft--Spektren und \bf B \it
HITRAN96--Spektren bei 1 *C und selbst gemessenen
H\down{2}O--Spektrum.}

Der mit gestrichelten Linien gezeigte statistische
Vertrauensbereich ist bei der Anpassung mit QASoft--Spektren
weitaus schlechter als mit angepassten HITRAN96--Daten, sowie
unter Verwendung eines mit dem K300--Spektrometer gemessenen
H\down{2}O--Spektrum. Der hierbei auch noch betrachtete
Temperatureffekt macht sich beim CO nur durch einen etwas größeren
Vertrauensbereich bei der Anpassung mit dem für 20* C berechneten
Spektrum bemerkbar (gestrichelte Linie gegenüber der
durchgezogenen beim 1*--Spektrum). Die ermittelten Konzentrationen
waren für die Anpassung mit beiden HITRAN96--Spektren gleich.\\

Bei der Anpassung des N\down{2}O--Spektrums ist wiederum eine
Diskrepanz zwischen den statistischen Fehlern unter Zugrundelegen
der QASoft-- und HITRAN96--Spektren zu erkennen. Es werden
Konzentrationsunterschiede von 30 ppb ermittelt. Diesmal ist aber
auch ein systematischer Fehler aufgrund der falschen Temperatur
des Referenzspektrums zu erkennen, was immerhin zu Abweichungen
von 5 ppb führt.\\

Abbildung \ref{rescon2o.wmf} zeigt die beachtlichen Unterschiede
im Residuum nach der Anpassung der Atmosphärenspektren mit den
QASoft--Spektren (\bf A\rm) und mit den optimal angepassten
HITRAN96--Spektren (\bf B\rm). Dieses Beispiel macht noch einmal
sehr deutlich, wie wichtig die Qualität der Referenzspektren
gerade bei kleinen Molek*len ist und welcher Wert vor allem auf
die richtige Wellenzahlanpassung zwischen Atmosphären-- und
Referenzspektren gelegt werden muss. Weiterhin muss auch darauf
geachtet werden, dass Temperaturunterschiede von 20 *C schon zu
deutlichen systematischen Fehlern führen können.\\

\bildlinks{htb}{schw1703.wmf}{275}{340}{120}{75}{-35}{\it
Statistische Vertrauensbereiche der betrachteten Leitsubstanzen
NH\down{3} und CH\down{4}. Ausgewertet wurde ein
MBA--\-De\-po\-nie\-spek\-trum mit einem selbst gemessenen
NH\down{3}--\-Re\-fe\-renz\-spek\-trum (20*C) und einem
CH\down{4}--\-Re\-fe\-renz\-spek\-trum (20*C) aus der
QASoft--\-Spek\-tren\-bi\-blio\-thek.}

Zur Charakterisierung des Mietenzustandes sind bei der Auswertung
dieser Messkampagne vor allem die statistischen Fehler der
Leitsubstanzen CH\down{4} und NH\down{3} von Bedeutung. Da diese
Molek*le eine nicht so deutlich ausgepr*gte schmale Linienstruktur
besitzen, sind die statistischen Fehler aufgrund von geringen
Abweichungen in der Wellenzahlachse und aufgrund von
Temperaturunterschieden nicht so signifikant wie im vorherigen
Beispiel. Abbildung \ref{schw1703.wmf} zeigt diese statistischen
Fehler. Angemerkt werden muss, dass das
NH\down{3}-Referenzspektrum im Labor selbst gemessen wurde, beim
CH\down{4} wieder auf ein QASoft--Spektrum zur*ckgegriffen werden
musste. Darauf ist auch der geringere statistische Fehler beim
NH\down{3} zur*ckzuführen. Beide Referenzspektren wurden bei 20 *C
gemessen, so dass die Temperatureffekte bei der Anpassung
weiterhin auftreten. Die systematischen Fehler sind aber weitaus
geringer, was an den auftretenden Residuen deutlich zu erkennen
ist.\\

\subsubsection{\label{miete4} Miete IV}

Die für die Miete III erhaltenen nur geringf*gig erh*hten
NH\down{3}-- und CH\down{4}--Kon\-zen\-tra\-tio\-nen entsprechen
den Erwartungen einer $3\frac{1}{2}$ Monate alten Miete. Die
Emissionen einer zwei Wochen alten Miete sollten deutlich höher
liegen.\\

\bildlinks{htb}{schw1901.wmf}{275}{340}{120}{75}{-35}{\it
Volumenanteile der Leitsubstanzen CH\down{4} und NH\down{3} an
verschiedenen Messstrecken der Miete IV. (Punkte: Spektrometer
ISAS, Dreiecke: Spektrometer LUA).}

Bei stabiler Windrichtung aus 313* (NW) erfolgten die Messungen an
vier verschiedenen Messstrecken. Die gemessenen
Methankonzentrationen (siehe Abb. \ref{schw1901.wmf}) lagen in der
windzugewandten Seite der Abzugsrohre mit 3.7 ppm etwa 0.9 ppm
höher als auf der abgewandten Seite (2.8 ppm). Direkt über den
Rohren wurde ein Wert von 3.3 ppm gefunden. Die
Ammoniakkonzentrationen zeigen einen *hnlichen Verlauf. Auf der
windzugewandten Seite der Abzugsrohre lassen sich höhere
Konzentrationen (135 ppb) als an der abgewandten (73 ppb) und
direkt über den Abzugsrohren (69 ppb) finden.\\

Deutlich höhere Methanwerte von etwa 6 ppm konnten dagegen auf der
Messstrecke 4 gefunden werden. Temperaturmessungen mit einem
Einstechthermometer in ca 1 bis 1.2 m Tiefe ergaben, dass ca. 1 m
vom Mietenrand sich Temperaturen zwischen 60 * und 70 *C
einstellten, wohingegen diese in der Mietenmitte auf ca. 40 *C
abnahmen. Direkt an den Abzugsrohren betrug die Temperatur ca. 50
*C. Diese Temperatureffekte konnten in beiden Richtungen
beobachtet werden.\\

Die deutlichen Konzentrationsunterschiede bei den Methanmessungen
an den verschiedenen Messstrecken der Miete IV und die o.g.
Temperatureffekte deuten darauf hin, dass die Sauerstoffversorgung
nicht ausreichend war, d.h. die Miete sich in einem anaeroben
Zustand befand. Sehr starke Regenf*lle Anfang November erh*rten
diesen Verdacht. Dies führte dazu, dass die Kompostierung nicht
optimal verlief und somit auch nicht die Mengen an Schadstoffen
emittiert wurden, wie für einen aeroben Zustand der Miete zu
erwarten ist.\\


\subsection{\label{bewertmba}Bewertung der Messkampagne}


Die hier vorgestellten Messungen haben gezeigt, dass die
FTIR--Lang\-weg\-ab\-sorp\-tions\-spek\-tros\-ko\-pie
grundsätzlich ein geeignetes Messverfahren ist, Untersuchungen
über Emissionen aus Fl*chenquellen wie Abfallmieten durchzuführen.
Es konnten wegstreckenintegrierte Konzentrationen von Methan und
Ammoniak mit einer zeitlichen Auflösung von wenigen Minuten
gemessen werden. Organische Stoffe wie Benzol, Toluol, Xylole u.a.
konnten dagegen nicht gefunden werden, obwohl sie, wie aus anderen
Untersuchungen bekannt ist, Bestandteile des Abgases sind. Es ist
davon auszugehen, dass die Nachweisgrenzen des
FTIR--Messverfahrens nicht ausreichen, diese Komponenten zu
bestimmen, da sie nach übertritt in die Atmosphäre sofort um das
100-- bis 1000--fache verdünnt werden.\\

Allgemein g*ltige Aussagen über das Emissionsverhalten der MBA
können aus den Feldmessungen leider nicht abgeleitet werden, da
aus o.g. Gr*nden keine Informationen über die Repr*sentativität
des Anlagenzustandes vorliegen. Die Messkampagne hat aber gezeigt,
dass zeitlich wesentlich umfangreichere Messungen dies leisten
k*nnten.\\

Weiterhin hat sich der Einsatz des verfeinerten CLS--Analysemoduls
bei der Auswertung der Spektren sehr bew*hrt. Die neuen Funktionen
zur Hintergrundberechnung waren unerl*sslich, um die
Deponiespektren überhaupt auswerten zu können. Ebenfalls wurde
sehr deutlich best*tigt, wie wichtig eigene Pr*fgasmessungen sind
und welche Probleme auftauchen können, wenn nur Referenzspektren
aus fremden Datenbanken zur Verfügung stehen, die nicht
spektrometerangepasste instrumentelle Linienprofile liefern, was
speziell für Komponenten mit aufgel*ster Rotations--Feinstruktur
unerl*ss\-lich ist.\\


\cleardoublepage
