\chapter{\label{umwelt}Umweltforschung}

Mehr als 99.9 \% der trockenen Atmosph*renluft bestehen aus
Stickstoff (78 \%), Sauerstoff (21 \%) und Argon (0.9 \%). Dennoch
haben die verbleibenden 0.1 \% der restlichen Gase einen
bedeutenden Anteil an f*r Lebewesen wichtigen Klimaver*nderungen
und die zus*tzliche Freisetzung von Schadgasen aus anthropogenen
Quellen kann erhebliche Auswirkungen in den unterschiedlichsten
Bereichen, z.B. toxische Wirkung auf den Menschen, andere
Lebewesen und Pflanzen haben. Aus diesem Grund ist eine
sorgf*ltige und kontinuierliche *berwachung dieser Gase von
*u*erster Wichtigkeit. In den folgenden Abschnitten wird ein
kurzer *berblick *ber Quellen und Wirkungen von Spuren-- und
Schadgasen und den sich daraus ergebenden Anforderungen an die
Schadgasanalytik gegeben.


\section{\label{atmosph*re}Klimarelevante Spuren-- und Schadgase}

Neben den obengenannten Gasen, finden sich eine ganze Reihe von
anderen Komponenten in den verbleibenden 0.1 \% reiner, trockener
Luft. Dabei werden nat*rlich vorkommende Gase wie z.B. CO\down{2},
O\down{3}, N\down{2}O, CO, CH\down{4} und NH\down{3} \it
Spurengase \rm und Gase, die die nat*rliche Gaszusammensetzung
*ndern, \it Schadgase \rm genannt. Aufgrund der geringen
Konzentration der Spuren-- und Schadgase, wird in der
Umweltanalytik das Vorkommen dieser Bestandteile auch oft in
Volumenanteilen ppm (parts per million) oder ppb (parts per
billion) angegeben. So finden sich neben N\down{2}, O\down{2} und
Ar auch die in Tabelle \ref{treibhausgase1} aufgef*hrten
klimarelevanten Gase in der Atmosph*re (sog. \it
Treib\-haus\-ga\-se\rm ). Die Treibhausgase tragen aufgrund ihrer
vergleichsweise starken Absorption von Strahlung im IR--Bereich zu
einem positiven Strahlungsantrieb (Erw*rmung) in der unteren
Troposph*re bei. Die Daten beziehen sich dabei auf das Jahr 1996,
in Klammern sind die vorindustriellen Werte von 1800 angegeben
\cite{schoenwiese99}.\\

\tabelle{htb}{1.2}{treibhausgase1}{{|l c c c c|}\hline & \bf
\small anthropogene & \bf \small atmosph*rischer & \bf \small
Treibhauseff. & \bf \small Treibhauseff.\\
\raisebox{2ex}[-0.8ex]{\bf \small Treibhausgas} & \bf \small
Emission & \bf \small Volumenanteil & \bf \small nat*rlich & \bf
\small anthrop.\\ \hline CO\down{2} & 29 Gt a\up{-1} & 362 (280)
ppm & 22 \% & 61 \%\\ CH\down{4} & 400 Mt a\up{-1} & 1.7 (0.7) ppm
& 2.5 \% & 15 \%\\ FCKW & 0.4 Mt a\up{-1} & F12: 0.5 ($\dagger$)
ppb &
--- & 11 \%\\ N\down{2}O & 15 Mt a\up{-1} & 0.31 (0.28) ppm & 4
\% & 4 \%\\ O\down{3} & 0.5 Gt a\up{-1} ($\star$) & 30 ($\star$)
ppb & 7 \% &
<9 \%\\ H\down{2}O & (vorw. indirekt) & 2.6 (2.6) \% & 62 \% &
indirekt
\\
\hline }{\it *bersicht der wichtigsten Treibhausgase. Die
Emissionen sind in Giga-- (G) oder Mega-- (M) Tonnen pro Jahr
angegeben. Die Emissionen und Volumenanteile beziehen sich auf das
Jahr 1996 (in Klammern die vorindustriellen Werte von 1800). F*r
O\down{3} und H\down{2}O sind die bodennahen Mittelwerte
angegeben. $\star$ Werte sind nicht bekannt bzw. $\dagger$ nicht
nachweisbar (aus: \cite{schoenwiese99}).}

\tabelle{htb}{1.2}{treibhausgase2}{{|l | c | r l|}\hline
\rule[-3mm]{0mm}{9mm} \bf Treibhausgas & \bf Gesamtemission &
\multicolumn{2}{c|}{\bf Aufschl*sselung}\\ \hline CO\down{2} & 29
$\pm$ 3 Gt a\up{-1} & \; 75 \% & fossile Energie\\ & & 20 \% &
Waldrodungen\\ & &  5 \% & Brennholznutzung\\ \hline

CH\down{4} & 400 $\pm$ 80 Mt a\up{-1} & 27 \% & fossile Energie\\
& & 23 \% & Viehhaltung\\ & & 17 \% & Reisanbau\\& & 11 \% &
Biomasseverbrennung\\ & & 8 \% & M*llhalden\\ & & 8  \% &
Abwasser\\ & & 7 \% & Tierexkremente\\ \hline

FCKW & $\approx$ 0.4 Mt a\up{-1} & & Spr*hdosen, K*ltetechnik,\\ &
& & D*mmmaterial, Reinigung\\ \hline

N\down{2}O & 15 $\pm$ 8 Mt a\up{-1} & 44 \% & Bodenbearbeitung\\ &
& 22 \% &  *berd*ngung\\ & & 15 \% & Nylonproduktion\\& & 10 \% &
fossile Energie\\ & & 9 \% & Biomasseverbrennung\\ \hline

O\down{3} (untere Atm.) & $\approx$ 0.5 Gt a\up{-1} & & Verkehr,
fossile Energie u.a.\\  \hline

H\down{2}O & nicht bekannt & & hochfliegender Flugverkehr\\
\hline
 }{\it Aufschl*sselung der anthropogenen Emissionen von Treibhausgasen
  (aus: \cite{schoenwiese99}).}

Es ist aus Tabelle \ref{treibhausgase1} zu entnehmen, dass der
anthropogene Anteil am Treibhauseffekt (mit steigender Tendenz)
gr**er als der nat*rliche ist. Vor allem der CO\down{2}--Gehalt
der Atmosph*re hat sich aufgrund der Industrialisierung seit 1800
von damals 280 ppm auf heute 362 ppm gesteigert. In Tabelle
\ref{treibhausgase2} sind die anthropogenen Emissionen der
wichtigsten Treibhausgase nach Ursachen aufgeschl*sselt. Eine
Erw*rmung der Erdatmosph*re geht vor allem mit Klima*nderungen
einher. Es ist aber wichtig, dabei festzustellen, dass der Mensch
keinesfalls als alleiniger Klimafaktor betrachtet werden darf.
Neben den Treibhausgasen, die einen mittleren Strahlungsantrieb
von z.Z. etwa 2.8 Wm\up{-2} verursachen, tragen auch die erh*hten
Sonnenaktivit*ten mit z.Z. etwa 0.5 Wm\up{-2}, Vulkanausbr*che mit
etwa 3 Wm\up{-2} und das El Ni$\rm \stackrel{\sim}{n}$o--Ph*nomen
(keine Zahlen f*r den Strahlungsantrieb bekannt) zum
Treibhauseffekt bei. Das El Ni$\rm \stackrel{\sim}{n}$o--Ph*nomen,
das aufgrund bestimmter Luftdruckkonstellationen der S*dhemisph*re
episodische Erw*rmung der tropischen Ozeane, und die
Vulkanausbr*che treten dabei in unbestimmten Zyklen, eine
ver*nderte Sonnenaktivit*t mit Zyklen von etwa 11 Jahren auf.
Neben diesen positiven Strahlungsantrieben f*hrt der anthropogene
Aussto* von SO\down{2} zur Bildung von Sulfatpartikeln in der
unteren Atmosph*re und damit zu einem negativen Strahlungsantrieb
von ca. 1.5 Wm\up{-2}. Hierbei ist allerdings nur ein
uneinheitlicher Trend zu beobachten.\\

Ohne an dieser Stelle im Detail auf s*mtliche Auswirkungen einer
globalen Erw*rmung eingehen zu wollen, sei an das Schmelzen
polaren Eises und somit das Steigen des Meeresspiegels erinnert,
was zur *berschwemmung heute k*stennaher Gebiete f*hren wird.
Extreme Naturkatastrophen, die fr*her eine mittlere statistische
Wiederkehr von 100 Jahren aufwiesen, scheinen nun in mittleren
Abst*nden von f*nf Jahren wiederzukehren \cite{caspary98}.
Abgesehen von den menschlichen Opfern, die diese Naturkatastrophen
fordern, sind desweiteren die dadurch entstehenden
volkswirtschaftlichen Sch*den nur schwer abzusch*tzen, da in
solchen Zahlen auch nicht klimabedingte Naturkatastrophen wie
Erdbeben enthalten sind. Das Sch*tzen der volkswirtschaftlichen
Sch*den wird weiterhin dadurch erschwert, weil das Bebauen
gef*hrdeter Fl*chen enorm zugenommen hat.\\

Eine weitere Folge ausschlie*lich anthropogener Freisetzung von
Gasen, wie z.B. der Fluorchlorkohlenwasserstoffe (FCKW), ist die
Zerst*rung der Ozonschicht der Stratosph*re. Aufgrund der Zunahme
der UV--B--Strahlungsintensit*t sind dadurch gravierende
Auswirkungen auf Mensch, Tier und Pflanzen zu erwarten.
Gegenw*rtige Modelluntersuchungen zeigen bereits jetzt eine
Zunahme der UV--B--Dosis von ca. 7 \%/Dekade in den mittleren
Breiten der Nordhemisph*re, der auf den Abbau der Ozonschicht
zur*ckzuf*hren ist \cite{seinfeld98}.\\

Das stratosph*rische Ozon wird ausschlie*lich photochemisch
gebildet. Ohne die anthropogene Zuf*hrung von Schadgasen laufen im
wesentlichen die Ozon--\-Bild\-ungs\-re\-aktion
\begin{eqnarray}\label{eqozon1}
  \text{O}_2 \: + \: \text{h} \, \nu \: (\lambda \,\leq \, 250 \: \text{nm})  & \rightarrow & 2 \:
  \text{O}
  \nonumber\\
  \text{O} \: + \: \text{O}_2 \: + \: \text{M}  & \rightarrow & \: \text{O}_3 \: + \:
  \text{M}
\end{eqnarray}
(M ist hierbei ein inerter Sto*partner, wie N\down{2} oder
O\down{2}) und die nat*rliche Abbaureaktion
\begin{eqnarray}\label{eqozon2}
  \text{O}_3 \: + \: \text{h} \, \nu \: (\lambda \, \leq \, 300 \: \text{nm}) & \rightarrow & \text{O}_2 \: + \:
  \text{O}\nonumber\\
  \text{O} \: + \: \text{O}_3 & \rightarrow & 2 \: \text{O}_2
\end{eqnarray}
ab. Das Ozon wird auf das Niveau eines Spurengases begrenzt.\\

Katalysatoren wie die FCKW und die Halone (mittlerweile sind etwa
30 verschiedene Spezies bekannt) f*hren zu einer weiteren
Reduktion des Ozons, ohne die Konzentration des Katalysators zu
beeinflussen \cite{zellner93}. Die Katalysatoren X k*nnen sich bis
zu mehreren Jahrzehnten in der Stratosph*re halten.
\begin{eqnarray}\label{eqozon3}
 \text{X} \: + \: \text{O}_3 & \rightarrow & \text{XO} \: + \:
 \text{O}_2\nonumber\\
 \text{O} \: + \: \text{XO} & \rightarrow & \text{X} \: + \:
 \text{O}_2\\
\end{eqnarray}
Dies ergibt folgende Nettoreaktion:
\begin{eqnarray}\label{eqozon4}
  \text{O} \: + \: \text{O}_3 & \rightarrow & 2 \: \text{O}_2
\end{eqnarray}

Etwa 90 \% des Ozons befindet sich in der Stratosph*re in H*hen
von 15--35 km, wohingegen nur 10 \% des Ozons sich in der
Troposph*re befinden. Diese Mengenverteilung ist verantwortlich
f*r die Temperaturverteilung in der Atmosph*re. So nimmt die
Temperatur in der Troposph*re mit der H*he st*ndig ab, in der
Stratosph*re hingegen steigt sie mit der H*he wieder an. Dies hat
zur Folge, dass sich zwischen Troposph*re und Stratosph*re eine
Sperrschicht ausbildet, die die Vermischung von Spurengasen in die
Troposph*re verhindert. Ein Abbau der Ozonkonzentration w*rde
diese Sperrwirkung verringern. Fourier--Transform--Spektrometer
sind im *brigen schon erfolgreich bei der Bestimmung von Ozon in
der Stratosph*re eingesetzt worden. Siehe hierzu den
*bersichtsartikel von \cite{traub98}.\\

Die in diesem Abschnitt aufgezeigten starken Auswirkungen der
Spuren-- und Schadgaskonzentrationen auf Klima und Umwelt weisen
auf die enorme Bedeutung von Messsystemen zur *berwachung dieser
Gase sowohl bodennah als auch in der oberen Stratosph*re hin. Die
Industriestaaten haben sich auf der Klimakonferenz in Kyoto im
Dezember 1997 rechtsverbindlich verpflichtet, die Emission der
Klimagase bis 2012 um 5 \% zu senken, wobei das Jahr 1990 als
Basis gilt. Zu den Klimagasen z*hlte man in Kyoto nicht nur die
urspr*nglich vorgesehenen drei Gase CO\down{2}, CH\down{4} und
N\down{2}O, sondern auch die FCKW, die perfluorierten
Kohlenwasserstoffe und SF\down{6}.


\section{\label{anthropogene}Weitere anthropogene Schadstoffemissionen}

Zus*tzlich zu den bereits im vorherigen Abschnitt aufgef*hrten
klimarelevanten Gasen, gibt es eine ganze Reihe weiterer Schadgase
anthropogenen Ursprungs, die bei Freisetzung in die Atmosph*re
stark toxisch auf Lebewesen und Pflanzen wirken k*nnen.
Unterschieden wird hier zwischen Gasen prim*rer und sekund*rer
Art. Die prim*ren Schadgase werden direkt in die offene Atmosph*re
emittiert, wohingegen die sekund*ren Schadgase aufgrund der
Reaktion von prim*ren Schadgasen untereinander und mit
Atmosph*rengasen entstehen. H*ufig sind dabei die sekund*ren
Schadgase wie z.B. Gase, die die Peroxyacetylnitrat--Gruppe (PAN)
enthalten, weit toxischer als die prim*ren Schadgase.\\

Die Hauptquellen industrieller Schadgasemissionen sind Fabriken,
die zur Produktion fossile Brennstoffe einsetzen (z.B. Stahlwerke,
Heizkraftwerke) und gerade in L*ndern mit geringen
Umweltschutzauflagen gro*e Mengen an Schwefeldioxid,
Kohlenmonoxid, Eisenoxid, teilweise auch Fluoriden und gro*e
Mengen an Asche und Staub emittieren. Hinzu kommen Raffinerien,
die vor allem Kohlenwasserstoffe und andere organische Komponenten
wie Amine, Mercaptane, Sulfide, Aldehyde, Ketone, Alkohole, S*uren
und Chlorkohlenwasserstoffe aussto*en. Ein weiterer gro*er
Emittent von Schadstoffen ist die chemische Industrie. Die Gr*nde
f*r die Emission von Schadstoffen k*nnen dabei recht vielf*ltig
sein. Sie gehen von der ungen*genden Umsetzung der
Ausgangsprodukte bis hin zu Leckagen an den Produktionsanlagen.
Vor allem bei St*rf*llen werden gro*e Mengen an Schadgasen
unkontrolliert freigesetzt. Eine weitere Quelle hohen
Schadgasaufkommens ist insbesondere in Sto*zeiten der
Stra*enverkehr, die Viehhaltung, D*ngung und Pestizidbek*mpfung in
der Landwirtschaft oder auch die vor allem in trockenen Regionen
oft auftretenden Waldbr*nde.\\

Die Zusammensetzung und Aufenthaltsdauer der Schadgase in der
Atmosph*re ist sehr unterschiedlich. Grunds*tzlich sammeln sich
die Schadgase an, wechselwirken miteinander, hydrolysieren und
oxidieren unter dem Einfluss von Feuchte und Sauerstoff und *ndern
ihre Zusammensetzung aufgrund von Strahlung. Infolgedessen ist die
Aufenthaltsdauer der Schadgase in der Atmosph*re durch ihre
chemischen Eigenschaften bestimmt. Diese betr*gt f*r
Schwefeldioxid 4 Tage, 2 Tage f*r Schwefelwasserstoff, 5 Tage f*r
Stickstoffoxid und 7 Tage f*r Ammoniak, wohingegen Methan und
Kohlenmonoxid inert sind und 3 Jahre und l*nger in der Atmosph*re
verbleiben.\\

Seit Anfang der 60er Jahre existiert in der Bundesrepublik
Deutschland ein stetig gewachsenes Netz an Punktmessstationen zur
fl*chendeckenden *berwachung ausgew*hlter Schadstoffe. Diese Daten
k*nnen den jeweiligen Jahresberichten der Landesumwelt*mter und
des Bundesumweltamtes oder aktuell im Internet (f*r NRW:
www.lua.nrw.de) entnommen werden. Abgesehen davon, dass diese
*berwachung nur wenige Schadgase erfasst, k*nnen damit in der
weitaus gr**ten Mehrzahl der F*lle keine R*ckschl*sse auf den
Emittenten gezogen werden.\\

Zu erw*hnen bleibt, dass neben den atmosph*rischen
Schadgasmessungen auch die Messung toxischer Gase am Arbeitsplatz,
vor allem auch in weitr*umigen Produktionshallen, f*r die
Gesunderhaltung des Menschen von gro*er Bedeutung ist. F*r die
Schadstoffe werden dabei jeweils maximale
Arbeitsplatz--Konzentrationen (MAK--Werte) angegeben. Weitere
Faktoren wie z.B. das aufgrund der Schwere der k*rperlichen Arbeit
abh*ngige Atemminutenvolumen k*nnen zudem ber*cksichtigt werden
(biologische Arbeitsstoff--Toleranz Werte, (BAT)--Werte)
\cite{dfg99}, \cite{heise96}. Eine Spezialanwendung dieser
Arbeitsplatzmessungen ist z.B. die Raumluft*berwachung in
Raumschiffen \cite{johansen97}.\\


\section{\label{anforderungen}Anforderungen an die
Spuren-- und Schadgasanalytik}

Die oben aufgef*hrten Beispiele zeigen, dass die Anforderungen an
die Spuren-- und Schadgasanalytik au*erordentlich hoch sind.
Aufgrund der im Vergleich zur reinen Luft sehr geringen Anteile
der Spuren-- und Schadgase sind sehr niedrige Nachweisgrenzen zu
fordern, die gerade auch bei den schon existierenden Messmethoden
weiter zu verbessern sind. Es ist hierbei zwischen Punkt-- und
integrierenden Messmethoden zu unterscheiden. Punktmessmethoden
k*nnen dann eingesetzt werden, wenn das zu untersuchende
Gasvolumen homogen von dem betrachteten Gas durchsetzt ist.
Andernfalls ist mit vorwiegend optischen Messmethoden *ber
ad*quate Messstrecken zu integrieren. Im aktuellen \it Konzept f*r
vorrangige mittelfristige deutsche Atmosph*renforschung \rm des
Bundesministeriums f*r Bildung und Forschung \cite{bmbf99} hei*t
es hierzu:\

\it "`Die Entwicklung und Nutzung geeigneter Me*verfahren stellt
ebenfalls einen wesentlichen Aufgabenbereich dar, der bei allen
vorrangigen atmosph*rischen Forschungsarbeiten beachtet werden
sollte... Dazu k*nnen Feldmessungen (Mesoskala) mit in situ-- und
Fernerkundungs--Me*ger*ten sowie Satellitenmessungen (globale
Skala) wesentlich beitragen."'\rm\\

Hierbei sind Messverfahren aus Kosten-- und
Praktikabilit*tsgr*nden klar im Vorteil, die die Konzentration
m*glichst vieler Gase simultan bestimmen k*nnen. Es w*re zu
w*nschen, dass die Messverfahren nicht nur am Boden, sondern auch
in Flugzeugen oder Raumschiffen eingesetzt werden k*nnen, um ein
repr*sentatives Konzentrationsh*henprofil der einzelnen Gase zu
erhalten (zu Fourier--Transform Messungen von Schadgasen in der
Stratosph*re siehe z.B. den *bersichtsartikel \cite{farmer97}).\\

F*r die Atmosph*renforschung sind zudem verl*ssliche
Ausbreitungsmodelle vonn*ten, die den Transport der Gase in der
Troposph*re und Stratosph*re beschreiben. In Kombination mit
Wetterdaten m*ssen die eingesetzten Messverfahren hierbei
verl*ssliches Datenmaterial an Konzentrationswerten *ber
verschiedene Messwegl*ngen liefern, um Quellen und Senken, aber
auch die homogene Vermischung der Schadgase in den
Ausbreitungsmodellen ansetzen zu k*nnen. Fernerkundungsmessungen
k*nnen bei der Aufstellung von Ausbreitungsmodellen auch helfen,
*ber gr**ere Fl*che zu integrieren und damit *bertragungen auf
schlecht untersuchte Gebiete der Erde zu erm*glichen.\\

Das in dieser Arbeit vor allem bzgl. der quantitativen Auswertung
weiter verbesserte FTIR--Fernerkundungsmessverfahren erf*llt eine
ganze Reihe der oben genannten Forderungen.\\


\cleardoublepage
