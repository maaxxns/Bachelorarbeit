\begin{appendix}

\chapter{\label{mathcad}LPSVD--Auswertungsprogramm}
\section{Implementierung in MathCad 5.0 Plus}

Die in Kap. \ref{barkh} beschriebenen Rechnungen wurden mit \it 
MathCad 5.0 Plus \rm durchgef�hrt.
Dadurch ist die Anzahl der einzulesenden Me�punkte auf 180 beschr�nkt. Da die
Me�punkte 3.65 ps auseinanderliegen (Kanalabstand des MCA), kann nur jeder zweite
Punkt ber�cksichtigt werden, um einen Zeitbereich von 1314 ps abzudecken. 
Dies ist jedoch in jedem Fall ausreichend.\\

Die Beispielabbildungen \ref{eigenw1} und \ref{eigenw2} stellen die Auswertungsergebnisse f�r 
die gleiche Messung dar. Im Signal--Rausch--Verh�ltnis liegt diese Messung
im Durchschnitt ($\approx$ 18 dB) aller im Rahmen dieser Diplomarbeit gemessenen 
Zeitkurven. F�r
Zeitkurven mit schlechterem Signal--Rausch--Verh�ltnis, f�r die die Abtrennung
der zum Signal geh�rigen Werte nicht so eindeutig wie in Abb. \ref{eigenw1} und 
Abb. \ref{eigenw2} vorgenommen werden kann, kann durch Variieren der zu ber�cksichtigenden
Eigenwerte (und dem Vergleich des mittleren Fehlerquadrates von gemessener und
berechneter Zeitkurve) sehr schnell der optimale Wert gefunden werden.\\

Auf den folgenden Seiten ist das Programm dokumentiert, mit dem die
Rechnungen durchgef�hrt wurden. Als Me�file wird dabei das Quantum--Beat--Signal
des freien Exzitons bei einem Magnetfeld von 3 Tesla gew�hlt (vgl. Tab. 
\ref{auswertxt}).

\clearpage
\ \ \

\clearpage
\ \ \

\clearpage
\ \ \

\clearpage
\ \ \

\clearpage
\ \ \

\clearpage
\ \ \

\clearpage
\ \ \
\end{appendix}